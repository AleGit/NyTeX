% !TeX encoding = UTF-8

%% ==============================================================
%% ### MY MATH ENVIRONMENTS ###

\theoremstyle{plain}
%\newtheorem{theorem}{Theorem}			% predefined in CL?
%\newtheorem{proposition}{Proposition}	% predefined in CL?
%\newtheorem{lemma}{Lemma}				% predefined in CL?
%\newtheorem*{corollary}{Corollary}		% predefined in CL?

\theoremstyle{definition}
%\newtheorem{definition}{Definition}	% predefined in CL?
\newtheorem{conjecture}{Conjecture}
%\newtheorem*{example}{Example}			% predefined in CL?
%\newtheorem{algorithm}{Algorithm}		% predefined in CL
\newtheorem{procedure}{Procedure}
\newtheorem{goal}{Goal}
\newtheorem{notation}{Notation}

\theoremstyle{remark}
\newtheorem*{remark}{Remark}			% predefined in CL?
\newtheorem*{observation}{Observation}
%\newtheorem*{note}{Note}
%\newtheorem{case}{Case}

%% ==============================================================
%% ### MY MATH DEFINITIONS ###

% math alphabets

\DeclareMathAlphabet{\mathpzc}{OT1}{pzc}{m}{it}	% \mathpzc
\DeclareMathAlphabet{\mathcll}{T1}{calligra}{m}{n}

% math operators

\DeclareMathOperator{\arity}{arity}		% arity of a symbol

\DeclareMathOperator{\var}{\mathpzc{Vars}}			% variables of a term
\DeclareMathOperator{\fun}{\mathpzc{Funs}}			% function symbols of a term
\DeclareMathOperator{\pos}{\mathpzc{Pos}}			% positions in a term
\DeclareMathOperator{\posT}{\mathpzc{{t-}Pos}}			% positions in a term
\DeclareMathOperator{\posS}{\mathpzc{Pos^F}}
\DeclareMathOperator{\fvar}{\mathpzc{Fvars}}
\DeclareMathOperator{\bvar}{\mathpzc{Bvars}}

\DeclareMathOperator{\domain}{dom}			% domain of an assignment
\DeclareMathOperator{\range}{rng}		% range of an assignment
\DeclareMathOperator{\image}{img}			% image of an assignment

\DeclareMathOperator{\mgu}{mgu}			% most general unifier
\DeclareMathOperator{\sel}{sel}			% selection function

%\DeclareMathOperator{\mul}{mul}
%\DeclareMathOperator{\add}{add}
\DeclareMathOperator{\head}{head}
\DeclareMathOperator{\tail}{tail}
\DeclareMathOperator{\length}{length}



\DeclareMathOperator{\UNIF}{unifiable}
\DeclareMathOperator{\INST}{instance}
\DeclareMathOperator{\GNRL}{generalization}
\DeclareMathOperator{\VRNT}{variant}
\DeclareMathOperator{\PSTR}{pstr}

%\DeclareMathOperator{\subterms}{\mathpzc{Subterms}}	
%\DeclareMathOperator{\termsize}{size}
%\DeclareMathOperator{\symbols}{symbols}
%\DeclareMathOperator{\subforms}{\mathpzc{Subforms}}	