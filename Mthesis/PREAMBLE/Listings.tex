% !TeX encoding = UTF-8
% !TeX spellcheck = en_US

\lstdefinelanguage{smtlib}{
	comment={[l];},
	keywords={assert,xor, or, and},
	otherkeywords={declare-fun, set-logic},
	emph={Int,QF_LIA},
}

\lstdefinelanguage{Yices}{
	language = C,
	morekeywords={},
%	comment={[l];},
%	keywords={assert,xor, or, and},
	keywords=[2]{type_t, uint32_t, term_t},
%	emph={Int,QF_LIA},
%	otherkeywords={yices_bool_type}
}

\lstdefinelanguage{flea}{
	language = C,
%	morekeywords={},
	%	comment={[l];},
	%	keywords={assert,xor, or, and}, 	%
	keywords=[2]{func, var, let, protocol, associatedtype, extension},	% Swift Keywords
	keywords=[3]{String, UInt32},					% Swift data types
	keywords=[4]{type_t, uint32_t, term_t,},		% yices data types
	%%% begin yices API keywords %%%
	keywords=[5]{yices_get_type_by_name, yices_new_uninterpreted_term, yices_new_uninterpreted_type}
	%%% end yices API keywords %%%
	%	emph={Int,QF_LIA},
	%	otherkeywords={yices_bool_type}
}

\lstset{
	backgroundcolor=\color{white},     	% choose the background color; you must add \usepackage{color} or \usepackage{xcolor}
	basicstyle=\ttfamily\footnotesize, 	% the size of the fonts that are used for the code
	breakatwhitespace=false,         	% sets if automatic breaks should only happen at whitespace
	breaklines=true,                 	% sets automatic line breaking
	caption=\lstname,
	captionpos=b,                    	% sets the caption-position to bottom
	commentstyle=\color{gray},    		% comment style
	deletekeywords={},            		% if you want to delete keywords from the given language
	emphstyle=\color{orange},
	escapeinside={\%*}{*)},         % if you want to add LaTeX within your code
	extendedchars=true,             % lets you use non-ASCII characters; for 8-bits encodings only, does not work with UTF-8
	frame=none,		% frame=single, % adds a frame around the code
	keepspaces=true,                % keeps spaces in text, useful for keeping indentation of code (possibly needs columns=flexible)
	keywordstyle=\color{OliveGreen},      % keyword style
	keywordstyle=[2]\color{Fuchsia},
	keywordstyle=[3]\color{RedViolet},
	keywordstyle=[4]\color{RoyalBlue},	% yices data types
	keywordstyle=[5]\color{NavyBlue},	% yices api
	% language=smtlib,	%Octave,    % the language of the code
	% literate={;},
	% otherkeywords={declare-fun,set-logic,assert,xor,or,and},            % if you want to add more keywords to the set
	%morecomment=[l]{;}				% line comment
	numbers=left,                   % where to put the line-numbers; possible values are (none, left, right)
	numbersep=5pt,                  % how far the line-numbers are from the code
	numberstyle=\tiny\color{gray}, 	% the style that is used for the line-numbers
	rulecolor=\color{black},        % if not set, the frame-color may be changed on line-breaks within not-black text (e.g. comments (green here))
	showspaces=false,               % show spaces everywhere adding particular underscores; it overrides 'showstringspaces'
	showstringspaces=false,         % underline spaces within strings only
	showtabs=false,                 % show tabs within strings adding particular underscores
	stepnumber=1,                   % the step between two line-numbers. If it's 1, each line will be numbered
	stringstyle=\color{orange},     % string literal style
	tabsize=2,	                   	% sets default tabsize to 2 spaces
%	title=\lstname,                  % show the filename of files included with \lstinputlisting; also try caption instead of title,
	mathescape=true
}
