% !TeX root = ../mthesis.tex
% !TeX encoding = UTF-8
% !TeX spellcheck = en_US

\begin{table}[hbt]
\begin{gather*}
\begin{array}{ccccc}
\infer[(\land i)]{F\land G}{F & G} 
&
\infer[(\land e_1)]{G}{F \land G} 
&
\infer[(\land e_2)]{F}{F \land G}
&
\infer[(\lnot\lnot i)]{\lnot\lnot F}{F} 
&
\infer[(\lnot\lnot e)]{F}{\lnot\lnot F}
\\[0.7em]
\infer[(\bot e)]{F}{\bot}
&
\infer[(\lnot e)]{\bot}{F & \lnot F}
&
\infer[\text{LEM}]{F\lor\lnot F}{}
&
\infer[(\lor i_1)]{F\lor G}{F}
&
\infer[(\lor i_2)]{F\lor G}{G}
\\[0.7em]
\infer[(\lnot i)]{\lnot F}{
	\boxed{\begin{array}{c}F\\\vdots\\\bot\end{array}}}
&
\infer[\text{PBC}]{F}{
	\boxed{\begin{array}{c}\lnot F\\\vdots\\\bot\end{array}}}
&
\infer[({\limp} i)]{F\limp G}{
	\boxed{\begin{array}{c}F\\\vdots\\G\end{array}
}}
&
\multicolumn{2}{l}{
	\infer[(\lor e)]{H}{
		F\lor G &
		\boxed{\begin{array}{c}F\\\vdots\\H\end{array}} &
		\boxed{\begin{array}{c}G\\\vdots\\H\end{array}}
	}	
}
\\[0.7em]
&
\multicolumn{3}{r}{
\infer[\text{modus}\atop\text{ponens}]{G}{F & F\limp G}
\qquad
\infer[\text{modus}\atop\text{tollens}]{\lnot F}{F\limp G & \lnot G}
}
&
\end{array}
\end{gather*}
\caption{Natural Deduction Rules for Connectives}
\label{tab:natural:deduction:connectives}
\end{table}
%\begin{itemize}
%	\item[($\land i$)] If $F$ and $G$ hold we may introduce the conjunction $F\land G$.
%	\item[($\land e$)] If $F \land G$ holds we may eliminate the conjunction and keep $F$ or $G$.
%	\item[($\lnot\lnot$)] We may introduce or eliminate double negation at any time.
%	\item[($\bot e$)] Ex falso sequitur quodlibet. 
%	\item[($\lnot e$)] Whenever $F$ and $\lnot F$ hold then we can introduce the contradiction.
%	\item[(LEM)] Tertium non datur. Law of excluded middle.
%	\item[($\lor i$)] If $F$ or $G$ hold we may introduce the disjunction $F\lor G$.
%	\item[($\lnot i$)] If we assume $F$ and conclude the contradiction, then we can introduce $\lnot F$.
%	\item[(PBC)] Proof by contradiction. If we assume $\lnot F$ and conclude the contradiction, then we can introduce $F$.
%	$(\lnot i,\lnot\lnot e)$. 
%	\item[($\limp i$)] If we assume $F$ and conclude $G$, then we can introduce the implication $F\limp G$.
%	\item[($\lor e$)] If $F\lor G$ holds and H is a conclusion of F, and H is a conclusion of G, then we can introduce $H$.
%	\item[(MP)] If $F$ and the implication $F\limp G$ hold, then we can introduce $G$.
%	\item[(MT)] If $\lnot G$ and the implication $F\limp G$ hold, then we can introduce $\lnot F$.
%\end{itemize}

\begin{table}[hbt]
	\begin{gather*}
	\infer[({=}i)]{t=t}{}
	\qquad
	\infer[({=}e)]{F\{x\mapsto t\}}{s=t & F\{x\mapsto s\}}
	\end{gather*}
	\caption{Natural Deduction Rules for Equality}
	\label{tab:natural:deduction:equality}
\end{table}

\begin{table}[hbt]
	\begin{gather*}
	\begin{array}{ccc}
	\infer[(\forall e)]{F\{x\to t\}}{
		\forall x F
	}
	&&
		\infer[(\exists i)]{\exists x F}{
		F\{x\mapsto x_0 \}
	}
	\\[1em]
	\infer[(\forall i)]{\forall x F}{
		\boxed{\begin{array}{cc}x_0\\&\vdots\\&F\{x\mapsto x_0\}\end{array}}
	}
	&&
	\infer[(\exists e)]{H}{
		\exists x F &
		\boxed{\begin{array}{cc}x_0&F\{x\mapsto x_0 \}\\&\vdots\\&H\end{array}}
	}
	\end{array}
	\end{gather*}
	\caption{Natural Deduction Rules for Quantifiers}
	\label{tab:natural:deduction:quantifiers}
\end{table}