% !TEX root = ../m2report.tex
% !TEX encoding = UTF-8 Unicode

%% ==============================================================


%% ==============================================================
%% ### MY COLOR DEFINITONS ###

\colorlet{col:em}{Blue}

\colorlet{col:a}{Fuchsia}
\colorlet{col:b}{Blue}
\colorlet{col:g}{Gray}
\colorlet{col:o}{Orange}
\colorlet{col:hi}{Green}
\colorlet{col:lo}{Red}
\colorlet{col:na}{Gray}
\colorlet{col:n}{Blue}
\colorlet{col:max}{Violet}
\colorlet{col:smx}{RoyalBlue}

\newcommand{\colA}{\color{col:a}}	% example a
\newcommand{\colB}{\color{col:b}}	% example b
\newcommand{\colG}{\color{col:g}}	% neutral
\newcommand{\colO}{\color{col:o}}	% old
\newcommand{\colN}{\color{col:n}}	% new
\newcommand{\colHI}{\color{col:hi}}	% hi, true
\newcommand{\colLO}{\color{col:lo}}	% lo, false
\newcommand{\colNA}{\color{col:na}}	% not available
\newcommand{\colMAX}{\color{col:max}}	% maximal
\newcommand{\colSMX}{\color{col:smx}}	% strictly maximal

%\newcommand{\MYA}[1]{{\color{col:a}#1}}
%\newcommand{\MYB}[1]{{\color{col:b}#1}}
%\newcommand{\MYG}[1]{{\color{col:g}#1}}
%\newcommand{\MYO}[1]{{\color{col:o}#1}}

%\newcommand{\MYHI}[1]{{\color{col:hi}#1}}
%\newcommand{\MYLO}[1]{{\color{col:lo}#1}}
%\newcommand{\MYNA}[1]{{\color{col:na}#1}}
%
%\newcommand{\MYS}[1]{{\color{Violet}#1}}
%\newcommand{\MYM}[1]{{\color{RoyalBlue}#1}}


%\colorlet{cMAX}{TealBlue}
%\colorlet{col:SAT}{RoyalBlue}
\colorlet{cSAT}{RoyalBlue}
%\colorlet{col:GEN}{Mulberry}
\colorlet{cUSP}{Purple}
\newcommand{\MaxSMT}{\textbf{\textcolor{TealBlue}{MAX}\textcolor{RoyalBlue}{SMT}}\!}
\newcommand{\MaxC}{\textbf{\textcolor{TealBlue}{Maximal}\textcolor{RoyalBlue}{\\Completion}}\!}


%% ==============================================================

\newcommand{\myem}{\em\color{col:em}} 	%\newcommand{\MYEM}[1]{\emph{\MYM{#1}}}

%% ==============================================================
%% ### MISC ###

\newcommand{\mLightning}{{\text{\Lightning}}}

\newcommand{\EQ}{\simeq}
\newcommand{\NE}{\not\simeq}

\newcommand{\foEQ}{\approx}		%	{\simeq}
\newcommand{\foNEQ}{\not\foEQ}
\newcommand{\trEQ}{\approx}
\newcommand{\trNEQ}{\not\trEQ}
\newcommand{\sEQ}{=}
\newcommand{\sNEQ}{\neq}

\newcommand{\disjointunion}{\uplus}

\newcommand{\myc}{\!:\!}

\newcommand\mytop[2]{\genfrac{}{}{0pt}{}{#1}{#2}}
\newcommand{\mygreek}[1]{\selectlanguage{polutonikogreek}#1\selectlanguage{english}}
%\newcommand{\mygreek}[1]{{\selectlanguage{polutonikogreek}#1}\selectlanguage{english}}
%\renewcommand{\mygreek}[1]{\foreignlanguage{polutonikogreek}{#1}}

\newcommand{\iSUB}[2]{#2\!\mapsto\!#1}
%\newcommand{\BgSyntaxTree}{\usebackgroundtemplate{\transparent{0.1}\includegraphics[width=\paperwidth]{SyntaxTreeBackground.png}}}




%% ==============================================================
%% ### MY MATH ENVIRONMENTS ###

% CL theorem environments: corollary, definition, example, exercise, lemma, proof, proposition, theorem

\theoremstyle{plain}

\theoremstyle{definition}
\newtheorem*{mydescription}{Description}

\theoremstyle{remark}
\newtheorem*{remark}{Remark}


%% ==============================================================
%% ### MY MATH DEFINITIONS ###

% constant (function, predicate) symbols

\newcommand{\mf}{\mathsf f}
\newcommand{\mg}{\mathsf g}
\newcommand{\mh}{\mathsf h}
\newcommand{\ma}{\mathsf a}
\newcommand{\mb}{\mathsf b}
\newcommand{\mc}{\mathsf c}
\newcommand{\md}{\mathsf d}
\newcommand{\mx}{\mathsf x}
\newcommand{\my}{\mathsf y}
\newcommand{\msucc}{\mathsf s}
\newcommand{\mpred}{\mathsf p}
\newcommand{\mA}{\mathsf A}
\newcommand{\mB}{\mathsf B}
\newcommand{\mF}{\mathsf F}
\newcommand{\mL}{\mathsf L}
\newcommand{\mP}{\mathsf P}
\newcommand{\mQ}{\mathsf Q}

% caligraphic symbols

\newcommand{\mcA}{\mathcal A}
\newcommand{\mcB}{\mathcal B}
\newcommand{\mcC}{\mathcal C}
\newcommand{\mcD}{\mathcal D}
\newcommand{\mcE}{\mathcal E}
\newcommand{\mcF}{\mathcal F}
\newcommand{\mcG}{\mathcal G}
\newcommand{\mcI}{\mathcal I}
\newcommand{\mcL}{\mathcal L}
\newcommand{\mcM}{\mathcal M}
\newcommand{\mcP}{\mathcal P}
\newcommand{\mcR}{\mathcal R}		% no need for extra { ... }
\newcommand{\mcRr}{\dot{\mcR}}
\newcommand{\mcRrr}{\ddot{\mcR}}
\newcommand{\mcS}{\mathcal S}
\newcommand{\mcT}{\mathcal T}
\newcommand{\mcV}{\mathcal V}



% math operators

\DeclareMathOperator{\arity}{arity}

\DeclareMathOperator{\var}{{\mcV}ar}
\DeclareMathOperator{\pos}{{\mcP}os}
\DeclareMathOperator{\dom}{{\mcD}om}
\DeclareMathOperator{\fun}{{\mcF}un}
\DeclareMathOperator{\subterms}{{\mcS}ub}
\DeclareMathOperator{\depth}{depth}



\DeclareMathOperator{\T}{T}
\DeclareMathOperator{\rng}{rng}
\DeclareMathOperator{\img}{img}
\DeclareMathOperator{\mgu}{mgu}
\DeclareMathOperator{\wgt}{W\!}
\DeclareMathOperator{\sel}{sel}
\DeclareMathOperator{\mul}{mul}
\DeclareMathOperator{\add}{add}
\DeclareMathOperator{\roo}{root}
\DeclareMathOperator{\NF}{NF}


\newcommand{\VAR}{\mcV\mathtt{ar}}
% fraktal symbols

\newcommand{\mfC}{{\mathfrak C}}
\newcommand{\mfE}{{\mathfrak E}}
\newcommand{\mfL}{{\mathfrak L}}
\newcommand{\mfR}{{\mathfrak R}}
\newcommand{\mfS}{{\mathfrak S}}
\newcommand{\mfT}{{\mathfrak T}}

% tt symbols

\newcommand{\mtS}{{\mathtt S}}
\newcommand{\sgr}{\succ_{\mathsf gr}}

% combined symbols
\newcommand{\fP}{\mathpzc{f}}
\newcommand{\gQ}{\mathpzc{g}}

\newcommand{\mcTFV}{{\mcT(\mcFf,\!\mcV)}}
\newcommand{\mcFf}{{\mcF_\mf}}
\newcommand{\mcFP}{{\mcF_\mP}}
\newcommand{\mcFfP}{\mcFf\,\disjointunion\,\mcFP}
\newcommand{\mfA}{{\mf\!_\mcA}}
\newcommand{\mPA}{{\mP\!\!_\mcA}}
%

\newcommand{\joins}{\rightarrow^*\cdot^*\!\!\leftarrow}
\newcommand{\meets}{^*\!\!\leftarrow \cdot \rightarrow^* }

\newcommand{\mNF}[1]{\mathsf{NF}(#1)}	% Normal Form
\newcommand{\mCP}[1]{\mathsf{CP}(#1)}		% Critical Pair
\newcommand{\mNFR}{\mNF{\mcR}}		% set of normal forms
\newcommand{\mCPR}{\mCP{\mcR}}		% set of critical pairs

\newcommand{\MUL}[2]	% multiplication
{\mf(#1,#2)}			% mul(x,y)
%{#1\cdot #2}			% x.y

\newcommand{\ADD}[2]	% addition
{\add(#1,#2)}			% add(x,y)
%{#1+#2}				% x+y

\newcommand{\MYPOS}[1]{{\mathtt #1}}
\newcommand{\overlap}[3]{\langle #1,\MYPOS{#2}, #3 \rangle}
\newcommand{\overlapN}[4]{{_{\overlap{#1}{#2}{#3}}}^{#4:\;}}

\newcommand{\gtkbo}{\succ_\texttt{kbo}}
\newcommand{\ltkbo}{\prec_\texttt{kbo}}
\newcommand{\gtlpo}{\succ_\texttt{lpo}}
\newcommand{\gtpre}{\succ}
\newcommand{\gtpreq}{\succcurlyeq}
\newcommand{\ngtpreq}{\not\succcurlyeq}
\newcommand{\GTKBOW}[2]{\texttt{SMT}(#1\!\succ_\texttt{kbo}\!#2)}
\newcommand{\GTKBOP}[2]{\texttt{SMT}(#1\!\succ_\texttt{kbo}'\!#2)}


\newcommand\encompeq{\mathrel{\ooalign{\( \trianglerighteq \)\cr\hidewidth\raise.1em\hbox{\( \cdot\mkern6mu \)}\cr}}}
\newcommand\encomp{\mathrel{\ooalign{\( \vartriangleright \)\cr\hidewidth\hbox{\( \cdot\mkern6mu \)}\cr}}}
\newcommand\nencompeq{\mathrel{\ooalign{\( \ntrianglerighteq \)\cr\hidewidth\raise.1em\hbox{\( \cdot\mkern6mu \)}\cr}}}
\newcommand\nencomp{\mathrel{\ooalign{\( \ntriangleright \)\cr\hidewidth\hbox{\( \cdot\mkern6mu \)}\cr}}}

%\newcommand{\encompeq}{\unrhd}
%\newcommand{\encomp}{\rhd}
%\newcommand{\nencompeq}{\ntrianglerighteq}
%\newcommand{\nencomp}{\ntriangleright}

%\newcommand{\rewrites}[2]{\overset{#1,#2}{\underset{}\longrightarrow}}
%\newcommand{\rewrites}[2]{\rightarrow_{#1,\cdot,#2}}

\newcommand{\UPL}{\infer
	[(\sigma)
		\quad\sigma=\mgu(l,l'), l'\!\not\in\mcV, l\sigma\rho\sgr r\sigma\rho
	]
	{L[r]\sigma}
	{l=r & L[l']}
}

\newcommand{\myemptyset}{\varnothing}
\newcommand{\emptyclause}{\square}
\newcommand{\ctxhole}{\boxdot}			% \square
\newcommand\binary{\mathrel{\odot}}
\newcommand\relation{\mathrel{\circledast}}	% \currency

\newcommand{\InstGen}{\textsf{Inst-Gen}\xspace}
\newcommand{\InstGenEq}{\textsf{Inst-Gen-Eq}\xspace}
\newcommand{\Maxcomp}{\textsf{Maxcomp}\xspace}
\newcommand{\Yices}{\textsf{Yices}\xspace}
\newcommand{\iProver}{\textsf{iProver}\xspace}
\newcommand{\iProverEq}{\textsf{iProver-Eq}\xspace}
\newcommand{\KBO}{\textsf{KBO}\xspace}
\newcommand{\LPO}{\textsf{LPO}\xspace}
\newcommand{\SAT}{\textsf{SAT}\xspace}
\newcommand{\SMT}{\textsf{SMT}\xspace}
%% ==============================================================
%% ### TIKZ ###

\usepackage{tikz}
\usetikzlibrary{shapes}

\tikzstyle{myrect} = [rectangle,draw=black,rounded corners, minimum height=3em, thick, text centered,text width=5.5em]
\tikzstyle{mykaro} = [diamond,draw=black,rounded corners, thick, text centered,text width=4em]
\tikzstyle{mycircle} = [circle,draw=black,thick, text centered, minimum height=3.5em,text width=4em,text width=3.5em]
\tikzstyle{myarrow} = [thick,->,>=stealth]

\tikzstyle{myframe} = [rectangle,draw=black,rounded corners, minimum height=3em, thick, text centered,text width=5.5em]

%% ==============================================================

%\author{Alexander Maringele}
%\title[InstGenEq meets Maxcomp]{Instantiation-Based Theorem Proving\\
%%\( \uparrow \)
%%\( \joins \)
%\( \meets \)
%\\Maximal Completion}
%%\subtitle{Dominik Klein and Nao Horkawa}
%\date{June 10th, 2015}
%
%%% ==============================================================

\lstdefinelanguage{smtlib}{
	comment={[l];},
	keywords={assert,xor, or, and},
	otherkeywords={declare-fun, set-logic},
	emph={Int,QF_LIA},
}

\lstset {
backgroundcolor=\color{white},   % choose the background color; you must add \usepackage{color} or \usepackage{xcolor}
 basicstyle=\ttfamily\footnotesize,        % the size of the fonts that are used for the code
 breakatwhitespace=false,         % sets if automatic breaks should only happen at whitespace
 breaklines=true,                 % sets automatic line breaking
 captionpos=b,                    % sets the caption-position to bottom
 commentstyle=\color{gray},    % comment style
 deletekeywords={...},            % if you want to delete keywords from the given language
 emphstyle=\color{orange},
 escapeinside={\%*}{*)},          % if you want to add LaTeX within your code
 extendedchars=true,              % lets you use non-ASCII characters; for 8-bits encodings only, does not work with UTF-8
 frame=none,	% frame=single,	                   % adds a frame around the code
 keepspaces=true,                 % keeps spaces in text, useful for keeping indentation of code (possibly needs columns=flexible)
 keywordstyle=\color{blue},       % keyword style
% language=smtlib,	%Octave,                 % the language of the code
% literate={;},
% otherkeywords={declare-fun,set-logic,assert,xor,or,and},            % if you want to add more keywords to the set
%morecomment=[l]{;}	% line comment
 numbers=left,                    % where to put the line-numbers; possible values are (none, left, right)
 numbersep=5pt,                   % how far the line-numbers are from the code
 numberstyle=\tiny\color{gray}, % the style that is used for the line-numbers
 rulecolor=\color{black},         % if not set, the frame-color may be changed on line-breaks within not-black text (e.g. comments (green here))
 showspaces=false,                % show spaces everywhere adding particular underscores; it overrides 'showstringspaces'
 showstringspaces=false,          % underline spaces within strings only
 showtabs=false,                  % show tabs within strings adding particular underscores
 stepnumber=2,                    % the step between two line-numbers. If it's 1, each line will be numbered
 stringstyle=\color{orange},     % string literal style
 tabsize=2,	                   % sets default tabsize to 2 spaces
 title=\lstname                   % show the filename of files included with \lstinputlisting; also try caption instead of title
}
