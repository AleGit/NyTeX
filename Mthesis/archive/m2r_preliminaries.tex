% !TEX encoding = UTF-8 Unicode
% !TEX root = ../m2report.tex
%
We assume familiarity with propositional and predicate logic \cite{Huth:2004:LCS:975331},
automated theorem proving \cite{Fitting:1996:FLA:230183},
term rewriting \cite{Baader:1998:TR:280474},
decision procedures \cite{Kroening:2008:DPA:1391237},
and (maximum) satisfiability testing \cite{Biere:2009:HSV:1550723}.
Even so, for clarity we introduce
definitions and state lemmas and theorems without proofs in this section.
% to be used in the following sections.
Notions and notations follow (with deviations) the lecture notes \cite{AM2015tr} and \cite{GM2013ar}.
%____________________________________________________________________________________________________________
%\subsection{Signatures}\label{sec:signature}
%% DEF %%
\begin{definition}\label{def:signature}
A \coloremph{signature} \( \mcF=\mcFfP \) is the disjoint union of \coloremph{function symbols} \( \mcFf \)
and \coloremph{predicate symbols} \( \mcFP \).
The \coloremph{arity} of a symbol is the number of its arguments.
With \( \mcF^{(i)} \) we denote symbols with arity \( i \).
Function symbols \( \mc\in\mcFf^{(0)} \) with no arguments are called \coloremph{constants}.
%Predicate symbols \( \mP\in\mcFP^{(0)} \) with no arguments are propositions.
%The special binary equality symbol \( {\foEQ}\in\mcFP \) is always available.
\end{definition}
%\begin{remark}
%We exclusively use the symbol \( \sEQ \) for identity in this report.
%\end{remark}
\begin{definition}
An \coloremph{interpretation} \( \mcI \) over a signature \( \mcF \) consists
of a \coloremph{universe}, i.e.~non-empty set  \( A \),
operations \( \mf^\mcI: A^n\rightarrow A \)  for every function symbol \( \mf\in\mcFf \),
and possibly empty relations \( \mP^\mcI\subseteq A^n \) for every predicate symbol \( \mP\in\mcFP \).
A \coloremph{consistent interpretation} of the equational symbol \( {\trEQ}\in\mcFP^{(2)} \) must respect the theory of equality
with reflexivity, symmetry, transitivity, function congruence, and predicate congruence.
\end{definition}
%____________________________________________________________________________________________________________
%\subsection{Terms}\label{sec:terms}
\begin{definition}
We build the set of \coloremph{(function) terms} \( \mcTFV \)
from a signature \( \mcF \) and
a countable set of \coloremph{variables} \( \mcV \) disjoint from \( \mcF \).
Every variable \( x\in\mcV \) is a term,
every constant \( c\in\mcFf \) is a term, and
every expression \( \mf(t_1,\ldots,t_n) \)
for every function symbol \( \mf\in\mcFf \) with arity \( n>0 \)
and terms \( t_1,\ldots,t_n \)
is a term.
%
We build the set of general or \coloremph{predicate} terms \( \mP(t_1,\ldots,t_n) \) from
predicate symbols \( \mP\in\mcFP \) with arity \( n \) and terms \( t_1,\ldots,t_n \).
%
A term is \coloremph{ground} if its set of variables \( \var(t)=\myemptyset \) is empty.
The set of all ground (function) terms is donated by \( \mcT(\mcFf) \).
\end{definition}

%
%\begin{remark}
%Most of the following definitions are easily applicable on predicate terms too.
%\end{remark}
%%
%% DEF %%
\begin{definition}\label{def:position}
A \coloremph{position} is a finite sequence of positive integers.
The root position is the empty sequence \( \epsilon \).
Position \( pq \) is obtained by concatenation of positions \( p \) and \( q \).
%
The expression \( t|_p \) denotes the subterm at position \( p \)
and \( t[s]_p \) is obtained by replacing the subterm of \( t \) at position \( p \) with the term \( s \).
%A (function) term is called \coloremph{linear} if it does not contain multiple occurrences of the same variable.
%
\end{definition}
%% DEF %%
%\begin{definition}\label{def:subterm}
%The~\coloremph{subterm} \( t|_p \) of term \( t \) \coloremph{at position} \( p\in\pos(t) \) is defined as
%\begin{gather*}
%t|_p = \left\{\begin{array}{ll}
%		t 		& \text{if }p=\epsilon \\
%		t_i|_q	& \text{if }t=\mf(t_1,\ldots,t_n)\text{ and }p=iq
%	\end{array}\right.
%\end{gather*}
%%
%If \( s \) is a term then \( t[s]_p \) represents the term build by replacing the subterm of \( t \) at position \( p \) with the term \( s \).
%\begin{gather*}
%t[s]_p = \left\{\begin{array}{ll}
%		s 		& \text{if }p=\epsilon \\
%		\mf(t_1,\ldots,t_i[s]_q,\ldots,t_n)	& \text{if }t=\mf(t_1,\ldots,t_n)\text{ and }p=iq
%	\end{array}\right.
%\end{gather*}
%A \coloremph{hole} denotes a special constant symbol \( \ctxhole\in\mcFf^{(0)} \).
%A \coloremph{context} is a term \( t \) with exactly one hole, i.e.~one occurrence \( |t|_{\ctxhole}=1 \).
%\end{definition}
%% DEF %%
\begin{definition}\label{def:substitution}
A \coloremph{substitution} is an assignment \( \alpha \) from variables \( x\in\mcV \) to terms in \( \mcTFV \)
with finite \coloremph{domain} \( \dom(\alpha) = \{ x\in\mcV\mid\alpha(x) \neq x \} \).
We write substitutions as mappings \( \sigma=\{ x_1\mapsto s_1,\ldots,x_n\mapsto s_n \} \)
where \( \dom(\sigma)=\{ x_1,\ldots,x_n \} \) and \( \sigma(x_i)=s_i \).
A \coloremph{variable substitution} is an assignment from \( \mcV \) to \( \mcV \).
A \coloremph{renaming} is a bijective variable substitution.
A \coloremph{proper instantiator} is a substitution that is not a variable substitution.
%We define the \coloremph{instance} \( t\sigma \) of a term \( t \)
%\begin{align*}
%	t\sigma &= \left\{\begin{array}{ll}
%		s_i 					& \text{if }t=x_i\in\dom(\sigma)\\
%		t					& \text{if }t\in\mcV\,\backslash\dom(\sigma)\\
%		\mf(t_1\sigma,\ldots,t_n\sigma)	&\text{if }t=\mf(t_1,\ldots,t_n)
%	\end{array}\right.
%\end{align*}
%and the \coloremph{composition} of two substitutions \( \sigma \) and \( \tau \) as
%\begin{align*}
%	\sigma\tau&=\{ x_i\mapsto s_i\tau\mid x_i\in\dom(\sigma) \}
%	\cup
%	\{ y_i\mapsto t_i\mid y_i\in\dom(\tau) \backslash \dom(\sigma) \}.
%\end{align*}
\end{definition}
%% DEF %%
\begin{definition}\label{def:unifier}
Two terms \( s \) and \( t \) are \coloremph{unifiable} if there exists a substitution such that \( s\sigma=t\sigma \).
They are \coloremph{variants} if their most general unifier is a renaming.
The \coloremph{most general unifier} \( \sigma=\mgu(s,t) \) is a unifier such that
for every other unifier \( \sigma' \) there exists a substitution \( \tau \) such that
\( \sigma' = \sigma \tau \).
\end{definition}
%% DEF %%
%\begin{example}
%The unifier \( \sigma'=\{ x\mapsto\mg(\ma),y\mapsto\mb,z\mapsto\mg(\mb)\} \)
%of terms \( s=\mf(x,\mg(y)) \)
%and  \( t=\mf(\mg(\ma),z) \)
%is less general than \( \sigma=\mgu(s,t)=\{ x\mapsto\mg(\ma), z\mapsto\mg(y) \):
%\( \sigma'=\sigma\{ y\mapsto\mb \} \).
%\end{example}
%%%%%%%%%%%%%%%%%%%%%%%%%%%%%%%%%%%%%%%%%%%%%%%%%%%
%\subsection{First order formulae}
%
\begin{definition}\label{def:first-order}
%Let \( \mcF \) be a signature.
A \coloremph{first order formula} \( F \) is either an
\coloremph{atomic formula}, i.e.~a predicate term or an equation, or a composition of formulae where \( \mP \) is a predicate symbol and \( s,t,t_1,\ldots,t_n \) are terms.
%
\[
 F ::= \mP(t_1,\ldots,t_n) \mid
	s \foEQ t \mid
	\lnot F \mid
	\forall x F \mid
	\exists x F \mid
	 F\lor F \mid
	 F\land F \mid
	 F\rightarrow F
\]
%negation \( \lnot F \),
%universal quantification \( \forall x F \),
%existential quantification \( \exists x F \),
%%of one formula \( \phi \);
%disjunction \(  F\lor F' \),
%conjunction \(  F\land F' \),
%or implication \(  F\rightarrow F' \).
%Exclusive disjunction \(  F\oplus F \) abbreviates \( ( F\lor F)\land(\lnot\phi\lor\lnot F) \).
%of two formulae \( \phi \) and \( \psi \).
\end{definition}
%% REM %%
%% DEF %%
\begin{definition}\label{def:literals}
A \coloremph{literal} \( L \) is either an atom
or the negation of an atom.
%, where \( l\not{}\hspace{-0.5mm}\circ \) abbreviates \( \lnot(l\circ r) \)
%for binary symobl \( \circ \).
%A literal is ground if all its terms are ground.
%The complement \( L^c \) of a literal \( L \) is defined as %with some atom \( A \):
%\[ L^c = \left\{ \begin{array}{rl}
%A & \text{if } L = \lnot A\\
%\lnot A & \text{if } L = A
%\end{array}\right.
%\]
Two literals are complementary if one is an atom and the other is the negation of this atom.
%
A \coloremph{clause}\ \ \( \mcC = L_1\lor\ldots\lor L_n \)  is a possible empty multiset of literals and is interpreted as disjunction of literals.
The \coloremph{empty clause} \( \emptyclause \) expresses a contradiction.
%A clause is ground if all its literals are ground.
A finite \coloremph{set of clauses} \( S=\{ \mcC_1,\ldots,\mcC_n \} \) 
is interpreted as an universally quantified conjunction of variable distinct clauses
\( S\equiv\forall\vec{x}_1\ldots\forall\vec{x}_n\left(\mcC_1\land\ldots\land\mcC_n\right) \) with
\( \var(\mcC_i)=\vec{x}_i \),
\( \vec{x}_i\cap\vec{x}_j=\myemptyset\lor i=j \).
%A set of clauses is ground if all its clauses are ground.
\end{definition}
%
\begin{remark}
When the same variable symbols are used in different clauses the distinguishing additions, e.g.~super- or subscripts, were omitted due to laziness.
\end{remark}
%
%\begin{definition}
%A {\em clause} \( \mathcal C = L_1\lor \ldots\lor L_n \) is a possible empty multiset of literals and is interpreted as disjunction of literals.
%\( \square \) denotes the {\em empty clause}, respectively false, contradiction, pair or conjunction of complementary literals.
%A clause is ground if all its literals are ground.
%A finite {\em set of clauses} \( S=\{ \mathcal C_1,\ldots \mathcal C_n \} \) is interpreted as universally quantified conjunction of all its clauses.
%\end{definition}
%
\begin{definition}\label{def:model}
%A ground literal \( L \) holds in an Herbrand \coloremph{interpretation} \( \mathcal I \) (a domain of ground terms and definitions for functions and predicates),
%if it is the logical consequence \( \mathcal I\models L \) of ground literals in \( \mathcal I \).
%A non-ground literal holds in \( \mathcal I \), if all its ground instances hold in \( \mathcal I \).
%A clause holds in \( \mathcal I \), if at least one of its literals holds in \( \mathcal I \).
%A set of clauses \( S \) holds in \( \mathcal I \), if every clause \( \mathcal C\in S \) holds in \( \mathcal I \).
%An interpretation is consistent \( \mathcal I\not\models\square \), if no pair of complementary literals holds in \( \mathcal I \).
%A set of clauses \( S \) is \coloremph{satisfiable} if it has a \coloremph{model} \( \mathcal M \), i.e.~a consistent interpretation \( \mathcal M\models S \).
A \coloremph{term model} \( \mcM \) is an consistent interpretation where the elements of the domain \( A \)
are equivalence classes of ground terms
and the interpretation of each ground term \( t^\mcM := [t]_\sim \) is its equivalence class.
An equation \( s\trEQ t \) of ground terms holds in \( \mcM \) if \( [s]_\sim=[t]_\sim \);
a ground predicate term \( \mP(t_1,\ldots,t_n) \) holds in \( \mcM \) if \( \mP([t_1]_\sim,\ldots,[t_n]_\sim)] \in A^n \).
A ground literal holds in \( \mcM \) if and only if its complementary literal does not hold in \( \mcM \).

A non-ground literal holds in \( \mcM \) if all its
%(infinitely many in the presence of function symbols with arity greater than one)
(perhaps infinitely many)
instances hold in \( \mcM \).
A clause holds in \( \mcM \), if at least one of its literals holds in \( \mcI \).
A set of clauses \( S \) holds in \( \mcM \), if every clause \( \mcC\in S \) holds in \( \mcM \).
Then we say \( \mcM \) is a model {\em for} \( S \).

\end{definition}
%
%
%%%%%%%%%%%%%%%%%%%%%%%%%%%%%%%%%%%%%%%
%%% SUBSEC %%

%%
%%%%%%%%%%%%%%%%%%%%%%%%%%%%%%%%%%%%%%%%%%%%%%%%%%%%%%%%%%%%%%%
%\subsection{Term rewrite systems}
%% DEF %%
\begin{definition}
A \coloremph{rewrite rule}  \( l \rightarrow r \) is an equation of terms where the left-hand side is not a variable
and the variables which occur in the right-hand side occur also in the left-hand side.
%\[
%	l\not\in\mcV\text{ and }\var(r)\subseteq\var(l)
%\]
A rewrite rule \( l'\rightarrow r' \) is a \coloremph{variant} of \( l\rightarrow r \) if there is a renaming \( \varrho \) such that
\( (l\rightarrow r)\varrho = (l\varrho\rightarrow r\varrho) = l'\rightarrow r' \).
A \coloremph{term rewrite system} is a set of rewrite rules without variants.
%\end{definition}
%% REM %%
%\begin{remark}
%It's easy to see (lemma \vref{lem:xa}) that not all equations \( s\foEQ t \) or their inverse \( t\foEQ s \) are rewrite rules.
%%A little sloppy we may write \( l\trEQ r \) to express that \( l\rightarrow r \) or its inverse \( l\leftarrow r \) must be rewrite rules,
%%e.g.~\( x\trEQ f(x) \).
%%If we do not care we write \( l\foEQ r \).
%\end{remark}
%
%\begin{definition}
The expression \( s\rightarrow t \) denotes a \coloremph{rewrite step}
from term \( s \) to term \( t \)
if there exists
a position \( p\in\pos(s) \),
a rewrite rule \( l\rightarrow r \),
and a substitution \( \sigma \) such that
\( s|_p=l\sigma \) and \( t=s[r\sigma]_p \).
The subterm \( l\sigma \) is called \coloremph{redex} and
\( s \) rewrites to \( t \) by \coloremph{contracting} \( l\sigma \) to \coloremph{contractum} \( r\sigma \).
%
We may write
\( s\rightarrow_{p,l\rightarrow r,\sigma}t \) to indicate position, rewrite rule and substitution
or \( s\rightarrow_\mcR t \) to indicate the origin of the rule.
%
A term \( s \) is \coloremph{irreducible} or in \coloremph{normal form} if there is no rewrite step \( s\rightarrow t \) for any term \( t \).
The set of normal forms \( \mNFR \) of a TRS \( \mcR \) contains all irreducible terms of the TRS.
%
\end{definition}
%%%
\begin{definition}
A term \( s \) can be rewritten to term \( t \) with notion \( s\rightarrow^* t \)
if there exists at least one \coloremph{rewrite sequence} \( (a_1,\ldots ,a_n) \) such that
\( s=a_1 \), \( a_n=t \), and \( a_i\rightarrow a_{i+1} \) are rewrite steps for \( 1\leq i<n \).
A TRS is \coloremph{terminating} if there is no infinite rewrite sequence of terms.
%
Two Terms \( s \) and \( t \) are \coloremph{joinable} with notion \( s\downarrow t \)
if both can be rewritten to some term \( c \), i.e.~\( s \rightarrow^*c\ \, ^*\!\!\leftarrow t \).
A TRS is \coloremph{confluent } if \( s \) and \( t \) are joinable whenever \( s\ ^*\!\!\leftarrow a \rightarrow^* t \) holds for some term \( a \).
%
Terms \( s \) and \( t \) are \coloremph{convertible} with notion \( s\leftrightarrow^* t \)
if there exists a sequence \( (a_1,\ldots ,a_n) \) such that
\( s=a_1 \), \( a_n=t \), and \( a_i\leftrightarrow a_{i+1} \), i.e.~\( a_i\rightarrow a_{i+1} \) or \( a_i\leftarrow a_{i+1} \) are rewrite steps for \( 1\leq i<n \).
\end{definition}
%
%% DEF %%
\begin{definition}\label{def:closed-under}
%A binary relation \( R \) on terms is \coloremph{closed under contexts}
%if \( u[s]|_p\,R\;u[t]|_p \) holds
%for all positions \( p\in\pos(u) \) and
%for all terms \( s,t,u \) with \( s\,R\;t \).
%It is \coloremph{closed under substitutions}
%if \( s\sigma\,R\;t\sigma \) holds for all substitutions \( \sigma \)
%and all terms \( s,t \) with \( s\,R\;t \).
A binary relation \( \relation \) on terms is called a \coloremph{rewrite relation} whenever it is \coloremph{closed under contexts},
i.e.~\( u[s]|_p\relation u[t]|_p \) %holds
for all positions \( p\in\pos(u) \) and
for all terms \( s,t,u \) with \( s\relation t \)
and \coloremph{closed under substitutions},
i.e.~\( s\sigma\relation t\sigma \) %holds
for all substitutions \( \sigma \)
and all terms \( s,t \) with \( s\relation t \).
\end{definition}
\begin{lemma}
The relations \( \rightarrow^*_\mcR \),
\( \rightarrow^+_\mcR \),
\( \downarrow_\mcR \), \( \uparrow_\mcR \) are rewrite relations on every TRS \( \mcR \).
\end{lemma}
%
\begin{definition}
A proper (i.e.~irreflexive and transitive) order on terms is called \coloremph{rewrite order} if it is a rewrite relation.
A \coloremph{reduction order} is a well-founded rewrite order,
i.e.~there is no infinite sequence
\( (a_i)_{i\in\mathbb{N}} \)
where \( a_i\gtpre a_{i+1} \) for all \( i \).
% with \( i\in\mathbb{N} \).
A \coloremph{simplification order} is a rewrite order with the \coloremph{subterm property},
i.e.~\( u[t]_p \gtpre t \) for all terms \( u \), \( t \) and positions \( p\neq\epsilon \).
\end{definition}
\begin{lemma}
Every simplification order is well-founded, i.e.~it is a reduction order.
\end{lemma}
%% DEF %%
%
\begin{theorem}
A TRS \( \mcR \) is terminating if and only if there exists a reduction order \( \gtpre \)
such that \( l\gtpre r \) for every rewrite rule \( l\rightarrow r\in\mcR \).
We call \( \mcR \) simply terminating if \( \gtpre \) is a simplification order.
\end{theorem}

%\begin{theorem}[Knuth-Bendix-Criterion]
%\end{theorem}
%% DEF %%

