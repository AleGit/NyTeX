% !TeX root = ../mthesis.tex
% !TeX encoding = UTF-8
% !TeX spellcheck = en_US

\chapter{Completeness}

We present the completeness proof from \cite{GK2004csl} in a more accessible form.

\section{Closures}

\begin{definition}
    A closure is a pair of a clause $C$ and a substitution $\sigma$,
    conveniently written as $C\cdot\sigma$. Two closures $C\cdot\sigma = D\cdot\tau$
    are the same if $C$ is a variant of $D$ and $C\sigma$ is a variant of $D\tau$.
    A closure $C\cdot\sigma$ represents a clause $C\sigma$,
    i.e.~the result of applying substitution $\sigma$ to $C$.
    A ground closure represents a ground clause.
\end{definition}

% \begin{center}
\begin{figure}\label{fig:simplification:order}
\begin{tikzpicture}
    \node (defCUC) at (-4,8.5) { $s\succ t\Rightarrow C[s]\succ C[t]$};
    \node (CUC) at (-4,8) { contexts };
    \node (defCUS) at (0,8.5) { $s\succ t\Rightarrow s\sigma\succ t\sigma$};
    \node (CUS) at (0,8) { substitutions };

        \node (CL) at (-2,6) { closed under };
        \node (defIRR) at (2,6.5) { $s\not>s$ };
        \node (IRR) at (6,6) { irreflexive };
        \node (defIRR) at (6,6.5) { $s>t>u\Rightarrow s>u$ };
        \node (TRA) at (2,6) { transitive };

        \node (PO) at (4,4.4) { $>$ };
        \node (PO) at (4,4) { proper order };
        \node (RWR) at (0,4) { rewrite relation };

        \node (defSTP) at (-2,2.5) { $C[s]\succ s$ };
        \node (STP) at (-2,2) { subterm property };
        \node (RWO) at (2,2) { rewrite order };
        \node (defWF) at (8,4.5) { $(s_i Rs_{i+1})_i$ is finite };
        \node (WF) at (8,4) { well-founded };

        \node (SO) at (0,0) {simplification order};
        \node (RO) at (4,0) {reduction order};

        \node (WFO) at (6,2) { well-founded order };


        \draw[->] (CL) -- (CUC);
        \draw[->] (CL) -- (CUS);

        \draw[->] (PO) -- (IRR);
        \draw[->] (PO) -- (TRA);π

        \draw[->] (RWR) -- (CL);

        \draw[->] (RWO) -- (PO);
        \draw[->] (RWO) -- (RWR);

        \draw[->] (SO) -- (STP);
        \draw[->] (RO) -- (RWO);
        \draw[->] (RO) -- (WFO);
        % \draw (RO) edge[out=0,in=-45,->] (WF);

        \draw[->] (SO) -- (RWO);
        \draw[->, dotted] (SO) -- (RO);

        \draw[->] (WFO) -- (WF);
        \draw[->] (WFO) -- (PO);

        \draw[->, dotted] (WF) -- (IRR);


    \end{tikzpicture}
\caption{Properties of a simplification order on terms}
\end{figure}
% \end{center}

\begin{lemma}
    A total simplification order
    – see Figure \vref{fig:simplification:order} –
    over ground terms always exists \cite{NR2001}.
\end{lemma}

\begin{definition}

    We assume $\succG$ as a
        total, well-founded, and monotone extension of a
        total simplification order over ground terms to ground (literal) clauses
        and ground (literal) closures such that
        \begin{align*}
            s\sigma\mNE t\sigma \quad&\succG\quad s\sigma\mEQ t\sigma
            \\
            L\sigma\lor C\sigma \quad&\succG\quad L\sigma
            \\[0.7em]
            L\sigma \succG L'\sigma'
            \quad&\Rightarrow\quad
            L\cdot\sigma \succL L'\cdot\sigma'
        \\
        \left.
        \begin{array}{r}
            C\tau\succG D\rho\text{ or}
            \\
            C\tau=D\rho, C\theta=D
        \end{array}
        \right\}
            \quad&\Rightarrow\quad
            C\cdot\tau\succC D\cdot\rho
        \end{align*}
        holds for terms $s,t$,
        literals $L, L'$,
        clauses $C,D$,
        ground substitutions $\sigma, \tau, \rho$
        and proper instantiator $\theta$.

\end{definition}

\begin{definition}[Unit paramodulation \cite{GK2004csl}]
    
\begin{gather*}
    \infer[\theta]
        {L[r]\theta\cdot\rho}
        {(\ell\mEQ r)\cdot\sigma & L[\ell']\cdot\sigma'}
        \qquad\qquad
        \infer[\mu]
        {\emptyclause}
        {(s\mNE t)\cdot\tau}
\end{gather*}
    where
    
\begin{itemize}

    \item
        \( \ell\sigma\succG r\sigma \),
        \( \theta=\mgu(\ell,\ell') \),
        \( \ell\sigma = \ell'\sigma' = \ell'\theta\rho \),
        \( \ell'\notin\mcV \)

    \item
        \( s\tau = t\tau \),
        \( \mu=\mgu(s,t) \)
\end{itemize}
    \end{definition}

    \begin{example}
        Let  $\mf(s,t,u)\succG\mg(s')$ for all ground terms
        $r,s,t,s'$.
        % $\succ_{kbo}$ order on terms.
        \begin{gather*}
            \infer[\theta]{
                \left(
                    \mh(\underbrace{
                        \boxed{\mg(x')}
                        }_{r\theta})\mNE\mg(x')
                    \right)\cdot\rho
            }{
                \left(
                    \overbrace{\mf(x,y,\mc)}^{\ell}
                    \mEQ
                    \overbrace{\mg(x)}^r
                \right)\cdot \sigma
            & \left(
                \mh(\overbrace{
                    \boxed{\mf(x',\mh(y'), z')}
                    }^{\ell'})\mNE\mg(x')
                \right)\cdot \sigma'
            }
        \end{gather*}
        \begin{align*}
            \sigma &= \{x\mapsto\ma, y\mapsto\mh(\mb)\} &
             \sigma' &= \{x'\mapsto\ma, y'\mapsto\mb, z'\mapsto\mc\}
             \\
             \theta &= \{x\mapsto x',y\mapsto\mh(y'),z'\mapsto\mc\}
              &
             \rho &=\{ x'\mapsto \ma, y'\mapsto \mb \}
             \\
             \ell'\theta &= \mf(x',\mh(y'),\mc)
             &
             \ell'\theta\rho &= \mf(\ma, \mh(\mb), \mc) = \ell\sigma = \ell'\sigma'
        \end{align*}
    \end{example}

    \begin{lemma}
        Let $R$ be a ground rewrite system and UP is applicable to
        \( (l\mEQ r)\cdot\sigma, L[l']\cdot\sigma' \)
        with conclusion
        \( L[r]\theta\cdot\rho \).
        If $\sigma,\sigma'$ are irreducible w.r.t.~R then $\rho$ is irreducible w.r.t.~R.
    \end{lemma}

    \begin{proof}
        Assume otherwise.
        Hence there is a $x\in\var(l'\theta)$ such that $x\theta\rho$ is reducible by $R$.

    \end{proof}



