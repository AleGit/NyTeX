% !TeX root = ../mthesis.tex
% !TeX encoding = UTF-8
% !TeX spellcheck = en_US

\chapter{Automated First Order Theorem Proving}

%\epigraph{All men are mortal
%	
%	Fosca is a man
%	
%	Therefore Fosca is mortal
%	}{Syllogism
%}

\epigraph{
	Tous les hommes sont mortels
	
	Fosca est un homme
	
	Par conséquent Fosca est mortel\footnotemark
}{
	Syllogism
}
\footnotetext{All men are mortal. Fosca is a man. Hence Fosca is mortal.}


\section{Motivation}

The central idea of instantiation-based first order theorem proving is the
translation of an undecidable problem into a sequence of effectivly decideable problems.

First it makes use of the fact that a first-order formula is valid if and only if its negation is unsatisfiable.
Second it uses the existing efficient transformation algorithms from arbitrary negated formulas to equisatisfiable sets of clauses.
Third it utilizes that is is sufficient to find one unsatisfiable set of ground instances 
to show that the original set of clauses is unsatisfiable.


\begin{example} Let $\mcF = \{\species^1, \fosca^0, \human^0 \} \disjointunion \{ \mortal^1  \} \disjointunion \{\mEQ^2\}$ be a signature.
	We translate the given syllogism into a formula $F$ in first order logic.
	\begin{align*}
	F &= A \limp (B \limp C) \\[0.5em]
	A &= \forall x\,(\species(x) \mEQ \human \limp \mortal(x)) 	\tag{theory}\\
	B &= \species(\fosca) \mEQ \human 							\tag{fact}\\
	C &= \mortal(\fosca)										\tag{conjecture}
	\end{align*}
%	 
	We transform the negated formula into an equivalent, hence equisatisfiable set of clauses.
%
\begin{align*}
\lnot F &\mEQ \{ \mcC_1, \mcC_2, \mcC_3 \} \\[0.5em]
\mcC_1 &= \species(x)\mNE\human\lor\mortal(x) \tag{theory}\\
\mcC_2 &= \species(\fosca)\mEQ\human \tag{fact} \\
\mcC_3 &= \lnot\mortal(\fosca) \tag{negated conjecture} \\
%\mcC_4 &:=L_2^\mcc\lor L_3^\mcc&\species(\fosca)\mNE\human\lor\mortal(\fosca)\tag*{$\mcC_1 \{x\mapsto\fosca\}$}
%\\
%4 &&\species(\fosca)\mNE\human\lor\mortal(\fosca) \tag*{instance}
\end{align*}
When we instantiate $\mcC_1$ with instantiator $\sigma = \{x\mapsto\fosca\}$ we get an unsatisfiable 
set of ground instances $\{ \mcC_2, \mcC_3, \mcC_1\sigma  \}$.
This can be easily checked by the DPLL 
method for ground clauses \cite{Davis:1962:MPT:368273.368557, Davis:1960:CPQ:321033.321034},
i.e.~with a modern satisfiability checker.

\section{Theory}



	An equational interpretation must satisfy the formulas for reflexivity, symmetry, transitivity, 
	function congruence for every function symbol $f\in\mcFf$, 
	and predicate congruence for every predicate symbol $P\in\mcFP$.
	\[
		\begin{array}[t]{c}
		x\mEQ x 
\qquad
		x\mNE y\lor y\mEQ x
\qquad		
		x\mNE y\lor y\mNE z\lor x\mEQ z \\[0.7em]
%		
x_1 \mNE y_1\lor\ldots\lor x_n \mNE y_n\lor f(x_1,\ldots,x_n)\mEQ f(y_1,\ldots,y_n) \\[0.7em]
%
x_1\mNE y_1\lor\ldots\lor x_n\mNE y_n\lor\lnot P_i(x_1,\ldots,x_n)\lor P(y_i,\ldots,y_n)

		\end{array}
	\]







\end{example}

