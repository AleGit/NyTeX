% !TeX root = ../mthesis.tex
% !TeX encoding = UTF-8
% !TeX spellcheck = en_US

\chapter{First Order Theorem Proving with Equality}

%\epigraph{All men are mortal
%	
%	Fosca is a man
%	
%	Therefore Fosca is mortal
%	}{Syllogism
%}

\epigraph{
	Tous les hommes sont mortels
	
	Fosca est un homme
	
	Par conséquent Fosca est mortel\footnotemark
}{
	Syllogism
}
\footnotetext{All men are mortal. Fosca is a man. Hence Fosca is mortal.}


The instantiation based idea to prove a first order theorem from a set of axioms is simple. 
It exploits the fact that a formula is valid if and only if its negation is unsatisfiable.
So we transform the problem, i.e.~the axioms and the negated conjecture(s) into an equisatisfiable set of clauses.
Afterwards we search for an unsatisfiable set of ground instances of these clauses.

\begin{example} Let $\mcF = \{\species^1, \fosca^0, \human^0 \} \disjointunion \{ \mortal^1  \} \disjointunion \{\mEQ^2\}$ be a signature. 
	Consider the set of clauses $\{ \mcC_1, \mcC_2, \mcC_3 \}$, i.e.~{\myem the problem}, with the derived instance $\mcC_4$.

%\begin{align*}
%	\forall x \left((\species(x)\mEQ\human)\limplies\mathsf{Mortal}(x) \tag*{theory}\right)\\
%	\species(\fosca)\mEQ\human \tag*{fact} \\
%	\mathsf{Mortal}(\fosca) \tag*{conjecture}
%\end{align*}

\begin{align*}
\mcC_1 &:= L_{11}\lor L_{12}&\species(x)\mNE\human\lor\mortal(x) \tag*{theory}\\
\mcC_2 &:= L_2&\species(\fosca)\mEQ\human \tag*{fact} \\
\mcC_3 &:=L_3&\lnot\mortal(\fosca) \tag*{negated conjecture} \\
\mcC_4 &:=L_2^\mcc\lor L_3^\mcc&\species(\fosca)\mNE\human\lor\mortal(\fosca)\tag*{$\mcC_1 \{x\mapsto\fosca\}$}
%\\
%4 &&\species(\fosca)\mNE\human\lor\mortal(\fosca) \tag*{instance}
\end{align*}

The set of ground instances $\{ \mcC_2, \mcC_3, \mcC_4  \}$ is unsatisfiable,
which is easily checked by a SMT solver.

\section{Theory}



	An equational interpretation must satisfy the formulas for reflexivity, symmetry, transitivity, 
	function congruence for every function symbol $f\in\mcFf$, 
	and predicate congruence for every predicate symbol $P\in\mcFP$.
	\[
		\begin{array}[t]{c}
		x\mEQ x 
\qquad
		x\mNE y\lor y\mEQ x
\qquad		
		x\mNE y\lor y\mNE z\lor x\mEQ z \\[0.7em]
%		
x_1 \mNE y_1\lor\ldots\lor x_n \mNE y_n\lor f(x_1,\ldots,x_n)\mEQ f(y_1,\ldots,y_n) \\[0.7em]
%
x_1\mNE y_1\lor\ldots\lor x_n\mNE y_n\lor\lnot P_i(x_1,\ldots,x_n)\lor P(y_i,\ldots,y_n)

		\end{array}
	\]







\end{example}

