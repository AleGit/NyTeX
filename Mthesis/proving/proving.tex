% !TeX root = ../mthesis.tex
% !TeX encoding = UTF-8
% !TeX spellcheck = en_US

\chapter{Automated Theorem Proving}

%\epigraph{All men are mortal
%	
%	Fosca is a man
%	
%	Therefore Fosca is mortal
%	}{Syllogism
%}

\epigraph{
	Tous les hommes sont mortels
	
	Fosca est un homme
	
	Par conséquent Fosca est mortel\footnotemark
}{
	Syllogism
}
\footnotetext{All men are mortal. Fosca is a man. Hence Fosca is mortal.}



The central idea of instantiation-based first order theorem proving is the
translation of an undecidable problem into a possible infinite sequence of decidable problems.
First we make use of the fact that a first-order formula is valid if and only if its negation is unsatisfiable.
Second we utilize the existing efficient transformation algorithms from arbitrary formulas to equisatisfiable sets of clauses. 
For a given set of clauses we create a sequence of finite subsets of all ground instances of this set
until we've created an unsatisfiable one. 




\section{Gilmore's Prover}

Gilmore's Prover transforms a negated formula into a skolem normal form.
Then it creates a sequence of ground instances $\mcC_i$ of this formula while building the Herbrand universe 
$H = \bigcup_{n\geq 0} H_i$ for the underlying signature
recursivly
\begin{align*}
H_0 &:= \left\{ 
	\begin{array}{ll}
	\{ c \mid \mc\in\mcFfn[0] \} 
	&\text{if constants exist}\\
	\{ c \}
	&\text{otherwise}
	\end{array}
\right. 
\\
H_{n+1} &:= \{\  
	\mf(t_1,\ldots,t_k) \mid
	\mf\in\mcFfn[k],
	t_1,\ldots,t_k \in H_n
\ \}
\end{align*}

$C_i'$ are ground instances that uses just elements of $H_i$. 





\begin{example}
	Let $\mcF = \{ \fosca \} \disjointunion \{ \mortal, \human \}$ be a signature without equality. 
	We translate the given syllogism into a formula $F$ in first order logic.
	\begin{align*}
		F &= A \limp (B \limp C) \equiv \lnot(A\land B) \lor C 
		\tag*{formula}
		\\[0.5em]
		A &= \forall x\, ( \human(x) \limp \mortal(x) 
		\tag*{theory}
		\\
		B &= \human(\fosca) 
		\tag*{fact}
		\\
		C &= \mortal(\fosca)
		\tag*{conjecture}
	\end{align*}
\end{example}
 
%We easily find a satisfying interpretation $\mcI$ with domain $\{ \fosca \}$ with 
%$\fosca^\mcI = \fosca$,
%$\human^\mcI = \{ (\fosca) \}$,
%$\mortal^\mcI = \{ (\fosca) \}$.


	We negate the formula $\lnot F \equiv A\land B \land\lnot C$ and since there is exactly one constant we get
	$H_0 = \{ \fosca \}$ for our Herbrand universe and get a single ground formula in $C_0'$
	which we will transform into 
	disjunctive normal form for satisfiability checking.
%
\begin{gather*}
(\lnot\human(\fosca)\land\human(\fosca)\land\lnot\mortal(\fosca))
\\ 
\lor
\\ 
(\mortal(\fosca)\land\human(\fosca)\land\lnot\mortal(\fosca))
\end{gather*}
Since both conjunctions contain complementary literals we conclude the negated formula is unsatisfiable.


\begin{gather*}
content...
\end{gather*}

%Since there is only one constant we instantiate $\mcC_1$ with instantiator $\sigma = \{x\mapsto\fosca\}$ and we get an unsatisfiable 
%set of ground instances $\{ \mcC_2, \mcC_3, \mcC_1\sigma  \}$.
%This can be easily checked by the DPLL 
%method for ground clauses \cite{Davis:1962:MPT:368273.368557, Davis:1960:CPQ:321033.321034},
%i.e.~with a modern satisfiability checker.

\begin{example}
	Consider the set of clauses 
	$S = \{ 
	\lnot\mE(\msucc(x), \mzero), 
	\lnot \mE(\msucc(x), \msucc(y)) \lor \mE(x,y)
	\}$ with binary predicate symbol $\mE$, unary function symbol $\msucc$, and constant symbol $\mzero$.
	\begin{align*}
	H_0 :=&\quad \{ \mzero \} & \mcC_0' &= \{  
	\lnot\mE(\msucc(\mzero), \mzero), 
	\lnot \mE(\msucc(\mzero), \msucc(\mzero)) \lor \mE(\mzero,\mzero)
	\}
	\\
	H_1 :=&\quad \{ \msucc(\mzero) \} & \mcC_1' &= \{ 
		\lnot\mE(\msucc(\msucc(\mzero), \mzero),
		\lnot \mE(\msucc(\msucc(\mzero)), \msucc(\msucc(\mzero))) \lor \mE(\msucc(\mzero),\msucc(\mzero))
	 \}
	 \\\ldots
	 \\
	 H_i :=&\quad \{ 
	 \msucc(\msucc^i(\mzero)) \} & \mcC_i' &= \{  
	 \lnot\mE(\msucc(\msucc^i(\mzero)), \mzero), 
	 \lnot \mE(\msucc(\msucc^i(\mzero)), \msucc(\msucc^i(\mzero)) \lor \mE(\msucc^i(\mzero),\msucc^i(\mzero))
	 \}
	\end{align*} 
	It is easy to see that each $\mcC_i'$ is satisfiable and Gilmore's prover will proceed for ever.
	\begin{align*}
	A := &\ \{ \mzero \} \\
	s := &\ \mzero\mapsto\mzero \\
	\mE := &\ \{\ \} \subseteq A\times A
	\end{align*}
\end{example}

\section{Theory}



	An equational interpretation must satisfy the formulas for reflexivity, symmetry, transitivity, 
	function congruence for every function symbol $f\in\mcFf$, 
	and predicate congruence for every predicate symbol $P\in\mcFP$.
%	\[

%		\begin{array}[t]{c}
%		x\mEQ x 
%\qquad
%		x\mNE y\lor y\mEQ x
%\qquad		
%		x\mNE y\lor y\mNE z\lor x\mEQ z \\[0.7em]
%%		
%x_1 \mNE y_1\lor\ldots\lor x_n \mNE y_n\lor f(x_1,\ldots,x_n)\mEQ f(y_1,\ldots,y_n) \\[0.7em]
%%
%x_1\mNE y_1\lor\ldots\lor x_n\mNE y_n\lor\lnot P_i(x_1,\ldots,x_n)\lor P(y_i,\ldots,y_n)
%
%		\end{array}
%	\]

\begin{align*}
	 x \mEQ x & \tag*{reflexivity} \\
	x \mNE y \lor y \mEQ x & \tag*{symmetry}\\
	x\mNE y\lor y\mNE z\lor x\mEQ z & \tag*{transitivity} \\
	x_1 \mNE y_1\lor\ldots\lor x_n \mNE y_n\lor f(x_1,\ldots,x_n)\mEQ f(y_1,\ldots,y_n) &\quad f\in\mcFfn
	\tag*{f-congruence}
	\\
	x_1\mNE y_1\lor\ldots\lor x_n\mNE y_n\lor\lnot P_i(x_1,\ldots,x_n)\lor P(y_i,\ldots,y_n) &\quad P\in\mcFPn
	\tag*{P-congruence}
	\\[0.7em]
	x_1\mNE y_1\lor x_2\mNE y_2
	\lor x_1\mNE x_2
	\lor y_1\mEQ y_2
	\tag*{$\mEQ$-congruence}
\end{align*}

\begin{align*}
	x \mEQ x & \tag*{reflexivity} 
	\\
	x \mEQ y \limp y \mEQ x & \tag*{symmetry}
	\\
	(x\mEQ y\land y\mEQ z)\limp x\mEQ z & \tag*{transitivity} 
	\\
	(x_1 \mEQ y_1\land\ldots\land x_n \mEQ y_n)\limp f(x_1,\ldots,x_n)\mEQ f(y_1,\ldots,y_n) &\quad f\in\mcFfn
	\tag*{f-congruence}
	\\
	x_1\mEQ y_1\land\ldots\land x_n\mNE y_n\land P_i(x_1,\ldots,x_n)
	\limp P(y_i,\ldots,y_n) &\quad P\in\mcFPn
	\tag*{P-congruence}
	\\[0.7em]
	(x_1\mEQ y_1
	\land x_2\mEQ y_2
	\land x_1\mEQ x_2)
	\limp y_1\mEQ y_2
	& \quad P\in\mcFPn[2]
		\tag*{$\mEQ$-congruence}
\end{align*}
	
	



\section{Inst-Gen}

\section{Equality predicate}

Consider the satisfiable set of clauses $S = \{a \relation b, b \relation c, \mP(\ma), \lnot \mP(\mc) \}$

When we use the equality symbol as a predicate symbol 
then at least we expect a congruence relation.



