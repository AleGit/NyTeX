% !TeX root = ../mthesis.tex
% !TeX encoding = UTF-8
% !TeX spellcheck = en_US

\chapter{Automated First Order Theorem Proving}

%\epigraph{All men are mortal
%	
%	Fosca is a man
%	
%	Therefore Fosca is mortal
%	}{Syllogism
%}

\epigraph{
	Tous les hommes sont mortels
	
	Fosca est un homme
	
	Par conséquent Fosca est mortel\footnotemark
}{
	Syllogism
}
\footnotetext{All men are mortal. Fosca is a man. Hence Fosca is mortal.}


From Gilmore's Prover \cite{5392528} in 1960 
to the superposition calculus  in the 1990s there has been many approaches for automated reasoning \cite{Robinson:2001:HAR:778522}. 

\section{Motivation}

The central idea of instantiation-based first order theorem proving is the
translation of an undecidable problem into a possible infinite sequence of effectively decidable problems.

First we make use of the fact that a first-order formula is valid if and only if its negation is unsatisfiable.
Second we utilize the existing efficient transformation algorithms from arbitrary negated formulas to equisatisfiable sets of clauses.
Through compactness and Hebrand's Theorem is is sufficient to find one unsatisfiable set of ground instances 
to show that the original set of clauses is unsatisfiable.


\begin{example} Let $\mcF = \{\species^1, \fosca^0, \human^0 \}_{\mcFf} \disjointunion \{ \mortal^1  \}_{\mP} \disjointunion \{\mEQ^2\}$ be a signature.
	We translate the given syllogism into a formula $F$ in first order logic.
	\begin{align*}
	F &= A \limp (B \limp C) \equiv \lnot(A\land B) \lor C\\[0.5em]
	A &= \forall x\,(\species(x) \mEQ \human \limp \mortal(x)) 	\tag{theory}\\
	B &= \species(\fosca) \mEQ \human 							\tag{fact}\\
	C &= \mortal(\fosca)										\tag{conjecture}
	\end{align*}
%	 
	We transform the negated formula into an equivalent, hence equisatisfiable set of clauses.
%
\begin{align*}
\lnot F &\equiv A\land B \land\lnot C
\mEQ \{ \mcC_1, \mcC_2, \mcC_3 \} \\[0.5em]
\mcC_1 &= \species(x)\mNE\human\lor\mortal(x) \tag{theory}\\
\mcC_2 &= \species(\fosca)\mEQ\human \tag{fact} \\
\mcC_3 &= \lnot\mortal(\fosca) \tag{negated conjecture} \\
%\mcC_4 &:=L_2^\mcc\lor L_3^\mcc&\species(\fosca)\mNE\human\lor\mortal(\fosca)\tag*{$\mcC_1 \{x\mapsto\fosca\}$}
%\\
%4 &&\species(\fosca)\mNE\human\lor\mortal(\fosca) \tag*{instance}
\end{align*}
When we instantiate $\mcC_1$ with instantiator $\sigma = \{x\mapsto\fosca\}$ we get an unsatisfiable 
set of ground instances $\{ \mcC_2, \mcC_3, \mcC_1\sigma  \}$.
This can be easily checked by the DPLL 
method for ground clauses \cite{Davis:1962:MPT:368273.368557, Davis:1960:CPQ:321033.321034},
i.e.~with a modern satisfiability checker.

\section{Theory}



	An equational interpretation must satisfy the formulas for reflexivity, symmetry, transitivity, 
	function congruence for every function symbol $f\in\mcFf$, 
	and predicate congruence for every predicate symbol $P\in\mcFP$.
%	\[

%		\begin{array}[t]{c}
%		x\mEQ x 
%\qquad
%		x\mNE y\lor y\mEQ x
%\qquad		
%		x\mNE y\lor y\mNE z\lor x\mEQ z \\[0.7em]
%%		
%x_1 \mNE y_1\lor\ldots\lor x_n \mNE y_n\lor f(x_1,\ldots,x_n)\mEQ f(y_1,\ldots,y_n) \\[0.7em]
%%
%x_1\mNE y_1\lor\ldots\lor x_n\mNE y_n\lor\lnot P_i(x_1,\ldots,x_n)\lor P(y_i,\ldots,y_n)
%
%		\end{array}
%	\]

\begin{align*}
	 x \mEQ x & \tag*{reflexivity} \\
	x \mNE y \lor y \mEQ x & \tag*{symmetry}\\
	x\mNE y\lor y\mNE z\lor x\mEQ z & \tag*{transitivity} \\
	x_1 \mNE y_1\lor\ldots\lor x_n \mNE y_n\lor f(x_1,\ldots,x_n)\mEQ f(y_1,\ldots,y_n) &\quad f\in\mcFfn
	\tag*{f-congruence}
	\\
	x_1\mNE y_1\lor\ldots\lor x_n\mNE y_n\lor\lnot P_i(x_1,\ldots,x_n)\lor P(y_i,\ldots,y_n) &\quad P\in\mcFPn
	\tag*{P-congruence}
	\\[0.7em]
	x_1\mNE y_1\lor x_2\mNE y_2
	\lor x_1\mNE x_2
	\lor y_1\mEQ y_2
	\tag*{$\mEQ$-congruence}
\end{align*}

\begin{align*}
	x \mEQ x & \tag*{reflexivity} 
	\\
	x \mEQ y \limp y \mEQ x & \tag*{symmetry}
	\\
	(x\mEQ y\land y\mEQ z)\limp x\mEQ z & \tag*{transitivity} 
	\\
	(x_1 \mEQ y_1\land\ldots\land x_n \mEQ y_n)\limp f(x_1,\ldots,x_n)\mEQ f(y_1,\ldots,y_n) &\quad f\in\mcFfn
	\tag*{f-congruence}
	\\
	x_1\mEQ y_1\land\ldots\land x_n\mNE y_n\land P_i(x_1,\ldots,x_n)
	\limp P(y_i,\ldots,y_n) &\quad P\in\mcFPn
	\tag*{P-congruence}
	\\[0.7em]
	(x_1\mEQ y_1
	\land x_2\mEQ y_2
	\land x_1\mEQ x_2)
	\limp y_1\mEQ y_2
	& \quad P\in\mcFPn[2]
		\tag*{$\mEQ$-congruence}
\end{align*}
	
	







\end{example}

