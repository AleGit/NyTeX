% !TeX root = ../mythesis.tex
% !TeX encoding = UTF-8
% !TeX spellcheck = en_US

\subsection{Provability}

In general a proof may be a finite sequence of proof steps
from none or some premises via intermediate statements
to a final, the then proven statement.
A formal proof system or logical calculus describes admissible basic proof steps
in the underlying logic of the statements, in our case first order logic.
A formal proof comprises only proof steps confirmed by rules of the applied logical calculus.

% !TeX root = ../mythesis.tex
% !TeX encoding = UTF-8
% !TeX spellcheck = en_US



\begin{table}[hbt]
    \begin{gather*}
    \begin{array}{ccccc}
    \infer[(\land i)]{F\land G}{F & G}
    &
    \infer[(\land e_1)]{G}{F \land G}
    &
    \infer[(\land e_2)]{F}{F \land G}
    &
    \infer[(\lnot\lnot i)]{\lnot\lnot F}{F}
    &
    \infer[(\lnot\lnot e)]{F}{\lnot\lnot F}
    \\[0.7em]
    \infer[(\bot e)]{F}{\bot}
    &
    \infer[(\lnot e)]{\bot}{F & \lnot F}
    &
    \infer[\text{LEM}]{F\lor\lnot F}{}
    &
    \infer[(\lor i_1)]{F\lor G}{F}
    &
    \infer[(\lor i_2)]{F\lor G}{G}
    \\[0.7em]
    \infer[(\lnot i)]{\lnot F}{
        \boxed{\begin{array}{c}F \\ \vdots \\ \bot \end{array}}}
    &
    \infer[\text{PBC}]{F}{
        \boxed{\begin{array}{c}\lnot F \\ \vdots \\ \bot \end{array}}}
    &
    \infer[({\limp} i)]{F\limp G}{
        \boxed{\begin{array}{c}F \\ \vdots \\ G \end{array}
    }}
    &
    \multicolumn{2}{l}{
        \infer[(\lor e)]{H}{
            F\lor G &
            \boxed{\begin{array}{c}F \\ \vdots \\ H \end{array}} &
            \boxed{\begin{array}{c}G \\ \vdots \\ H \end{array}}
        }
    }
    \\[0.7em]
    &
    \multicolumn{3}{r}{
    \infer[\text{modus}\atop\text{ponens}]{G}{F & F\limp G}
    \qquad
    \infer[\text{modus}\atop\text{tollens}]{\lnot F}{F\limp G & \lnot G}
    }
    &
    \end{array}
    \end{gather*}
    \caption{Natural Deduction Rules for Connectives}\label{tab:natural:deduction:connectives}
    \end{table}

\begin{definition}
	We say a term \( t \) is \coloremph{free} for variable \( x \)
	in a formula \( F \) if for every variable \( y \in \var(t) \)
	and every subformula \( G = \quantify y H \) of \( F \),
	the variable \( x \not\in\fvar(H) \).
\end{definition}

\begin{table}[hbt]
	\begin{gather*}
	\infer[({=}e)]{F'\{x\mapsto t\}}{s=t & F'\{x\mapsto s\}}
	\qquad
	\infer[({=}i)]{t=t}{}
	\end{gather*}
	\caption{Natural Deduction Rules for Equality}\label{tab:natural:deduction:equality}
\end{table}


\begin{definition}[Natural deduction~\cite{Huth:2004:LCS:975331}]\label{def:natural:deduction}
	Let \( F \), \( G \), and \( H \) be first order formulae and
	\(s\) and \(t\) be first order terms.
	We recall the rules of \coloremph{natural deduction} for connectives
	in Table~\ref{tab:natural:deduction:connectives},
	the rules for equality in Table~\ref{tab:natural:deduction:equality}
	where \( s \) and \( t \) are free for variable \( x \) in formula \( F \),
	and the rules for quantifiers in
	Table~\ref{tab:natural:deduction:quantifiers}
	where
	\( x_0 \) is a \emph{fresh} variable symbol,
	i.e.~\( x_0 \) did not occur in any formula so far.
\end{definition}

\begin{table}[hbt]
	\begin{gather*}
	\begin{array}{ccc}
	\infer[(\forall e)]{F'\{x\to t\}}{
		\forall x F'
	}
	&&
		\infer[(\exists i)]{\exists x F'}{
		F'\{x\mapsto t \}
	}
	\\[1em]
	\infer[(\forall i)]{\forall x F'}{
		\boxed{
			\begin{array}{cc}
				x_0
				\\
				&\vdots
				\\
				&F' \{x\mapsto x_0 x\}
			\end{array}} % end of boxed
	}
	&&
	\infer[(\exists e)]{H}{
		\exists x F' &
		\boxed{
			\begin{array}{cc}
				x_0
				&F' \{ x\mapsto x_0 \}
				\\
				&\vdots
				\\
				&H
			\end{array}} % end of boxed
	}
	\end{array}
	\end{gather*}
	\caption{Natural Deduction Rules for Quantifiers}\label{tab:natural:deduction:quantifiers}
\end{table}




\begin{definition}
	A sentence in first order logic is \coloremph{provable}
	if their exists a proof in a formal proof system for first order logic,
	e.g.~natural deduction.
	We write
	\( F_1, \ldots, F_n \proves G \)
	when we can prove \(G\) from premises \( F_1,\ldots,F_n \), \(n\in\mathbb{N}\).
\end{definition}

A natural deduction proof starts with a (possible empty) set of sentences (the premises) and infer other sentences (the conclusions) by applying the syntactic proof inference rules.
A box must be opened for each assumption, e.g.~a term or a sentence.
Closing the box discards the assumption and all its conclusions within the box{(es)},
but may introduce a derived sentence outside the box{(es)}.
Then \( F_1,\ldots,F_n \proves H \) claims that there is a proof where
\( H \) is outside of any box
and has been inferred (via a finite set of intermediate formulae)
from \( \{ F_1,\ldots,F_n\} \).

\begin{example}We show \( \forall x (\mP(x)\land\lnot\mQ(x)) \proves \forall x(\lnot\mQ(x)\land\mP(x)) \) with natural deduction in Figure~\ref{fig:natural:dedction:proof}.
	First we state our premise (1), then we open a box and assume a fresh variable \( x_0 \)(2),
	we create an instance of our premise with quantifier elimination and the variable \( x_0 \) (3),
	we extract the literals with both variants of conjunction elimination (4, 5),
	we introduce a conjunction of the literals (6),
	and close the box to introduce the universally quantified conjunction (7).
	\begin{figure}
		\begin{center}
	\begin{gather*}
	\begin{BMAT}{rcrclccl}{ccccccccc}
1 && 		&& \forall x(\mP(x)\land\lnot\mQ(x)) 	& && \texttt{premise}\\
2 && x_0 	&& 										& && \\
3 && 	 	&& \mP(x_0)\land\lnot\mQ(x_0)			& && 1: {\forall}e\\
4 && 		&& \lnot\mQ(x_0) 						& && 3: \land e_1\\
5 && 		&& \mP(x_0) 							& && 3: \land e_2\\
6 &&		&& \lnot\mQ(x_0)\land\mP(x_0) 			& && 4+5: \land i\\
7 && 	 	&&	\forall x(\lnot\mQ(x)\land\mP(x))	& && 2-6: \forall i
\addpath{(2,1,1)rrrruuuuullllddddd}
\end{BMAT}
\end{gather*}
\caption{\( \forall x (\mP(x)\land\lnot\mQ(x)) \proves \forall x(\lnot\mQ(x)\land\mP(x)) \)}
\end{center}
\end{figure}\label{fig:natural:dedction:proof}
\end{example}

\begin{definition}
	We write \( F \biprove G \) as shorthand whenever 
	\( F \proves G \) and \( G \proves F \) hold
	which denotes \coloremph{provable equivalence} of first order formulae.
\end{definition}

\begin{lemma}\cite{Huth:2004:LCS:975331}
	Let \( F \), \( G \), and \( H \) be first oder formulae.
	\begin{align*}
		F \land G &\biprove G \land F 
		&
		F \land ( G \lor H) &\biprove (F\land G) \lor (F\land H)
		\\
		F \lor G &\biprove G \lor F
		&
		F \lor ( G \land H) &\biprove (F \lor G) \land (F \lor H)
		\\
		F \limp &\biprove \lnot F \lor G
		&
		\lnot( F \land G) &\biprove \lnot F \lor \lnot G
		\\
		\lnot \lnot F &\biprove F
		&
		\lnot( F \lor G) &\biprove \lnot F \land \lnot G
	\end{align*}
% 
	\begin{align*}
		\lnot \forall x F &\biprove \exists \lnot F & 
		\lnot \exists x F &\biprove \forall \lnot F 
		\\[0.5em]
		(\forall x F) \land G &\biprove \forall x ( F \land G) &
		(\exists x F) \land G &\biprove \exists x ( F \land G ) 
		\tag*{if \( x\not\in\fvar(G)\)} 
		\\
		(\forall x F) \lor G &\biprove \forall x ( F \lor G) &
		(\exists x F) \lor G &\biprove \exists x ( F \lor G )
		\tag*{if \( x\not\in\fvar(G)\)} 
		\\[0.5em]
		(\forall x F) \land (\forall x G) &\biprove \forall x (F \land G) &
		(\exists x F) \lor (\exists x G) &\biprove \exists x (F \lor G)
		\\
		\forall x \forall y F &\biprove \forall y \forall x F &
		\exists x \exists y F &\biprove \exists y \exists x F
	\end{align*}
\end{lemma}
\begin{proof}
	By natural deduction.
\end{proof}