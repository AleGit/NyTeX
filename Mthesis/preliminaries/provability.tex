% !TeX root = ../mthesis.tex
% !TeX encoding = UTF-8
% !TeX spellcheck = en_US

\subsection{Provability}

In general a proof may be a finite sequence of proof steps 
from none or some premises via intermediate statements 
to a final, the then proven statement. 
A formal proof system or logical calculus describes admissible basic proof steps 
in the underlying logic of the statements, in our case first order logic.
A formal proof comprises only proof steps confirmed by rules of the applied logical calculus.

% !TeX root = ../mthesis.tex
% !TeX encoding = UTF-8
% !TeX spellcheck = en_US

\begin{definition}[\cite{Huth:2004:LCS:975331}]\label{def:natural:deduction}
	We recall the rules of \coloremph{natural deduction} for connectives
	in Table \ref{tab:natural:deduction:connectives},
	for equality in Table \ref{tab:natural:deduction:equality},
	and for quantifiers in Table \ref{tab:natural:deduction:quantifiers}.
	Natural deduction provides a logical calculus,
	i.e.~a formal proof system for first order logic.
	The formulae $F$ and $G$ in these rules are sentences,
	the bound variable in  $\forall x F'$ occurs free in $F'$,
	and terms $s$ and $t$ are ground.

\begin{table}[hbt]
\begin{gather*}
\begin{array}{ccccc}
\infer[(\land i)]{F\land G}{F & G}
&
\infer[(\land e_1)]{G}{F \land G}
&
\infer[(\land e_2)]{F}{F \land G}
&
\infer[(\lnot\lnot i)]{\lnot\lnot F}{F}
&
\infer[(\lnot\lnot e)]{F}{\lnot\lnot F}
\\[0.7em]
\infer[(\bot e)]{F}{\bot}
&
\infer[(\lnot e)]{\bot}{F & \lnot F}
&
\infer[\text{LEM}]{F\lor\lnot F}{}
&
\infer[(\lor i_1)]{F\lor G}{F}
&
\infer[(\lor i_2)]{F\lor G}{G}
\\[0.7em]
\infer[(\lnot i)]{\lnot F}{
	\boxed{\begin{array}{c}F\\\vdots\\\bot\end{array}}}
&
\infer[\text{PBC}]{F}{
	\boxed{\begin{array}{c}\lnot F\\\vdots\\\bot\end{array}}}
&
\infer[({\limp} i)]{F\limp G}{
	\boxed{\begin{array}{c}F\\\vdots\\G\end{array}
}}
&
\multicolumn{2}{l}{
	\infer[(\lor e)]{H}{
		F\lor G &
		\boxed{\begin{array}{c}F\\\vdots\\H\end{array}} &
		\boxed{\begin{array}{c}G\\\vdots\\H\end{array}}
	}
}
\\[0.7em]
&
\multicolumn{3}{r}{
\infer[\text{modus}\atop\text{ponens}]{G}{F & F\limp G}
\qquad
\infer[\text{modus}\atop\text{tollens}]{\lnot F}{F\limp G & \lnot G}
}
&
\end{array}
\end{gather*}
\caption{Natural Deduction Rules for Connectives}
\label{tab:natural:deduction:connectives}
\end{table}

\begin{table}[hbt]
	\begin{gather*}
	\infer[({=}e)]{F'\{x\mapsto t\}}{s=t & F'\{x\mapsto s\}}
	\qquad
	\infer[({=}i)]{t=t}{}
	\end{gather*}
	\caption{Natural Deduction Rules for Equality}
	\label{tab:natural:deduction:equality}
\end{table}

\begin{table}[hbt]
	\begin{gather*}
	\begin{array}{ccc}
	\infer[(\forall e)]{F'\{x\to t\}}{
		\forall x F'
	}
	&&
		\infer[(\exists i)]{\exists x F'}{
		F'\{x\mapsto t \}
	}
	\\[1em]
	\infer[(\forall i)]{\forall x F'}{
		\boxed{\begin{array}{cc}t\\&\vdots\\&F'\{x\mapsto t\}\end{array}}
	}
	&&
	\infer[(\exists e)]{H}{
		\exists x F' &
		\boxed{\begin{array}{cc}t&F'\{x\mapsto t \}\\&\vdots\\&H\end{array}}
	}
	\end{array}
	\end{gather*}
	\caption{Natural Deduction Rules for Quantifiers}
	\label{tab:natural:deduction:quantifiers}
\end{table}
\end{definition}

\begin{definition}A sentence in first order logic is provable 
	if their exists a proof in a formal proof system for first order logic, e.g. natural deduction.
	We write
	$F_1, \ldots, F_n \proves G$
	when we can prove G from premisis $F_1,\ldots,F_n$.
\end{definition}

A natural deduction proof starts with a (possible empty) set of sentences -- the premises -- 
and infer other sentences -- the conclusions -- by applying the syntactic proof inference rules.
A box must be opened for each assumption, e.g.~a term or a sentence. 
Closing the box discards the assumption and all its conclusions within the box,
but introduces a conclusion outside the box. 
Then $F_1,\ldots,F_n \proves H$ claims that $H$ 
is in the transitive closure of inferable formulae from $\{ F_1,\ldots,F_n\}$ outside of any box.

\begin{example}We show $\forall x (\mP(x)\land\lnot\mQ(x)) \proves \forall x(\lnot\mQ(x)\land\mP(x))$ with natural deduction.
	We note our premise (1), we open a box and assume an arbitrary constant (2), 
	we create a ground instance of our premise with quantifier elimination and the constant (3),
	we extract the literals with both variants of conjunction elimination (4, 5),
	we introduce a conjunction of the ground literals (6),
	and close the box to introduce the universal quantified conjunction (7).
	\begin{gather*}
	\begin{BMAT}{rcrclccl}{ccccccccc}
1 && 		&& \forall x(\mP(x)\land\lnot\mQ(x)) 	& && \texttt{premise}\\
2 && \mc 	&& 										& && \\
3 && 	 	&& \mP(\mc)\land\lnot\mQ(\mc)			& && 1: {\forall}e\\
4 && 		&& \lnot\mQ(\mc) 						& && 3: \land e_1\\
5 && 		&& \mP(\mc) 							& && 3: \land e_2\\
6 &&		&& \lnot\mQ(\mc)\land\mP(\mc) 			& && 4+5: \land i\\
7 && 	 	&&	\forall x(\lnot\mQ(x)\land\mP(x))	& && 2-6: \forall i
\addpath{(2,1,1)rrrruuuuullllddddd}	
\end{BMAT}
\end{gather*}
\end{example}

%	\infer[\forall i]
%	{
%		\forall x( \lnot\mQ(x) \land \mP(y))
%	}
%	{
%		\infer[\land i]
%		{ \lnot\mQ(\mc) \land \mP(\mc) }
%		{
%			\infer[\land e_1]
%			{\mQ(\mc)} 
%			{ 
%				\infer[\forall e]
%				{ \mP(\mc)\land\lnot\mQ(\mc)}
%				{ \forall x (\mP(x)\land\lnot\mQ(x)) }
%			}
%			&
%			\infer[\land e_2]
%			{\mP(\mc)}
%			{ \colG
%				\infer[\forall e]
%				{ \mP(\mc)\land\lnot\mQ(\mc)}
%				{ \forall x (\mP(x)\land\lnot\mQ(x)) }
%			}
%		}
%	}
%\\