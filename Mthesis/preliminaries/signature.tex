% !TeX root = ../mthesis.tex
% !TeX encoding = UTF-8
% !TeX spellcheck = en_US

\begin{definition}\label{def:signature}
A
first order
{\myem signature} with equality
%with equality
$\mcF = \mcFfPE$
is the disjoint union of
a set of {\myem function symbols} $\mcFf$,
a set of {\myem predicate symbols} $\mcFP$,
and one distinct equality symbol.
%
The {\myem arity} of a symbol determines the number of its arguments in a first order expression.
With $\mcFn = \{ \mcf\in\mcF \mid \arity(\mcf) = n \}$ we denote symbols with arity $n$.
\end{definition}

\begin{remark}
    We use $\mEQ$ as equality symbol in our signatures to emphasize
    that at this point it is just a highlighted symbol
    without “meaning”.
    On the other hand we use $=$ to express “identity” of objects
    like formualae or sets
    without actually defining how this identity can be determined.
\end{remark}
