% !TeX root = ../mthesis.tex
% !TeX encoding = UTF-8
% !TeX spellcheck = en_US

\begin{definition}\label{def:position}
	A {\myem position} is a finite sequence of positive integers.
	The root position is the empty sequence $\epsilon$.
	The position $pq$ is obtained by concatenation of positions $p$ and $q$.
	%
	A position $p$ is above a position $q$ if $p$ is a prefix of $q$, 
	i.e.~there exists a unique position $r$ such that $pr = q$, 
	we write $p\leq q$ and $q\backslash r = p$.
	We write $p<q$ if $p$ is a proper prefix of $q$, i.e.~$p\leq q$ but $p\neq q$.
	Two positions are parallel if none is above the other.
	
\end{definition}
\begin{definition}
	
%	We define the set of positions of a term $t$,
%	\DEFINE{ 
%		\pos(t) }
%	{
%		\{ \epsilon \} 		
%		& \text{if }t = x \in \mcV\\
%		%		
%		\{ \epsilon \} \cup \bigcup_{i=1}^{n} \{ iq\mid q\in\pos(t_i) \}	
%		&	\text{if }t=\mf(t_1,\ldots,t_n)\text{ where } \mf\in\mcFfn.\\
%	}
	%
	
	\DEFINE{
		t|_p
	}{
		t 		& \text{if }p=\epsilon \\
		t_i|_q	& \text{if }t=\mf(t_1,\ldots,t_n), p=iq, \mf\in\mcFfn, 0 < i \leq n
	}
	%
	and the insertion of a term at a position
	\DEFINE{
		t[s]_p}
	{
		s 		& \text{if }p=\epsilon \\
		\mf(t_1,\ldots,t_i[s]_q,\ldots,t_n)	& \text{if }t=\mf(t_1,\ldots,t_n), p=iq, \mf\in\mcFfn, 0 < i \leq n
	}
	%A {\myem hole} denotes a special constant symbol $\ctxhole\in\mcFf^{(0)}$. 
	%A {\myem context} is a term $t$ with exactly one hole, i.e.~one occurrence $|t|_{\ctxhole}=1$.
\end{definition}