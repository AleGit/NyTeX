% !TeX root = ../mythesis.tex
% !TeX encoding = UTF-8
% !TeX spellcheck = en_US

\begin{definition}\label{def:terms}
	We build the set of (first order) \coloremph{terms }\( \mcTf = \mcTFfV \)
	from function symbols and a
	countable set of \coloremph{variables }\( \mcV \) disjoint from \( \mcF \) \!.
	Every variable \( x\in\mcV \) is a term,
	every \coloremph{constant} \( \mc\in\mcFf^{(0)} \) is a term,
	and every expression \( \mf(t_1,\ldots,t_n) \) is a term
	for \( n>0 \), function symbol \( \mf\in\mcFfn \),
	and arbitrary terms \( t_1,\ldots,t_n \).
\end{definition}

\begin{definition}\label{def:term:vars}\label{def:term:ground}
	We define the set of variables of a first order term \( t \) as follows:
	\begin{gather*}
	\MDEFINE{\var(t)}{ll}{
		\{ x \} & \text{if } t = x \in \mcV \\
		\bigcup_{i=1}^n \var(t_i) & \text{if }  t = \mf(t_1, \ldots t_n)
	}
	\end{gather*}
	A term \( t' \) is \coloremph{ground} if and only if \( \var(t') = \emptyset \),
	i.e.~it does not contain any variables.
\end{definition}

\begin{definition}\label{def:unary:power}
	For a unary function symbol \( \mg\in\mcFfn[1] \), a natural number
	\( i\in\mathbb{N} \), and an arbitrary term \( t\in\mcTf \) we introduce the convenience notation \( \mg^i(t) \) defined as follows:
	\[
	\mg^0(t) := t \qquad
	\mg^{i+1}(t) := \mg(\mg^i(t))
	\]
\end{definition}
