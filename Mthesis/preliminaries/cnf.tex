% !TeX root = ../mthesis.tex
% !TeX encoding = UTF-8
% !TeX spellcheck = en_US

\begin{definition}\label{def:literals}
A {\myem literal} $L$ is either an atom or the negation of an atom.
%
A {\myem clause}\ \ $\mcC = L_1\lor\ldots\lor L_n$  is a possible empty multiset of literals 
and is equivalent to the universally quantified disjunction of its literals.
The {\myem empty clause} $\emptyclause$ expresses a contradiction. 
%
A finite {\myem set of clauses} $S=\{ \mcC_1,\ldots,\mcC_n \}$ is equivalent to the conjunction of its clauses.
The empty set of clauses expresses a tautology.

%We define the {\myem complement }of a literal $L$ as follows
%\[
%L^c = \left\{
%\begin{array}{rl}
%\lnot L & \text{if $L$ is an atom} \\
%A 	& \text{if }L = \lnot A
%\end{array}
%\right.
%\]
%Two literals are complementary if one is an atom, i.e. a positive literal and the other is the negation of this atom.
\end{definition}

\begin{definition}We define the set of subterms, the set of variables of a first order expression $\mkt$ (i.e. term, atom, literal, clause)
	
	\begin{align*}
		\var(\mkt) &= \left\{\begin{array}{ll}
			\{ \mkt \} & \text{if } \mkt  \in \mcV)\\
			\{\ \} &\text{if } \mkt \in \mcFO\\
			\bigcup & \text{if }  \mkt = \mkf(t_1, \ldots t_n), \mkf \in \mcFn, n > 0
%			\emptyset &\text{if } \mkt \in \mcFO
		\end{array}\right.	
	\end{align*}
	
\end{definition}