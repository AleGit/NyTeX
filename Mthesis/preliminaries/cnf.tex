% !TeX root = ../mthesis.tex
% !TeX encoding = UTF-8
% !TeX spellcheck = en_US

\begin{definition}[\CNF]\label{def:literals}\label{def:syntax:CNF}
A (first order) {\myem literal} $L$ is either an atom $A$ or the negation\footnote{
	We usually abbreviate $\lnot(s\mEQ t)$ with $s\mNE t$.
} 
$(\lnot A)$ of an atom.
%
The {\myem complement} $L^\mcc$ of an atom (positive literal) is the negation of the atom.
The complement of a negated atom (negative literal) is the atom itself. 
%\DEFINE{
%	L^\mcc 
%}{
%	\lnot L & \text{if } L \text{ is an atom} \\
%	A 		& \text{if } L = \lnot A, \text{ the negation of an atom } A
%}
%
A {\myem clause}\ \ $\mcC = L_1\lor\ldots\lor L_n$  is a possible empty multiset of literals 
and is equivalent to an universally quantified disjunction of its literals.
%The {\myem empty clause} $\emptyclause$ expresses a contradiction. 
%
A finite {\myem set of clauses} $S=\{ \mcC_1,\ldots,\mcC_n \}$ is equivalent to a conjunction of all its clauses.
%
%Two literals are complementary if one is an atom, i.e. a positive literal and the other is the negation of this atom.
We call this quantifier-free representation {\myem clausal normal form}.
\end{definition}

\begin{example}The set of clauses is equivalent to a universally quantified formula in conjunctive normal form.
	\begin{align*}
		\{\, \mP(x)\lor\mR(x,y), \lnot \mE(x,\mf(x)) \, \}
		&\equiv
		\forall x \forall y\, (\mP(x)\lor\mR(x,y))
		\land
		\forall x\, (\lnot \mE(x,\mf(x))) 
		\\
		{\colG
			\{\, \mP(x)\lor\mR(x,y), \lnot \mE(z,\mf(z)) \, \}}
		&\equiv
		\forall x \forall y\, (\mP(x)\lor\mR(y))
		\land
		\forall z\,(\lnot \mE(z,\mf(z)) 
		\\
		&\equiv
		\forall x \forall y\forall z\, 
		(\mP(x)\lor\mR(y)\land \lnot \mE(z,\mf(z))
	\end{align*}
\end{example}

\begin{definition}
	We denote
	terms, atoms, and literals as first order expressions.
	We call a first order expression without variables {\myem ground} or {\myem closed}, 
	i.e.~we build our ground first order expressions over an empty set of variables.
\end{definition}

