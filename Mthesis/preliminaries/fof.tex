% !TeX root = ../mthesis.tex
% !TeX encoding = UTF-8
% !TeX spellcheck = en_US

\begin{definition}[\index{FOF}\FOF]\label{def:syntax:FOF}
	Atoms from the previous definition are {\myem first order formulae}.  
	The universal quantification $(\forall x F)$ 
	and the existential quantification $(\exists x F)$ 
	of a first order formula are (quantified) first order formulae
	with {\myem bound} variable $x\in\mcV$.
	The negation $(\lnot F)$ of a first order formula
	is a (composite) first order formula.
	Further, the disjunction $(F \lor F')$, 
	the conjunction $(F \land F') $, 
	and the implication $(F \limp F')$ 
	of two first order formulae
	are (composite) first order formulae.
\end{definition}

\begin{definition}\label{def:term:vars}\label{def:fof:fvars}\label{def:fof:sentence}
	We define the set of variables of a first order term $t$ and the set of {\myem free} variables of a first order formula $F$ as follows:
\DEFINE{\var(t)} {
		\{ x \} & \text{if } t = x \in \mcV \\
		\bigcup_{i=1}^n \var(t_i) & \text{if }  t = \mf(t_1, \ldots t_n), \mf \in \mcFfn
	}
\DEFINE{\fvar(F)}{
		\bigcup_{i=1}^n \var(t_i) &\text{if }F = P(t_1,\ldots,t_n)\text{ or }t_1\mEQ t_{n=2} 
		\\
		\fvar (G) \setminus \{ x\} &\text{if } F \in\{\, \forall x\,G, \exists x\,G\,\}
		\\
		\fvar(G) \cup \fvar(H)&\text{if }F \in\{\,G\land H,G\lor H,G\limp H\,\}
%		\\
%		\var(s) \cup \var(t) &\text{if }F = s\mEQ t
		\\
		\fvar(G)&\text{if }F = \lnot G
	}

\noindent A first order {\myem sentence} is a closed formula, 
i.e.~a formula without free variables, 
over an implicit or given signature.
\end{definition}

\begin{remark} We often will write formulae without a given signature.
The reader can easily deduce the underlying {\myem implicit} signature with arities and the set of variables
by applying the definitions of the syntax for first order formulae.
In the presence of free variables we follow the convention to use $x,y,z$ for variables 
and $\ma,\mb,\mc$ for constant function symbols to avoid ambiguity. 
For easier readability we will use uppercase for predicate symbols and lowercase for function symbols .
\end{remark}