% !TeX root = ../mthesis.tex
% !TeX encoding = UTF-8
% !TeX spellcheck = en_US

\begin{definition}[\index{FOF}\FOF]\label{def:syntax:FOF}
	Atoms from the previous definition are {\myem first order formulae}.  
	The universal quantification $(\forall x F)$ 
	and the existential quantification $(\exists x F)$ 
	of a first order formula are (quantified) first order formulae
	with {\myem bound} variable $x\in\mcV$.
	The negation $(\lnot F)$ of a first order formula
	is a (composite) first order formula.
	Further, the disjunction $(F \lor F')$, 
	the conjunction $(F \land F') $, 
	and the implication $(F \limp F')$ 
	of two first order formulae
	are (composite) first order formulae.
\end{definition}

\begin{definition}\label{def:fof:fvars}\label{def:fof:bvars}
	We define the set of {\myem free} variables and the set of {\myem bound} variables 
	of a first order formula $F$ as follows:
\DEFINE{\fvar(F)}{
		\bigcup_{i=1}^n \var(t_i) &\text{if }F = \mP(t_1,\ldots,t_n)\text{ or }t_1\mEQ t_{n=2} 
		\\
		\fvar(G)&\text{if }F = \lnot G
		\\
		\fvar(G) \cup \fvar(H)&\text{if }F \in\{\,G\land H,G\lor H,G\limp H\,\}
		\\
		\fvar (G) \setminus \{ x\} &\text{if } F \in\{\, \forall x\,G, \exists x\,G\,\}
	}
\DEFINE{\bvar(F)}{
	\emptyset &\text{if }F = \mP(t_1,\ldots,t_n)\text{ or }t_1\mEQ t_{n=2} 
	\\
	\bvar(G)&\text{if }F = \lnot G
	\\
	\bvar(G) \cup \bvar(H)&\text{if }F \in\{\,G\land H,G\lor H,G\limp H\,\}
	\\
	\bvar (G) \cup \{ x\} &\text{if } F \in\{\, \forall x\,G, \exists x\,G\,\}
}
\end{definition}

\begin{example}
	With formula $F = (\forall x (x\mEQ y)) \lor (\exists y \mP(x,y)$ we have $\fvar(F) = \bvar(F)$.
	%	 = \{ x, y\}$.
\end{example}

%\begin{remark} 
We often will write formulae or sentences 
without stating the signature.
The reader can easily deduce the underlying {\myem implicit} signature with arities 
and the set of variables by applying the definitions of the syntax for first order formulae.
We follow the convention to use $x,y,z$ for variables 
and $\ma,\mb,\mc$ for constant function symbols 
(which avoids ambiguity in the presence of free variables). 
For easier readability we will use uppercase predicate symbols and lowercase function symbols.
We may denote $\mcF(F)$ for the implicit signature of an formula $F$.
%\end{remark}

\begin{definition}\label{def:fof:closed}\label{def:fof:sentence}
	A first order formula is closed, i.e.~a first order {\myem sentence}, 
	if it does not contain free variables -- all occurring variables are bound.
%	A quantifier free sentence is ground. 
	Additionally we assume for any sentence 
	that each variable is bound exactly once, 
	but occurs as free variable in a subformula, 
	or more formally
	\begin{align*}
	&x\in\fvar(G), x\not\in\bvar(G)& \text{if }&F \in\{\, \forall x\,G, \exists x\,G\,\}
	 \\
	&\bvar(G)\cap\bvar(H)=\emptyset& \text{if }&F \in\{\,G\land H,G\lor H,G\limp H\,\}
	\end{align*}
\end{definition}

\begin{example}
	\[
	\begin{array}{clcl}
		\forall x ({\mP(x) \lor \forall x {\mQ(x)}}) && \forall x ({\mP(x) \lor \forall y {\mQ(y)}}) &  \checkmark\\
		(\forall x {\mP(x)}) \lor (\forall x {\mQ(x)}) & & (\forall x {\mP(x)}) \lor (\forall y {\mQ(y)}) & \checkmark
	\end{array}
	\]
	
\end{example}

\begin{definition}[\PNF]
	A first order sentence $F = \quantify_1 x_1 \ldots \quantify_n x_n\, G$ 
	with $n$ quantifiers $\quantify_i \in \{\exists,\forall\}$,
	$n$ bound and distinct variables $x_i$, 
	and quantifier free subformula $G$ with 
	a matching set of free variables, i.e.~$\fvar(G) = \bvar(F)$,
	is in {\myem prenex normal form}.
\end{definition}


