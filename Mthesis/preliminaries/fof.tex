% !TeX root = ../mthesis.tex
% !TeX encoding = UTF-8
% !TeX spellcheck = en_US

\begin{definition}[\FOF]\label{def:syntax:FOF}
	Atoms from the previous definition are {\myem first order formulae}.  
	The universal quantification $(\forall x F)$ 
	and the existential quantification $(\exists x F)$ 
	of a first order formula are (quantified) first order formulae.
	The negation $(\lnot F)$ of a first order formula
	is a (composite) first order formula.
	Further, the disjunction $(F \lor F')$, 
	the conjunction $(F \land F') $, 
	and the implication $(F \limp F')$ 
	of two first order formulae
	are (composite) first order formulae.
\end{definition}

\begin{definition}\label{def:term:vars}\label{def:fof:fvars}\label{def:fof:sentence}
	We define the set of variables of a first order term $t$
	\DEFINE{\var(t)} {
		\{ x \} & \text{if } t = x \in \mcV \\
		\bigcup_{i=1}^n \var(t_i) & \text{if }  t = \mf(t_1, \ldots t_n), \mf \in \mcFfn
	}
and the set of {\myem free} variables of a first order formula $F$.
\DEFINE{\fvar(F)}{
		\bigcup_{i=1}^n \var(t_i) &\text{if }F = P(t_1,\ldots,t_n)\text{ or }t_1\mEQ t_2 
		\\
		\fvar (G) \setminus \{ x\} &\text{if } F = \forall x\, G \text{ or } \exists x\, G
		\\
		\fvar(G) \cup \fvar(H)&\text{if }F = G\land H, G\lor H\text{ or } G\limp H
%		\\
%		\var(s) \cup \var(t) &\text{if }F = s\mEQ t
		\\
		\fvar(G)&\text{if }F = \lnot G
	}
A first order {\myem sentence} is a closed formula, 
i.e.~it has no free variables, 
over a signature.
\end{definition}