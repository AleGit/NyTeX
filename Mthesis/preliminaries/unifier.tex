% !TeX root = ../mthesis.tex
% !TeX encoding = UTF-8
% !TeX spellcheck = en_US

\begin{definition}\label{def:unifier}
Two first order expressions $\mct, \mcu$ are {\myem unifiable} if there exists a {\myem unifier}, 
i.e.~a substitution $\sigma$ such that $\mct\sigma=\mcu\sigma$.
The {\myem most general unifier} $\mgu(\mct,\mcu)$ is a unifier such that
for every other unifier $\sigma'$ there exists a substitution $\tau$ where
$\sigma' = \sigma \tau$. 
Two literals are variants if their most general unifier is a renaming.
Two literals are {\myem clashing} when the first literal 
and the complement of the second literal are unifiable, 
i.e.~literals $L'$ and $L$ are clashing if $\mgu(L', L^c)$ exists.
\end{definition}

\begin{example}
	Literals $\mP(x)$ and $\lnot\mP(\mf(\ma,y))$ 
	are clashing by unifier $\{ x\mapsto \mf(\ma,y)\}$.
\end{example}