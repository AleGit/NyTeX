% !TeX root = ../mythesis.tex
% !TeX encoding = UTF-8
% !TeX spellcheck = en_US

\begin{definition}\label{def:unifier}
Two first order expressions \( \mct, \mcu \) are \coloremph{unifiable} if there exists a \coloremph{unifier},
i.e.~a substitution \( \sigma \) such that \( \mct\sigma=\mcu\sigma \).
The \coloremph{most general unifier} \( \mgu(\mct,\mcu) \) is a unifier such that
for every other unifier \( \sigma' \) there exists a substitution \( \tau \) where
\( \sigma' = \sigma \tau \).
Two literals are variants if their most general unifier is a renaming.
Two literals are \coloremph{clashing} when the first literal
and the complement of the second literal are unifiable,
i.e.~literals \( L' \) and \( L \) are clashing if \( \mgu(L', L^c) \) exists.
\end{definition}

\begin{remark}
	The unification of quantified formulae remains undefined in this thesis.
\end{remark}

\begin{example}
	Literals \( \mP(x) \) and \( \lnot\mP(\mf(\ma,y)) \)
	are clashing by unifier \( \{ x\mapsto \mf(\ma,y)\} \).
\end{example}