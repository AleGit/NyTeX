% !TeX root = ../mthesis.tex
% !TeX encoding = UTF-8
% !TeX spellcheck = en_US

\begin{definition}\label{def:unifier}
Two terms or literals $T, U$ are {\myem unifiable} if there exists a substitution $\sigma$ such that $T\sigma=U\sigma$.
The {\myem most general unifier} $\mgu(T,U)$ is a unifier such that
for every other unifier $\sigma'$ there exists a substitution $\tau$ where
$\sigma' = \sigma \tau$. 
Two literals are variants if their $\mgu$ is a renaming.
Two literals are {\myem clashing} when the complement of one literal 
and the unchanged other literal are unifiable, i.e.~$L$ and $L'$ are clashing if $\mgu(L^c, L')$ exists.
\end{definition}