% !TeX root = ../mthesis.tex
% !TeX encoding = UTF-8
% !TeX spellcheck = en_US

\chapter{Preliminaries}





\epigraph{
	A good theory starts 
	
	with a good definition
}{
	unknown
}

We assume familiarity with propositional and predicate logic \cite{Huth:2004:LCS:975331}, 
automated theorem proving \cite{Fitting:1996:FLA:230183}, 
term rewriting \cite{Baader:1998:TR:280474}, 
decision procedures \cite{Kroening:2008:DPA:1391237}, 
and (maximum) satisfiability testing \cite{Biere:2009:HSV:1550723}.
Even so, for clarity we introduce basic
definitions and state basic lemmas and theorems without proofs.
This section is an extension of the same section in our seminar report \cite{axm:SR2}
where notions and notations largely follow those in lecture notes \cite{AM2015tr} and \cite{GM2013ar}.


\section{Syntax}

% !TeX root = ../mthesis.tex
% !TeX encoding = UTF-8
% !TeX spellcheck = en_US

\begin{definition}\label{def:signature}
A 
%first order 
{\myem signature} 
%with equality
$\mcF = \mcFfPE$ 
is the disjoint union of 
{\myem function symbols} $\mcFf$, 
{\myem predicate symbols} $\mcFP$,
and the equality symbol.
%
The {\myem arity} of a symbol is the number of its arguments.
With $\mcFn$ we denote symbols with arity $n$.
\end{definition}


% !TeX root = ../mthesis.tex
% !TeX encoding = UTF-8
% !TeX spellcheck = en_US

\begin{definition}\label{def:terms}
	We build the set of (first order) {\myem terms }$\mcTf = \mcTFfV$ 
	from function symbols and a
	countable set of {\myem variables }$\mcV$ disjoint from $\mcF$\!.
	Every variable $x\in\mcV$ is a term,
	every {\myem constant} $\mc\in\mcFf^{(0)}$ is a term, 
	and every expression $\mf(t_1,\ldots,t_n)$ is a term
	for $n>0$, function symbol $\mf\in\mcFfn$,  
	and arbitrary terms $t_1,\ldots,t_n$.
\end{definition}



% !TeX root = ../mthesis.tex
% !TeX encoding = UTF-8
% !TeX spellcheck = en_US

\begin{definition}\label{def:predicates}
	We build the set of {\myem predicate} $\mcPT$
	from predicate symbols and terms. 
	Every proposition $p\in \mcFP$ is a predicate, 
	and every expression $P(t_1,\ldots,t_n)$ is a predicate for $n>0$,
	predicate symbol $P\in\mcFPn$ and arbitrary terms $t_1,\ldots,t_n$.
%	
	We build the set of {\myem equations }$\mcET$ from the equality symbol and pairs of terms.
	Every expression $s\mEQ t$ is an equation\footnote{
		We use prefix and infix notation interchangable, e.g.~${\mEQ}(s,t)$ represents the same equation as $s\mEQ t$.} 
	for arbitrary terms $s$ and $t$.
%	
	The set of atomic formulas (or {\myem atoms }for short) is the union of predicates and equations.
\end{definition}

\input{preliminaries/ground}

In Section \ref{sec:syntax} we have defined atomic formulae in Definition \ref{def:atoms}, 
but we can only build formulae in clausal normal form (\CNF) with Definition \vref{def:syntax:CNF}.
Now we will define arbitrary first order formulae (\FOF).

\begin{definition}[\FOF]\label{def:syntax:FOF}
	Predicates and equations are (atomic) first order formulae. 
	The negation $(\lnot F)$, 
	the universal quantification $(\forall x F)$, 
	and the existential quantification $(\exists x F)$ 
	of a given formula $F$ are (composite) first order formulae.
	Further, the disjunction $(F \lor F')$, 
	the conjunction $(F \land F') $, 
	and the implication $(F \limp F') $ 
	of two given formulae $F$ and $F'$ 
	are (composite) first order formulae.
\end{definition}

We've already defined when an atom holds for an assignment $\alpha_\mcI$ 
in an interpretation $\mcI$ within Definition \vref{def:model}.
Now we extend these definitions to arbitrary formulae.

\begin{definition}[Semantics of \FOF]\label{def:semantics:FOF}
	
	A universally quantified formula $\forall x F$ holds in $\mcI$ if its subformula $F$ holds for all assignments for $x$.
	An existential quantified formula $\exists xF$ holds if its subformula $F$ holds for at least one assignment for $x$.
	A negation $\lnot F$ holds if its subformula $F$ does not hold, 
	a disjunction $F\lor F'$ holds if one or both of its subformulae $F$ or $F'$ hold,
	a conjunction $F\land F'$ holds, if both of its subformualae $F$ and $F'$ hold, 
	an implication $F\limp F'$ holds if its first subformula $F$ does not hold or its second subformula $F'$ holds (or both).
\end{definition}

\begin{remark}Usually we us precedences on connectives to omit parentheses 
	and some heuristics to structure the formulae for readability 
	without introducing semantic ambiguity.
%
	Beside the obvious semantically indistinguishable formulae with double negations, conjunctions, and disjunctions 
	we have introduced new ones.
	\begin{enumerate}
		\item $\forall x F$, $\exists x F$, and $F$ are indistinguishable if $x\not\in\var(F)$. 
		We usually omit quantifiers with variables that do not occur in subformulae.
		\item In general $\exists x F$ is different from $F$ if $x\in\var(F)$, e.g. $\exists x(x\mNE\ma)$ is satisfiable and $x\mNE\ma$ isn't.
		\item $\forall x F$ and $F$ are equivalent even if $x\in\var(F)$, 
		because in both cases we demand that $F$ holds in all assignments in our model.
		Usually we keep these universal quantifiers in \FOF.
		
		A first order formulae without quantifiers is in {\myem clausal form}, 
		but not necessarily in \CNF, e.g.~a weakened version of symmetry $(x\mEQ \ma)\limp (\ma\mEQ x)$ 
		is equisatisfiable to $\forall x ((x\mEQ \ma)\limp (\ma\mEQ x))$ 
		or $\exists a (\forall x ((x\mEQ a)\limp (a\mEQ x))$. 
	\end{enumerate}

\end{remark}

% !TeX root = ../mthesis.tex
% !TeX encoding = UTF-8
% !TeX spellcheck = en_US

\begin{definition}\label{def:literals}
A {\myem literal} $L$ is either an atom $A$ or a negated atom $\lnot A$.
The {\myem complement } $L^\mcc$ of an atom (positive literal) is the negation of the atom.
The complement of a negated atom (negative literal) is the atom itself. 
%\DEFINE{
%	L^\mcc 
%}{
%	\lnot L & \text{if } L \text{ is an atom} \\
%	A 		& \text{if } L = \lnot A, \text{ the negation of an atom } A
%}
%
A {\myem clause}\ \ $\mcC = L_1\lor\ldots\lor L_n$  is a possible empty multiset of literals 
and is equivalent to a universally quantified disjunction of its literals.
%The {\myem empty clause} $\emptyclause$ expresses a contradiction. 
%
A finite {\myem set of clauses} $S=\{ \mcC_1,\ldots,\mcC_n \}$ is equivalent to a conjunction of all its clauses.
%
%Two literals are complementary if one is an atom, i.e. a positive literal and the other is the negation of this atom.
\end{definition}

\begin{definition}
	We call a well formed first order expression without variables {\myem ground}.
\end{definition}



% !TeX root = ../mthesis.tex
% !TeX encoding = UTF-8
% !TeX spellcheck = en_US

% In the following definitions $\mcf$ denotes a function symbol, a predicate symbol or the equality sign.


\begin{definition}\label{def:substitution}
	A \coloremph{substitution} $\sigma$ is a mapping from variables $x\in\mcV$ to terms in $\mcTFfV$
	where \coloremph{domain }$\domain(\sigma) = \{ x\in\mcV\mid\sigma(x) \neq x \}$
	and image $\image(\sigma) =\{ \sigma(x) \mid x\in V, \sigma(x)\neq x \}$ are finite.
	We write substitutions as bindings, e.g.~$\sigma=\{ x_1\mapsto s_1,\ldots,x_n\mapsto s_n \}$
	where $\domain(\sigma)=\{ x_1,\ldots,x_n \}$ and $\sigma(x_i)=s_i$.
	A \coloremph{variable substitution} is a mapping from $\mcV$ to $\mcV\subseteq\mcTFfV$.
	A \coloremph{renaming} is a bijective variable substitution.
	A \coloremph{proper instantiator} is a substitution that is not a variable substitution
	(at least one variable is mapped to a non-variable term).
\end{definition}
\begin{definition}
	We define the instance $\mct\sigma$
	respectively the application of a substitution $\sigma$ to a literal or term $\foxt$ as follows
	\DEFINE{
		\foxt\sigma
	}{
			s_i 				& \text{if }\foxt=x_i\in\domain(\sigma), \sigma(x_i) = s_i
			\\
			y					& \text{if }\foxt=y\in\mcV\,\backslash\domain(\sigma)
			\\
			\foxf(\foxt_1\sigma,\ldots,\foxt_n\sigma)	&\text{if }\foxt=\foxf(\foxt_1,\ldots,\foxt_n)
			\text{ where  }
			 \foxf\in\mcFn \\
			 \lnot (A\sigma) &\text{if }\foxt = \lnot A, \text{ where } A\text{ is an atom}
			 }

\noindent Further we define the instance of a clause as the multiset of the instances of its literals.
\end{definition}

\begin{definition}
	We can easily extend our definition to composite first order formulae,
	but the cases of quantified formulae need more consideration.
	So we only partially define $F\sigma$ for first order formulae $F$ and substitution $\sigma$ as follows (if $G\sigma$ and $H\sigma$ are defined in the respective cases).
	\DEFINE{
		F\sigma
	}
{
		\lnot(G\sigma) & \text{if }F = \lnot G
		\\
		(G\sigma) * (H\sigma) & \text{if }F = G * H, *\in \{ \land, \lor, \limp \}\\
		\quantify x(G\sigma) &\text{if }F = \quantify x G,
		\quantify\in\{ \forall,\exists \},
		x\not\in\domain(\sigma)
		% \\
		% \texttt{undefind} & \texttt{otherwise}
		% x\not\in\bvar(G)
	}
\end{definition}

\begin{definition}
	A clause $\mcC$ \coloremph{strictly subsumes} a clause $\mcD$ if their exists a substitution $\theta$ such that $\mcC\theta \subsetneq \mcD$,
	e.g.~when clause $\mcD = \mcC\theta \lor \mcD'$ is a weakened instance of clause $\mcC$.
\end{definition}
%\begin{example}
%	Trivially any clause $\mcC$ subsumes all its instances $\mcC\sigma$ or weakenings $\mcC\lor\mcD$.
%	Further $x\mEQ y$ subsumes $x\mEQ x \lor \mP(y)$.
%\end{example}

\begin{definition}\label{def:substcomp}
We define the \coloremph{composition} of two substitutions $\sigma$ and $\tau$ as follows
	\begin{align*}
		\sigma\tau&=\{ x_i\mapsto s_i\tau\mid x_i\in\domain(\sigma) \}
		\cup
		\{ y_i\mapsto t_i\mid y_i\in\domain(\tau) \backslash \domain(\sigma) \}.
	\end{align*}
\end{definition}

\begin{lemma}\label{lem:substitution}
	With the definitions in \ref{def:substitution} and \ref{def:substcomp} the equation
	$(\mct\sigma)\tau = \mct(\sigma\tau)$ holds for
	term, atoms, and literals.
\end{lemma}

\begin{proof}
	Assume $\sigma$ and $\tau$ are substitutions.
	Then we use induction on the structure of the 	expression $\mct$
	that the equation $(\mct\sigma)\tau =  \mct(\tau\sigma)$ holds in all possible cases.
	\begin{itemize}
		\item (base case) Let $\mct = x_i \in\domain(\sigma)$ then
		$((x_i)\sigma)\tau\defEQ s_i\tau\defEQ x_i(\sigma\tau)$ holds.

		\item (base case) Let $\mct = y \not\in\domain(\sigma)$ then
		$(y\sigma)\tau \defEQ y\tau \defEQ y(\sigma\tau)$ holds.

		\item (step case) Let $\mct = \mcf(t_1,\ldots,t_n)$
		then
		$((\mcf(t_1,\ldots,t_n))\sigma)\tau
		\defEQ
		(\mcf(t_1\sigma,\ldots,t_n\sigma))\tau
		\defEQ
		\mcf((t_1\sigma)\tau,\ldots,(t_n\sigma)\tau)
		\defEQ[IH]
		\mcf(t_1(\sigma\tau),\ldots,t_n(\sigma\tau))
		\defEQ
		(\mcf(t_1,\ldots,t_n))(\sigma\tau)
		$ holds.

		\item (step case) Let $\mct = \lnot A$ then
		$((\lnot A)\sigma)\tau
		\defEQ
		(\lnot(A\sigma))\tau
		\defEQ
		\lnot((A\sigma)\tau)
		\defEQ[IH]
		\lnot(A(\sigma\tau))
		\defEQ
		(\lnot A)(\sigma\tau)
		$ holds.
	\end{itemize}
\end{proof}


% !TeX root = ../mthesis.tex
% !TeX encoding = UTF-8
% !TeX spellcheck = en_US

\begin{definition}\label{def:unifier}
Two terms $s$ and $t$ are {\myem unifiable} if there exists a substitution $\sigma$ such that $s\sigma=t\sigma$.
They are {\myem variants} if their most general unifier is a renaming.
The {\myem most general unifier} $\sigma=\mgu(s,t)$ is a unifier such that
for every other unifier $\sigma'$ there exists a substitution $\tau$ such that
$\sigma' = \sigma \tau$. 
\end{definition}

\section{Semantics}

\begin{definition}
	An {\myem interpretation} $\mcI$ over a signature $\mcF$ consists of a
	non-empty set $A$ -- the {\myem universe} or {\myem domain},
	definitions of functions $\mf^\mcI: A^n\rightarrow A$ for every function symbol $\mf\in\mcFf$, 
	and definitions of possibly empty relations 
	 ${\mP^\mcI}\subseteq A^n$ for every predicate symbol $\mP\in\mcFP$
	 (and ${\mEQ^\mcI}\subseteq A^2$ for the equality symbol)
	
	A ground predicate $\mP(t_1,\ldots,t_n)$ holds in interpretation $\mcI$ 
	if $\mP(t_1,\ldots,t_n) \in A^n$.
	A ground literal holds in $\mcI$ if and only if its complementary literal does not hold in $\mcI$.
	
	A non-ground literal holds in $\mcI$ if all its 
	(perhaps infinitely many)
	instances hold in $\mcI$.
	A clause holds in $\mcI$, if at least one of its literals holds in $\mcI$.
	A set of clauses $S$ holds in $\mcI$, if every clause $\mcC\in S$ holds in $\mcI$.
	We say $\mcI$ is a model {\em for} $S$ if $S$ holds in $\mcI$. 
\end{definition}

\begin{definition}\label{def:hk}
	An {\myem Herbrand universe} is the smallest set that contains all $H_k\ge 0$ of
	\begin{align*}
	H_0 &:= \left\{ 
	\begin{array}{ll}
	\{ \mc \mid \mc\in\mcFfn[0] \} 
	&\text{if } \mcFfn[0]\not=\emptyset\\
	\{ \mc \}
	&\text{if } \mcFfn[0]=\emptyset, \mc\not\in\mcF
	\end{array}
	\right. 
	\\
	H_{k+1} &:= H_k \cup \{\  
	\mf(t_1,\ldots,t_k) \mid
	\mf\in\mcFfn[k],
	t_1,\ldots,t_k \in H_k
	\ \}
	\end{align*}
	An {\myem Herbrand interpretation} $\mcH$ is an interpretation where the domain 
	is an Hebrand unverse
\end{definition}

\begin{definition}
	and the interpretation of each ground term $t^\mcH := t$ is the term itself.
	An {\myem equational} interpretation interprets the equality symbol as equality on its domain.
\end{definition}

\begin{definition}
	An {\myem equational term interpretation} $\mcM$ is an equational interpretation 
	where the elements of the domain $A$ 
		are equivalence classes of ground terms
		and the interpretation of each ground term $t^\mcM := [t]_\sim$ is its equivalence class.
		An equation $s\mEQ t$ of ground terms holds in $\mcM$ if $[s]_\sim=[t]_\sim$;
	A ground predicate $\mP(t_1,\ldots,t_n)$ holds in $\mcM$ if $\mP([t_1]_\sim,\ldots,[t_n]_\sim)] \in A^n$.
	A ground literal holds in $\mcM$ if and only if its complementary literal does not hold in $\mcM$.
	
\end{definition}

\begin{example}
	
\end{example}

\section{Term Rewriting}




\begin{definition}
	A {\myem rewrite rule} is an equation of terms where the left-hand side is not a variable
	and the variables occuring in the right-hand side occur also in the left-hand side.
	%	\[
	%		\ell\rwEQ r \text{ is rewrite rule }\quad :\Longleftrightarrow\quad\ell\not\in\mcV\text{ and }\var(r)\subseteq\var(l)
	%	\] 
	A rewrite rule $\ell'\rwEQ r'$ is a {\myem variant} of $\ell\rwEQ r$ if there is a renaming $\varrho$
	such that 
	$(l\rightarrow r)\varrho = l'\rightarrow r'$.
	A {\myem term rewrite system} is a set of rewrite rules without variants.
\end{definition}

\begin{definition}\label{def:position}
	A {\myem position} is a finite sequence of positive integers.
	The root position is the empty sequence $\epsilon$.
	The position $pq$ is obtained by concatenation of positions $p$ and $q$.
	%
	The set of all positions of an expression $\mct$ is defined as 
	\DEFINE{ 
		\pos(\mct) }
	{
		\{ \epsilon \} 		
		& \text{if }\mct = x \in \mcV \\
%		
		\{ \epsilon \} \cup \bigcup_{i=1}^{n} \{ iq\mid q\in\pos(t_i) \}	
		& \text{if }\mct=\mcf(t_1,\ldots,t_n)\text{ where } \mcf\in\mcF^{(n)}.
	}
The~{\myem subexpression} $\mct|_p$ of an expression $\mct$ {\myem at position} $p\in\pos(\mct)$ is defined as
%
\DEFINE{
	t|_p
}{
	t 		& \text{if }p=\epsilon \\
	t_i|_q	& \text{if }t=\mcf(t_1,\ldots,t_n)\text{ and }p=iq
}
%
If $s$ is an expression then $t[s]_p$ represents the expression build by replacing the subexpression of $t$ at position $p$ with the term $s$.
\DEFINE{
	t[s]_p}
{
	s 		& \text{if }p=\epsilon \\
	\mf(t_1,\ldots,t_i[s]_q,\ldots,t_n)	& \text{if }t=\mf(t_1,\ldots,t_n)\text{ and }p=iq
}
%A {\myem hole} denotes a special constant symbol $\ctxhole\in\mcFf^{(0)}$. 
%A {\myem context} is a term $t$ with exactly one hole, i.e.~one occurrence $|t|_{\ctxhole}=1$.
\end{definition}

\begin{definition}
	We say $s\rightarrow_\mcR t$ is a 
	{\myem rewrite step} 
	 with respect to TRS $\mcR$ 
	when there is a position $p \in \pos(s)$, 
	a rewrite rule $l\rwEQ r\in\mcR$, 
	and a substitution $\sigma$ such that
	$s|_p=l\sigma$ and $s[r\sigma]_p = t$.
		The subterm $l\sigma$ is called {\myem redex} and
	$s$ rewrites to $t$ by {\myem contracting} $l\sigma$ to {\myem contractum} $r\sigma$.
	%
	We say a term $s$ is {\myem irreducible} or in {\myem normal form} with respect to TRS $\mcR$ if there is no rewrite step $s\rightarrow_\mcR t$ for any term $t$. 
	The set of normal forms $\mNFR$ contains all irreducible terms of the TRS $\mcR$.
	\end{definition}

%\begin{definition}
%	A {\myem rewrite relation} is a binary relation on terms that is closed under contexts and substitutions.
%	A {\myem rewrite order} is a proper order (i.e. irreflexive and transitive relation) and a rewrite relation.
%	A {\myem reduction order} is a well-founded rewrite order.
%\end{definition}

\begin{definition}
	A term $s$ can be rewritten to term $t$ with notion $s\rightarrow^*_\mcR t$ 
	if there exists at least one {\myem rewrite sequence} $(a_1,\ldots ,a_n)$ such that
	$s=a_1$, $a_n=t$, and $a_i\rightarrow_\mcR a_{i+1}$ are rewrite steps for $1\leq i<n$.
	A TRS is {\myem terminating} if there is no infinite rewrite sequence of terms.
	%
	Two Terms $s$ and $t$ are {\myem joinable} with notion $s\downarrow t$ 
	if both can be rewritten to some term $c$, i.e.~$s \rightarrow^*c\ \, ^*\!\!\leftarrow t$.
%	
	Two Terms $s$ and $t$ are {\myem meetable} with notion $s\uparrow t$ 
	if both can be rewritten from some common ancestor term $a$, i.e.~$s \leftarrow^*a\ \, ^*\!\!\rightarrow t$.
%
	A TRS is {\myem confluent } if $s$ and $t$ are joinable whenever $s\ ^*\!\!\leftarrow a \rightarrow^* t$ holds for some term $a$.
	%
	Terms $s$ and $t$ are {\myem convertible} with notion $s\leftrightarrow^* t$ 
	if there exists a sequence $(a_1,\ldots ,a_n)$ such that
	$s=a_1$, $a_n=t$, and $a_i\leftrightarrow a_{i+1}$, i.e.~$a_i\rightarrow a_{i+1}$ or $a_i\leftarrow a_{i+1}$ are rewrite steps for $1\leq i<n$.
\end{definition}

\begin{definition}\label{def:closed-under}
	A {\myem rewrite relation} is a binary relation $\relation$ on terms thats is {\myem closed under contexts},
	i.e.~$u[s]_p\relation u[t]_p$ %holds 
	for all positions $p\in\pos(u)$ and
	for all terms $s,t,u$ whenever $s\relation t$
	and {\myem closed under substitutions}, 
	i.e.~$s\sigma\relation t\sigma$ %holds
	for all substitutions $\sigma$
	and all terms $s,t$ whenever $s\relation t$.
\end{definition}
\begin{lemma}
	The relations $\rightarrow^*_\mcR$, 
	$\rightarrow^+_\mcR$,
	$\downarrow_\mcR$, $\uparrow_\mcR$ are rewrite relations on every TRS $\mcR$.
\end{lemma}
%
\begin{definition}
	A proper (i.e.~irreflexive and transitive) order on terms is called {\myem rewrite order} if it is a rewrite relation.
	A {\myem reduction order} is a well-founded rewrite order,
	i.e.~there is no infinite sequence 
	$(a_i)_{i\in\mathbb{N}}$
	where $a_i\gtpre a_{i+1}$ for all $i$.
	% with $i\in\mathbb{N}$.
	A {\myem simplification order} is a rewrite order with the {\myem subterm property},
	i.e.~$u[t]_p \gtpre t$ for all terms $u$, $t$ and positions $p\neq\epsilon$.
\end{definition}
\begin{lemma}
	Every simplification order is well-founded, i.e.~it is a reduction order.
\end{lemma}
%% DEF %%
%
\begin{theorem}
	A TRS $\mcR$ is terminating if and only if there exists a reduction order $\gtpre$
	such that $l\gtpre r$ for every rewrite rule $l\rightarrow r\in\mcR$.
	We call $\mcR$ simply terminating if $\gtpre$ is a simplification order.
\end{theorem}

\section{Usefull theorems}

\begin{theorem}[Compactness]\label{the:compactness}
	If every finite subset of a set of formulas $S$ has a model then $S$ has a model. 
\end{theorem}

\begin{theorem}[Löwenheim Skolem]\label{the:loewenheim}
	If a set of formulas $S$ has a model then $S$ has a countable model.
\end{theorem}

\begin{definition}
A Herbrand universe 
\end{definition}

\begin{definition}
	
\end{definition}

\begin{theorem}[Herbrand]\label{the:herbrand}
	Let $S$ be a set of clauses without equality. Then the following statements are equivalent.
	\begin{itemize}
		\item $S$ is satisfiable.
		\item $S$ has a Herbrand model.
		\item Every finite subset of all ground instances of $S$ has a Herbrand model.
	\end{itemize} 
\end{theorem}

\begin{corollary}
	Let $S$ be a set of clauses without equality. 
	Then $S$ is unsatisfiable if and only if there exists 
	an unsatisfiable finite set of ground instances of $S$.
\end{corollary}

\section{Conventions}

\begin{align*}
\mc, \md &\ \in\mcFfn[{0}]\tag*{constant symbols} \\
\mf, \mg, \mh &\ \in\mcFfn[{n>0}]\tag*{function symbols} \\
\mpp, \mq, \mr &\ \in\mcFPn[{0}]\tag*{propositional symbols}\\
\mP, \mQ, \mR &\ \in\mcFPn[{n>0}]\tag*{predicate symbols}\\
s,t,u &\ \in\mcTf = \mcTFfV\tag*{terms}
\end{align*}


