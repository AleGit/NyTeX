% !TeX root = ../mythesis.tex
% !TeX encoding = UTF-8
% !TeX spellcheck = en_US

% In the following definitions \( \mcf \) denotes a function symbol, a predicate symbol or the equality sign.


\begin{definition}\label{def:substitution}
	A \coloremph{substitution} \( \sigma \) is a mapping from variables \( x\in\mcV \) to terms in \( \mcTFfV \)
	where \coloremph{domain }\( \domain(\sigma) = \{ x\in\mcV\mid\sigma(x) \neq x \} \)
	and \coloremph{image} \( \image(\sigma) =\{ \sigma(x) \mid x\in V, \sigma(x)\neq x \} \) are finite.
	We write substitutions as bindings, e.g.~\( \sigma=\{ x_1\mapsto s_1,\ldots,x_n\mapsto s_n \} \)
	where \( \domain(\sigma)=\{ x_1,\ldots,x_n \} \) and \( \sigma(x_i)=s_i \).
	A \coloremph{variable substitution} is a mapping from \( \mcV \) to \( \mcV\subseteq\mcTFfV \).
	A \coloremph{renaming} is a bijective variable substitution.
	A \coloremph{proper instantiator} is a substitution that is not a variable substitution
	(at least one variable is mapped to a non-variable term).
\end{definition}
\begin{definition}\label{def:term:literal:clause:substitution}
	We define the application of a substitution \( \sigma \)
	to term \( t \), a literal \( L \), and clauses \( C \) as follows
	\begin{align*}
	t\sigma &=
	\left \{
		\begin{array}{ll}
			s_i & \text{if }t=x_i\in\domain(\sigma), \sigma(x_i) = s_i
			\\
			y & \text{if }t=y\in\mcV \, \backslash\domain(\sigma)
			\\
			\mf(t_1\sigma,\ldots,t_n\sigma)	&\text{if }t=\mf(t_1,\ldots,t_n)
			\text{ where  }\mf\in\mcFfn
		\end{array}
	\right.
			 \\
	L\sigma{} &=
	\left \{
		\begin{array}{ll}
			\mP(t_1\sigma,\ldots,t_n\sigma)
			&\text{if }L=\mP(t_1,\ldots,t_n)
			\text{ where  }\mP\in\mcFPn
			\\
			t_1\sigma\mEQ t_2\sigma
			&\text{if } L = t_1\mEQ t_2
			\\
			\lnot(A\sigma)&\text{if } L=\lnot A\text{ where }A\text{ is an atom}
		\end{array}
	\right.
	\\
	C \sigma{} &=
\bigvee_{
	L\in{} C
}
	L \sigma{}
	\end{align*}
\end{definition}

\begin{definition}\label{def:grounding:substitution}
	We define the application of the special “grounding substitution” \( \subsbot \) to a term \( t \) with distinct constant symbol \( \consbot \not\in\mcF \)
	as follows
	\begin{gather*}
	\MDEFINE{t\sigma}{ll}{
			\consbot & \text{if }t=x\in\mcV
			\\
			\mf(t_1\consbot,\ldots,t_n\consbot)	&\text{if }t=\mf(t_1,\ldots,t_n)
			\text{ where  }
			 \mf\in\mcFfn
			 }
	\end{gather*}
	We define the application of \( \subsbot \) to literals and clauses as in
	Definition~\vref{def:term:literal:clause:substitution}.
\end{definition}

\begin{definition}
	We can easily extend the application of substitutions
	to composite first order formulae.
	The cases of quantified formulae need more consideration ---
	we must not substitute bound variables.
	% So we only partially define \( F\sigma \) for first order formulae \( F \) and substitution \( \sigma \) as follows (if \( G\sigma \) and \( H\sigma \) are defined in the respective cases).
	\begin{gather*}
	\MDEFINE{F\sigma}{ll}{
		\lnot(G\sigma) & \text{if }F = \lnot G
		\\
		(G\sigma) \connect (H\sigma) & \text{if }F = G * H,
		\connect\in\SetOfConnectives
		\\
		\quantify x (G\sigma')
		&
		\text{if }
		\left\{  % chktex 21
			\begin{array}{l}
				F = \quantify x\,G,
				\quantify\in\SetOfQuantifiers
				\text{ and}
				\\
				\sigma'(x) = x, \sigma'(y) = \sigma(y)
				\text{ for all } y \neq x.
		\end{array}
		\right.
	}
	\end{gather*}
\end{definition}

\begin{definition}
	A formula \( G' \) is an \emph{instance} of formula \( F = \quantify x (G) \)
	if there exists a term \( t \) such that \( G' = G\{ x\mapsto t \} \).
	If \( F \) is closed (i.e.\ \( \fvar(G) = \{ x \} \)) % chktex 21
	and \( t \) is a ground term (i.e.\ \( \var(t) = \emptyset \)) then % chktex 21
	\( G' \) is a ground instance of \( F \).
\end{definition}

\begin{definition}\label{def:strictly:subsumes}
	A clause \( \mcC \) \coloremph{strictly subsumes} a clause \( \mcD \) if there exists a substitution \( \theta \) such that \( \mcC\theta \subsetneq \mcD \),
	e.g.~when clause \( \mcD = \mcC\theta \lor \mcD' \) is a weakened instance of clause \( \mcC \).
\end{definition}

\begin{definition}\label{def:substcomp}
We define the \coloremph{composition} of two substitutions \( \sigma \) and \( \tau \) as follows
	\begin{align*}
		\sigma\tau&=\{ x_i\mapsto s_i\tau\mid x_i\in\domain(\sigma) \}
		\cup
		\{ y_i\mapsto t_i\mid y_i\in\domain(\tau) \backslash \domain(\sigma) \}.
	\end{align*}
\end{definition}

\begin{lemma}\label{lem:substitution}
	With the Definitions in~\ref{def:substitution} and~\ref{def:substcomp} the resulting expressions for
	\( (\mct\sigma)\tau \) and \( \mct( \sigma\tau) \) coincide for
	any term or literal \( \mct \), i.e.~\( (\mct\sigma)\tau = \mct( \sigma\tau) \).
\end{lemma}

\begin{proof}
	Assume \( \sigma \) and \( \tau \) are substitutions.
	We use induction on the structure of \( \mct \)
	to show that
	\( (\mct\sigma)\tau = \mct(\tau\sigma) \) in all possible cases.
	\begin{itemize}
		\item (base case) If \( \mct = x_i \in\domain(\sigma) \) then
		\( ((x_i)\sigma)\tau\defEQ s_i\tau\defEQ x_i(\sigma\tau) \).

		\item (base case) If \( \mct = y \not\in\domain(\sigma) \) then
		\( (y\sigma)\tau \defEQ y\tau \defEQ y(\sigma\tau) \).

		\item (step case) If \( \mct = \Pf(t_1,\ldots,t_n) \)
		then
		\(
			((\Pf(t_1,\ldots,t_n))\sigma)\tau
		\defEQ
		(\Pf(t_1\sigma,\ldots,t_n\sigma))\tau
		\defEQ
		\Pf((t_1\sigma)\tau,\ldots,(t_n\sigma)\tau)
		\defEQ[IH]
		\Pf(t_1(\sigma\tau),\ldots,t_n(\sigma\tau))
		\defEQ
		(\Pf(t_1,\ldots,t_n))(\sigma\tau)
		\).

		\item (step case) If \( \mct = \lnot A \) then
		\(
		((\lnot A)\sigma)\tau
		\defEQ
		(\lnot(A\sigma))\tau
		\defEQ
		\lnot((A\sigma)\tau)
		\defEQ[IH]
		\lnot(A(\sigma\tau))
		\defEQ
		(\lnot A)(\sigma\tau)
		\).
	\end{itemize}
\end{proof}
