% !TeX root = ../mthesis.tex
% !TeX encoding = UTF-8
% !TeX spellcheck = en_US

\begin{definition}\label{def:substitution}
	A {\myem substitution} is an assignment $\alpha$ from variables $x\in\mcV$ to terms in $\mcTfFV$
	with a finite {\myem domain} $\dom(\alpha) = \{ x\in\mcV\mid\alpha(x) \neq x \}$.
	We write substitutions as mappings $\alpha=\{ x_1\mapsto s_1,\ldots,x_n\mapsto s_n \}$
	where $\dom(\alpha)=\{ x_1,\ldots,x_n \}$ and $\alpha(x_i)=s_i$.
	A {\myem variable substitution} is an assignment from $\mcV$ to $\mcV$.
	A {\myem renaming} is a bijective variable substitution.
	A {\myem proper instantiator} is a substitution that is not a variable substitution.
	%We define the {\myem instance} $t\sigma$ of a term $t$ 
	%\begin{align*}
	%	t\sigma &= \left\{\begin{array}{ll}
	%		s_i 					& \text{if }t=x_i\in\dom(\sigma)\\
	%		t					& \text{if }t\in\mcV\,\backslash\dom(\sigma)\\
	%		\mf(t_1\sigma,\ldots,t_n\sigma)	&\text{if }t=\mf(t_1,\ldots,t_n)
	%	\end{array}\right.
	%\end{align*}
	%and the {\myem composition} of two substitutions $\sigma$ and $\tau$ as
	%\begin{align*}
	%	\sigma\tau&=\{ x_i\mapsto s_i\tau\mid x_i\in\dom(\sigma) \}
	%	\cup
	%	\{ y_i\mapsto t_i\mid y_i\in\dom(\tau) \backslash \dom(\sigma) \}.
	%\end{align*}
\end{definition}