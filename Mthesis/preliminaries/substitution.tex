% !TeX root = ../mthesis.tex
% !TeX encoding = UTF-8
% !TeX spellcheck = en_US

\begin{definition}\label{def:substitution}
	A {\myem substitution} $\alpha$ is a mapping from variables $x\in\mcV$ to terms in $\mcTfFV$
	where the {\myem domain }$\dom(\alpha) = \{ x\in\mcV\mid\alpha(x) \neq x \}$ is finite.
	We write substitutions as bindings $\alpha=\{ x_1\mapsto s_1,\ldots,x_n\mapsto s_n \}$
	where $\dom(\alpha)=\{ x_1,\ldots,x_n \}$ and $\alpha(x_i)=s_i$.
	A {\myem variable substitution} is a mapping from $\mcV$ to $\mcV$.
	A {\myem renaming} is a bijective variable substitution.
	A {\myem proper instantiator} is a substitution that is not a variable substitution.
	For substitution $\sigma$ we define the {\myem instances }$t\sigma$, $A\sigma$ of term $t$, 
	predicate $A$ as follows:
	\begin{align*}
		t\sigma &= \left\{\begin{array}{ll}
			s_i 					& \text{if }t=x_i\in\dom(\sigma)\\
			t					& \text{if }t\in\mcV\,\backslash\dom(\sigma)\\
%			c					& \text{if }tc \in \mcFf^{\!0}\\
			f(t_1\sigma,\ldots,t_n\sigma)	&\text{if }t=f(t_1,\ldots,t_n), f\in\mcFf^{(n)}
		\end{array}\right.\\[1em]
%		
		A\sigma &= P(t_1\sigma,\ldots,t_n\sigma)\qquad\text{if }A=P(t_1,\ldots,t_n), P\in\mcF_{\mP}^{(n)} \\
	\end{align*}
	and the {\myem composition} of two substitutions $\sigma$ and $\tau$ as
	\begin{align*}
		\sigma\tau&=\{ x_i\mapsto s_i\tau\mid x_i\in\dom(\sigma) \}
		\cup
		\{ y_i\mapsto t_i\mid y_i\in\dom(\tau) \backslash \dom(\sigma) \}.
	\end{align*}
	
\end{definition}