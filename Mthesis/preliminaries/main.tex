% !TeX root = ../mthesis.tex
% !TeX encoding = UTF-8
% !TeX spellcheck = en_US

\section{Preliminaries}

We assume familiarity with propositional and predicate logic \cite{Huth:2004:LCS:975331}, 
automated theorem proving \cite{Fitting:1996:FLA:230183}, 
term rewriting \cite{Baader:1998:TR:280474}, 
decision procedures \cite{Kroening:2008:DPA:1391237}, 
and (maximum) satisfiability testing \cite{Biere:2009:HSV:1550723}.
Even so, for clarity we introduce
definitions and state lemmas and theorems without proofs.
Notions and notations follow (with small deviations) the lecture notes \cite{AM2015tr} and \cite{GM2013ar}.
This section is mostly an extension of the same section in our seminar report \cite{axm:SR2}.

% !TeX root = ../mthesis.tex
% !TeX encoding = UTF-8
% !TeX spellcheck = en_US

\begin{definition}\label{def:signature}
A
first order
{\myem signature} with equality
%with equality
$\mcF = \mcFfPE$
is the disjoint union of
a set of {\myem function symbols} $\mcFf$,
a set of {\myem predicate symbols} $\mcFP$,
and one distinct equality symbol.
%
The {\myem arity} of a symbol determines the number of its arguments in a first order expression.
With $\mcFn = \{ \mcf\in\mcF \mid \arity(\mcf) = n \}$ we denote symbols with arity $n$.
\end{definition}

\begin{remark}
    We use $\mEQ$ as equality symbol in our signatures to emphasize
    that at this point it is just a highlighted symbol
    without “meaning”.
    On the other hand we use $=$ to express “identity” of objects
    like formualae or sets
    without actually defining how this identity can be determined.
\end{remark}


% !TeX root = ../mthesis.tex
% !TeX encoding = UTF-8
% !TeX spellcheck = en_US

\begin{definition}
	We build the set of (function) {\myem terms} $\mcTfFV$ 
	from function symbols $\mcFf \subseteq \mcF$ and 
	a countable set of {\myem variables} $\mcV$ disjoint from $\mcF$.
	Every variable $x\in\mcV$ is a term,
	every constant $c\in\mcFf^{(0)}$ is a term, 
	and every expression $f(t_1,\ldots,t_n)$
	for every function symbol $f\in\mcFf$ 
	with arity $n>0$ 
	and terms $t_1,\ldots,t_n$
	is a term.
%	
	We build the set of predicate terms or {\myem atoms} $\mcTPFV$
	from predicate symbols $\mcFP \subseteq \mcF$ and terms $t\in\mcTfFV$. 
	Every proposition $p\in\mcFP^{(0)}$ is an atom 
	and every expression $P(t_1,\ldots,t_n)$
	for every predicate symbol $P\in\mcFP$ with arity $n>0$ and terms $t_1,\ldots,t_n$ is an atom.
%	Every equation $s\ \mEQ\ t$ for terms $s,t$ is an atom too.
%
%	We speak of the set of general terms 
%	$\mcTFV = \mcTfFV\disjointunion\mcTPFV$ as the disjoint union of terms and atoms.
\end{definition}



In examples we usually use $\ma,\mb,\mc,\md$ for constants $c\in\mcFf^{(0)}$,
$\mf,\mg,\mh$ for function symbols $f\in\mcFf^{(>0)}$,
$\mpp,\mq,\mr$ for propositions $p\in\mcFP^{(0)}$, and 
$\mP,\mQ,\mR$ for predicate symbols $P\in\mcFP^{(>0)}$.
The special binary equality symbol $\mEQ$ is always available.
% and can be seen as special predicate ${\mEQ}\in\mcFP^{(2)}$.