% !TeX root = ../mthesis.tex
% !TeX encoding = UTF-8
% !TeX spellcheck = en_US

\chapter{Preliminaries}

We assume familiarity with propositional and predicate logic \cite{Huth:2004:LCS:975331}, 
automated theorem proving \cite{Fitting:1996:FLA:230183}, 
term rewriting \cite{Baader:1998:TR:280474}, 
decision procedures \cite{Kroening:2008:DPA:1391237}, 
and (maximum) satisfiability testing \cite{Biere:2009:HSV:1550723}.
Even so, for clarity we introduce
definitions and state lemmas and theorems without proofs.
Notions and notations follow (with small deviations) the lecture notes \cite{AM2015tr} and \cite{GM2013ar}.
This section is mostly an extension of the same section in our seminar report \cite{axm:SR2}.

\section{Syntax}

% !TeX root = ../mthesis.tex
% !TeX encoding = UTF-8
% !TeX spellcheck = en_US

\begin{definition}\label{def:signature}
A
first order
{\myem signature} with equality
%with equality
$\mcF = \mcFfPE$
is the disjoint union of
a set of {\myem function symbols} $\mcFf$,
a set of {\myem predicate symbols} $\mcFP$,
and one distinct equality symbol.
%
The {\myem arity} of a symbol determines the number of its arguments in a first order expression.
With $\mcFn = \{ \mcf\in\mcF \mid \arity(\mcf) = n \}$ we denote symbols with arity $n$.
\end{definition}

\begin{remark}
    We use $\mEQ$ as equality symbol in our signatures to emphasize
    that at this point it is just a highlighted symbol
    without “meaning”.
    On the other hand we use $=$ to express “identity” of objects
    like formualae or sets
    without actually defining how this identity can be determined.
\end{remark}


% !TeX root = ../mthesis.tex
% !TeX encoding = UTF-8
% !TeX spellcheck = en_US

\begin{definition}
	We build the set of (function) {\myem terms} $\mcTfFV$ 
	from function symbols $\mcFf \subseteq \mcF$ and 
	a countable set of {\myem variables} $\mcV$ disjoint from $\mcF$.
	Every variable $x\in\mcV$ is a term,
	every constant $c\in\mcFf^{(0)}$ is a term, 
	and every expression $f(t_1,\ldots,t_n)$
	for every function symbol $f\in\mcFf$ 
	with arity $n>0$ 
	and terms $t_1,\ldots,t_n$
	is a term.
%	
	We build the set of predicate terms or {\myem atoms} $\mcTPFV$
	from predicate symbols $\mcFP \subseteq \mcF$ and terms $t\in\mcTfFV$. 
	Every proposition $p\in\mcFP^{(0)}$ is an atom 
	and every expression $P(t_1,\ldots,t_n)$
	for every predicate symbol $P\in\mcFP$ with arity $n>0$ and terms $t_1,\ldots,t_n$ is an atom.
%	Every equation $s\ \mEQ\ t$ for terms $s,t$ is an atom too.
%
%	We speak of the set of general terms 
%	$\mcTFV = \mcTfFV\disjointunion\mcTPFV$ as the disjoint union of terms and atoms.
\end{definition}



% !TeX root = ../mthesis.tex
% !TeX encoding = UTF-8
% !TeX spellcheck = en_US

\begin{definition}\label{def:predicates}\label{def:equations}\label{def:atoms}
	We build the set of (first order) {\myem predicates} $\mcPT$
	from predicate symbols and terms.
	Every proposition $\mpp\in \mcFPn[0]$ is a predicate,
	and every expression $\mP(t_1,\ldots,t_n)$ is a predicate for $n>0$,
	predicate symbol $\mP\in\mcFPn$ and arbitrary terms $t_1,\ldots,t_n$.
%
	We build the set of (first order) {\myem equations }$\mcET$ from the equality symbol and terms.
	Every pair $s\mEQ t$ is an equation %\footnote{
%		We use prefix and infix notation interchangeable,
%		e.g.~${\mEQ}(s,t)$ represents the same equation as $s\mEQ t$.}
	for arbitrary terms $s$ and $t$.
%
	The set of atomic formulas (or {\myem atoms} for short) is the (distinct) union of predicates and equations.
\end{definition}

%In examples we usually use $\ma,\mb,\mc,\md$ for constants $c\in\mcFfO$,
%$\mf,\mg,\mh$ for function symbols $f\in\mcFfgo$,
%$\mpp,\mq,\mr$ for propositions $p\in\mcFPO$, and 
%$\mP,\mQ,\mR$ for predicate symbols $P\in\mcFPgo$.
%%The special binary equality symbol $\mEQ$ is always available.
%% and can be seen as special predicate ${\mEQ}\in\mcFP^{(2)}$.

% !TeX root = ../mthesis.tex
% !TeX encoding = UTF-8
% !TeX spellcheck = en_US

\begin{definition}[\index{FOF}\FOF]\label{def:syntax:FOF}
	Atoms from the previous definition are {\myem first order formulae}.  
	The universal quantification $(\forall x F)$ 
	and the existential quantification $(\exists x F)$ 
	of a first order formula are (quantified) first order formulae
	with {\myem bound} variable $x\in\mcV$.
	The negation $(\lnot F)$ of a first order formula
	is a (composite) first order formula.
	Further, the disjunction $(F \lor F')$, 
	the conjunction $(F \land F') $, 
	and the implication $(F \limp F')$ 
	of two first order formulae
	are (composite) first order formulae.
\end{definition}

\begin{definition}\label{def:term:vars}\label{def:fof:fvars}\label{def:fof:sentence}
	We define the set of variables of a first order term $t$ and the set of {\myem free} variables of a first order formula $F$ as follows:
\DEFINE{\var(t)} {
		\{ x \} & \text{if } t = x \in \mcV \\
		\bigcup_{i=1}^n \var(t_i) & \text{if }  t = \mf(t_1, \ldots t_n)
	}
\DEFINE{\fvar(F)}{
		\bigcup_{i=1}^n \var(t_i) &\text{if }F = \mP(t_1,\ldots,t_n)\text{ or }t_1\mEQ t_{n=2} 
		\\
		\fvar (G) \setminus \{ x\} &\text{if } F \in\{\, \forall x\,G, \exists x\,G\,\}
		\\
		\fvar(G) \cup \fvar(H)&\text{if }F \in\{\,G\land H,G\lor H,G\limp H\,\}
%		\\
%		\var(s) \cup \var(t) &\text{if }F = s\mEQ t
		\\
		\fvar(G)&\text{if }F = \lnot G
	}
\end{definition}

\begin{definition}\label{def:fof:closed}
	A closed first order formula or in other words
	a first order {\myem sentence} 
	is a first order formula without free variables,
	i.e.~all occurring variables are bound.
	\DEFINE{\bvar(F)}{
		\emptyset &\text{if }F = \mP(t_1,\ldots,t_n)\text{ or }t_1\mEQ t_{n=2} 
		\\
		\bvar (G) \cup \{ x\} &\text{if } F \in\{\, \forall x\,G, \exists x\,G\,\}
		\\
		\bvar(G) \cup \bvar(H)&\text{if }F \in\{\,G\land H,G\lor H,G\limp H\,\}
		%		\\
		%		\var(s) \cup \var(t) &\text{if }F = s\mEQ t
		\\
		\bvar(G)&\text{if }F = \lnot G
	}
\end{definition}

%\begin{remark} 
	We often will write formulae or sentences 
	without stating the first order signature.
The reader can easily deduce the underlying {\myem implicit} signature with arities 
and the set of variables by applying the definitions of the syntax for first order formulae.
We follow the convention to use $x,y,z$ for variables 
and $\ma,\mb,\mc$ for constant function symbols 
(which avoids ambiguity in the presence of free variables). 
For easier readability we will use uppercase predicate symbols and lowercase function symbols.
%\end{remark}

% !TeX root = ../mthesis.tex
% !TeX encoding = UTF-8
% !TeX spellcheck = en_US

\begin{definition}\label{def:literals}
A {\myem literal} $L$ is either an atom
or the negation of an atom.
%, where $l\not{}\hspace{-0.5mm}\circ$ abbreviates $\lnot(l\circ r)$
%for binary symobl $\circ$.
%A literal is ground if all its terms are ground.
%The complement $L^c$ of a literal $L$ is defined as %with some atom $A$: 
%\[ L^c = \left\{ \begin{array}{rl}
%A & \text{if } L = \lnot A\\
%\lnot A & \text{if } L = A
%\end{array}\right.
%\]
Two literals are complementary if one is an atom and the other is the negation of this atom.
%
A {\myem clause}\ \ $\mcC = L_1\lor\ldots\lor L_n$  is a possible empty multiset of literals 
and is equivalent to the universally quantified disjunction of its literals.
The {\myem empty clause} $\emptyclause$ expresses a contradiction. 
%A clause is ground if all its literals are ground.
A finite {\myem set of clauses} $S=\{ \mcC_1,\ldots,\mcC_n \}$ 
is equivalent to the conjunction of its clauses.
$S\equiv(\forall\vec{x}_1\mcC_1)\land\ldots\land(\forall\vec{x}_n\mcC_n)$ with 
$\vec{x}_i = \var(\mcC_i)$.
%$\vec{x}_i\cap\vec{x}_j=\emptyclause\lor i=j$.
%A set of clauses is ground if all its clauses are ground.
\end{definition}

% !TeX root = ../mthesis.tex
% !TeX encoding = UTF-8
% !TeX spellcheck = en_US

\begin{definition}\label{def:substitution}
	A {\myem substitution} $\sigma$ is a mapping from variables $x\in\mcV$ to terms in $\mcTFfV$
	where the {\myem domain }$\dom(\sigma) = \{ x\in\mcV\mid\sigma(x) \neq x \}$ is finite.
	We write substitutions as bindings $\sigma=\{ x_1\mapsto s_1,\ldots,x_n\mapsto s_n \}$
	where $\dom(\sigma)=\{ x_1,\ldots,x_n \}$ and $\sigma(x_i)=s_i$.
	A {\myem variable substitution} is a mapping from $\mcV$ to $\mcV$.
	A {\myem renaming} is a bijective variable substitution.
	A {\myem proper instantiator} is a substitution that is not a variable substitution.
	We define the instance $\mkt\sigma$ of a term, literal or clause $\mkt$,
	respectively the application of a substitution $\sigma$ to terms, literals, and clauses as follows:
	\begin{align*}
		\mkt\sigma &= \left\{\begin{array}{ll}
			s_i 					& \text{if }T=x_i\in\dom(\sigma)\\
			y					& \text{if }T=y\in\mcV\,\backslash\dom(\sigma)\\
%			c					& \text{if }tc \in \mcFf^{\!0}\\
			f(t_1\sigma,\ldots,t_n\sigma)	&\text{if }T=f(t_1,\ldots,t_n), f\in\mcFfn \\[0.5em]
			P(t_1\sigma,\ldots,t_n\sigma)	&\text{if }T=P(t_1,\ldots,t_n), P\in\mcFPn \\
			s\sigma\mEQ t\sigma			&\text{if }T=s\mEQ t \\[0.5em]
			\lnot(A\sigma)					&\text{if }T=\lnot A \\
			L_1\sigma\lor\ldots\lor L_n\sigma	&\text{if }T=L_1\lor\ldots\lor L_n \\
		\end{array}\right.	
	\end{align*}
%	
	We define the {\myem composition} of two substitutions $\sigma$ and $\tau$ as follows
	\begin{align*}
		\sigma\tau&=\{ x_i\mapsto s_i\tau\mid x_i\in\dom(\sigma) \}
		\cup
		\{ y_i\mapsto t_i\mid y_i\in\dom(\tau) \backslash \dom(\sigma) \}.
	\end{align*}
	
\end{definition}

\begin{lemma}\label{lem:substitution}
	With the definitions in \ref{def:substitution} the identity
%	$(\mkt\sigma)\tau = \mkt(\sigma\tau)$ holds for
	arbitrary terms, literals, clauses, and substitutions.
\end{lemma}

% !TeX root = ../mthesis.tex
% !TeX encoding = UTF-8
% !TeX spellcheck = en_US

\begin{definition}\label{def:unifier}
Two terms $s$ and $t$ are {\myem unifiable} if there exists a substitution $\sigma$ such that $s\sigma=t\sigma$.
They are {\myem variants} if their most general unifier is a renaming.
The {\myem most general unifier} $\sigma=\mgu(s,t)$ is a unifier such that
for every other unifier $\sigma'$ there exists a substitution $\tau$ such that
$\sigma' = \sigma \tau$. 
\end{definition}