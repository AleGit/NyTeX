% !TeX root = ../mthesis.tex
% !TeX encoding = UTF-8
% !TeX spellcheck = en_US

\section{Preliminaries}

We assume familiarity with propositional and predicate logic \cite{Huth:2004:LCS:975331}, 
automated theorem proving \cite{Fitting:1996:FLA:230183}, 
term rewriting \cite{Baader:1998:TR:280474}, 
decision procedures \cite{Kroening:2008:DPA:1391237}, 
and (maximum) satisfiability testing \cite{Biere:2009:HSV:1550723}.
Even so, for clarity we introduce
definitions and state lemmas and theorems without proofs.
Notions and notations follow (with small deviations) the lecture notes \cite{AM2015tr} and \cite{GM2013ar}.
This section is mostly an extension of the same section in our seminar report \cite{axm:SR2}.

% !TeX root = ../mthesis.tex
% !TeX encoding = UTF-8
% !TeX spellcheck = en_US

\begin{definition}\label{def:signature}
A 
%first order 
{\myem signature} 
%with equality
$\mcF = \mcFfPE$ 
is the disjoint union of 
{\myem function symbols} $\mcFf$, 
{\myem predicate symbols} $\mcFP$,
and the equality symbol.
%
The {\myem arity} of a symbol is the number of its arguments.
With $\mcFn$ we denote symbols with arity $n$.
\end{definition}


% !TeX root = ../mthesis.tex
% !TeX encoding = UTF-8
% !TeX spellcheck = en_US

\begin{definition}\label{def:terms}
	We build the set of (first order) {\myem terms }$\mcTf = \mcTFfV$ 
	from function symbols and a
	countable set of {\myem variables }$\mcV$ disjoint from $\mcF$\!.
	Every variable $x\in\mcV$ is a term,
	every {\myem constant} $\mc\in\mcFf^{(0)}$ is a term, 
	and every expression $\mf(t_1,\ldots,t_n)$ is a term
	for $n>0$, function symbol $\mf\in\mcFfn$,  
	and arbitrary terms $t_1,\ldots,t_n$.
\end{definition}



In examples we usually use $\ma,\mb,\mc,\md$ for constants $c\in\mcFf^{(0)}$,
$\mf,\mg,\mh$ for function symbols $f\in\mcFf^{(>0)}$,
$\mpp,\mq,\mr$ for propositions $p\in\mcFP^{(0)}$, and 
$\mP,\mQ,\mR$ for predicate symbols $P\in\mcFP^{(>0)}$.
The special binary equality symbol $\mEQ$ is always available.
% and can be seen as special predicate ${\mEQ}\in\mcFP^{(2)}$.