% !TeX root = ../mthesis.tex
% !TeX encoding = UTF-8
% !TeX spellcheck = en_US

\section{Preliminaries}

We assume familiarity with propositional and predicate logic \cite{Huth:2004:LCS:975331}, 
automated theorem proving \cite{Fitting:1996:FLA:230183}, 
term rewriting \cite{Baader:1998:TR:280474}, 
decision procedures \cite{Kroening:2008:DPA:1391237}, 
and (maximum) satisfiability testing \cite{Biere:2009:HSV:1550723}.
Even so, for clarity we introduce
definitions and state lemmas and theorems without proofs.
Notions and notations follow (with small deviations) the lecture notes \cite{AM2015tr} and \cite{GM2013ar}.
This section is mostly an extension of the same section in our seminar report \cite{axm:SR2}.

% !TeX root = ../mthesis.tex
% !TeX encoding = UTF-8
% !TeX spellcheck = en_US

\begin{definition}\label{def:signature}
A 
%first order 
{\myem signature} 
%with equality
$\mcF = \mcFfPE$ 
is the disjoint union of 
{\myem function symbols} $\mcFf$, 
{\myem predicate symbols} $\mcFP$,
and the equality symbol.
%
The {\myem arity} of a symbol is the number of its arguments.
With $\mcFn$ we denote symbols with arity $n$.
\end{definition}


% !TeX root = ../mthesis.tex
% !TeX encoding = UTF-8
% !TeX spellcheck = en_US

\begin{definition}\label{def:terms}
	We build the set of (first order) {\myem terms }$\mcTf = \mcTFfV$ 
	from function symbols and a
	countable set of {\myem variables }$\mcV$ disjoint from $\mcF$\!.
	Every variable $x\in\mcV$ is a term,
	every {\myem constant} $\mc\in\mcFf^{(0)}$ is a term, 
	and every expression $\mf(t_1,\ldots,t_n)$ is a term
	for $n>0$, function symbol $\mf\in\mcFfn$,  
	and arbitrary terms $t_1,\ldots,t_n$.
\end{definition}



In examples we usually use $\ma,\mb,\mc,\md$ for constants $c\in\mcFfO$,
$\mf,\mg,\mh$ for function symbols $f\in\mcFfgo$,
$\mpp,\mq,\mr$ for propositions $p\in\mcFPO$, and 
$\mP,\mQ,\mR$ for predicate symbols $P\in\mcFPgo$.
%The special binary equality symbol $\mEQ$ is always available.
% and can be seen as special predicate ${\mEQ}\in\mcFP^{(2)}$.

In Section \ref{sec:syntax} we have defined atomic formulae in Definition \ref{def:atoms}, 
but we can only build formulae in clausal normal form (\CNF) with Definition \vref{def:syntax:CNF}.
Now we will define arbitrary first order formulae (\FOF).

\begin{definition}[\FOF]\label{def:syntax:FOF}
	Predicates and equations are (atomic) first order formulae. 
	The negation $(\lnot F)$, 
	the universal quantification $(\forall x F)$, 
	and the existential quantification $(\exists x F)$ 
	of a given formula $F$ are (composite) first order formulae.
	Further, the disjunction $(F \lor F')$, 
	the conjunction $(F \land F') $, 
	and the implication $(F \limp F') $ 
	of two given formulae $F$ and $F'$ 
	are (composite) first order formulae.
\end{definition}

We've already defined when an atom holds for an assignment $\alpha_\mcI$ 
in an interpretation $\mcI$ within Definition \vref{def:model}.
Now we extend these definitions to arbitrary formulae.

\begin{definition}[Semantics of \FOF]\label{def:semantics:FOF}
	
	A universally quantified formula $\forall x F$ holds in $\mcI$ if its subformula $F$ holds for all assignments for $x$.
	An existential quantified formula $\exists xF$ holds if its subformula $F$ holds for at least one assignment for $x$.
	A negation $\lnot F$ holds if its subformula $F$ does not hold, 
	a disjunction $F\lor F'$ holds if one or both of its subformulae $F$ or $F'$ hold,
	a conjunction $F\land F'$ holds, if both of its subformualae $F$ and $F'$ hold, 
	an implication $F\limp F'$ holds if its first subformula $F$ does not hold or its second subformula $F'$ holds (or both).
\end{definition}

\begin{remark}Usually we us precedences on connectives to omit parentheses 
	and some heuristics to structure the formulae for readability 
	without introducing semantic ambiguity.
%
	Beside the obvious semantically indistinguishable formulae with double negations, conjunctions, and disjunctions 
	we have introduced new ones.
	\begin{enumerate}
		\item $\forall x F$, $\exists x F$, and $F$ are indistinguishable if $x\not\in\var(F)$. 
		We usually omit quantifiers with variables that do not occur in subformulae.
		\item In general $\exists x F$ is different from $F$ if $x\in\var(F)$, e.g. $\exists x(x\mNE\ma)$ is satisfiable and $x\mNE\ma$ isn't.
		\item $\forall x F$ and $F$ are equivalent even if $x\in\var(F)$, 
		because in both cases we demand that $F$ holds in all assignments in our model.
		Usually we keep these universal quantifiers in \FOF.
		
		A first order formulae without quantifiers is in {\myem clausal form}, 
		but not necessarily in \CNF, e.g.~a weakened version of symmetry $(x\mEQ \ma)\limp (\ma\mEQ x)$ 
		is equisatisfiable to $\forall x ((x\mEQ \ma)\limp (\ma\mEQ x))$ 
		or $\exists a (\forall x ((x\mEQ a)\limp (a\mEQ x))$. 
	\end{enumerate}

\end{remark}

% !TeX root = ../mthesis.tex
% !TeX encoding = UTF-8
% !TeX spellcheck = en_US

\begin{definition}\label{def:literals}
A {\myem literal} $L$ is either an atom $A$ or a negated atom $\lnot A$.
The {\myem complement } $L^\mcc$ of an atom (positive literal) is the negation of the atom.
The complement of a negated atom (negative literal) is the atom itself. 
%\DEFINE{
%	L^\mcc 
%}{
%	\lnot L & \text{if } L \text{ is an atom} \\
%	A 		& \text{if } L = \lnot A, \text{ the negation of an atom } A
%}
%
A {\myem clause}\ \ $\mcC = L_1\lor\ldots\lor L_n$  is a possible empty multiset of literals 
and is equivalent to a universally quantified disjunction of its literals.
%The {\myem empty clause} $\emptyclause$ expresses a contradiction. 
%
A finite {\myem set of clauses} $S=\{ \mcC_1,\ldots,\mcC_n \}$ is equivalent to a conjunction of all its clauses.
%
%Two literals are complementary if one is an atom, i.e. a positive literal and the other is the negation of this atom.
\end{definition}

\begin{definition}
	We call a well formed first order expression without variables {\myem ground}.
\end{definition}



% !TeX root = ../mthesis.tex
% !TeX encoding = UTF-8
% !TeX spellcheck = en_US

% In the following definitions $\mcf$ denotes a function symbol, a predicate symbol or the equality sign.


\begin{definition}\label{def:substitution}
	A \coloremph{substitution} $\sigma$ is a mapping from variables $x\in\mcV$ to terms in $\mcTFfV$
	where \coloremph{domain }$\domain(\sigma) = \{ x\in\mcV\mid\sigma(x) \neq x \}$
	and image $\image(\sigma) =\{ \sigma(x) \mid x\in V, \sigma(x)\neq x \}$ are finite.
	We write substitutions as bindings, e.g.~$\sigma=\{ x_1\mapsto s_1,\ldots,x_n\mapsto s_n \}$
	where $\domain(\sigma)=\{ x_1,\ldots,x_n \}$ and $\sigma(x_i)=s_i$.
	A \coloremph{variable substitution} is a mapping from $\mcV$ to $\mcV\subseteq\mcTFfV$.
	A \coloremph{renaming} is a bijective variable substitution.
	A \coloremph{proper instantiator} is a substitution that is not a variable substitution
	(at least one variable is mapped to a non-variable term).
\end{definition}
\begin{definition}
	We define the instance $\mct\sigma$
	respectively the application of a substitution $\sigma$ to a literal or term $\foxt$ as follows
	\DEFINE{
		\foxt\sigma
	}{
			s_i 				& \text{if }\foxt=x_i\in\domain(\sigma), \sigma(x_i) = s_i
			\\
			y					& \text{if }\foxt=y\in\mcV\,\backslash\domain(\sigma)
			\\
			\foxf(\foxt_1\sigma,\ldots,\foxt_n\sigma)	&\text{if }\foxt=\foxf(\foxt_1,\ldots,\foxt_n)
			\text{ where  }
			 \foxf\in\mcFn \\
			 \lnot (A\sigma) &\text{if }\foxt = \lnot A, \text{ where } A\text{ is an atom}
			 }

\noindent Further we define the instance of a clause as the multiset of the instances of its literals.
\end{definition}

\begin{definition}
	We can easily extend our definition to composite first order formulae,
	but the cases of quantified formulae need more consideration.
	So we only partially define $F\sigma$ for first order formulae $F$ and substitution $\sigma$ as follows (if $G\sigma$ and $H\sigma$ are defined in the respective cases).
	\DEFINE{
		F\sigma
	}
{
		\lnot(G\sigma) & \text{if }F = \lnot G
		\\
		(G\sigma) * (H\sigma) & \text{if }F = G * H, *\in \{ \land, \lor, \limp \}\\
		\quantify x(G\sigma) &\text{if }F = \quantify x G,
		\quantify\in\{ \forall,\exists \},
		x\not\in\domain(\sigma)
		% \\
		% \texttt{undefind} & \texttt{otherwise}
		% x\not\in\bvar(G)
	}
\end{definition}

\begin{definition}
	A clause $\mcC$ \coloremph{strictly subsumes} a clause $\mcD$ if their exists a substitution $\theta$ such that $\mcC\theta \subsetneq \mcD$,
	e.g.~when clause $\mcD = \mcC\theta \lor \mcD'$ is a weakened instance of clause $\mcC$.
\end{definition}
%\begin{example}
%	Trivially any clause $\mcC$ subsumes all its instances $\mcC\sigma$ or weakenings $\mcC\lor\mcD$.
%	Further $x\mEQ y$ subsumes $x\mEQ x \lor \mP(y)$.
%\end{example}

\begin{definition}\label{def:substcomp}
We define the \coloremph{composition} of two substitutions $\sigma$ and $\tau$ as follows
	\begin{align*}
		\sigma\tau&=\{ x_i\mapsto s_i\tau\mid x_i\in\domain(\sigma) \}
		\cup
		\{ y_i\mapsto t_i\mid y_i\in\domain(\tau) \backslash \domain(\sigma) \}.
	\end{align*}
\end{definition}

\begin{lemma}\label{lem:substitution}
	With the definitions in \ref{def:substitution} and \ref{def:substcomp} the equation
	$(\mct\sigma)\tau = \mct(\sigma\tau)$ holds for
	term, atoms, and literals.
\end{lemma}

\begin{proof}
	Assume $\sigma$ and $\tau$ are substitutions.
	Then we use induction on the structure of the 	expression $\mct$
	that the equation $(\mct\sigma)\tau =  \mct(\tau\sigma)$ holds in all possible cases.
	\begin{itemize}
		\item (base case) Let $\mct = x_i \in\domain(\sigma)$ then
		$((x_i)\sigma)\tau\defEQ s_i\tau\defEQ x_i(\sigma\tau)$ holds.

		\item (base case) Let $\mct = y \not\in\domain(\sigma)$ then
		$(y\sigma)\tau \defEQ y\tau \defEQ y(\sigma\tau)$ holds.

		\item (step case) Let $\mct = \mcf(t_1,\ldots,t_n)$
		then
		$((\mcf(t_1,\ldots,t_n))\sigma)\tau
		\defEQ
		(\mcf(t_1\sigma,\ldots,t_n\sigma))\tau
		\defEQ
		\mcf((t_1\sigma)\tau,\ldots,(t_n\sigma)\tau)
		\defEQ[IH]
		\mcf(t_1(\sigma\tau),\ldots,t_n(\sigma\tau))
		\defEQ
		(\mcf(t_1,\ldots,t_n))(\sigma\tau)
		$ holds.

		\item (step case) Let $\mct = \lnot A$ then
		$((\lnot A)\sigma)\tau
		\defEQ
		(\lnot(A\sigma))\tau
		\defEQ
		\lnot((A\sigma)\tau)
		\defEQ[IH]
		\lnot(A(\sigma\tau))
		\defEQ
		(\lnot A)(\sigma\tau)
		$ holds.
	\end{itemize}
\end{proof}


% !TeX root = ../mthesis.tex
% !TeX encoding = UTF-8
% !TeX spellcheck = en_US

\begin{definition}\label{def:unifier}
Two terms $s$ and $t$ are {\myem unifiable} if there exists a substitution $\sigma$ such that $s\sigma=t\sigma$.
They are {\myem variants} if their most general unifier is a renaming.
The {\myem most general unifier} $\sigma=\mgu(s,t)$ is a unifier such that
for every other unifier $\sigma'$ there exists a substitution $\tau$ such that
$\sigma' = \sigma \tau$. 
\end{definition}