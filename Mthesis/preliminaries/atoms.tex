% !TeX root = ../mythesis.tex
% !TeX encoding = UTF-8
% !TeX spellcheck = en_US

\begin{definition}\label{def:predicates}\label{def:equations}\label{def:atoms}
	We build the set of (first order) \coloremph{predicates} \( \mcPT \)
	from predicate symbols and terms.
	Every proposition \( \mpp\in \mcFPn[0] \) is a predicate,
	and every expression \( \mP(t_1,\ldots,t_n) \) is a predicate for \( n>0 \),
	predicate symbol \( \mP\in\mcFPn \) and arbitrary terms \( t_1,\ldots,t_n \).
%
	We build the set of (first order) \coloremph{equations }\( \mcET \) from the equality symbol and terms.
	Every pair \( s\mEQ t \) is an equation %\footnote{
%		We use prefix and infix notation interchangeable,
%		e.g.~\( {\mEQ}(s,t) \) represents the same equation as \( s\mEQ t \).}
	for arbitrary terms \( s \) and \( t \).
%
	The set of atomic formulas (or \coloremph{atoms} for short) is the (distinct) union of predicates and equations.
\end{definition}