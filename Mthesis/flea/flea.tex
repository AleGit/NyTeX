% !TeX root = ../mthesis.tex
% !TeX encoding = UTF-8
% !TeX spellcheck = en_US

\chapter{FLEA}

\epigraph{
	Grau, teurer Freund, ist alle Theorie.
	Und grün des Lebens goldner Baum.\footnotemark\ 
}{\textit{
%	Faust 1, Studierzimmer. (Mephistopheles) \\ Johann Wolfgang von Goethe}}
	Faust 1 (Mephistopheles) \\ Johann Wolfgang von Goethe}}
\footnotetext{
	All theory is gray, my friend. 
	But forever green is the tree of life.
}

In this chapter we introduce \FLEA --
our \textbf{F}irst Order \textbf{L}ogic with \textbf{E}quality Theorem \textbf{A}ttester. 
It is 
-- as the reader may allready suspect --
a modest implementation of an instantiation based theorem prover for first order clauses with equality.

%We disucss its general architecture and elaborate on some implementation details.

\section{Installation}

\FLEA is open source (but still without license) and available on GitHub\footnote{
	\href{https://github.com/AleGit/FLEA}{github.com/AleGit/FLEA}
}. It depends on a working \Swift\footnote{
	Instructions and binaries available on \href{https://swift.org}{swift.org}
} toolchain at build time and the local installation of 
\Yices\footnote{
	Instruction and binary available on \href{http://yices.csl.sri.com}{yices.csl.sri.com}
}
and \Ziii\footnote{
	Instructions and source code available on \href{https://github.com/Z3Prover/z3}{github.com/Z3Prover/z3}
} at build and run time. After installation of these three prerequisites you may check the success with commands and results from Listing \ref{lst:check:Swyz}. 

Not a strict prerequisite but we recommend to download, to unpack \href{http://www.cs.miami.edu/~tptp/TPTP/Distribution/TPTP-v6.4.0.tgz}{TPTP-v6.4.0.tgz} (or newer)
from the \TPTP website\footnote{
	\href{http://www.cs.miami.edu/~tptp/}{www.cs.miami.edu/\textasciitilde tptp}
}, and to create a symbolic link to the unpacked library in your home directory.
\begin{lstlisting}[
	label=lst:check:Swyz,
	caption={Check toolchain and libraries},
	language=bash]
$\prompt$ yices - V
Yices 2.5.1          # or newer
$\prompt$ z3 --version
Z3 version 4.5.1     # or newer
$\prompt$ swift -version
Swift version 3.1    # or newer
$\prompt$ ls $\text{\textasciitilde}$/TPTP
Axioms    Documents    Generators    Problems    README    Scripts
\end{lstlisting}

After we have downloaded and installed external tool chain and libraries we clone \FLEA and install the parsing lib as in Listing \ref{lst:install:parsing}.

\begin{lstlisting}[
label=lst:install:parsing,
caption={Download \FLEA and install parsing lib},
language=bash]
$\prompt$ git clone https://github.com/AleGit/FLEA # download sources
$\prompt$ cd FLEA
$\prompt$ swift build                              # downloads dependencies, but fails
$\prompt$ pushd Packages/CTptpParsing-1.0.0        # or 1.0.1 or ...
$\prompt$ sudo make install                        # install parsing lib
$\prompt$ popd
\end{lstlisting}

\begin{lstlisting}[
label=lst:test:all,
caption={Build and and run all \FLEA tests.},
language=bash]
$\prompt$ Scripts/z3headers.sh                     # workaround for headers
$\prompt$ Scripts/ctests.sh                        # build and run all tests
\end{lstlisting}



\section{Usage}

\section{Data struture}

\begin{lstlisting}[language=flea]{Simplified defintion of general terms}}
protocol Node: Hashable {
	associatedtype Symbol : Hashable
	var symbol: Symbol { get set }
	var nodes: [Self]? { get set }
}
\end{lstlisting}

\section{Encodings}

%\begin{figure}
%	test
%\end{figure}

We can simply encode first order atoms purely propositional 
when we construct the name for the propositional atom 
from a predicate or equation recursively as concatination of symbols.
After installation of these prerequisites you may check if everything went well.



\begin{definition}
	We derive the propositional identifier of a general term as follows
\DEFINE{\xi(\mct)}{
	\bot &\text{if } \mct=x\in\mcV, \bot\not\in\mcF 
	\tag*{}\\
	\mc &\text{if } \mct=c\in\mcFn[0]\\
	\mcf\encsep\xi(t_1)\encsep\ldots\encsep\xi(t_n) &\text{if }\mct=\mcf(t_1,\ldots,t_n), \mcf\in\mcFn
}
where $\bot, \encsep\not\in\mcF$ are distinct symbols that do not occur in the signature of the set of clauses.
\end{definition}

\begin{example}
	We encode a simple predicate and an simple equation as follows.
	\begin{align*}
	\xi(\mpp(\mf(x,y),g(y)) &= \mpp\encsep\mf\encsep\bot\encsep\bot\encsep\mg\encsep\bot\\
	\xi(\mf(x,y)\mEQ \mg(y)) &= {\mEQ}\encsep\mf\encsep\bot\encsep\bot\encsep\mg\encsep\bot\\
	\end{align*}
\end{example}

\DEFINE{\Xi(\mct)}{
	\mc_0 : U &\text{if } \mct=x\in\mcV, \mc_0\not\in\mcFn[0]  
	\\
	\mc  : U &\text{if } \mct=c\in\mcFfn[0]
	\\
	\mf
	: (U^n\rightarrow U)\ \Xi(t_1)\ \ldots\ \Xi(t_n)
	&\text{if }\mct=\mf(t_1,\ldots,t_n), \mf\in\mcFfn 
	\\
	\mP  : \mathsf{Bool} &\text{if } \mct=p\in\mcFPn[0]
	\\
	\mP
	: (U^n\rightarrow \mathsf{Bool})\ \Xi(t_1)\ \ldots\ \Xi(t_n)
	&\text{if }\mct=\mP(t_1,\ldots,t_n), \mP\in\mcFPn 
}

\begin{lstlisting}[language=FLEA]
struct Yices {
 static func setUp() {
  yices_init()
 }
 static func tearDown() {
  yices_exit()
 }
}
\end{lstlisting}

\begin{lstlisting}[language=FLEA]
extension Yices {
 static var bool_tau = yices_bool_type()
 static var free_tau: type_t { return namedType("$\tau$") }
 
 /// Get or create (uninterpreted) type `name`.
 static func namedType(_ name: String) -> type_t {
  var tau = yices_get_type_by_name(name)
  if tau == NULL_TYPE {
   tau = yices_new_uninterpreted_type()
   yices_set_type_name(tau, name)
  }
  return tau
 }
 
 /// Get or create an uninterpreted global `symbol` of type `term_tau`.
 static func typedSymbol(symbol: String, term_tau: type_t) -> term_t {
  var c = yices_get_term_by_name(symbol)
  if c == NULL_TERM {
   c = yices_new_uninterpreted_term(term_tau)
   yices_set_term_name(c, symbol)
  } 
  return c
 }
\end{lstlisting}

\begin{lstlisting}[language=FLEA] 
 /// Get or create a global constant `symbol` of type `term_tau`
 static func constant(_ symbol: String, term_tau: type_t) -> term_t {
  return typedSymbol(symbol, term_tau: term_tau)
 }
 
 /// Create a homogenic domain tuple
 static func domain(_ count: Int, tau: type_t) -> [type_t] {
  return [type_t](repeating: tau, count: count)
 }
 
 /// Get or create a function symbol of type domain -> range
 static func function(_ symbol: String, domain: [type_t], range: type_t) -> term_t {
  let f_tau = yices_function_type(UInt32(domain.count), domain, range)
  return typedSymbol(symbol, term_tau: f_tau)
 }
}
\end{lstlisting}

\begin{lstlisting}[language=FLEA, mathescape=true]{SMT encoding}
yices_init()
let B = yices_bool_type()
let U = Yices.namedType(name: "$\tau$")


if foo
list= { $S_1,S_2,S_3$ }
\end{lstlisting}

\begin{lstlisting}[language=FLEA]{Propositional encoding}
func encodeSAT<N:Node>(term: N) -> term_t {
	switch n.symbol.type {
		case .negation:
			return yices_not( encode( term.nodes!.first!) )
		
		case .predicate, .equation:
			return typedSymbol( $\xi$(term), term_tau: boolType)
	}	
}
\end{lstlisting}

\begin{lstlisting}[language=FLEA]{EUF encoding}
func encodeEUF<N:Node>(term: N) -> term_t {
	switch term.symbol.type {
	case .negation:
		return yices_not( encodeEUF( term.nodes.first!) )

	case .predicate:
		return application(term.symbol, nodes:term.nodes!, term_tau: boolType)
		
	case .equation:
		return typedSymbol( $\xi$(term), term_tau: boolType)
	}	
}
\end{lstlisting}


\begin{lstlisting}[language=flea]{Yices types, symbols, and constants}
let boolType : type_t = yices_bool_type(void)

let freeType : type_t = yices_new_uninterpreted_type(void)
yices_set_type_name(freeType, "$\tau$") // pretty printing

func typedSymbol(_ symbol: String, term_tau: type_t) -> term_t {
	var t = yices_get_term_by_name(symbol)
	if t == NULL_TERM {
		t = yices_new_uninterpreted_term(term_tau)
		yices_set_term_name(t, symbol)
	}
	return t
}

func constant(_ symbol: String, term_tau: type_t) -> term_t {
	return typedSymbol(symbol, term_tau: term_tau)
}
\end{lstlisting}




\subsection{QF\_EUF}

\begin{lstlisting}[language=flea]
func domain(_ count: Int, tau: type_t) -> [type_t] {
	return [type_t](repeating: tau, count: count)
}
\end{lstlisting}

\begin{lstlisting}[language=flea]
func function(_ symbol: String, 
		domain: [type_t], range: type_t) -> term_t {
	let f_tau = yices_function_type(UInt32(domain.count), domain, range)
	return typedSymbol(symbol, term_tau: f_tau)
}
\end{lstlisting}

\begin{lstlisting}[language=flea]
func application(_ symbol: String, 
			args: [term_t], term_tau: type_t) -> term_t {
	let f = function(symbol, 
	domain:domain(args.count, tau:Yices.free_tau), range: term_tau)
	return yices_application(f, UInt32(args.count), args)
}
\end{lstlisting}


\begin{definition}[Flattening of clauses]
	
	\end{definition}

\begin{example}
	\begin{align*}
	\mP(t_1,\ldots, t_n) &\Rightarrow y_1\neq t_1 \lor \ldots y_n\neq t_n \lor \mP(y_1,\ldots,y_n) \\
	s \mNE t &\Rightarrow y_s\mNE s\lor y_t\mNE t \lor y_s \mEQ y_t \\
%	\mf(s_1,\ldost,s_m) \mEW \mg(t_1,\ldots,t_n) &\Rightarrow x_1\mNE s_1 \lor\ldots\lor x_m\mNE s_m \lor y_1\mNE t_1\lor\ldots\lor y_n\mNE t_n \lor \mf(x_1,\ldots,x_m)\
	\end{align*}
	\end{example}


\section{Implementation}


\section{Experiments}

\subsection{The TPTP library}

\begin{align*}
F 
&=
(F_1 \land \ldots \land F_n) \rightarrow G
\\
\lnot F 
&\equiv
\lnot\left( \lnot(F_1 \land \ldots \land F_n) \lor G \right)
\\
&\equiv 
(F_1 \land \ldots \land F_n) \land \lnot G
\end{align*}

