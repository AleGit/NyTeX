% !TeX root = ../mythesis.tex
% !TeX encoding = UTF-8
% !TeX spellcheck = en_US



\begin{table}[hbt]
    \begin{gather*}
    \begin{array}{ccccc}
    \infer[(\land i)]{F\land G}{F & G}
    &
    \infer[(\land e_1)]{G}{F \land G}
    &
    \infer[(\land e_2)]{F}{F \land G}
    &
    \infer[(\lnot\lnot i)]{\lnot\lnot F}{F}
    &
    \infer[(\lnot\lnot e)]{F}{\lnot\lnot F}
    \\[0.7em]
    \infer[(\bot e)]{F}{\bot}
    &
    \infer[(\lnot e)]{\bot}{F & \lnot F}
    &
    \infer[\text{LEM}]{F\lor\lnot F}{}
    &
    \infer[(\lor i_1)]{F\lor G}{F}
    &
    \infer[(\lor i_2)]{F\lor G}{G}
    \\[0.7em]
    \infer[(\lnot i)]{\lnot F}{
        \boxed{\begin{array}{c}F \\ \vdots \\ \bot \end{array}}}
    &
    \infer[\text{PBC}]{F}{
        \boxed{\begin{array}{c}\lnot F \\ \vdots \\ \bot \end{array}}}
    &
    \infer[({\limp} i)]{F\limp G}{
        \boxed{\begin{array}{c}F \\ \vdots \\ G \end{array}
    }}
    &
    \multicolumn{2}{l}{
        \infer[(\lor e)]{H}{
            F\lor G &
            \boxed{\begin{array}{c}F \\ \vdots \\ H \end{array}} &
            \boxed{\begin{array}{c}G \\ \vdots \\ H \end{array}}
        }
    }
    \\[0.7em]
    &
    \multicolumn{3}{r}{
    \infer[\substack{\text{modus}\\\text{ponens}}]{G}{F & F\limp G}
    \qquad
    \infer[\substack{\text{modus}\\\text{tollens}}]{\lnot F}{F\limp G & \lnot G}
    }
    &
    \end{array}
    \end{gather*}
    \caption{Natural Deduction Rules for Connectives}\label{tab:natural:deduction:connectives}
    \end{table}

\begin{definition}
	We say a term \( t \) is \coloremph{free} for variable \( x \)
	in a formula \( F \) if for every variable \( y \in \var(t) \)
	and every subformula \( G = \quantify y H \) of \( F \),
	the variable \( x \not\in\fvar(H) \).
\end{definition}


\begin{table}[hbt]
	\begin{gather*}
	\infer[({=}e)]{F'\{x\mapsto t\}}{s=t & F'\{x\mapsto s\}}
	\qquad
	\infer[({=}i)]{t=t}{}
	\end{gather*}
	\caption{Natural Deduction Rules for Equality}\label{tab:natural:deduction:equality}
\end{table}


\begin{definition}[Natural deduction~\cite{Huth:2004:LCS:975331}]\label{def:natural:deduction}
	Let \( F \), \( G \), and \( H \) be first order formulae and
	\(s\) and \(t\) be first order terms.
	We recall the rules of \coloremph{natural deduction} for connectives
	in Table~\ref{tab:natural:deduction:connectives},
	the rules for equality in Table~\ref{tab:natural:deduction:equality}
	where \( s \) and \( t \) are free for variable \( x \) in formula \( F \),
	and the rules for quantifiers in
	Table~\ref{tab:natural:deduction:quantifiers}
	where
	\( x_0 \) is a \emph{fresh} variable symbol,
	i.e.~\( x_0 \) did not occur in any formula so far.
\end{definition}

\begin{table}[hbt]
	\begin{gather*}
	\begin{array}{ccc}
	\infer[(\forall e)]{F'\{x\to t\}}{
		\forall x F'
	}
	&&
		\infer[(\exists i)]{\exists x F'}{
		F'\{x\mapsto t \}
	}
	\\[1em]
	\infer[(\forall i)]{\forall x F'}{
		\boxed{
			\begin{array}{cc}
				x_0
				\\
				&\vdots
				\\
				&F' \{x\mapsto x_0 x\}
			\end{array}} % end of boxed
	}
	&&
	\infer[(\exists e)]{H}{
		\exists x F' &
		\boxed{
			\begin{array}{cc}
				x_0
				&F' \{ x\mapsto x_0 \}
				\\
				&\vdots
				\\
				&H
			\end{array}} % end of boxed
	}
	\end{array}
	\end{gather*}
	\caption{Natural Deduction Rules for Quantifiers}\label{tab:natural:deduction:quantifiers}
\end{table}


