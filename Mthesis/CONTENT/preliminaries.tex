% !TeX root = ../mythesis.tex
% !TeX encoding = UTF-8
% !TeX spellcheck = en_US

\chapter{Preliminaries}

%



\epigraph{
	A good theory starts 
	
	with a good definition
}{
	unknown
}

In this thesis we assume the reader's familiarity with
propositional and first order logic~\cite{Huth:2004:LCS:975331},
%automated theorem proving \cite{Fitting:1996:FLA:230183},
term rewriting (TRW~\cite{Baader:1998:TR:280474}),
decision procedures (DP~\cite{Kroening:2008:DPA:1391237}),
and satisfiability checking modulo theories (SMT~\cite{Biere:2009:HSV:1550723}).
Nevertheless --- for clarity --- we state basic notions and  definitions
of first order logic with equality in Section~\ref{sec:syntax},
introduce basic concepts of first order semantics in Section~\ref{sec:semantics},
and describe basic term rewriting terminology in Section~\ref{sec:termrewriting}.
%This section is an extension of the same section in our seminar report \cite{axm:SR2}
These notions and notations largely follow the lecture notes to term rewriting and automated reasoning~\cite{AM2015tr, GM2013ar}.
% Figure \vref{fig:conventions} in the appendix lists typographical conventions.


\section{Syntax}\label{sec:syntax}

In this section we introduce the syntax of arbitrary first order formulae (\FOF),
prenex normal form (\PNF),
and clausal normal form (\CNF).

% !TeX root = ../mthesis.tex
% !TeX encoding = UTF-8
% !TeX spellcheck = en_US

\begin{definition}\label{def:signature}
A 
%first order 
{\myem signature} 
%with equality
$\mcF = \mcFfPE$ 
is the disjoint union of 
{\myem function symbols} $\mcFf$, 
{\myem predicate symbols} $\mcFP$,
and the equality symbol.
%
The {\myem arity} of a symbol is the number of its arguments.
With $\mcFn$ we denote symbols with arity $n$.
\end{definition}


% !TeX root = ../mthesis.tex
% !TeX encoding = UTF-8
% !TeX spellcheck = en_US

\begin{definition}\label{def:terms}
	We build the set of (first order) {\myem terms }$\mcTf = \mcTFfV$ 
	from function symbols and a
	countable set of {\myem variables }$\mcV$ disjoint from $\mcF$\!.
	Every variable $x\in\mcV$ is a term,
	every {\myem constant} $\mc\in\mcFf^{(0)}$ is a term, 
	and every expression $\mf(t_1,\ldots,t_n)$ is a term
	for $n>0$, function symbol $\mf\in\mcFfn$,  
	and arbitrary terms $t_1,\ldots,t_n$.
\end{definition}



% !TeX root = ../mthesis.tex
% !TeX encoding = UTF-8
% !TeX spellcheck = en_US

\begin{definition}\label{def:predicates}
	We build the set of {\myem predicate} $\mcPT$
	from predicate symbols and terms. 
	Every proposition $p\in \mcFP$ is a predicate, 
	and every expression $P(t_1,\ldots,t_n)$ is a predicate for $n>0$,
	predicate symbol $P\in\mcFPn$ and arbitrary terms $t_1,\ldots,t_n$.
%	
	We build the set of {\myem equations }$\mcET$ from the equality symbol and pairs of terms.
	Every expression $s\mEQ t$ is an equation\footnote{
		We use prefix and infix notation interchangable, e.g.~${\mEQ}(s,t)$ represents the same equation as $s\mEQ t$.} 
	for arbitrary terms $s$ and $t$.
%	
	The set of atomic formulas (or {\myem atoms }for short) is the union of predicates and equations.
\end{definition}

\subsection{Formulae and Normal forms}

In Section \ref{sec:syntax} we have defined atomic formulae in Definition \ref{def:atoms}, 
but we can only build formulae in clausal normal form (\CNF) with Definition \vref{def:syntax:CNF}.
Now we will define arbitrary first order formulae (\FOF).

\begin{definition}[\FOF]\label{def:syntax:FOF}
	Predicates and equations are (atomic) first order formulae. 
	The negation $(\lnot F)$, 
	the universal quantification $(\forall x F)$, 
	and the existential quantification $(\exists x F)$ 
	of a given formula $F$ are (composite) first order formulae.
	Further, the disjunction $(F \lor F')$, 
	the conjunction $(F \land F') $, 
	and the implication $(F \limp F') $ 
	of two given formulae $F$ and $F'$ 
	are (composite) first order formulae.
\end{definition}

We've already defined when an atom holds for an assignment $\alpha_\mcI$ 
in an interpretation $\mcI$ within Definition \vref{def:model}.
Now we extend these definitions to arbitrary formulae.

\begin{definition}[Semantics of \FOF]\label{def:semantics:FOF}
	
	A universally quantified formula $\forall x F$ holds in $\mcI$ if its subformula $F$ holds for all assignments for $x$.
	An existential quantified formula $\exists xF$ holds if its subformula $F$ holds for at least one assignment for $x$.
	A negation $\lnot F$ holds if its subformula $F$ does not hold, 
	a disjunction $F\lor F'$ holds if one or both of its subformulae $F$ or $F'$ hold,
	a conjunction $F\land F'$ holds, if both of its subformualae $F$ and $F'$ hold, 
	an implication $F\limp F'$ holds if its first subformula $F$ does not hold or its second subformula $F'$ holds (or both).
\end{definition}

\begin{remark}Usually we us precedences on connectives to omit parentheses 
	and some heuristics to structure the formulae for readability 
	without introducing semantic ambiguity.
%
	Beside the obvious semantically indistinguishable formulae with double negations, conjunctions, and disjunctions 
	we have introduced new ones.
	\begin{enumerate}
		\item $\forall x F$, $\exists x F$, and $F$ are indistinguishable if $x\not\in\var(F)$. 
		We usually omit quantifiers with variables that do not occur in subformulae.
		\item In general $\exists x F$ is different from $F$ if $x\in\var(F)$, e.g. $\exists x(x\mNE\ma)$ is satisfiable and $x\mNE\ma$ isn't.
		\item $\forall x F$ and $F$ are equivalent even if $x\in\var(F)$, 
		because in both cases we demand that $F$ holds in all assignments in our model.
		Usually we keep these universal quantifiers in \FOF.
		
		A first order formulae without quantifiers is in {\myem clausal form}, 
		but not necessarily in \CNF, e.g.~a weakened version of symmetry $(x\mEQ \ma)\limp (\ma\mEQ x)$ 
		is equisatisfiable to $\forall x ((x\mEQ \ma)\limp (\ma\mEQ x))$ 
		or $\exists a (\forall x ((x\mEQ a)\limp (a\mEQ x))$. 
	\end{enumerate}

\end{remark}

% !TeX root = ../mthesis.tex
% !TeX encoding = UTF-8
% !TeX spellcheck = en_US

\begin{definition}\label{def:literals}
A {\myem literal} $L$ is either an atom $A$ or a negated atom $\lnot A$.
The {\myem complement } $L^\mcc$ of an atom (positive literal) is the negation of the atom.
The complement of a negated atom (negative literal) is the atom itself. 
%\DEFINE{
%	L^\mcc 
%}{
%	\lnot L & \text{if } L \text{ is an atom} \\
%	A 		& \text{if } L = \lnot A, \text{ the negation of an atom } A
%}
%
A {\myem clause}\ \ $\mcC = L_1\lor\ldots\lor L_n$  is a possible empty multiset of literals 
and is equivalent to a universally quantified disjunction of its literals.
%The {\myem empty clause} $\emptyclause$ expresses a contradiction. 
%
A finite {\myem set of clauses} $S=\{ \mcC_1,\ldots,\mcC_n \}$ is equivalent to a conjunction of all its clauses.
%
%Two literals are complementary if one is an atom, i.e. a positive literal and the other is the negation of this atom.
\end{definition}

\begin{definition}
	We call a well formed first order expression without variables {\myem ground}.
\end{definition}



\subsection{Substitution}

% !TeX root = ../mthesis.tex
% !TeX encoding = UTF-8
% !TeX spellcheck = en_US

% In the following definitions $\mcf$ denotes a function symbol, a predicate symbol or the equality sign.


\begin{definition}\label{def:substitution}
	A \coloremph{substitution} $\sigma$ is a mapping from variables $x\in\mcV$ to terms in $\mcTFfV$
	where \coloremph{domain }$\domain(\sigma) = \{ x\in\mcV\mid\sigma(x) \neq x \}$
	and image $\image(\sigma) =\{ \sigma(x) \mid x\in V, \sigma(x)\neq x \}$ are finite.
	We write substitutions as bindings, e.g.~$\sigma=\{ x_1\mapsto s_1,\ldots,x_n\mapsto s_n \}$
	where $\domain(\sigma)=\{ x_1,\ldots,x_n \}$ and $\sigma(x_i)=s_i$.
	A \coloremph{variable substitution} is a mapping from $\mcV$ to $\mcV\subseteq\mcTFfV$.
	A \coloremph{renaming} is a bijective variable substitution.
	A \coloremph{proper instantiator} is a substitution that is not a variable substitution
	(at least one variable is mapped to a non-variable term).
\end{definition}
\begin{definition}
	We define the instance $\mct\sigma$
	respectively the application of a substitution $\sigma$ to a literal or term $\foxt$ as follows
	\DEFINE{
		\foxt\sigma
	}{
			s_i 				& \text{if }\foxt=x_i\in\domain(\sigma), \sigma(x_i) = s_i
			\\
			y					& \text{if }\foxt=y\in\mcV\,\backslash\domain(\sigma)
			\\
			\foxf(\foxt_1\sigma,\ldots,\foxt_n\sigma)	&\text{if }\foxt=\foxf(\foxt_1,\ldots,\foxt_n)
			\text{ where  }
			 \foxf\in\mcFn \\
			 \lnot (A\sigma) &\text{if }\foxt = \lnot A, \text{ where } A\text{ is an atom}
			 }

\noindent Further we define the instance of a clause as the multiset of the instances of its literals.
\end{definition}

\begin{definition}
	We can easily extend our definition to composite first order formulae,
	but the cases of quantified formulae need more consideration.
	So we only partially define $F\sigma$ for first order formulae $F$ and substitution $\sigma$ as follows (if $G\sigma$ and $H\sigma$ are defined in the respective cases).
	\DEFINE{
		F\sigma
	}
{
		\lnot(G\sigma) & \text{if }F = \lnot G
		\\
		(G\sigma) * (H\sigma) & \text{if }F = G * H, *\in \{ \land, \lor, \limp \}\\
		\quantify x(G\sigma) &\text{if }F = \quantify x G,
		\quantify\in\{ \forall,\exists \},
		x\not\in\domain(\sigma)
		% \\
		% \texttt{undefind} & \texttt{otherwise}
		% x\not\in\bvar(G)
	}
\end{definition}

\begin{definition}
	A clause $\mcC$ \coloremph{strictly subsumes} a clause $\mcD$ if their exists a substitution $\theta$ such that $\mcC\theta \subsetneq \mcD$,
	e.g.~when clause $\mcD = \mcC\theta \lor \mcD'$ is a weakened instance of clause $\mcC$.
\end{definition}
%\begin{example}
%	Trivially any clause $\mcC$ subsumes all its instances $\mcC\sigma$ or weakenings $\mcC\lor\mcD$.
%	Further $x\mEQ y$ subsumes $x\mEQ x \lor \mP(y)$.
%\end{example}

\begin{definition}\label{def:substcomp}
We define the \coloremph{composition} of two substitutions $\sigma$ and $\tau$ as follows
	\begin{align*}
		\sigma\tau&=\{ x_i\mapsto s_i\tau\mid x_i\in\domain(\sigma) \}
		\cup
		\{ y_i\mapsto t_i\mid y_i\in\domain(\tau) \backslash \domain(\sigma) \}.
	\end{align*}
\end{definition}

\begin{lemma}\label{lem:substitution}
	With the definitions in \ref{def:substitution} and \ref{def:substcomp} the equation
	$(\mct\sigma)\tau = \mct(\sigma\tau)$ holds for
	term, atoms, and literals.
\end{lemma}

\begin{proof}
	Assume $\sigma$ and $\tau$ are substitutions.
	Then we use induction on the structure of the 	expression $\mct$
	that the equation $(\mct\sigma)\tau =  \mct(\tau\sigma)$ holds in all possible cases.
	\begin{itemize}
		\item (base case) Let $\mct = x_i \in\domain(\sigma)$ then
		$((x_i)\sigma)\tau\defEQ s_i\tau\defEQ x_i(\sigma\tau)$ holds.

		\item (base case) Let $\mct = y \not\in\domain(\sigma)$ then
		$(y\sigma)\tau \defEQ y\tau \defEQ y(\sigma\tau)$ holds.

		\item (step case) Let $\mct = \mcf(t_1,\ldots,t_n)$
		then
		$((\mcf(t_1,\ldots,t_n))\sigma)\tau
		\defEQ
		(\mcf(t_1\sigma,\ldots,t_n\sigma))\tau
		\defEQ
		\mcf((t_1\sigma)\tau,\ldots,(t_n\sigma)\tau)
		\defEQ[IH]
		\mcf(t_1(\sigma\tau),\ldots,t_n(\sigma\tau))
		\defEQ
		(\mcf(t_1,\ldots,t_n))(\sigma\tau)
		$ holds.

		\item (step case) Let $\mct = \lnot A$ then
		$((\lnot A)\sigma)\tau
		\defEQ
		(\lnot(A\sigma))\tau
		\defEQ
		\lnot((A\sigma)\tau)
		\defEQ[IH]
		\lnot(A(\sigma\tau))
		\defEQ
		(\lnot A)(\sigma\tau)
		$ holds.
	\end{itemize}
\end{proof}


% !TeX root = ../mthesis.tex
% !TeX encoding = UTF-8
% !TeX spellcheck = en_US

\begin{definition}\label{def:unifier}
Two terms $s$ and $t$ are {\myem unifiable} if there exists a substitution $\sigma$ such that $s\sigma=t\sigma$.
They are {\myem variants} if their most general unifier is a renaming.
The {\myem most general unifier} $\sigma=\mgu(s,t)$ is a unifier such that
for every other unifier $\sigma'$ there exists a substitution $\tau$ such that
$\sigma' = \sigma \tau$. 
\end{definition}

% !TeX root = ../mthesis.tex
% !TeX encoding = UTF-8
% !TeX spellcheck = en_US

\begin{definition}\label{def:position}
	A {\myem position} is a finite sequence of positive integers.
	The root position is the empty sequence $\epsilon$.
	The position $pq$ is obtained by concatenation of positions $p$ and $q$.
	%
	A position $p$ is above a position $q$ if $p$ is a prefix of $q$, 
	i.e.~there exists a unique position $r$ such that $pr = q$, 
	we write $p\leq q$ and $q\backslash r = p$.
	We write $p<q$ if $p$ is a proper prefix of $q$, i.e.~$p\leq q$ but $p\neq q$.
	$(head, tail)(iq) = (i,q)$ for $i\in\mathbb{N}, q\in\mathbb{N}^*$.
	$length(\epsilon)=0, length(iq) = 1 + length(q)$
%	
	Two positions $p\parallel q$ are parallel if none is above the other,
	i.e.~for any common prefix $r$ both remaining tails
	$p\backslash r$ and $q\backslash r$ are different and not root positions.
	A position $p$ is left of position $q$ if $\mathtt{head}\ p\backslash r < \mathtt{head}\ p\backslash r$ 
	for maximal common prefix $r$. 
	
\end{definition}
\begin{definition}
	
	We define the set of {\myem term positions} of a term or atom,
	\DEFINE{ 
		\pos(\foxt) }
	{
		\{ \epsilon \} 		
		& \text{if }\foxt = x \in \mcV\\
		%		
		\{ \epsilon \} \cup \bigcup_{i=1}^{n} \{ iq\mid q\in\pos(t_i) \}	
		&	\text{if }\foxt=\mf(t_1,\ldots,t_n), \mf\in\mcFfn\\
		%		
		\bigcup_{i=1}^{n} \{ iq\mid q\in\pos(t_i) \}	
		&	\text{if }\foxt=\mP(t_1,\ldots,t_n), \mP\in\mcFPn\\
		%		
		\{ 1q \mid q\in\pos(t_1) \} \cup \{ 2q \mid q\in\pos(t_2) \}	
		&	\text{if }\foxt=t_1\mEQ t_2
	}
	the extraction of a subterm from a term or atom at a term position,
%	
	\DEFINE{
		\foxt|_p
	}{
		\foxt 		& \text{if }p=\epsilon \\
		\foxt_i|_q	& \text{if }t=\foxf(t_1,\ldots,t_n), p=iq, \foxf\in\mcFn
	}
	%
	and the insertion of a term $s$ into a term or atom at a term position.
	\DEFINE{
		\foxt[s]_p}
	{
		s 		& \text{if }p=\epsilon \\
		\foxf(\foxt_1,\ldots,\foxt_i[s]_q,\ldots,\foxt_n)	& \text{if }t=\foxf(\foxt_1,\ldots,\foxt_n), p=iq, 
		\foxf\in\mcFn, 0 < i \leq n
	}
where $p\neq\epsilon$ if $\foxt$ is an atom.
	%A {\myem hole} denotes a special constant symbol $\ctxhole\in\mcFf^{(0)}$. 
	%A {\myem context} is a term $t$ with exactly one hole, i.e.~one occurrence $|t|_{\ctxhole}=1$.
\end{definition}

% !TeX root = ../mthesis.tex
% !TeX encoding = UTF-8
% !TeX spellcheck = en_US

\subsection{Provability}

In general a proof may be a finite sequence of proof steps 
from none or some premises via intermediate statements 
to a final, the then proven statement. 
A formal proof system or logical calculus describes admissible basic proof steps 
in the underlying logic of the statements, in our case first order logic.
A formal proof comprises only proof steps confirmed by rules of the applied logical calculus.

% !TeX root = ../mthesis.tex
% !TeX encoding = UTF-8
% !TeX spellcheck = en_US

\begin{definition}[\cite{Huth:2004:LCS:975331}]\label{def:natural:deduction}
	We recall the rules of \coloremph{natural deduction} for connectives
	in Table \ref{tab:natural:deduction:connectives},
	for equality in Table \ref{tab:natural:deduction:equality},
	and for quantifiers in Table \ref{tab:natural:deduction:quantifiers}.
	Natural deduction provides a logical calculus,
	i.e.~a formal proof system for first order logic.
	The formulae $F$ and $G$ in these rules are sentences,
	the bound variable in  $\forall x F'$ occurs free in $F'$,
	and terms $s$ and $t$ are ground.

\begin{table}[hbt]
\begin{gather*}
\begin{array}{ccccc}
\infer[(\land i)]{F\land G}{F & G}
&
\infer[(\land e_1)]{G}{F \land G}
&
\infer[(\land e_2)]{F}{F \land G}
&
\infer[(\lnot\lnot i)]{\lnot\lnot F}{F}
&
\infer[(\lnot\lnot e)]{F}{\lnot\lnot F}
\\[0.7em]
\infer[(\bot e)]{F}{\bot}
&
\infer[(\lnot e)]{\bot}{F & \lnot F}
&
\infer[\text{LEM}]{F\lor\lnot F}{}
&
\infer[(\lor i_1)]{F\lor G}{F}
&
\infer[(\lor i_2)]{F\lor G}{G}
\\[0.7em]
\infer[(\lnot i)]{\lnot F}{
	\boxed{\begin{array}{c}F\\\vdots\\\bot\end{array}}}
&
\infer[\text{PBC}]{F}{
	\boxed{\begin{array}{c}\lnot F\\\vdots\\\bot\end{array}}}
&
\infer[({\limp} i)]{F\limp G}{
	\boxed{\begin{array}{c}F\\\vdots\\G\end{array}
}}
&
\multicolumn{2}{l}{
	\infer[(\lor e)]{H}{
		F\lor G &
		\boxed{\begin{array}{c}F\\\vdots\\H\end{array}} &
		\boxed{\begin{array}{c}G\\\vdots\\H\end{array}}
	}
}
\\[0.7em]
&
\multicolumn{3}{r}{
\infer[\text{modus}\atop\text{ponens}]{G}{F & F\limp G}
\qquad
\infer[\text{modus}\atop\text{tollens}]{\lnot F}{F\limp G & \lnot G}
}
&
\end{array}
\end{gather*}
\caption{Natural Deduction Rules for Connectives}
\label{tab:natural:deduction:connectives}
\end{table}

\begin{table}[hbt]
	\begin{gather*}
	\infer[({=}e)]{F'\{x\mapsto t\}}{s=t & F'\{x\mapsto s\}}
	\qquad
	\infer[({=}i)]{t=t}{}
	\end{gather*}
	\caption{Natural Deduction Rules for Equality}
	\label{tab:natural:deduction:equality}
\end{table}

\begin{table}[hbt]
	\begin{gather*}
	\begin{array}{ccc}
	\infer[(\forall e)]{F'\{x\to t\}}{
		\forall x F'
	}
	&&
		\infer[(\exists i)]{\exists x F'}{
		F'\{x\mapsto t \}
	}
	\\[1em]
	\infer[(\forall i)]{\forall x F'}{
		\boxed{\begin{array}{cc}t\\&\vdots\\&F'\{x\mapsto t\}\end{array}}
	}
	&&
	\infer[(\exists e)]{H}{
		\exists x F' &
		\boxed{\begin{array}{cc}t&F'\{x\mapsto t \}\\&\vdots\\&H\end{array}}
	}
	\end{array}
	\end{gather*}
	\caption{Natural Deduction Rules for Quantifiers}
	\label{tab:natural:deduction:quantifiers}
\end{table}
\end{definition}

\begin{definition}A sentence in first order logic is provable 
	if their exists a proof in a formal proof system for first order logic, e.g. natural deduction.
	We write
	$F_1, \ldots, F_n \proves G$
	when we can prove G from premisis $F_1,\ldots,F_n$.
\end{definition}

A natural deduction proof starts with a (possible empty) set of sentences -- the premises -- 
and infer other sentences -- the conclusions -- by applying the syntactic proof inference rules.
A box must be opened for each assumption, e.g.~a term or a sentence. 
Closing the box discards the assumption and all its conclusions within the box,
but introduces a conclusion outside the box. 
Then $F_1,\ldots,F_n \proves H$ claims that $H$ 
is in the transitive closure of inferable formulae from $\{ F_1,\ldots,F_n\}$ outside of any box.

\begin{example}We show $\forall x (\mP(x)\land\lnot\mQ(x)) \proves \forall x(\lnot\mQ(x)\land\mP(x))$ with natural deduction.
	We note our premise (1), we open a box and assume an arbitrary constant (2), 
	we create a ground instance of our premise with quantifier elimination and the constant (3),
	we extract the literals with both variants of conjunction elimination (4, 5),
	we introduce a conjunction of the ground literals (6),
	and close the box to introduce the universal quantified conjunction (7).
	\begin{gather*}
	\begin{BMAT}{rcrclccl}{ccccccccc}
1 && 		&& \forall x(\mP(x)\land\lnot\mQ(x)) 	& && \texttt{premise}\\
2 && \mc 	&& 										& && \\
3 && 	 	&& \mP(\mc)\land\lnot\mQ(\mc)			& && 1: {\forall}e\\
4 && 		&& \lnot\mQ(\mc) 						& && 3: \land e_1\\
5 && 		&& \mP(\mc) 							& && 3: \land e_2\\
6 &&		&& \lnot\mQ(\mc)\land\mP(\mc) 			& && 4+5: \land i\\
7 && 	 	&&	\forall x(\lnot\mQ(x)\land\mP(x))	& && 2-6: \forall i
\addpath{(2,1,1)rrrruuuuullllddddd}	
\end{BMAT}
\end{gather*}
\end{example}

%	\infer[\forall i]
%	{
%		\forall x( \lnot\mQ(x) \land \mP(y))
%	}
%	{
%		\infer[\land i]
%		{ \lnot\mQ(\mc) \land \mP(\mc) }
%		{
%			\infer[\land e_1]
%			{\mQ(\mc)} 
%			{ 
%				\infer[\forall e]
%				{ \mP(\mc)\land\lnot\mQ(\mc)}
%				{ \forall x (\mP(x)\land\lnot\mQ(x)) }
%			}
%			&
%			\infer[\land e_2]
%			{\mP(\mc)}
%			{ \colG
%				\infer[\forall e]
%				{ \mP(\mc)\land\lnot\mQ(\mc)}
%				{ \forall x (\mP(x)\land\lnot\mQ(x)) }
%			}
%		}
%	}
%\\





\section{Semantics}\label{sec:semantics}

In this section we recall some basic aspects and definitions of semantics in first order logic.
We state satisfiability and validity of arbitrary first order formulae or sets of clauses.
%See the appendix for more details and mathematics.

\subsection{Models}

\begin{definition}\label{def:interpretation}
	An \coloremph{interpretation} \( \mcI \) over a signature \( \mcF \) consists of a
	non-empty set \( A \) (i.e.~the \coloremph{universe} or \coloremph{domain}),
	definitions of mappings
	\( \mf_\mcI: A^n\rightarrow A \)
	for every function symbol \( \mf\in\mcFf \),
	and definitions of (possibly empty) n-ary relations
	 \( {\mP_\mcI}\subseteq A^n \) for every predicate symbol \( \mP\in\mcFP \)
	 and the definition of a binary relation \( {\mEQ_\mcI}\subseteq A^2 \) for the equality symbol.
	 A (variable) \coloremph{assignment} is a mapping from variables to elements of the domain.
	 We define the \coloremph{evaluation} \( \alpha_\mcI \) of a term \( t \)
	 for assignment \( \alpha \) and interpretation \( \mcI \):
	 \begin{gather*}
	 \MDEFINE{\alpha_\mcI(t)}{ll}{
	 	\alpha_\mcI(x)
	 	&\text{if }t=x\in\mcV \\
	 	\mc_\mcI
	 	&\text{if }t=\mc\in\mcFfn[0]
	 	\\
	 	\mf_\mcI (\alpha_\mcI(t_1),\ldots,\alpha_\mcI(t_n))
	 	&\text{if }t=\mf(t_1,\ldots,t_n), \mf\in\mcFfn[n>0], t_i\in\mcTf
 }
	 \end{gather*}
\begin{remark}
	The evaluation of ground terms does not depend on variable assignments.
\end{remark}
\end{definition}

\begin{definition}\label{def:semantics:atoms}
	A predicate \( \mP(t_1,\ldots,t_n) \)
	\coloremph{holds} for an assignment \( \alpha_\mcI \)
	if and only if the evaluation of its n-tuple \(
	(\alpha_\mcI({t_1}),\ldots,\alpha_\mcI({t_n})) \)
	is an element of the relation \( \mP_\mcI\subseteq A^n \).
	Similar an equation \( s\mEQ t \)
	holds if \( \alpha_\mcI(s) \mEQ_\mcI \alpha_\mcI(t) \).
\end{definition}

\begin{definition}
	[Semantics of \FOF{}]\label{def:semantics:FOF}
	A universally quantified sentence \( \forall x F \)
	holds in an interpretation if its subformula \( F \) holds for all assignments for \( x \).
	An existential quantified sentence \( \exists xF \) holds if its subformula \( F \) holds for at least one assignment for \( x \).
	For a given interpretation and predefined assignments for all occurring free variables
	a negation \( \lnot F \) holds if its subformula \( F \) does not hold,
	a disjunction \( F\lor G \) holds if one or both of its subformulae \( F \) or \( G \) hold,
	a conjunction \( F\land G \) holds, if both of its subformulae \( F \) and \( G \) hold,
	an implication \( F\limp G \) holds if its first subformula \( F \) does not hold or its second subformula \( F' \) holds (or both).

	\begin{remark}Usually we use precedences on connectives to omit parentheses
		and some heuristics to structure the formulae for readability
		without introducing semantic ambiguity.
	\end{remark}
\end{definition}

\begin{definition}[Semantics of \CNF{}]\label{def:semantics:CNF}
	An atom holds in an interpretation if and only if it
	holds with all possible assignments.
	A literal holds if and only if its complement does not hold.
	A clause holds if at least one of its literals holds,
	hence the empty clause \( \square \) does not hold in any interpretation.
	A set of clauses holds if and only if every clause in the set holds.
\end{definition}

\begin{definition}
	A \coloremph{model} \( \mcM \) for a set of clauses \( S \) (for a sentence \( F \))
	is an interpretation that
	\coloremph{satisfies} the set of clauses (the sentence),
	i.e.~the set of clauses (the sentence) holds in that interpretation \( \mcM \).
	We write \( \mcM\models S \) or \( \mcM\models F \).
	% i.e.~the set of clauses (the sentence) holds in this interpretation.

	A set of clauses (a sentence) is \coloremph{satisfiable} if there exists at least one model for it.
	A set of clauses (a sentence) is \coloremph{valid} if and only if every interpretation is a model.
\end{definition}

\begin{definition}\label{def:hk}
	The \coloremph{Herbrand universe} for a first order signature \( \mcF \)
	is the smallest set of terms that contains all \( H_{i\geq 0} \) defined inductively as
	\begin{align*}
	H_0 &=
	\left \{
	\begin{array}{ll}
	\{ \, \mc \mid \mc\in\mcFfn[0] \, \} 
	&\text{if } \mcFfn[0]\not=\emptyset
	\\
	\{ \, \mc_0 \, \}
	&\text{if } \mcFfn[0]=\emptyset, \mc_0\not\in\mcF
	\end{array}
	\right.
	&
	H'_0 &= H_0
	\\
	H_{k+1} &= \bigcup_{n> 0}
	\{ \, \mf(t_1, \ldots, t_n) \mid{} \mf\in\mcFfn[n],
	t_1, \ldots, t_n \in{} H_k' \, \}
	&
	H'_{k+1} &= H_k \cup{} H'_{k+1}
	\end{align*}

\end{definition}

\begin{definition}
	An \coloremph{Herbrand interpretation} \( \mcH \) is an interpretation where the domain
	is an Herbrand universe
	and the interpretation of each ground term \( t_\mcH := t \) is the term itself.
\end{definition}





\subsection{Equivalence and Equisatisfiability}

\begin{definition}\label{def:entailment}\label{def:equivalence}
	% [Consequence and entailment]
	A (first order) sentence \( G \) is a \coloremph{semantic consequence}
	of a set of sentences
	\( \Gamma = \{ F_1,\ldots,F_n \} \) if \( G \) holds in all models for \( F_1,\ldots,F_n \).
	We write \( \Gamma\entails G \) and also say that \( \Gamma \) entails G.
	Two sentences \( F \equival G \) are \coloremph{equivalent}
	if an only if the first sentence entails the second and vice versa.
	An \coloremph{equivalence transformation} morphs an arbitrary sentence \(F\)
	to another sentence \(F'\) such that \( F\equiv F'\).
	Equivalence transformations preserve
	validity \coloremph{and} satisfiability
	of sentences.
\end{definition}

\begin{example}
	We can easily see that not satisfiable \( F\land\lnot F \) entails every formula,
	that valid \( F\lor\lnot F \) is entailed by every sequence, further that
	\begin{align*}
	\Delta, F&\entails G  &\text{if and only if} && \Delta &\entails \lnot F\lor G \\
	F_1,\ldots,F_n&\entails G  &\text{if and only if} && F_1\land\ldots\land F_n&\entails G\\
	F\land G&\equival G\land F &F\lor G\equival G\lor F && F\limp G &\equival \lnot F\lor \lnot G
	\end{align*}
	by the definitions for semantics, entailment, and equivalence.\hfill \( \qed \)
\end{example}

\begin{definition}\label{def:equisatisfiable}
	Two sentences \( F \equisat G \) are \coloremph{equisatisfiable}
	if \( F \) is satisfiable whenever \( G \) is satisfiable
	and the other way round.
	An \coloremph{satisfiable transformation} morphs an arbitrary sentence
	\( F \) to a sentence \( F' \) such that \( F \equisat F' \).
	Equisatisfiable transformations respect the satisfiability of sentences.
\end{definition}



% \begin{example}
% 	\begin{align*}
% 		\{ \mcC\lor p, \mcD\lor\lnot p \} &\entails \{ \mcC\lor p, \mcD\lor\lnot p, \mcC\lor\mcD \}
% 		\\
% 		\{ \mcC\lor p, \mcD\lor\lnot p \} &\equisat \{ \mcC\lor p, \mcD\lor\lnot p, \mcC\lor\mcD \}
% 		\\
% 		\{ \mcC\lor p, \mcD\lor\lnot p \} &\not\equiv \{ \mcC\lor p, \mcD\lor\lnot p, \mcC\lor\mcD \}
% 	\end{align*}
% \end{example}

\begin{example}
	Let \(F\) be an arbitrary formula with \( \fvar(F) = \{ x \} \).
	It is easy to see that in general
	\( F\{ x\mapsto\ma \} \not\equiv F\{ x\mapsto\mb \} \)
	but
	\( F\{ x\mapsto\ma \} \equisat F\{ x\mapsto\mb \} \),
	e.g.~we can construct a model such that
	\(\mP(\ma)\) holds but \(\mP(\mb)\) does not.
	But undoubtedly \(\mP(\ma)\) is as satisfiable as
	 \(\mP(\mb)\).
\end{example}

\begin{example}
	\( \exists x \, \mP(x) \equisat \mP(\ma) \), but only
	\( \mP(\ma) \entails \exists x \, \mP(x) \) holds.
\end{example}

\begin{example}
	\[
	\{ \, \mP(x, \mf(x)), \, \mQ(y,\ma)\, \}
	\equisat \forall x\exists y(\mP(x,y)) \land \exists a\forall y(\mQ(y,a)) .
	\]
\end{example}


\subsection{Equality}

In Definition~\ref{def:semantics:atoms} we have interpreted the equality symbol as binary relation without restrictions.
This allows unwelcome models as in Example~\ref{ex:a:neq:a}.
Hence we state useful definitions to deal with this situation and demonstrate their usage in an example.

\begin{example}\label{ex:a:neq:a}
	Any interpretation \( \mcI \)
	with \( {\mEQ_\mcI} = \emptyset \) satisfies \( \ma\mNE\ma \).
\end{example}

\begin{definition}\label{def:normal:interpreation}
	An \coloremph{normal} interpretation defines \( \mEQ_\mcI \) as identity on its domain,
	e.g.~the equation of terms \( s \mEQ_\mcI t \) holds if and only
	if any evaluation of its terms are equal \( \alpha_\mcI(s) = \alpha_\mcI(t) \)
	for all assignments \( \alpha \).
	In other words a normal interpretation yields different elements
	for ground terms \( s' \) and \( t' \) if and only if \( s'\mNE_\mcI t' \).
\end{definition}

\begin{definition}
	A \coloremph{term interpretation}
	\( \mcI_t \)
	is an interpretation
	where the elements of its domain \( A = \mcTFf/_\sim \)
	are equivalence classes of ground terms
	and the interpretation of each ground term \( t^{\mcI_t} := {[t]}_\sim \) is its equivalence class.
	%		An equation \( s\mEQ t \) of ground terms holds in if \( [s]_\sim=[t]_\sim \).
	A ground predicate \( \mP(t_1,\ldots,t_n) \) holds if
	\( ({[t_1]}_\sim,\ldots,{[t_n]}_\sim) \in \mP^{\mcI_t} \subseteq A^n \).
	%	A ground literal does not hold if and only if its complementary literal holds.
	%	In an \coloremph{equational} term interpretation an equation \( s\mEQ t \) holds if an only if \( s\sim t \).
\end{definition}


\begin{example}
	Consider the satisfiable set of clauses \( S = \{ \mf(x) \mEQ x \} \).
	%
	We easily find a Herbrand model \( \mcH \) with
	predicate definition
	\( \mEQ_\mcH = \{ (\mf^{i+1}(\ma)), \mf^{i}(\ma) \mid i\geq 0  \}  \).
	However \( \mcH \) is not a normal model because obviously \( \mf(\ma)\neq\ma \) in its domain.
	%
	Further on we easily find an normal model \( \mcM \)
	with domain \( \{ \mc \} \), function definition \( \mf_\mcM(\mc) \mapsto \mc \),
	and the relation \( {\mEQ_\mcM} = \{ (\mc,\mc)\} \) coincides with identity in its domain.
	Certainly this model \( \mcM \) is not an Herbrand model
	because the interpretation of ground term \( {\mf(\mc)}_\mcM = \mc \) is not the ground term \( \mf(\mc) \) itself.
	%
	On the other hand we easily construct a normal term model
	\( \mcM_t \) with domain \( \{ {[\ma]}_\sim \} \),
	a plain function definition
	\( \mf_{\mcM_t}({[\ma]}_\sim) \mapsto {[\ma]}_\sim \)
	with equivalence relation
	\( \ma \sim \mf(\ma) \).
	Hence \( \mEQ_{\mcM_t} \) agrees to equality in its domain of equivalence classes of ground terms.
\end{example}

%How we find a normal model or justify its existence we will discuss in Section \vref{sec:automation} about Automation.

\section{Term Rewriting and Orderings}\label{sec:termrewriting}

\begin{definition}\label{def:rewrite:signature}
	A term rewrite signature \( \mcFf \) is a set of function symbols
	with associated arities
	as in Definition~\ref{def:signature}.
	Terms, term variables, ground terms and unary function symbol notations
	are defined as in Definitions~\ref{def:terms} to~\ref{def:unary:power}.
\end{definition}


\begin{definition}\label{def:rewriterule}
	A \coloremph{rewrite rule} is an equation of terms where the left-hand side is not a variable
	and the variables occuring in the right-hand side occur also in the left-hand side.
	%	\[
	%		\ell\rwEQ r \text{ is rewrite rule }\quad :\Longleftrightarrow\quad\ell\not\in\mcV\text{ and }\var(r)\subseteq\var(l)
	%	\]
	A rewrite rule \( \ell'\rwStep r' \) is a \coloremph{variant} of \( \ell\rwStep r \) if there is a variable renaming \( \varrho \)
	such that
	\( (\ell\rightarrow r)\varrho := \ell\varrho\rightarrow r\varrho = l'\rightarrow r' \).
	A \coloremph{term rewrite system} is a set of rewrite rules without variants.
	In a \coloremph{ground} term rewrite system every term on every side in every rule is a ground term.
\end{definition}

Although we use the the same symbol for implications \( F\limp G \) between first order formulae
and rewrite rules \( s\rwStep t \) or rewrite steps \( s'\rwStep_\mcR t \) between first order terms,
there will not arise any ambiguity for the reader about the role of the symbol.

%
\begin{definition}
	We say \( s\rightarrow_\mcR t \) is a
	\coloremph{rewrite step}
	 with respect to TRS \( \mcR \)
	when there is a position \( p \in \pos(s) \),
	a rewrite rule \( \ell\rwStep r\in\mcR \),
	and a substitution \( \sigma \) such that
	\( s|_p=\ell\sigma \) and \( s{[r\sigma]}_p = t \).
		The subterm \( \ell\sigma \) is called \coloremph{redex} and
	\( s \) rewrites to \( t \) by \coloremph{contracting} \( \ell\sigma \) to \coloremph{contractum} \( r\sigma \).
	%
	We say a term \( s \) is \coloremph{irreducible} or in \coloremph{normal form} with respect to TRS \( \mcR \) if there is no rewrite step \( s\rightarrow_\mcR t \) for any term \( t \).
	The set of normal forms \( \mNFR \) contains all irreducible terms of the TRS \( \mcR \).

	\end{definition}
%
%\begin{definition}
%	A \coloremph{rewrite relation} is a binary relation on terms that is closed under contexts and substitutions.
%	A \coloremph{rewrite order} is a proper order on terms
%	(i.e. an irreflexive and transitive relation on terms) and a rewrite relation.
%	A \coloremph{reduction order} is a well-founded rewrite order.
%\end{definition}
%
\begin{definition}
	A term \( s \) can be rewritten to term \( t \) with notion \( s\rightarrow^*_\mcR t \)
	if there exists at least one \coloremph{rewrite sequence} \( (a_1,\ldots ,a_n) \) such that
	\( s=a_1 \), \( a_n=t \), and \( a_i\rightarrow_\mcR a_{i+1} \) are rewrite steps for \( 1\leq i<n \).
	A TRS is \coloremph{terminating} if there is no infinite rewrite sequence of terms.
	%
% 	Two Terms \( s \) and \( t \) are \coloremph{joinable} with notion \( s\downarrow t \)
% 	if both can be rewritten to some term \( c \), i.e.~\( s \rightarrow^*c\ \, ^*\!\!\leftarrow t \).
% %
% 	Two Terms \( s \) and \( t \) are \coloremph{meetable} with notion \( s\uparrow t \)
% 	if both can be rewritten from some common ancestor term \( a \), i.e.~\( s \leftarrow^*a\ \, ^*\!\!\rightarrow t \).
% %
% 	A TRS is \coloremph{confluent } if \( s \) and \( t \) are joinable whenever \( s\ ^*\!\!\leftarrow a \rightarrow^* t \) holds for some term \( a \).
% 	%
% 	Terms \( s \) and \( t \) are \coloremph{convertible} with notion \( s\leftrightarrow^* t \)
% 	if there exists a sequence \( (a_1,\ldots ,a_n) \) such that
% 	\( s=a_1 \), \( a_n=t \), and \( a_i\leftrightarrow a_{i+1} \), i.e.~\( a_i\rightarrow a_{i+1} \) or \( a_i\leftarrow a_{i+1} \) are rewrite steps for \( 1\leq i<n \).
\end{definition}
%
\begin{definition}\label{def:closed-under}
	A \coloremph{rewrite relation} is a binary relation
	\( \relation \) on arbitrary terms \( s \) and \( t \),
	which additionally is \coloremph{closed under contexts}
	(whenever \( s\relation t \) then \( u{[s]}_p\relation u{[t]}_p \)
	for an arbitrary term \( u \) and any position \( p\in\pos(u) \))
	and \coloremph{closed under substitutions}
	(whenever \( s\relation t \) then \( s\sigma\relation t\sigma \)
	for an arbitrary substitution \( \sigma \)).
\end{definition}
\begin{lemma}
	The relations \( \rightarrow^*_\mcR \),
	\( \rightarrow^+_\mcR \),
	\( \downarrow_\mcR \), \( \uparrow_\mcR \) are rewrite relations on every TRS \( \mcR \).
\end{lemma}
%%
\begin{definition}\label{def:simplification:order}
	A proper (i.e.~irreflexive and transitive) order on terms is called \coloremph{rewrite order} if it is a rewrite relation.
	A \coloremph{reduction order} is a well-founded rewrite order,
	i.e.~there is no infinite sequence
	\( {(a_i)}_{i\in\mathbb{N}} \)
	where \( a_i\succ a_{i+1} \) for all \( i \).
	% with \( i\in\mathbb{N} \).
	A \coloremph{simplification order} is a rewrite order with the \coloremph{subterm property},
	i.e.~\( u{[t]}_p \succ t \) for all terms \( u \), \( t \) and positions \( p\neq\epsilon \).
\end{definition}

\begin{figure}[htb]\label{fig:simplification:order}
	\begin{tikzpicture}
    \node (defCUC) at (-4,8.5) { \( s\succ t\Rightarrow \ctx[s]\succ \ctx[t] \)};
    \node (CUC) at (-4,8) { contexts };
    \node (defCUS) at (0,8.5) { \( s\succ t\Rightarrow s\sigma\succ t\sigma \)};
    \node (CUS) at (0,8) { substitutions };

    \node (ASYM) at (4,8.5) { \( s\succ t\Rightarrow t\not\succ s \) };
    \node (ASYM) at (4,8) { asymmetric };

        \node (CL) at (-2,6) { closed under };
        \node (defIRR) at (2,6.5) { \( s\succ t\succ u\Rightarrow s\succ u \) };
        \node (IRR) at (6,6) { irreflexive };
        \node (defIRR) at (6,6.5) { \( s\nsucc s \) };
        \node (TRA) at (2,6) { transitive };

        \node (POx) at (4,4.4) { \( > \) };
        \node (PO) at (4,4) { proper order };
        \node (RWRx) at (0,4.4) { \( \rightarrow^*_\mcR \) };
        \node (RWR) at (0,4) { rewrite relation };

        \node (defSTP) at (-2,2.5) { \( \ctx\neq\ctxhole\Rightarrow \ctx[s]\succ s \) };
        \node (STP) at (-2,2) { subterm property };
        \node (RWO) at (2,2) { rewrite order };
        \node (defWF) at (8,4.5) { \( \nexists s_0(s_i\succ s_{i+1})_{i=0}^{\infty} \) };
        \node (WF) at (8,4) { well-founded };

        \node (SO) at (0,0) {simplification order};
        \node (RO) at (4,0) {reduction order};

        \node (WFO) at (6,2) { well-founded order };


        \draw[->] (CL) -- (CUC);
        \draw[->] (CL) -- (CUS);

        \draw[->] (PO) -- (IRR);
        \draw[->] (PO) -- (TRA);

        \draw[->] (RWR) -- (CL);

        \draw[->] (RWO) -- (PO);
        \draw[->] (RWO) -- (RWR);

        \draw[->] (SO) -- (STP);
        \draw[->] (RO) -- (RWO);
        \draw[->] (RO) -- (WFO);
        % \draw (RO) edge[out=0,in=-45,->] (WF);

        \draw[->] (SO) -- (RWO);
        \draw[->, dotted] (SO) -- (RO);

        \draw[->] (WFO) -- (WF);
        \draw[->] (WFO) -- (PO);

        \draw[->, dotted] (WF) -- (IRR);

        % \node (TRAIRR) at (4,7) { \( \bullet \) };
        \draw[->, dotted] (4,7) -- (ASYM);

        \draw[dotted] (TRA) edge [out=-10, in=-90] (4,7);
        \draw[dotted] (IRR) edge [out=190, in=-90] (4,7);

        \draw[->, dotted] (ASYM) -- (IRR);
        % \draw[->, dotted] (ASYM) edge [out=0, in=0] (IRR);


    \end{tikzpicture}
	\caption{Properties of relations on terms}
\end{figure}

Figure~\vref{fig:simplification:order} summarizes the properties of relations on terms.
The solid arrows mark definitions,
e.g.~a rewrite order is closed under contexts and substitutions (Definition~\ref{def:closed-under});
a simplification order is a rewrite order
that respects the subterm property
(Definition~\ref{def:simplification:order}).
The dotted arrows mark derived properties,
e.g.~every simplification order is a reduction order
(Lemma~\ref{lem:siplifiaction:order:well:founded});
transitive and irreflexive relations are always asymmetric,
etc.

\begin{lemma}\label{lem:siplifiaction:order:well:founded}
	Every simplification order is well-founded, hence it is a reduction order.
\end{lemma}

\begin{theorem}
	A TRS \( \mcR \) is terminating if and only if there exists a reduction order \( \succ \)
	such that \( l\succ r \) for every rewrite rule \( l\rightarrow r\in\mcR \).
	We call \( \mcR \) simply terminating if \( \succ \) is a simplification order.
\end{theorem}

\subsection{Clause and literal orderings}

\begin{lemma}
	Any ordering \( \succ \) on a set \( C \) can be extended to an ordering on multisets over \( C \)
	as follows \( N \succ M \) if \( N \neq M \)
	and whenever there is \( x\in C \) with \( N(x) < M(x) \)
	then there is \( y \succ x \) with \( N(y) > M(y) \).

	An ordering \( \succ \) on terms can be extended to orderings on literals and clauses.
\end{lemma}
% 2.4
For the following definition we assume
\( \succG \) as a a total, well-founded and monotone extension
from a total simplification ordering on ground terms
to ground clauses~\cite{NR2001}.

\begin{definition}
	% 4.3
    We define an order \( \succL \) on ground closures of literals
    as an arbitrary total well-founded extension of \( \succG \)
    such that
    \( L\cdot\sigma\succL L'\cdot\sigma' \) whenever
    \( L\sigma\succG L'\sigma' \).

	% 5.1
    We define an order \( \succC \) on ground closures
    as an arbitrary total well-founded extension of
    \( \succC' \) --- an inherently well-founded order defined as extension of \( \succG \)
    such that
    \( C\cdot\tau\succC' D\cdot\rho \) whenever
    \( C\tau\succG D\rho \) or \( C\theta = D \) for an proper instantiator \( \theta \).
\end{definition}

\begin{lemma}A well-founded and total order on general ground terms always exists.\end{lemma}

	\begin{definition}[Order on literals]\label{def:orders-on-literals}
		We extend a well-founded and total order \( \succ \) on general ground terms,
		i.e~general atoms to a well-founded proper order \( \succ_\mL \)
		on literals such that for all atoms \( A \) and \( B \) with \( A\succ B \)
		the relations \( A\succ_\mL B \),
		\( \lnot A\succ_\mL\lnot B \) and
		\( \lnot A\succ_\mL A \) hold.
		%
		A (non-ground) literal \( L \) is \coloremph{(strictly) maximal} if there exists a ground substitution \( \tau \)
		such for no other literal \( L' \) the relation \( L'\tau\succ L\tau \) (strictly: \( \succcurlyeq \)) holds.
		We write \( \succ_{gr} \) to suggest the existence of such a ground substitution \( \tau \).
	\end{definition}



