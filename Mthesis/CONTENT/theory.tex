% !TeX root = ../mthesis.tex
% !TeX encoding = UTF-8
% !TeX spellcheck = en_US

\chapter{Soundness and Completeness}

In the last chapter with introduced \InstGenEQ.
We can/could easily deduce soundness of this approach
from the definition of satisfiability of a set of clauses.
The set is satisfiable
if all possible ground instances of all its clauses
hold in at least one model.
We add proper instances of already present clauses to the set of clauses.
Hence the set of all possible ground instances does not change
the growing sets of clauses are all equisatisfiable.
Additionally the procedure stays sound when we remove clauses that are
entailed by other clauses still in the set.

But proper instances of clauses
are entailed by there origins. So a carless removal of entailed clauses
can neutralize our effort towards an unsatisfiable set of ground instances.
%
But even when we carefully craft our redundancy criteria for the removal of clauses
it is not easily seen that the procedure is complete,
i.e.~that for any unsatisfiable set of clauses an unsatisfiable
set of ground clauses will eventually fall out.

For that we are going to show
that any saturated -- no more non-redundant instances are derivable --
set of clauses is satisfiable if the set of grounded clauses is satisfiable.

First we discuss a calculus for sets of literal closures.
and state redundancy criteria.
After that we show that if no more steps are applicable
the set of literals is satisfiable
if the set of grounded literals is satisfiable.
At the end of this chapter we lift the result to predicates and clauses.


\IMPROVE{
For example we must not remove the derived proper instance
\(
    \mP(x)\{x\mapsto\mf(z)\}
\) from
\(
    \{\, \mP(\mf(x'))\lor\mQ(y'), \mP(x), \lnot\mP(\mf(y))\, \}
    \disjointunion\,
    \{\, \mP(\mf(z))\, \}
\) but there is no need to keep
\(
    \mP(\mf(x'))\lor\mQ(y')
\).}
% -- \cite{GK2004csl} 1 Introduction

% -- \cite{GK2004csl} 2 Preliminaries

% % -- \cite{GK2004csl} 2.3

\section{Unit Paramodulation}
\begin{definition}
    A closure is a pair \( (C,\sigma) \) of a clause \( C \) and a substitution \( \sigma \),
    conveniently written as \( C\cdot\sigma \). Two closures \( C\cdot\sigma = D\cdot\tau \)
    are the same if \( C \) is a variant of \( D \) and \( C\sigma \) is a variant of \( D\tau \).
    Note the different meaning of a closure \( C\cdot\sigma \) representing a clause \( C\sigma \)
    --- the result of applying substitution \( \sigma \) to \( C \).
    A ground closure represents a ground clause.
\end{definition}

% \cite{GK2004csl, NR2001}

% 4.1
\begin{definition}[Unit paramodulation \cite{GK2004csl}]
% \begin{gather*}
%     \begin{array}{ccl}
%         \infer[\theta]
%         {L[r]\theta\cdot\rho}
%         {(l\mEQ r)\cdot\sigma & L[l']\cdot\sigma'}
%         &\text{if}&
%         l\sigma\succG r\sigma,
%         \theta=\mgu(l,s),
%         l\sigma = l'\sigma' = l'\theta\rho,
%         l'\notin\mcV
%         \\[0.7em]
%         \infer[\mu]
%         {\emptyclause}
%         {(s\mNE t)\cdot\tau}
%         &\text{if}&
%         s\tau = t\tau,
%         \mu=\mgu(s,t)
%     \end{array}
% \end{gather*}

\begin{gather*}
    \infer[\theta]
        {L[r]\theta\cdot\rho}
        {(\ell\mEQ r)\cdot\sigma & L[\ell']\cdot\sigma'}
        \qquad\qquad
        \infer[\mu]
        {\emptyclause}
        {(s\mNE t)\cdot\tau}
\end{gather*}
where

\begin{itemize}

    \item
        $\ell\sigma\succG r\sigma$,
        $\theta=\mgu(\ell,s)$,
        $\ell\sigma = \ell'\sigma' = \ell'\theta\rho$,
        $\ell'\notin\mcV$

    \item
        $s\tau = t\tau$,
        $\mu=\mgu(s,t)$
\end{itemize}
\end{definition}

\section{Redundancy and Saturation}
    We assume \( \succ, \succG, \succL, \succC, \succ_\mL \)
    as used and defined in
    Section \vref{sec:literal:clause:orderings}.
    \IMPROVE{use consolidated ordering symbols}

\begin{definition}
    A substitution \( \sigma \) is irreducible w.r.t.~TRS \( R \)
    if
    \( \image(\sigma)\subseteq\mNFR \),
    i.e.~all terms \(\sigma(x)\) with
    \( x\in\domain(\sigma) \) are irreducible (normal forms) w.r.t.~TRS \(R\).
    % 4.2
    Let \( R \) be an arbitrary ground rewrite system
    and \( \mcL \) a set of literals closures.
    With
    \begin{align*}
        \irred_R(\mcL) &= \{ L\cdot\sigma \mid L\cdot\sigma\in\mcL,
        \sigma\textsf{ irreducible w.r.t.~}R \}
    \\
        \mcL_{L\cdot\sigma\succL} &= \{
            L'\cdot\sigma'\mid L'\cdot\sigma'\in\mcL, L\cdot\sigma\succL L'\cdot\sigma'
            \}
    \end{align*}
    % 4.4 %
    we call a literal closure \( L\cdot\sigma \) \coloremph{UP-redundant}
    in set of literal closures \( \mcL \)
    if
    \[
        R \cup \irred_R(\mcL_{L\cdot\sigma\succL}) \entails L\sigma
    \]
    holds for every ground rewrite system \( R \) oriented by \( \succG \)
    where \( \sigma \) is irreducible w.r.t.\,\( R \).
    We distill all UP-redundant closures in \( \mcL \) to the set
    \( \mcR_{UP}(\mcL) \).
    \[
        \mcR_{UP}(\mcL) = \{
            L\cdot\sigma
            \mid
            L\cdot\sigma \in \mcL,
            R \cup \irred_R(\mcL_{L\cdot\sigma\succL}) \entails L\sigma
        \}
    \]

    % denotes the set of all closures in \( \mcL \) less than
\end{definition}

% 5.3
\begin{lemma}
    Let \( R \) be a ground rewrite system and UP is applicable to
    \( (l\mEQ r)\cdot\sigma, L[l']\cdot\sigma' \)
    with conclusion
    \( L[r]\theta\cdot\rho \).
    If \( \sigma,\sigma' \) are irreducible w.r.t.~R then \( \rho \) is irreducible w.r.t.~R.
\end{lemma}

\begin{proof}
\end{proof}

\begin{definition}[Non proper demodulation \cite{GK2004csl}]
    \begin{gather*}
            \infer[\theta\text{ non-proper instantiator}]
            {L[r]\theta\cdot\sigma'}
            {(l\mEQ r)\cdot\sigma & L[l']\cdot\sigma'}
    \end{gather*}
    where
        \( l'=l\theta \),
        \( l\sigma\succG r\sigma \),
        \( l\sigma = l'\sigma \),
        \( \var(r)\subseteq\var(l) \),
        \( L[l']\sigma'\succG(l\mEQ r)\sigma \).
    \end{definition}

\begin{example}
    Let \( g\succ f \),
    \( \sigma = \{ x\mapsto \ma, y\mapsto \mb \} \),
    \( \sigma' = \{ x'\mapsto \ma, y'\mapsto \mb \} \)
    \begin{align*}
    \infer[
        \theta = \{ x\mapsto x', y\mapsto y'\}
    ]{
        \mf(x') \mEQ y'\cdot\sigma'
    }{
        \mg(x,y)\mEQ \mf(x)\cdot\sigma & \mg(x',y')\mEQ y'\cdot\sigma'
    }
    \end{align*}
\end{example}

\begin{lemma}Non-proper demodulation is a simplification rule.\end{lemma}
    \begin{proof}
        ...
    \end{proof}

\begin{definition}
    A UP-\coloremph{saturation process} is
    a sequence \( \{ \mcL_i \}_{i=0}^\infty \)
    of sets of literal closures where
    \( \mcL_{i+1} \) can be obtained from
    \( \mcL_i \)
    by \coloremph{adding} a conclusion of an UP-inference with premises in \( \mcL_i \)
    or by \coloremph{removing} a UP-redundant w.r.t.~\(\mcL_i\) closure:
    \begin{gather*}
        \mcL_{i+1} = \left\{
            \begin{array}{lclc}
                \mcL_i \cup\,\emptyclause
                &\text{if}
                &
                    \mcL_i\ni(s\mNE t)\cdot\tau,\ {}
                    s\tau = t\tau,\ {}
                    \mu=\mgu(s,t)
                \\
                \mcL_i \backslash L\cdot\sigma
                &\text{if}
                &R \cup \irred_R(\mcL_{L\cdot\sigma\succL}) \entails L\sigma
                \\
                \mcL_i \cup\, L[r]\theta\cdot\rho
                &\text{if}
                &\left\{\begin{array}{l}
                    \mcL_i\ni(l\mEQ r)\cdot\sigma,\ {}
                    l\sigma\succG r\sigma,\ {}
                    \theta=\mgu(l,l'),
                    \\
                    \mcL_i\ni L[l']\cdot\sigma',\ {}
                    l'\notin\mcV,\ {}
                    l\sigma = l'\sigma' = l'\theta\rho,
                \end{array}\right.
                \\
                \mcL_i
                &&\text{otherwise}
            \end{array}
        \right.
    \end{gather*}

    Let \( \mcL^\infty \) be the set of persistent closures,
    i.e.~the lower limit of the sequence.
    The process is \coloremph{fair} if for every UP-inference
    with premesis in \( \mcL^\infty \) the conclusion is UP-redundant
    w.r.t.~\(\mcL_j\) for some \(j\).

    For a set of literals \( \mcL \) we define
    the saturated set of literal closures
    \( \mcL^{sat} = \mcL^\infty\backslash\mcR_{UP}(\mcL^\infty) \)
    for some UP-saturation process
    \( \{ \mcL_i\}_{i=0}^\infty \)
    with \( \mcL_0 = \mcL \).
\end{definition}

\begin{lemma}
    The set \( \mcL^{sat} \) is unique because
    for any two UP-fair saturation processes
    \(\{ \mcL_i
        \}_{i=0}^\infty\) and
        \(\{ \mcL'_i
        \}_{i=0}^\infty\)
        with \( \mcL_0 = \mcL'_0 \) we have
        \begin{gather*}
            \mcL^\infty \backslash \mcR_{UP}(\mcL^\infty)
            =
            \mcL'^\infty \backslash \mcR_{UP}(\mcL'^\infty)
        \end{gather*}
\end{lemma}

% 5.2
\begin{definition}
    Let \( S \) be a set of clauses.
    A ground closure \( C \) is \coloremph{Inst-redundant} in \( S \)
    if there exist closures \( C_1,\ldots,C_k \)
    that are ground instances of \( S \) such that
    \begin{align*}
        C \succC C_i \tag*{for all \( i=1,\ldots,k \)} \\
        C_1,\ldots,C_k\entails C
    \end{align*}

    A (possibly non-ground) clause \( C' \) is called Inst-redundant
    if each ground closure \( C\cdot\sigma \) is Inst-redundant in S.
    The set \( \mcR_{Inst}(S) \) denotes the set of all Inst-redundant clauses/closures in S.
\end{definition}

\begin{definition}
    Consider a set of clauses \( S \), a model \( I_\bot \) of \( S_\bot \),
    and a \coloremph{selection funtion} \( \sel \) based on \( I_\bot \)
    that maps clauses to literals such that for all \( C\in S \)
    \begin{gather*}
        \sel(C)=L \in C
        \qquad
        \text{and}
        \qquad
        I_\bot \entails L\bot
    \end{gather*}
    % 5.4
    We define a set of \coloremph{S-relevant} instances of literals \( \mcL_S \)
    as the set of all literal closures \( L\cdot\sigma \) such that
    \begin{gather*}
        L\lor C\in S
        \qquad
        (L\lor C)\cdot\sigma\text{ is not Inst-redundant in }S
        \qquad
        L = \sel(L\lor C)
    \end{gather*}
    Then let \( \mcL_S^{sat} \) denote the UP-saturation process of \( \mcL_S \).
    % 5.5
    A set of clauses is \coloremph{Inst-saturated} w.r.t.~a selection function \( \sel \)
    if \( \emptyclause\not\in\mcL_S^{sat} \).
\end{definition}

% 5.7 b
\begin{definition}[model construction]
    By induction on \( \succL \)
    a candidate model for \( S \) is constructed
    based on \( \mcL_S^{sat} \).

    \begin{align*}
        \mcL_S^{sat} &=
        \\
        I_L &= \bigcup_{L \succL M}
            \epsilon_M
        \\
        R_L &=  \{
            s\mEQ t \mid s\succL t
        \} \subseteq I_L
    \end{align*}

\end{definition}

% 5.7
\begin{theorem}
    An Inst-saturated set of clauses \( S \) is satisfiable if \( S\bot \) is satisfiable.
    % If for a given set of clauses \( S \)
    % the set of ground instances \( S\bot \) is satisfiable
    % and \( S \) is Inst-saturated
    % with selection function \( \sel_\bot \) based
    % on model \( I_\bot \) for \( S\bot \)
    % then S is satisfiable.
\end{theorem}



\begin{proof}
    We assume an Inst-saturated set of clauses \( S \),
    where \( S\bot \) is satisfiable with a model \( I_\bot \),
    a function \( \sel \) from clauses to literals is based on \( I_\bot \),
    \( \mcL_S \) contains the S-relevant instances of (selected) literals,
    and \( \mcL_S^{sat} \) does not contain the empty clause.



\end{proof}

\begin{align*}
    I_S &= \bigcup_{L\in\mcL_S^{sat}} \epsilon_L
    \\
    R_S &= \bigcup_{L\in\mcL_S^{sat}} R_L
\end{align*}

% 5.7k
\begin{lemma}
    If \( M\cdot\tau \in \mcL_S^{sat} \),
    \( M\cdot\tau \) is irreducible by \( R_s \),
    and \( M\cdot \) is not productive, then
    \begin{gather}
        I_{M\cdot\tau} \models (\lnot M)\tau
    \end{gather}
\end{lemma}

% \begin{proof}[by case analysis]
%     \begin{enumerate}
%         \item[M = (s\mEQ t)]

%         \item[M = (s\mNE t)]
% \end{proof}

