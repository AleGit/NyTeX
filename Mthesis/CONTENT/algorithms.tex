% !TeX root = ../mthesis.tex
% !TeX encoding = UTF-8
% !TeX spellcheck = en_US

\chapter{Algorithms and Data Structures}

\section{Terms, Literals and Clauses}




\section{Given Clause Algorithm}

In the previous chapter we just randomly applied derivation rules
on pairs of clauses from the set of clauses 
to derive and add new clauses 
until we could conclude unsatisfiability of the set of clauses.

In this section we will discuss the given clause algorithm 
as a saturation process that eventually determines all possible derivate clauses
with respect to a given calculus, e.g. ordered resolution.




\begin{procedure}[Basic Given Clause Algorithm]
	We start with a set of clauses $S = P \disjointunion U$, 
	where $P$ is a set of {\myem passive} clauses 
	and $U$ is the initially empty set of {\myem active} (or used) clauses.
	\begin{enumerate}
		\item[(\Lightning?)] Whenever we can conclude unsatisfiability of $S$ 
		we exit the procedure with $\lnot\SAT$.
		\setcounter{enumi}{0}
		\item If $P$ is empty we exit and return \SAT.
		\item We select a passive clause, i.e.~the given clause. (\Lightning?)
		\item We iterate over the active clauses and check for applicable inference rule
		for each pair of an active and the given clause
		to derive additional clauses which we add to the set of passive clauses.
		(\Lightning?)
		\item We move the given clause from passive to active and continue with step 1.
	\end{enumerate}
\end{procedure}


In the following examples the given clause is boxed, 
the active clauses are on the left of the given clause, 
and the passive clauses are on the right. 
Each line represents one iteration of our procedure 3.
%
In both examples we start with the unsatisfiable set of clauses $S=\{\, \mP(\ma)\lor\mQ(a), \mP(\ma)\lor\lnot\mQ(y), \lnot\mP(x) \,\}$.

\subsection{Ordered Resolution}

(\Lightning?) We conclude unsatisfiability of $S$ whenever the empty clause is derived (or already present).
New clauses are disjunctions of an active and the given clause where conflicting literals were removed.

\begin{example}[Ordered Resolution] We assume $\mP(\ma)\succ\mQ(\ma)$,
	underline the maximal literal the given clauses,
	color conflicts red,
	 and derive the empty clause in the fifth iteration.
	
	\begin{align*}
	^{1:}&\boxed{\underline{\mP(\ma)}\lor\mQ(a)} 
	& ^{2:}&\mP(\ma)\lor\lnot\mQ(y) & ^{3:}&\lnot\mP(x) 
	\\
	^{1:}&\mP(\ma)\lor\mQ(a) 
	& ^{2:}&\boxed{\underline{\mP(\ma)}\lor\lnot\mQ(y)} & ^{3:}&\lnot\mP(x) 
	\\
	^{1:}&{\colLo\mP(\ma)}\lor\mQ(a) 
	& ^{2:}&{\colLo\mP(\ma)}\lor\lnot\mQ(y) 
	& ^{3:}&\boxed{\underline{\lnot\mP(x)}} 
	& ^{1,3:}&\mQ(\ma) 
	& ^{2,3:}&\lnot\mQ(y)
	\\
	^{1:}&\mP(\ma)\lor\mQ(a) 
	& ^{2:}&\mP(\ma)\lor{\colLo\lnot\mQ(y)} 
	& ^{3:}&\lnot\mP(x)
	& ^{1,3:}&\boxed{\underline{\mQ(\ma)}} 
	& ^{2,3:}&\lnot\mQ(y) 
	& ^{2,(1,3):}&\mP(\ma)
	\\
	^{1:}&\mP(\ma)\lor{\colLo\mQ(a)} 
	& ^{2:}&\mP(\ma)\lor\lnot\mQ(y) 
	& ^{3:}&\lnot\mP(x)
	& ^{1,3:}&\colLo\mQ(\ma) 
	& ^{2,3:}&\boxed{\underline{\lnot\mQ(y)}} 
	& ^{2,(1,3):}_{1,(2,3):}&\mP(\ma)
	& ^{(1,3),(2,3):}&\emptyclause
	\\
	\end{align*}
\end{example}

\subsection{InstGen}

(\Lightning?) We conclude unsatisfiability of $S$ whenever a subset of $S_i\bot$ is unsatisfiable. 
New clauses are proper instances of an active or the given clause.


\begin{example}[InstGen]The active and given clauses are already encoded and given to a \SAT solver.
	The basic given clause algorithm would stop and fail after $(4)$
	since $S_4\bot$ is still satisfiable and there is no conflict between the underlined selected literal of the given clause 
	and any selected literals of any of the active clauses.
	
	
	\begin{align*}
	(1)\quad^{1:}&\boxed{\underline{\mP(\ma)}\lor\mQ(\ma)} 
	& ^{2:}&\mP(\ma)\lor\lnot\mQ(\consbot/y) 
	& ^{3:}&\lnot\mP(\consbot/x)
	\\
	(2)\quad^{1:}&{\colHi\underline{\mP(\ma)}}\lor\mQ(\ma) 
	& ^{2:}&\boxed{\underline{\mP(\ma)}\lor\lnot\mQ(\consbot/y)} 
	& ^{3:}&\lnot\mP(\consbot/x) 
	\\
	(3)\quad^{1:}&{\colHi\underline{\mP(\ma)}}\lor\mQ(\ma) 
	& ^{2:}&{\colHi\underline{\mP(\ma)}}\lor\lnot\mQ(\consbot/y) 
	& ^{3:}&\boxed{\underline{\lnot\mP(\consbot/x)}} 
	& ^{1,3:}_{2,3:}&\lnot\mP(\ma)
	\\
	(4)\quad
	^{1:}&{\colLo\mP(\ma)}\lor\colHi\mQ(\ma) 
	& ^{2:}&{\colLo\mP(\ma)}\lor\colHi\lnot\mQ(\consbot/y) 
	& ^{3:}&{\colHi\underline{\lnot\mP(\consbot/x)}} 
	& ^{1,3:}_{2,3:}&\boxed{\underline{\lnot\mP(\ma)}}
	\end{align*}
	But the model did change in (4) and the selected literals of two of the active clauses changed too.
	Active clauses with changed selected literals have to be moved back to the passive clauses.
	Then we hit a contradiction of ground instances in (7').
	\begin{align*}
	(4')\quad^{3:}&{\underline{\colHi\lnot\mP(\consbot/x)}} 
	& ^{1,3:}_{2,3:}&\boxed{\underline{\lnot\mP(\ma)}}
	& ^{1:}&{\colLo\mP(\ma)}\lor\colHi\mQ(\ma) 
	& ^{2:}&{\colLo\mP(\ma)}\lor\colHi\lnot\mQ(\consbot/y)
	\\
	(5')\quad^{3:}&{\underline{\colHi\lnot\mP(\consbot/x)}} 
	& ^{1,3:}_{2,3:}&\colHi\underline{\lnot\mP(\ma)}
	& ^{1:}&\boxed{{\colLo\mP(\ma)}\lor\colHi\underline{\mQ(\ma)}} 
	& ^{2:}&{\colLo\mP(\ma)}\lor\colHi\lnot\mQ(\consbot/y)
	\\
	(6')\quad^{3:}&{\colHi\underline{\lnot\mP(\consbot/x)}} 
	& ^{1,3:}_{2,3:}&\colHi\underline{\lnot\mP(\ma)}
	& ^{1:}&{\colLo\mP(\ma)}\lor\colHi\underline{\mQ(\ma)} 
	& ^{2:}&\boxed{{\colLo\mP(\ma)}\lor\colHi\underline{\lnot\mQ(\consbot/y)}}
	& ^{2,1:}&\mP(\ma)\lor\lnot\mQ(\ma)
	\\
	(7')\quad^{3:}&{\lnot\mP(\consbot/x)} 
	& ^{1,3:}_{2,3:}&\colHi\lnot\mP(\tikzmark{NP}\ma)
	& ^{1:}&{\colLo\mP(\tikzmark{PP}\ma)}\lor\colN\mQ(\tikzmark{PQ}\ma) 
	& ^{2:}&{\mP(\ma)\lor\lnot\mQ(\consbot/y)}
	& ^{2,1:}&\boxed{{\colLo\mP(\tikzmark{PA}\ma)}\lor\colN\lnot\mQ(\tikzmark{NQ}\ma)}
%	\begin{tikzpicture}[overlay,remember picture,out=315,in=225,distance=0.4cm]
%	\draw[<->,red,shorten >=3pt,shorten <=3pt] (PQ.center) to (NQ.center);
%	\end{tikzpicture}
	\end{align*}
	\begin{tikzpicture}[overlay,remember picture, out=340, in=200 ]
	\draw[->, thick, dotted] (NP.south) to (PP.south);
	\draw[->, thick, dotted] (NP.south) to (PA.south);
	\draw[<->,red, thick] (PQ.south) to (NQ.south);
	\end{tikzpicture}
\end{example}

\begin{figure}[hbt]
	\begin{center}
		\begin{tikzpicture}[scale=0.95, transform shape]
		\node[rectangle] (start) at (0,-4em) {};
		\node (nc) [myrect] at (0,0) {new\\clauses};
		\node (pc) [myrect] at (0,8em) {passive clauses};
		\node (ab) [myrect] at (-8em,3.5em) {instantiation};
		\node (gs) [myrect] at (-8em,8em) { \SAT};
		\node (un) [mycircle] at (-15em,8em) {un\-satis\-fiable};
		\node (se) [myrect] at (-8em,12.5em) {selection};
		\node (gc) [myrect] at (0em,16em) {given clause};
		\node (ac) [myrect,dashed] at (-13em,16.5em) {active clauses};
		
		\node (sl) [myrect, thick] at (8.5em,12.5em) {selected literal};
		\node (us) [myrect,very thick, 
		%		minimum width=8em, minimum height=3.5em, text width=7em
		] at (8.5em,8em) { \InstGen};
		\node (sa) [mycircle] at (16em,8em) {satis\-fiable};
		\node (su) [myrect] at (8.5em,3.5em) {substitution};
		
		\draw[myarrow] (start) to (nc);
		\draw[myarrow] (nc) to (pc);
		\draw[myarrow] (nc.west)  to [bend left=10] (ab);
		\draw[myarrow] (ab) to (gs);
		\draw[myarrow] (gs) to (un);
		\draw[myarrow] (gs) to (se);
		\draw[myarrow] (pc) to (gc);
		\draw[myarrow] (se.north) to [bend left=10] (gc.west);
		\draw[myarrow] (us) to (sa);
		\draw[myarrow] (us) to (su);
		\draw[myarrow] (su) to [bend left=10] (nc.east);
		\draw[myarrow] (sl) to (us);
		\draw[myarrow] (gc.east) to [bend left=10] (sl.north);
		\draw[myarrow,dashed] (gc.north) to [bend right=15](ac);
		
		\draw[myarrow,dotted] (ac.east) to [bend left=25](pc);
		
		\end{tikzpicture}
		\caption{Proving loop with \SAT and \InstGen}
		\label{fig:inst-gen-maxcomp}
	\end{center}
\end{figure}

\section{Term Indexing}

\subsection{Clashing literals}

\subsection{Variants}

\subsection{Instances}



