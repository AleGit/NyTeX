% !TeX root = ../mthesis.tex
% !TeX encoding = UTF-8
% !TeX spellcheck = en_US

\chapter{other}

%\section{Purely equational logic}

We can transform a set of selected literals into a equational system 
when we introduce one new constant symbol $\bullet$,
and a new function symbol $f\!_P$ for every predicate symbol $P$:
\begin{align*}
	P(t_1,\ldots,t_n) \quad\Rightarrow&\quad f\!_P(t_1,\ldots,t_n) \mEQ \bullet \\ 
	\lnot P(t_1,\ldots,t_n) \quad\Rightarrow&\quad f\!_P(t_1,\ldots,t_n) \mNE \bullet
\end{align*}

%\section{Term Rewrite System}

We can transform a set of selected literals into a term rewrite system 
when we introduce two new constant symbols $\mathsf{true}$ and $\mathsf{false}$,
and a new function symbol $f\!_P$ for every predicate symbol $P$:
\begin{align*}
P(t_1,\ldots,t_n) \quad\Rightarrow &\quad f\!_P(t_1,\ldots,t_n) \rwEQ \mathsf{true} 
\\ 
\lnot P(t_1,\ldots,t_n) \quad\Rightarrow &\quad f\!_P(t_1,\ldots,t_n) \rwEQ \mathsf{false} 
\\
s \mEQ t \quad\Rightarrow&\quad s \mEQ t\rwEQ \mathsf{true} 
\\ 
r \mNE u \quad\Rightarrow&\quad r \mEQ u\rwEQ  \mathsf{false}
\end{align*}
When we find a not unifiable critical peek, then the set of literals is incosistent.




