% !TeX root = ../mthesis.tex
% !TeX encoding = UTF-8
% !TeX spellcheck = en_US


\chapter{Completeness and Undecidability}

%\begin{definition}
	We call a logical calculus, i.e.~a formal proof system, {\myem complete} if every valid formula is provable.
	In other words any expression that is satisfied by all interpretations 
	is deducible or justifiable by applying rules of the formal system only.
	We expect a useful calculus to be {\myem sound}, that is, every proven formula is valid, 
	i.e.~it holds in all interpretations.
%\end{definition}

\begin{lemma}[Refutation]
	By definition of the semantics of negation a formula is valid if and only if its negation is not satisfiable.
\end{lemma}

\begin{theorem}[Completeness, Gödel 1929]
	Natural deduction, a sound formal system, is complete.
\end{theorem}

\begin{theorem}
	[Undecidability, Church 1936, Turing 1937]
	The satisfiability problem for first order logic is undecidable.
\end{theorem}

\begin{theorem}[Trakhtenbort 1950, Craig 1950]
	The satisfiability problem for first order logic on {\myem finite} structures (domains) is undecidable.
\end{theorem}

\begin{theorem}[Finite model property]
\end{theorem}








\begin{theorem}[Compactness]\label{the:compactness}
	If every finite subset of a set of formulas $S$ has a model then $S$ has a model. 
\end{theorem}

\begin{theorem}[Löwenheim Skolem]\label{the:loewenheim}
	If a set of formulas $S$ has a model then $S$ has a countable model.
\end{theorem}

\begin{theorem}[Herbrand]\label{the:herbrand}
	Let $S$ be a set of clauses without equality. Then the following statements are equivalent.
	\begin{itemize}
		\item $S$ is satisfiable.
		\item $S$ has a Herbrand model.
		\item Every finite subset of all ground instances of $S$ has a Herbrand model.
	\end{itemize} 
\end{theorem}

\begin{corollary}
	Let $S$ be a set of clauses without equality. 
	Then $S$ is unsatisfiable if and only if there exists 
	an unsatisfiable finite set of ground instances of $S$.
\end{corollary}

\begin{lemma}
With Skolemization and Tseitin transformation we can effectively transform a arbitrary first order formula into an equisatisfiable set of clauses.	
\end{lemma}



\section{Decidable fragments of first order logic}

Since we have limited our first order syntax to \CNF in Definition \vref{def:syntax:CNF} 
we transform all the pure syntactical requirements for decidability of first order formulae 
to this representation, which introduces otherwise absent function symbols, albeit with restricted arities.


\begin{lemma}[$(\exists^*\forall^*)^*$ --- $P(x)$]\ref{lem:monadic}
	Satisfiability is decidable for monadic first order logic, 
	i.e.~the clauses 
	may contain constant and unary predicate symbols, 
	constant function symbols,
	and arbitrary many distinct variables 
	(each perhaps on multiple positions).
\end{lemma}


\begin{lemma}[$\exists^*\!\vec{x}\,\forall y\,\exists^*\!\vec{z}$ --- $P(\vec{x},y,\vec{z})$]\label{lem:ackermann}
	Satisfiability is decidable for the Ackermann class, 
	i.e.~the clauses
	may contain arbitrary predicate symbols, 
	constant and unary function symbols, 
	and at most one variable 
	(each perhaps on multiple positions) per clause.
\end{lemma}


\begin{lemma}[$ \exists^*\!\vec{x}\,\forall y_1\forall y_2\,\exists^*\!\vec{z}$ --- $P(\vec{x},y,\vec{z})$]\label{lem:goedel}
	Satisfiability is decidable for the Gödel class, 
	i.e.~the clauses
	may contain arbitrary predicate symbols, 
	constant, unary, and binary function symbols, 
	and at most two distinct variables 
	(each perhaps on multiple positions) per clause.
	
\end{lemma}


\begin{lemma}[$\exists^*\!\vec{x}\,\forall^*\!\vec{z}$ --- $P(\vec{x},\vec{z})$]\label{lem:schoenfinkel}
	Satisfiability is decidable for the Schönfinkel-Bernays class, 
	i.e.~the clauses
	may contain arbitrary predicate symbols, 
	constant function symbols, 
	and arbitrary many distinct variables 
	(each perhaps on multiple positions).
	
\end{lemma}

\begin{lemma}
\end{lemma}





\begin{definition}Finite model property
\end{definition}


\begin{theorem}
	content...
\end{theorem}



\section{Decidable theories in first oder logic}

praesburger arithmetic

successor



