% !TeX root = ../mthesis.tex
% !TeX encoding = UTF-8
% !TeX spellcheck = en_US


\chapter{Completeness and Undecidability}

%\begin{definition}
	We call a logical calculus, i.e.~a formal proof system, {\myem complete} if every valid formula is provable.
	In other words any expression that is satisfied by all interpretations 
	is deducible or justifiable by applying rules of the formal system only.
	We expect a useful calculus to be {\myem sound}, that is, every proven formula is valid, 
	i.e.~it holds in all interpretations.
%\end{definition}

\begin{lemma}[Refutation]
	By definition of the semantics of negation a formula is valid if and only if its negation is not satisfiable.
\end{lemma}

\begin{theorem}[Completeness, Gödel 1929]
	Natural deduction, a sound formal system, is complete.
\end{theorem}

\begin{theorem}
	[Undecidability, Church 1936, Turing 1937]
	The satisfiability problem for first order logic is undecidable.
\end{theorem}










\begin{theorem}[Compactness]\label{the:compactness}
	If every finite subset of a set of formulas $S$ has a model then $S$ has a model. 
\end{theorem}

\begin{theorem}[Löwenheim Skolem]\label{the:loewenheim}
	If a set of formulas $S$ has a model then $S$ has a countable model.
\end{theorem}

\begin{theorem}[Herbrand]\label{the:herbrand}
	Let $S$ be a set of clauses without equality. Then the following statements are equivalent.
	\begin{itemize}
		\item $S$ is satisfiable.
		\item $S$ has a Herbrand model.
		\item Every finite subset of all ground instances of $S$ has a Herbrand model.
	\end{itemize} 
\end{theorem}

\begin{corollary}
	Let $S$ be a set of clauses without equality. 
	Then $S$ is unsatisfiable if and only if there exists 
	an unsatisfiable finite set of ground instances of $S$.
\end{corollary}

\begin{lemma}
With Skolemization and Tseitin transformation we can effectively transform a arbitrary first order formula into an equisatisfiable set of clauses.	
\end{lemma}



\section{Decidable fragments and complexity}

\begin{definition}Finite model property
\end{definition}


\begin{theorem}
	content...
\end{theorem}



In this chapter we will outline fragments and theories in first order logic, 
where validity and satisfiability are decidable. 
We will focus on satisfiability and will present the fragments, theories and problems in clausal form.






