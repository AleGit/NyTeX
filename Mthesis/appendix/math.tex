% !TeX root = ../mthesis.tex
% !TeX encoding = UTF-8
% !TeX spellcheck = en_US

%\begin{definition}
%%A Cartesian product is a mathematical operation that returns a set of tuples from an ordered list of sets.
%%\end{definition}
%%\begin{definition}
%A binary relation on $A$ is a subset of $A \times A$, the Cartesian product of $A$ and $A$, i.e.~all pairs of elements of $A$.
%\end{definition}

%For the following definitions we express the conditions as predicates clausal form, 
%i.e.~the variables are universally quantified. 


%\begin{definition}
%We label a binary relation $R$ on an arbitrary domain $A$ if for all $x,y,z\in A$ the following holds
%\begin{align*}
%R(x,x)
%\tag*{reflexive}
%\\
%\lnot R(x,x) 
%\tag*{irreflexive}
%\\
%\lnot R(x,y) \lor \lnot R(y,z) \lor R(x,z)
%\tag*{transitive}
%\\
%\lnot R(x,y) \lor R(y,x)
%\tag*{symmetric}
%\\
%\lnot R(x,y) \lor \lnot R(y,x)
%\tag*{asymmetric}
%\\
%\lnot R(x,y) \lor \lnot R(y,x) \lor x = y
%\tag*{antisymmetric}
%\end{align*}
%
%\noindent We recognize the difference between {\em not symmetric} and {\em asymmetric}.
%%The first expresses that at least one pair of elements is not symmetrically related. 
%%The second one expresses that all pairs of elements are not symmetrically related.
%	\begin{align*}
%	\lnot\left( \forall x \forall y\ \lnot R(x,y) \lor R(y,x) \right)
%	&\equiv
%	\exists x\exists y\ R(x,y) \land \lnot R(y,x) 
%	\tag*{not symmetric}
%	\\
%	&\not\equiv
%	\\
%	\lnot\left( \exists x \exists y\ R(x,y) \land R(y,x)\right)
%	&\equiv
%	\forall x\forall y\ \lnot R(x,y) \lor \lnot R(y,x)
%	\tag*{asymmetric}
%	\end{align*} 
%%\end{remark}
%\end{definition}


\begin{definition}
	A n-ary relation is a subset of the Cartesian product of $n$ sets.
%	$A^n = \{ (a_1, \ldots, a_n) \mid a_i \in A\text{ for }0\leq i \leq n \}$.
	A {\myem binary relation} $R$ on domain $A$ 
	is a subset of $A^2 = A \times A = \{ (a,b) \mid a\in A, b\in A \}$ -- 
	the set of all pairs of elements in $A$.
	We write $R(a,b)$ or $a\ R\ b$ to express that $(a,b)\in R \subseteq A\times A$.
\end{definition}
\begin{definition}
	An {\myem equivalence relation} $\sim$ on an arbitrary domain $A$ 
	is a reflexive, transitive, and symmetric binary relation on $A$, 
	i.e.~for arbitrary elements $x,y,z$ in $A$ the following three clauses hold.
	\begin{align*}
	x\sim x
	\tag*{reflexivivity}
	\\
	x\not\sim y \lor  y \not\sim z \lor x\sim z
	\tag*{transitivity}
	\\
	x\not\sim y \lor y\sim x
	\tag*{symmetry}
	\end{align*}
	The {\myem equivalence class} $[a]_\sim$ of an element $a\in A$ contains all elements $b\in A$ that are equivalent to $a$.
	Trivially $[a]_\sim = [b]_\sim$ if $a\sim b$.
	The quotient set of $A$ modulo $\sim$ is the set of all equivalent classes $A/_\sim$. 
\end{definition}