% !TeX root = ../mthesis.tex
% !TeX encoding = UTF-8
% !TeX spellcheck = en_US

%\begin{definition}
%%A Cartesian product is a mathematical operation that returns a set of tuples from an ordered list of sets.
%%\end{definition}
%%\begin{definition}
%A binary relation on $A$ is a subset of $A \times A$, the Cartesian product of $A$ and $A$, i.e.~all pairs of elements of $A$.
%\end{definition}

%For the following definitions we express the conditions as predicates clausal form, 
%i.e.~the variables are universally quantified. 


%\begin{definition}
%We label a binary relation $R$ on an arbitrary domain $A$ if for all $x,y,z\in A$ the following holds
%\begin{align*}
%R(x,x)
%\tag*{reflexive}
%\\
%\lnot R(x,x) 
%\tag*{irreflexive}
%\\
%\lnot R(x,y) \lor \lnot R(y,z) \lor R(x,z)
%\tag*{transitive}
%\\
%\lnot R(x,y) \lor R(y,x)
%\tag*{symmetric}
%\\
%\lnot R(x,y) \lor \lnot R(y,x)
%\tag*{asymmetric}
%\\
%\lnot R(x,y) \lor \lnot R(y,x) \lor x = y
%\tag*{antisymmetric}
%\end{align*}
%
%\noindent We recognize the difference between {\em not symmetric} and {\em asymmetric}.
%%The first expresses that at least one pair of elements is not symmetrically related. 
%%The second one expresses that all pairs of elements are not symmetrically related.
%	\begin{align*}
%	\lnot\left( \forall x \forall y\ \lnot R(x,y) \lor R(y,x) \right)
%	&\equiv
%	\exists x\exists y\ R(x,y) \land \lnot R(y,x) 
%	\tag*{not symmetric}
%	\\
%	&\not\equiv
%	\\
%	\lnot\left( \exists x \exists y\ R(x,y) \land R(y,x)\right)
%	&\equiv
%	\forall x\forall y\ \lnot R(x,y) \lor \lnot R(y,x)
%	\tag*{asymmetric}
%	\end{align*} 
%%\end{remark}
%\end{definition}


\begin{definition}
	A n-ary relation is a subset of the Cartesian product of $n$ sets.
%	$A^n = \{ (a_1, \ldots, a_n) \mid a_i \in A\text{ for }0\leq i \leq n \}$.
	A {\myem binary relation} $R$ on domain $A$ 
	is a subset of $A^2 = A \times A = \{ (a,b) \mid a\in A, b\in A \}$ -- 
	the set of all pairs of elements in $A$.
	We write $R(a,b)$ or $a\ R\ b$ to express that $(a,b)\in R \subseteq A\times A$.
\end{definition}
\begin{definition}
	An {\myem equivalence relation} 
	is a reflexive, transitive, and symmetric binary relation, 
	i.e.~$\sim$ is an equivalence relation on $A$ if the following three clauses
	\begin{align*}
	x\sim x
	\tag*{reflexivivity}
	\\
	x\not\sim y \lor  y \not\sim z \lor x\sim z
	\tag*{transitivity}
	\\
	x\not\sim y \lor y\sim x
	\tag*{symmetry}
	\end{align*}
	always hold for arbitrary elements $x,y,z$ in $A$.
	The {\myem equivalence class} $[a]_\sim$ of an element $a\in A$ contains all elements $b\in A$ that are equivalent to $a$.
	Trivially $[a]_\sim = [b]_\sim$ if $a\sim b$.
	The quotient set of $A$ modulo $\sim$ is the set of all equivalent classes $A/_\sim$. 
\end{definition}

\begin{lemma}
	The identity relation $=$ is an equivalence relation.
\end{lemma}

\begin{definition}
	A {\myem partial order} is a reflexive, transitive, and antisymmetric binary relation, 
	i.e.~$\sqsubseteq$ is a partial order on $A$ if
	the following three clauses
	\begin{align*}
	x\sqsubseteq x
	\tag*{reflexivivity}
	\\
	x\not\sqsubseteq y \lor  y \not\sqsubseteq z \lor x\sqsubseteq z
	\tag*{transitivity}
	\\
	x\not\sqsubseteq y \lor y\not\sqsubseteq x \lor x = y
	\tag*{antisymmetry}
	\end{align*}
	always hold for arbitrary elements $x,y,z$ in $A$.
\end{definition}

\begin{definition}
	A {\myem proper order} is a irreflexive and transitive binary relation, 
	i.e.~$\sqsubset$ is a proper order on $A$ if the following two clauses
	\begin{align*}
	x\not\sqsubset x
	\tag*{irreflexivivity}
	\\
	x\not\sqsubset y \lor  y \not\sqsubset z \lor x\sqsubset z
	\tag*{transitivity}
	\end{align*}
	always hold for arbitrary elements $x,y,z$ in $A$.
	A proper order is a {\myem total order} if additionally the following clause
	\[
		x \sqsubset y \lor y \sqsubset x  \lor x=y \tag*{totality}
	\]
	always holds for arbitrary $x$ and $y$ in $A$.
\end{definition}

\begin{example}
	By definition the empty relation $\emptyset \subseteq A \times A$ is a
	irreflexive,
	transitive,
	symmetric,
	antisymmetric,
	and asymmetric
	relation. 
	Hence it is a proper order, but not total.
\end{example}

\begin{example}
	Any proper order is asymmetric, e.g.~the clause
	\begin{align*}
	x\not\sqsubset y &\lor y\not\sqsubset x \tag*{asymmetry}
	\end{align*}
	always holds for arbitrary $x$ and $y$ in $A$.
\end{example}

\begin{example}The strict subset relation $\subsetneq$ over a power set is a proper order, but not a total, 
	e.g. for the power set over natural numbers we have 
	\[
	\{ 1 \} \not\subsetneq \{ 2 \}
	\land \{ 2 \} \not\subsetneq \{ 1 \}
	\land 	\{ 1 \} \neq \{ 2 \}
	\tag*{non-totality}
	\]
\end{example}



\begin{definition}
	An binary relation $\supset$ is {\myem well-founded} on a set $A$ if there is no infinite sequence 
	$(a_i)_{i\in\mathbb{N}}$ of $a_i\in A$
	with $a_i\supset a_{i+1}$ for all $i\in\mathbb{N}$.
\end{definition}

\begin{example}
	The strict superset relation $\supsetneq$ on final sets is well-founded.
	The canonical greater than relation $>$ is well founded on natural numbers,
	but not on integers or positive rational numbers.
	\begin{gather*}
	-1 > -2 > \ldots > -(2^i) > -(2^{i+1}) > \ldots\\
	1 > \frac{1}{2} > \frac{1}{4} > \ldots > \frac{1}{2^i} > \frac{1}{2^{i+1}} > \ldots
	\end{gather*}
\end{example}

\begin{definition}\label{def:prec}
	A {\myem precedence} $\gtpre$ is a proper order 
	on a signature.
\end{definition}

\begin{definition}[LPO]\label{def:lpo}
	Let $\gtpre$ be a precedence. In a {\myem lexicographic path order} $\gtpre_{lpo}$ on (general) terms the relation $s\gtpre_{lpo} t$ holds,
	if one of these three cases holds:
	\begin{enumerate}
		\item $s=\mcf(s_1,\ldots,s_n)$, $t=\mcf(t_1,\ldots,t_n)$ and for some $1\leq i\leq n$:
		\[
		(s_j=t_j\lor j\geq i) \land (s_i\gtpre_{lpo} t_i) \land (s\gtpre_{lpo} t_i \lor j\leq i)
		\]
		\item $s=\mcf(s_1,\ldots,s_n)$, $t=\mcg(t_1,\ldots,t_m)$, $\mcf\gtpre\mcg$, and $s\gtpre_{lpo} t_i$ for all $1\leq i\leq m$.
		\item $s=\mcf(s_1,\ldots,s_n)$ and $(s_i=t \lor s_i\gtpre_{lpo} t)$ for some $1\leq i\leq m$.
	\end{enumerate}
\end{definition}

\begin{definition}\label{def:weight}
	Let $\mcF$ be a signature.
	A {\myem weight function} is a pair $(w,w_0)$. 
	The first member $w$ is a function that maps every symbol $\mcf\in\mcF$ to a natural number $w(\mcf)$,
	the second is a constant $w_0>0$ such that $w(\mc)\geq w_0$ for every constant $c\in\mcF$. 
	The weight of a (general) term $t$ is defined as:
	\begin{gather*}
	w(t) = \left\{ \begin{array}{ll} 
	w_0 & \text{if } t\in\mcV\\
	w(\mcf)+\sum_{i=1}^n w(t_i) & \text{if }t=\mcf(t_1,\ldots,t_n)
	\end{array}\right.
	\end{gather*}
\end{definition}

\begin{definition}[KBO]\label{def:kbo}
	Let $\gtpre$ be a precedence and $(w,w_0)$ a weight function.
	In a Knuth-Bendix order $\gtpre_{kbo}$ on (general) terms the relation $s\gtpre_{kbo} t$ holds if
	$|s|_x\geq|t|_x$ for all $x\in\mcV$ and one of these two cases holds:
	\begin{enumerate}
		\item $w(s) > w(t)$
		\item $w(s) = w(t)$ and one of these three sub cases holds:
		\begin{enumerate}
			\item $t\in\mcV$ and $s=\mcf^n(t)$ for some unary symbol $\mcf$ and $n\gtpre0$,
			\item $s=\mcf(s_1,\ldots,s_n)$, $t=\mcf(t_1,\ldots,t_n)$, and for some $1\leq i\leq n$:
			\[
			(s_j=t_j \lor j\geq i) \land s_i\gtpre_{kbo} t_i
			\]
			\item $s=\mcf(s_1,\ldots,s_n)$, $t=\mcg(t_1,\ldots,t_n)$, and $\mcf\gtpre_{kbo}\mcg$.
		\end{enumerate}
	\end{enumerate}
\end{definition}

\begin{lemma}
	LPO and KBO are simplification orders on (general) terms.
\end{lemma}

\begin{lemma}
	\begin{align*}
	\bigwedge_{i=1}^{n} 
	\left(	
		\bigvee_{j=1}^{c_i}\,p_{i,j} 
	\right)
	\ &\equiv
	\bigvee_{(j_1,\ldots,j_n)}
	\left(
		\bigwedge_{i=1}^{n}\,a_{(i,j_i)}
	\right)
	&\text{with }(j_1,\ldots,j_n)\in\prod_{i=1}^{n}\{ 1,\ldots,c_i \}
	\end{align*}
\end{lemma}
\begin{proof}By induction on $n$
	\begin{itemize}
		\item (base) $n=1$.
		\begin{align*}
		\bigvee_{j=1}^{c_1} p_{i,j}\ &= \bigvee_{(j_1)} p_{(1,j_1)} 
		&
		(j_1) \in \{ 1,\ldots, c_1 \}
		\end{align*}
		
		\item (step) $n+1$
		
		\begin{align*}
		\bigwedge_{i=1}^{n+1} 
		\left(	
			\bigvee_{j=1}^{c_i}\,p_{i,j} 
		\right)
		\ \defEQ&\quad
		\left( 
			\bigwedge_{i=1}^{n} 
			\left(
				\bigvee_{j=1}^{c_i}\,p_{i,j}
			\right)
		\right)
%		
		\land 
		\left(\bigvee_{j=1}^{c_{n+1}} p_{n+1,j}\right)
		\\
		\defEV[I.H.]&\ 
		\left(
		\bigvee_{(j_1,\ldots,j_n)}
		\bigwedge_{i=1}^{n}\,a_{(i,j_i)}
		\right)
		\land 
		\left(\bigvee_{j=1}^{c_{n+1}} p_{n+1,j}\right)
%		&(j_1,\ldots,j_n)\in\prod_{i=1}^{n}\{ 1,\ldots,c_i \}
		\\
		\defEV[DIST]&\ 
		\bigvee_{j_{n+1}=1}^{c_{n+1}}
		\left(
		\bigvee_{(j_1,\ldots,j_n)}
		\left(\left(
		\bigwedge_{i=1}^{n}\,p_{(i,j_i)}
		\right)\right)
		\land p_{(n+1,j_{n+1})}
		\right)
		\end{align*}
		
		
	\end{itemize}
\end{proof}
