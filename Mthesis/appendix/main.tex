% !TeX root = ../mthesis.tex
% !TeX encoding = UTF-8
% !TeX spellcheck = en_US

\chapter{Additional Definitions}

{
	\setlength\epigraphwidth{.5\textwidth}
	\setlength\epigraphrule{0pt}



\epigraph{Eine Menge ist die Zusammenfassung bestimmter, wohlunterschiedener Objekte unserer Anschauung oder unseres Denkens -- welche Elemente der Menge genannt werden – zu einem Ganzen.\footnotemark}{--- G.~Cantor}
\footnotetext{
A set is a gathering together into a whole of definite, 
distinct objects of our perception or of our thought -- which are called elements of the set.	
}

}



In this chapter we recall some mathematical and logical notions we have used but not defined the main part of our thesis.
We assume at least a {\myem basic} understanding of sets as described by Georg Cantor in the 19th century. 
However we ignore all paradoxes in {\myem naive} set theory.
Instead we just assume that our sets are definable within a consistent set theory (e.g.~Zermalo-Fraenkel).
Thus we have well defined {\myem membership} relations between our sets and objects,
well defined {\myem subset} relations between our sets, 
and well defined {\myem Cartesian products} over sets.

In section \ref{sec:app:maths} we {\em finally} define {\myem relations} and {\myem orders}. 
In section \ref{sec:app:fol} we describe sets and numbers general calculated from first order terms.

\begin{figure}[hbt]\label{fig:conventions}
	\begin{align*}
	x,y,z &\ \in\mcV \tag*{variables} \\
	\ma,\mb,\mc,\md &\ \in\mcFfn[{0}]\tag*{constant symbols} \\
	\mf, \mg, \mh &\ \in\mcFfn[{n>0}]\tag*{function symbols} \\
	\mpp, \mq, \mr &\ \in\mcFPn[{0}]\tag*{propositional symbols}\\
	\mP, \mQ, \mR &\ \in\mcFPn[{n>0}]\tag*{predicate symbols}\\
	\mcf, \mcg, \mch &\ \in\mcF^{n>0} \tag*{predicate or functions symbols}\\
	\ell, r, s,t,u &\ \in\mcTf = \mcTFfV\tag*{(first order) terms}\\
	\mcs, \mct, \mcu &\ \in \mcPT \cup \mcET \cup \mcTf \tag*{predicates, equations, terms}\\
	\mu, \nu, \sigma, \tau &\ \tag*{substitutions, unifiers}\\
	A,B,C,D &\ \tag*{atoms} \\
	\mcL, L &\ \tag*{literals} \\
%	\mcC, \mcD &\ \tag*{clauses} \\
	S &\ \tag*{sets of clauses}
	\end{align*}
	\caption{Typographical conventions}
\end{figure}

\section{Relations and orders}\label{sec:app:maths}


\begin{definition}
A Cartesian product is a mathematical operation that returns a set of tuples from an ordered list of sets.
%\end{definition}
%\begin{definition}
A {\myem relation R} on a Cartesian product $A_1 \times \ldots \times A_n$ 
is a subset of this Cartesian product.

For the following definitions we express the conditions as predicates clausal form, 
i.e.~the variables are universally quantified. 

\end{definition}
A binary relation $R$ on domain $A$
\begin{align*}
R(x,x)
\tag*{reflexive}
\\
\lnot R(x,x) 
\tag*{irreflexive}
\\
\lnot R(x,y) \lor \lnot R(y,z) \lor R(x,z)
\tag*{transitive}
\\
\lnot R(x,y) \lor R(y,x)
\tag*{symmetric}
\\
\lnot R(x,y) \lor \lnot R(y,x)
\tag*{asymmetric}
\\
\lnot (x,y) \lor \lnot (y,x) \lor x = y
\tag*{antisymmetric}
\end{align*}

\begin{corollary}
	Not symmetric is different from asymmetric.
	\begin{align*}
	\lnot\left( \forall x \forall y\ (\lnot R(x,y) \lor R(y,x) \right)
	&\equiv
	\exists x\exists y\ R(x,y) \land \lnot R(y,x) \\
	&\not\equiv
	\forall x\forall y\ \lnot R(x,y) \lor \lnot R(y,x)
	\end{align*} 
\end{corollary}


\begin{definition}
	
\end{definition}

\section{First order logic}\label{sec:app:fol}

% !TeX root = ../mthesis.tex
% !TeX encoding = UTF-8
% !TeX spellcheck = en_US

\begin{definition}
	We define
	the set of function symbols
	\DEFINE{\fun(t)}{
		\emptyset & \text{if } t = x \in \mcV \\
		\{ \mf \} \cup \bigcup_{i=1}^n \fun(t_i) & \text{if }  t = \mf(t_1, \ldots t_n), \mf \in \mcFfn
	}
	and the set of subterms
		\DEFINE{\subterms(t)}{
			\{ t \} & \text{if } t \in \mcV \\
			\{ t \} \cup \bigcup_{i=1}^n \subterms(t_i) & \text{if }  t = f(t_1, \ldots t_n), f \in \mcFn
	}
of a first order term \( t \).

\end{definition}

\begin{definition}
	We define
	the set of subformulae
\DEFINE[=\{F\}\disjointunion]{\subforms(F)}{
	\emptyset &\text{if }F\text{ is an atomic formula}
	\\
	\subforms(G)&\text{if }F = \lnot G
	\\
	\subforms(G) \cup \subforms(H)&\text{if }F = G\land H, G\lor H\text{ or } G\limp H
	\\
	\subforms (G)  &\text{if } F = \forall x G \text{ or } \exists x G
}
of a first order formula \( F \).
\end{definition}










