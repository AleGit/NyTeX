% !TeX root = ../mthesis.tex
% !TeX encoding = UTF-8
% !TeX spellcheck = en_US


\begin{definition}
	A n-ary relation is a subset of the Cartesian product of $n$ sets.
%	$A^n = \{ (a_1, \ldots, a_n) \mid a_i \in A\text{ for }0\leq i \leq n \}$.
	A {\myem binary relation} $R$ on domain $A$ 
	is a subset of $A \times A := \{ (a,b) \mid a\in A, b\in A \}$ -- 
	the set of all pairs of elements in $A$.
	We write $R(a,b)$ or $a\ R\ b$ to express that $(a,b)\in R \subseteq A\times A$.
\end{definition}

\begin{example}
	Pythagorean triples define a relation on natural numbers. 
	\[(3,4,5)\in\{ 
		(a,b,c) \mid 
		a\in\mathbb{N}\land 
		b\in\mathbb{N}\land 
		c\in\mathbb{N}\land
		a^2 + b^2 = c^3
	\}
	\subsetneq \mathbb{N}^3
	\] 
\end{example}

\begin{definition}
	An {\myem equivalence relation} 
	is a reflexive, transitive, and symmetric binary relation, 
	i.e.~$\sim$ is an equivalence relation if the following three clauses hold.
	\begin{align*}
	x\sim x
	\tag*{reflexivivity}
	\\
	x\not\sim y \lor  y \not\sim z \lor x\sim z
	\tag*{transitivity}
	\\
	x\not\sim y \lor y\sim x
	\tag*{symmetry}
	\end{align*}
	The {\myem equivalence class} $[x]_\sim$
	of an element $x$ is a subset that contains all elements that are equivalent to $x$.
	Trivially the equivalence $x\sim y$ implies the identity $[x]_\sim = [y]_\sim$.
	The quotient set modulo an equivalence relation is the set of a set's equivalent classes.
	\begin{align*} 
	[x]_\sim &:= \{\, y \mid x \sim y\} &
	A/_{\!\sim} &:= \{\, [x]_\sim \mid x \in A \,\} 
	\end{align*} 
\end{definition}

\begin{lemma}
	The identity relation $=$ is an equivalence relation.
\end{lemma}

