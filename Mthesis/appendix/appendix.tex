% !TeX root = ../mthesis.tex
% !TeX encoding = UTF-8
% !TeX spellcheck = en_US


\chapter{Appendix}

{
	\setlength\epigraphwidth{.5\textwidth}
	\setlength\epigraphrule{0pt}



\epigraph{Eine Menge, ist die Zusammenfassung bestimmter, wohlunterschiedener Objekte unserer Anschauung oder unseres Denkens -- welche Elemente der Menge genannt werden – zu einem Ganzen.\footnotemark}{--- G.~Cantor}
\footnotetext{
A set is a gathering together into a whole of definite, 
distinct objects of our perception or of our thought -- which are called elements of the set.	
}

}

In this appendix we recall some mathematical and logical concepts and notions 
we have used but not defined in the main part.
We assume at least a {\myem basic} understanding of sets as already 
described by Georg Cantor in the 19th century. 
However we just assume that our sets are definable within a consistent set theory.
Thus we have well defined {\myem membership} relationships between objects and sets,
well defined {\myem subset} relationships between sets, 
and well defined {\myem Cartesian products} over sets.

?? In the presentations of the equality axioms we have chosen the clausal normal form 
from Definition \vref{def:syntax:CNF} 
with the semantics of Definition \vref{def:model}.

\section{Mathematics}\label{sec:app:mathematics}

\subsection{Relations}

% !TeX root = ../mthesis.tex
% !TeX encoding = UTF-8
% !TeX spellcheck = en_US


\begin{definition}
	A n-ary relation is a subset of the Cartesian product of $n$ sets.
%	$A^n = \{ (a_1, \ldots, a_n) \mid a_i \in A\text{ for }0\leq i \leq n \}$.
	A {\myem binary relation} $R$ on domain $A$ 
	is a subset of $A \times A := \{ (a,b) \mid a\in A, b\in A \}$ -- 
	the set of all pairs of elements in $A$.
	We write $R(a,b)$ or $a\ R\ b$ to express that $(a,b)\in R \subseteq A\times A$.
\end{definition}

\begin{example}
	Pythagorean triples define a relation on natural numbers. 
	\[(3,4,5)\in\{ 
		(a,b,c) \mid 
		a\in\mathbb{N}\land 
		b\in\mathbb{N}\land 
		c\in\mathbb{N}\land
		a^2 + b^2 = c^3
	\}
	\subsetneq \mathbb{N}^3
	\] 
\end{example}

\begin{definition}
	An {\myem equivalence relation} 
	is a reflexive, transitive, and symmetric binary relation, 
	i.e.~$\sim$ is an equivalence relation if the following three clauses hold.
	\begin{align*}
	x\sim x
	\tag*{reflexivivity}
	\\
	x\not\sim y \lor  y \not\sim z \lor x\sim z
	\tag*{transitivity}
	\\
	x\not\sim y \lor y\sim x
	\tag*{symmetry}
	\end{align*}
	The {\myem equivalence class} $[x]_\sim$
	of an element $x$ is a subset that contains all elements that are equivalent to $x$.
	Trivially the equivalence $x\sim y$ implies the identity $[x]_\sim = [y]_\sim$.
	The quotient set modulo an equivalence relation is the set of a set's equivalent classes.
	\begin{align*} 
	[x]_\sim &:= \{\, y \mid x \sim y\} &
	A/_{\!\sim} &:= \{\, [x]_\sim \mid x \in A \,\} 
	\end{align*} 
\end{definition}

\begin{lemma}
	The identity relation $=$ is an equivalence relation.
\end{lemma}



\subsection{Orders}\label{sec:app:orders}

% !TeX root = ../mthesis.tex
% !TeX encoding = UTF-8
% !TeX spellcheck = en_US


\begin{definition}
	A {\myem partial order} is a reflexive, transitive, and antisymmetric binary relation, 
	i.e.~$\sqsubseteq$ is a partial order on $A$ if
	the following three clauses hold for arbitrary elements $x,y,z$ in $A$.
	\begin{align*}
	x\sqsubseteq x
	\tag*{reflexivivity}
	\\
	x\not\sqsubseteq y \lor  y \not\sqsubseteq z \lor x\sqsubseteq z
	\tag*{transitivity}
	\\
	x\not\sqsubseteq y \lor y\not\sqsubseteq x \lor x = y
	\tag*{antisymmetry}
	\end{align*}
\end{definition}

\begin{definition}
	A {\myem proper order} is a irreflexive and transitive binary relation, 
	i.e.~$\sqsubset$ is a proper order on $A$ if the following two clauses hold.
	\begin{align*}
	x\not\sqsubset x
	\tag*{irreflexivivity}
	\\
	x\not\sqsubset y \lor  y \not\sqsubset z \lor x\sqsubset z
	\tag*{transitivity}
	\end{align*}
	always hold for arbitrary elements $x,y,z$ in $A$.
\end{definition}

\begin{definition}
	A {\myem total order} is a proper order where the following clause holds.
	\[
		x \sqsubset y \lor y \sqsubset x  \lor x=y \tag*{totality}
	\]
\end{definition}

\begin{example}
	By definition the empty relation $\emptyset \subseteq A \times A$ is a
	irreflexive,
	transitive,
	symmetric,
	antisymmetric,
	and asymmetric
	relation. 
	Hence it is a proper order, but not total.
\end{example}

\begin{lemma}
	Any proper order is asymmetric, e.g.~the clause
	\begin{align*}
	x\not\sqsubset y &\lor y\not\sqsubset x \tag*{asymmetry}
	\end{align*}
	always holds for arbitrary $x$ and $y$ in $A$.
\end{lemma}

\begin{proof} We use resolution to derive asymmetry from irreflectivity and transitivity.
	\[
			\infer[\{x'\mapsto x, z\mapsto x \}]{
			x\not\sqsubset y \lor y\not\sqsubset x }{
			{\colHi x\not\sqsubset x} & x'\not\sqsubset y \lor  y \not\sqsubset z \lor {\colLo x'\sqsubset z}
		}
	\]
\end{proof}

\begin{example}The strict subset relation $\subsetneq$ over a power set is a proper order, but not a total, 
	e.g. for the power set over natural numbers we have 
	\[
	\{ 1 \} \not\subsetneq \{ 2 \}
	\land \{ 2 \} \not\subsetneq \{ 1 \}
	\land 	\{ 1 \} \neq \{ 2 \}
	\tag*{non-totality}
	\]
\end{example}



\begin{definition}
	A binary relation $\supset$ is {\myem well-founded} on a set $A$ if there is no infinite sequence 
	$(a_i)_{i\in\mathbb{N}}$ of $a_i\in A$
	with $a_i\supset a_{i+1}$ for all $i\in\mathbb{N}$.
\end{definition}

\begin{example}
	The strict superset relation $\supsetneq$ on finite sets is well-founded.
	The canonical greater than relation $>$ is well founded on natural numbers,
	but not on integers or positive rational numbers.
	\begin{gather*}
	-1 > -2 > \ldots > -(2^i) > -(2^{i+1}) > \ldots\\
	1 > \frac{1}{2} > \frac{1}{4} > \ldots > \frac{1}{2^i} > \frac{1}{2^{i+1}} > \ldots
	\end{gather*}
\end{example}




\subsection{Term Rewriting}

% !TeX root = ../mthesis.tex
% !TeX encoding = UTF-8
% !TeX spellcheck = en_US

\begin{definition}
	A binary relation $R$ on terms has the {\myem subterm property} if 
	$s[t]_p\mathbin{R}t$ for all terms $s,t$ and all positions $p\in\pos(s) \setminus \{\epsilon\}$.
\end{definition}

\begin{definition}\label{def:prec}
	A {\myem precedence} $\succ$ is a proper order 
	on a signature.
\end{definition}

\begin{definition}[LPO]\label{def:lpo}
	Let $\succ$ be a precedence. In a {\myem lexicographic path order} $\succ_{lpo}$ on (general) terms the relation $s\succ_{lpo} t$ holds,
	if one of these three cases holds:
	\begin{enumerate}
		\item $s=\mcf(s_1,\ldots,s_n)$, $t=\mcf(t_1,\ldots,t_n)$ and for some $1\leq i\leq n$:
		\[
		(s_j=t_j\lor j\geq i) \land (s_i\succ_{lpo} t_i) \land (s\succ_{lpo} t_i \lor j\leq i)
		\]
		\item $s=\mcf(s_1,\ldots,s_n)$, $t=\mcg(t_1,\ldots,t_m)$, $\mcf\succ_\mcg$, and $s\succ_{lpo} t_i$ for all $1\leq i\leq m$.
		\item $s=\mcf(s_1,\ldots,s_n)$ and $(s_i=t \lor s_i\succ_{lpo} t)$ for some $1\leq i\leq m$.
	\end{enumerate}
\end{definition}

\begin{definition}\label{def:weight}
	Let $\mcF$ be a signature.
	A {\myem weight function} is a pair $(w,w_0)$. 
	The first member $w$ is a function that maps every symbol $\mcf\in\mcF$ to a natural number $w(\mcf)$,
	the second is a constant $w_0>0$ such that $w(\mc)\geq w_0$ for every constant $c\in\mcF$. 
	The weight of a (general) term $t$ is defined as:
	\begin{gather*}
	w(t) = \left\{ \begin{array}{ll} 
	w_0 & \text{if } t\in\mcV\\
	w(\mcf)+\sum_{i=1}^n w(t_i) & \text{if }t=\mcf(t_1,\ldots,t_n)
	\end{array}\right.
	\end{gather*}
\end{definition}

\begin{definition}[KBO]\label{def:kbo}
	Let $\succ$ be a precedence and $(w,w_0)$ a weight function.
	In a Knuth-Bendix order $\succ_{kbo}$ on (general) terms the relation $s\succ_{kbo} t$ holds if
	$|s|_x\geq|t|_x$ for all $x\in\mcV$ and one of these two cases holds:
	\begin{enumerate}
		\item $w(s) > w(t)$
		\item $w(s) = w(t)$ and one of these three sub cases holds:
		\begin{enumerate}
			\item $t\in\mcV$ and $s=\mcf^n(t)$ for some unary symbol $\mcf$ and $n\succ0$,
			\item $s=\mcf(s_1,\ldots,s_n)$, $t=\mcf(t_1,\ldots,t_n)$, and for some $1\leq i\leq n$:
			\[
			(s_j=t_j \lor j\geq i) \land s_i\succ_{kbo} t_i
			\]
			\item $s=\mcf(s_1,\ldots,s_n)$, $t=\mcg(t_1,\ldots,t_m)$, and $\mcf\succ\mcg$.
		\end{enumerate}
	\end{enumerate}
\end{definition}

\begin{lemma}
	LPO and KBO are simplification orders on (general) terms.
\end{lemma}

\begin{lemma}
	\begin{align*}
	\bigwedge_{i=1}^{n} 
	\left(	
		\bigvee_{j=1}^{c_i}\,p_{i,j} 
	\right)
	\ &\equiv
	\bigvee_{(j_1,\ldots,j_n)}
	\left(
		\bigwedge_{i=1}^{n}\,a_{(i,j_i)}
	\right)
	&\text{with }(j_1,\ldots,j_n)\in\prod_{i=1}^{n}\{ 1,\ldots,c_i \}
	\end{align*}
\end{lemma}
\begin{proof}By induction on $n$
	\begin{itemize}
		\item (base) $n=1$.
		\begin{align*}
		\bigvee_{j=1}^{c_1} p_{i,j}\ &= \bigvee_{(j_1)} p_{(1,j_1)} 
		&
		(j_1) \in \{ 1,\ldots, c_1 \}
		\end{align*}
		
		\item (step) $n+1$
		
		\begin{align*}
		\bigwedge_{i=1}^{n+1} 
		\left(	
			\bigvee_{j=1}^{c_i}\,p_{i,j} 
		\right)
		\ \defEQ&\quad
		\left( 
			\bigwedge_{i=1}^{n} 
			\left(
				\bigvee_{j=1}^{c_i}\,p_{i,j}
			\right)
		\right)
%		
		\land 
		\left(\bigvee_{j=1}^{c_{n+1}} p_{n+1,j}\right)
		\\
		\defEV[I.H.]&\ 
		\left(
		\bigvee_{(j_1,\ldots,j_n)}
		\bigwedge_{i=1}^{n}\,a_{(i,j_i)}
		\right)
		\land 
		\left(\bigvee_{j=1}^{c_{n+1}} p_{n+1,j}\right)
%		&(j_1,\ldots,j_n)\in\prod_{i=1}^{n}\{ 1,\ldots,c_i \}
		\\
		\defEV[DIST]&\ 
		\bigvee_{j_{n+1}=1}^{c_{n+1}}
		\left(
		\bigvee_{(j_1,\ldots,j_n)}
		\left(\left(
		\bigwedge_{i=1}^{n}\,p_{(i,j_i)}
		\right)\right)
		\land p_{(n+1,j_{n+1})}
		\right)
		\end{align*}
		
		
	\end{itemize}
\end{proof}

\subsection{First Order Formulae}

In Section \ref{sec:syntax} we have defined atomic formulae in Definition \ref{def:atoms}, 
but we can only build formulae in clausal normal form (\CNF) with Definition \vref{def:syntax:CNF}.
Now we will define arbitrary first order formulae (\FOF).

\begin{definition}[\FOF]\label{def:syntax:FOF}
	Predicates and equations are (atomic) first order formulae. 
	The negation $(\lnot F)$, 
	the universal quantification $(\forall x F)$, 
	and the existential quantification $(\exists x F)$ 
	of a given formula $F$ are (composite) first order formulae.
	Further, the disjunction $(F \lor F')$, 
	the conjunction $(F \land F') $, 
	and the implication $(F \limp F') $ 
	of two given formulae $F$ and $F'$ 
	are (composite) first order formulae.
\end{definition}

We've already defined when an atom holds for an assignment $\alpha_\mcI$ 
in an interpretation $\mcI$ within Definition \vref{def:model}.
Now we extend these definitions to arbitrary formulae.

\begin{definition}[Semantics of \FOF]\label{def:semantics:FOF}
	
	A universally quantified formula $\forall x F$ holds in $\mcI$ if its subformula $F$ holds for all assignments for $x$.
	An existential quantified formula $\exists xF$ holds if its subformula $F$ holds for at least one assignment for $x$.
	A negation $\lnot F$ holds if its subformula $F$ does not hold, 
	a disjunction $F\lor F'$ holds if one or both of its subformulae $F$ or $F'$ hold,
	a conjunction $F\land F'$ holds, if both of its subformualae $F$ and $F'$ hold, 
	an implication $F\limp F'$ holds if its first subformula $F$ does not hold or its second subformula $F'$ holds (or both).
\end{definition}

\begin{remark}Usually we us precedences on connectives to omit parentheses 
	and some heuristics to structure the formulae for readability 
	without introducing semantic ambiguity.
%
	Beside the obvious semantically indistinguishable formulae with double negations, conjunctions, and disjunctions 
	we have introduced new ones.
	\begin{enumerate}
		\item $\forall x F$, $\exists x F$, and $F$ are indistinguishable if $x\not\in\var(F)$. 
		We usually omit quantifiers with variables that do not occur in subformulae.
		\item In general $\exists x F$ is different from $F$ if $x\in\var(F)$, e.g. $\exists x(x\mNE\ma)$ is satisfiable and $x\mNE\ma$ isn't.
		\item $\forall x F$ and $F$ are equivalent even if $x\in\var(F)$, 
		because in both cases we demand that $F$ holds in all assignments in our model.
		Usually we keep these universal quantifiers in \FOF.
		
		A first order formulae without quantifiers is in {\myem clausal form}, 
		but not necessarily in \CNF, e.g.~a weakened version of symmetry $(x\mEQ \ma)\limp (\ma\mEQ x)$ 
		is equisatisfiable to $\forall x ((x\mEQ \ma)\limp (\ma\mEQ x))$ 
		or $\exists a (\forall x ((x\mEQ a)\limp (a\mEQ x))$. 
	\end{enumerate}

\end{remark}

\subsection{Properties of first order terms and formulae}

% !TeX root = ../mthesis.tex
% !TeX encoding = UTF-8
% !TeX spellcheck = en_US

\begin{definition}We define the set of variables
	
	\begin{align*}
		\var(t) &= \left\{\begin{array}{ll}
			\{ t \} & \text{if } t \in \mcV \\
			\bigcup_{i=1}^n \var(t_i) & \text{if }  t = f(t_1, \ldots t_n), f \in \mcFn
%			\emptyset &\text{if } \mkt \in \mcFO
		\end{array}\right.	
	\end{align*}
	the set of subterms
	\begin{align*}
	\subterms(t) &= \left\{\begin{array}{ll}
	\{ t \} & \text{if } t \in \mcV \\
	\{ t \} \cup \bigcup_{i=1}^n \subterms(t_i) & \text{if }  t = f(t_1, \ldots t_n), f \in \mcFn
	%			\emptyset &\text{if } \mkt \in \mcFO
	\end{array}\right.	
	\end{align*}
	of a term $t$.
	
\end{definition}


%\begin{definiton}
	\begin{align*}
	\forall x\ &\ x = x \\
	\forall x,y\ &\ x = y \Leftrightarrow y = x \\
	\forall x,y,z\ &\ x = y \land y = z \Rightarrow x = y \\
	\forall x_1,\ldots,x_n,y_1,\ldots,y_n\ &\ x_1=y_1\land\ldots x_n=y_n\Rightarrow f(x_1,\ldots,x_n)=f(y_1,\ldots,y_n) \\
	\forall x_1,\ldots,x_n,y_1,\ldots,y_n\ &\ x_1=y_1\land\ldots x_n=y_n\Rightarrow P(x_1,\ldots,x_n)\Leftrightarrow P(y_1,\ldots,y_n) \\
	\end{align*}
%\end{definition}



\subsection{Normal Forms}

% !TeX root = ../mthesis.tex
% !TeX encoding = UTF-8
% !TeX spellcheck = en_US

\noindent We can assume that each variable of a formula is quantified exactly once. 
We can achieve this by carefully renaming variables to unused variable names.

\begin{example}We replace formulae at the left hand side with equivalent formulae.
	\begin{align*}
	\forall x (\boxed{\mP(x) \land \forall {\colLo x} \mQ({\colLo x}, {\colG y})}) 
	&\equiv 
	\forall x (\mP(x) \land \forall {\colHi z}{\colN \forall y} \mQ({\colHi z}, {\colN y}))
	\\
	\boxed{\forall x \mP(x)} \land \boxed{\forall {\colLo x} \mQ({\colLo x}, {\colG y})} 
	&\equiv 
	\forall x \mP(x) \land \forall {\colHi z}{\colN \forall y} \mQ({\colHi z}, {\colN y}) 
	)\\
	\end{align*}
\end{example}

\begin{definition}\label{def:syntax:PNF}
	A first order formula $F = Q_1 x_1 \ldots Q_n x_n\, F'$ 
	with quantifiers $Q_i\in\{\forall,\exists\}$, 
	quantifier-free subformula $F'$ with $\var(F') = \{ x_1,\ldots,x_n \}$
	is in {\myem prenex normal form}.
\end{definition}

\begin{definition}\label{def:syntax:PNF}
	{\myem negational prenex normal form}.
\end{definition}

\begin{definition}\label{def:syntax:PNF}
	{\myem conjunctive prenex normal form}.
\end{definition}

\begin{lemma}
	Any first order formula can be transformed 
	into an equivalent prenex normal form 
	by exhaustively using the following equivalences from left to right.
	\begin{align*}
	\lnot(\forall x F) &\equiv \exists x (\lnot F)
	&
	\lnot(\exists x F) &\equiv \forall x (\lnot F)
	\\
	{\colG G \land (\forall x F) \equiv{}} (\forall x F) \land G &\equiv \forall x (F \land G)  
	&
	{\colG G \land (\exists x F) \equiv{}} (\exists x F) \land G &\equiv \exists x (F \land G)  
	\\
	{\colG G \lor (\forall x F) \equiv{}} (\forall x F) \lor G &\equiv \forall x (F \lor G) 
	&
	{\colG G \lor (\exists x F) \equiv{}} (\exists x F) \lor G &\equiv \exists x (F \lor G)
	\end{align*}
	\begin{remark}
		The equivalences depend on $y\not\in\var(F)$ and $x\not\in\var(G)$.
	\end{remark}
\end{lemma}

\begin{lemma}We just state the following useful equivalences.
	\begin{align*}
		(\forall x F) \land (\forall y G) &\equiv \forall x (F\land G\{y\mapsto x\})
		&
		(\exists x F) \lor (\exists y G) &\equiv \exists x (F\lor G\{y\mapsto x\})
		\\
		(\forall x F) \lor (\forall y G) &\equiv \forall x\forall y (F\lor G)
		&
		(\exists x F) \land (\exists y G) &\equiv \exists x\exists y (F\land G)
		\\
		\forall x \forall y F &\equiv \forall y \forall x F
		&
		\exists x \exists y F &\equiv \exists y \exists x F
		\\
		\forall x G &\equiv G & \exists x G &\equiv G
		\\
		(\forall x F) \land (\exists y G) &\equiv \exists y \forall x (F\land G)
		&
		(\forall x F) \lor (\exists y G) &\equiv \exists y \forall x (F\lor G)
	\end{align*}
\end{lemma}

\begin{lemma}Just in case, you are wondering.
	\begin{align*}
	(\forall x F) \lor (\forall y G) &\not\equiv \forall x (F\lor G\{y\mapsto x\})
	&
	(\exists x F) \land (\exists y G) &\not\equiv \exists x (F\land G\{y\mapsto x\})
	\end{align*}
\end{lemma}

\subsection{Fragments of first order logic}

% !TeX root = ../mthesis.tex
% !TeX encoding = UTF-8
% !TeX spellcheck = en_US

%\ref{tab:decidedable:FiniteModelProperty} and
%\ref{tab:decidable:InfinityAxioms}
%\ref{tab:undecidable:PurePredicateLogic}, and
%\ref{tab:undecidable:FunctionsAndEquations}
This subsection presents fragments of first order logic where satisfiability is decidable.\footnote{
	Definition and compact overview from presentation
	“\href{http://logic.rwth-aachen.de/~graedel/kalmar.pdf}{Decidable fragments of first-order and fixed-point logic}”
	by E.~Grädel (\url{http://logic.rwth-aachen.de/~graedel/}).	
}  


\begin{definition}\label{def:prefix:class}
	We describe classes of first order formulae in \PNF with triples
	\[
		[\, \Pi, (p_1,p_2,\ldots), (f_1,f_2,\ldots)\,]_{(=)}
	\]
	where $\Pi$ describes the quantifier prefix,
	$p_i$ the maximal number of predicates symbols of arity $i$,
	$f_i$ the maximal number of function symbols of arity $i$.
	The equality symbol is not counted as binary predicate symbol.
	Instead the subscript $=$ indicates its presence.
	
\end{definition}

\begin{example}
	The monadic predicate calculus includes formulae with arbitrary quantifier prefixes, 
	arbitrary many unary predicate symbols, the equality symbol, but no function symbols.
	\begin{align*}
	\colG [\,all, (\omega), (1)\,]_= 
	\quad\supsetneq& &
		[\,all, (\omega), (0 )&\,]_=
		 \tag{Löwenheim 1925, Kalmár 1929}
	\\
	\colG [\,\exists^*\forall\exists^*,all,(1)\,]_=
	\quad\supsetneq& &
	[\,\exists^*\forall\exists^*,all,(0)&\,]_{=} \tag{Ackermann 1928}
	\end{align*}
	The Ackermann prefix class contains formulae with arbitrary many existential quantifiers, 
	but just one universal quantifier. It contains arbitrary many predicate symbols
	with arbitrary arities, the equality symbol, but no function symbols. 
\end{example}

%Goldfarb, Gurevich, Rabin, Shelah completly characterized decidable and undecidable prefix classes.

\begin{table}[hbt]
\begin{align*}
%\colG [\,all, (\omega), (0 )&\colG\,]_= \tag{Löwenheim 1925, Kalmár 1929}
%\\
[\,\exists^{∗}\forall^{∗}, all, (0)&\,]_{=} \tag{Bernays, Schönfinkel 1928, Ramsey 1932}
\\
[\,\exists^{∗}\forall^2\exists^{∗} , all, (0)&\,] \tag{Gödel 1932, Kalmár 1933, Schütte 1934}
\\
[\,all, (\omega), (\omega)&\,] \tag{Löb 1967, Gurevich 1969}
\\
[\,\exists^{∗}\forall\exists^{∗}, all, all&\,] \tag{Gurevich 1973}
\\
[\,\exists^{∗}, all, all&\,]_{=} \tag{Gurevich 1976}
\end{align*}
%\caption[Decidable prefix classes]{Decidable prefix classes in first order logic}
\caption[Decidable prefix classes]{Decidable prefix classes with final model property }
\label{tab:decidedable:FiniteModelProperty}
\end{table}

\begin{table}[hbt]
	\begin{align*}
	[\,all, (\omega), (1)&\,]_{=} \tag{Rabin 1969} 
	\\
	[\,\exists^{∗}\forall\exists^{∗}, all, (1)&\,]_{=} \tag{Shelah 1977}
	\end{align*}
	%\caption[Decidable prefix classes]{Decidable prefix classes in first order logic}
	\caption[Decidable prefix classes]{Decidable prefix classes with infinity axioms. }
	\label{tab:decidable:InfinityAxioms}
\end{table}

Now we look how formulae from these decidable fragments can be transformed nicely into equisatisfiable set of clauses.
The transformation is easy for formulae where the quantifiers follow a specific pattern where the appearances of the universal quantifier is limited. First we remove leading existential quantifiers and replace their variables with Skolem constants.
Then we remove the trailing existential quantifiers and replace there variables with Skolem functions where the variables of the universal quantifiers are the arguments. After that we remove the remaining universal quantifiers. We transform the remaining formula with constants, functions and a restricted number of free variables into (an equisatisfiable) conjunctive normal form with a suitable procedure. 

\begin{definition}\label{def:prefix:class}
	We describe classes of first order formulae in clausal normal form with triples
	\[
	_{_\CNF}[\, \pi, (p_1,p_2,\ldots), (f_1,f_2,\ldots)\,]_{(=)}
	\]
	where $\pi$ is the maximal number of distinct variables (with multiple occurrences),
	$p_i$ the maximal number of predicates symbols of arity $i$,
	$f_i$ the maximal number of function symbols of arity $i$.
	The equality symbol is not counted as binary predicate symbol.
	Instead the subscript $=$ indicates its presence.
\end{definition}

\begin{table}[hbt]
\begin{align}
\PNF& & \CNF& \tag*{}\\
[\,\exists^{∗}\forall^{∗}, all, (0)&\,]_{=} 
& _{_\CNF}[\,\omega,all, (0)&\,]_{=}
\\
[\,\exists^{∗}\forall^2\exists^{∗} , all, (0)&\,] 
& _{_\CNF}[\,2, all, (0,\omega)&\,]
\\
[\,all, (\omega), (\omega)&\,]
& _{_\CNF}[\,\omega, (\omega), (\omega)&\,]
\\
[\,\exists^{∗}\forall\exists^{∗}, all, all&\,]
& _{_\CNF}[\,1,all, all&\,]
\\
[\,\exists^{∗}, all, all&\,]_{=}
& _{_\CNF}[\, 0, all, all&\,]_{=}
\\
[\,all, (\omega), (1)&\,]_{=} 
& _{_\CNF}[\,\omega, (\omega), (1)&\,]_{=} 
\\
[\,\exists^{∗}\forall\exists^{∗}, all, (1)&\,]_{=}
& _{_\CNF}[\,1, all, (1)&\,]_{=}
\end{align}
\caption[Transformation]{Transformation into equisatisfiable clausal forms}
\label{tab:decidedable:CNF}
\end{table}


\begin{proof}We use induction on the structure of $F$.
	\begin{enumerate}
		\item (base) A literal,
		i.e.a (negated) predicate with one argument term $t$.
		We abbreviate the literal with $L(z)$ if $\var(t) = \{ z \}$. 
		\begin{gather*}
			\forall x \exists y L(x)  \equiv \forall x L(x) \equiv {\colG \exists y} \forall x L(x)
			\qquad\qquad
			\forall x \exists y L(y)  \equiv \exists y L(y) \equiv \exists y{\colG \forall x}  L(y)
		\end{gather*}
		\item (step)A negation.
		\begin{gather*}
		\forall x \exists y\, (\lnot G) 
		\equiv \lnot (\exists x \forall y\, G)
		\defEV[I.H.] \lnot (\forall y\exists x \, G) 
		\equiv \exists y \forall x\, (\lnot G)
		\end{gather*}
		\item (step) A conjunction 
		of a literal with the universal quantified variable and a formula.
		\begin{align*}
		\forall x \exists y\,({L(x)\land H}) 
		&\equiv \forall x\,({L(x)} \land \exists y\,{H})
		\equiv \forall x\,L(x)\land \forall x\exists y\,H
		\\&\defEV[IH] \forall x\,{L(x)} \land \exists y\forall x\,{H}
%		\equiv \exists y\forall x\,({L(x)\land H})
		\equiv\exists y \forall x\,({L(x)\land H})
		\end{align*}
		\item A disjunction of a literal with the existential quantified variable and a formula.
		\begin{align*}
		\forall y \exists x\,({L(x)\lor H}) 
		&\equiv\lnot(\exists y \forall x (L^c(x)\land\lnot H)))\\
		&\defEV[3.]\lnot(\forall x\exists y  (L^c(x)\land\lnot H))) 
		\equiv \exists x\forall y\, (L(x) \land H)
		\end{align*}
		\item A conjunction of a literal with the existential quantified variable and a formula.
		\begin{align*}
		\forall x \exists y\,({L(y)\land H}) 
		&\equiv\lnot(\exists x\forall y \,({L^c(y) \lor \lnot H})) \\
		&\defEV[4.]\lnot(\forall x\exists y \,({L^c(y) \lor \lnot H}) )
		\equiv \exists x\forall y\,(L(y)\land H)
		\end{align*}
		\item A disjunction of a literal with universal quantified variable and a formula.
		\begin{align*}
		\forall x \exists y\,(L(x)\lor H) 
		&\equiv \lnot (\exists x\forall y\,(L^c(x) \land H)) \\
		&\defEV[4.]\lnot (\forall y\exists x\,(L^c(x) \land\lnot H)) 
		\equiv \exists y \exists x\,(L(x)\lor H))\\
		\end{align*}
	\end{enumerate}
\end{proof}


%%% ADDITIONAL MATERIAL %%%%

%\begin{table}[hbt]
%	\begin{align*}
%	[\,\forall\exists\forall, (\omega, 1), (0)&\,] \tag{Kahr 1962}
%	\\
%	[\,\forall^3 \exists, (\omega, 1), (0)&\,] \tag{Surányi 1959}
%	\\
%	[\,\forall^{∗} \exists, (0, 1), (0)&\,] \tag{Kalmár-Surányi 1950}
%	\\
%	[\,\forall\exists\forall^{∗} , (0, 1), (0)&\,]  \tag{Denton 1960}
%	\\
%	[\,\forall\exists\forall\exists^{∗}, (0, 1), (0)&\,] \tag{Gurevich 1966}
%	\\
%	[\,\forall^3 \exists^{∗} , (0, 1), (0)&\,] \tag{Kalmár-Surányi 1947}
%	\\
%	[\,\forall\exists^{∗} \forall, (0, 1), (0)&\,] \tag{Kostyrko-Genenz 1964}
%	\\
%	[\,\exists^{∗} \forall\exists\forall, (0, 1), (0)&\,] \tag{Surányi 1959}
%	\\
%	[\,\exists^{∗} \forall^3 \exists, (0, 1), (0)&\,] \tag{Surányi 1959}
%	\end{align*}
%	\caption{Undecidable prefix classes in pure predicate logic}
%	\label{tab:undecidable:PurePredicateLogic}
%\end{table}

%\begin{table}[hbt]
%	\begin{align*}
%	[\,\forall, (0), (2)&\,]_{=} \tag{Gurevich 1976}
%	\\
%	[\,\forall, (0), (0, 1)&\,]_{=} \tag{Gurevich 1976}
%	\\
%	[\,\forall^2 , (0, 1), (1)&\,] \tag{Gurevich 1969}
%	\\
%	[\,\forall^2 , (1), (0, 1)&\,] \tag{Gurevich 1969}
%	\\
%	[\,\forall^2\exists, (\omega, 1), (0)&\,]_{=} \tag{Goldfarb 1984}
%	\\
%	[\,\exists^{∗}\forall^2\exists, (0, 1), (0)&\,]_{=} \tag{Goldfarb 1984}
%	\\
%	[\,\forall^2\exists^{∗}, (0, 1), (0)&\,]_{=} \tag{Goldfarb 1984}
%	\end{align*}
%	\caption{Undecidable prefix classes with functions or equality}
%	\label{tab:undecidable:FunctionsAndEquations}
%\end{table}


%\begin{table}[hbt]
%	\begin{gather*}
%		\begin{array}{rcccl}
%		\text{undecidable} &&\multicolumn{1}{c}{decidable}  & &  \text{classification}
%		\\ \\{}
%[\,\Pi_?, (p_1, {\colLo 1}), (0)\,] &\supsetneq&
%[\,\Pi_?, (p_1, {\colHi 0}), (0)\,]&\subseteq&
%[\,all,(\omega),(\omega)\,]
%%\\[0.5em]		
%%		\\ \\{}
%%		[\,\forall\exists\forall, (\omega, 1), (0)\,] &\multirow{9}{*}{$\supsetneq$}&
%%		\multirow{9}{*}{$[\,\Pi, (p_1, {\colHi 0}), (0)\,]$} &\multirow{9}{*}{$\subseteq$}&
%%		\multirow{9}{*}{$[\,all,(\omega),(\omega)\,]$}
%%	\\{}
%%	[\,\forall^3 \exists, (\omega, 1), (0)\,] &&
%%	&&
%%	\\{}
%%	[\,\forall^{∗} \exists, (0, 1), (0)\,] &&
%%	&&
%%	\\{}
%%	[\,\forall\exists\forall^{∗} , (0, 1), (0)\,]&&
%%	&&
%%	\\{}
%%	[\,\forall\exists\forall\exists^{∗}, (0, 1), (0)\,]&&
%%	&&
%%	\\{}
%%	[\,\forall^3 \exists^{∗} , (0, 1), (0)\,] &&
%%	&&
%%	\\{}
%%	[\,\forall\exists^{∗} \forall, (0, 1), (0)\,] &&
%%	&&
%%	\\{}
%%	[\,\exists^{∗} \forall\exists\forall, (0, 1), (0)\,]&&
%%	&&
%%	\\{}
%%	[\,\exists^{∗} \forall^3 \exists, (0, 1), (0)\,] &&
%%	&&
%%\\{}
%\\[0.5em]
%	[\,\forall, (0), ({\colLo 2})\,]_{=} &\supsetneq&
%	[\,\forall, (0), ({\colHi 1})\,]_{=} &\subseteq&[\,all, (\omega), (1)\,]_{=}
%	\\{}
%	[\,\forall, (0), (0, {\colLo 1})\,]_{=} &\supsetneq&
%	[\,\forall, (0), (0, {\colHi 0})\,]_{=}	&\subseteq&[\,all, (\omega), (1)\,]_{=}
%	\\[0.5em]
%	[\,\forall^2 , (0, {\colN 1}), ({\colLo 1})\,]_{\color{white}=} &\supsetneq&
%	[\,\forall^2 , (0, {\colN 1}), ({\colHi 0})\,]_{\color{white}=} &\subseteq&[\,\exists^{∗}\forall^2\exists^{∗} , all, (0)\,]_{\color{white}=}
%	\\{}
%	%
%	{[\,\forall^2 , ({\colN 1}), (0, {\colLo 1})\,]_{\color{white}=}}&\supsetneq&
%	[\,\forall^2 , ({\colN 1}), (0, {\colHi 0})\,]_{\color{white}=}&\subseteq&[\,\exists^{∗}\forall^2\exists^{∗} , all, (0)\,]_{\color{white}=}
%	\\[0.5em]
%	[\,\forall^2\exists, (\omega, {\colLo 1}), (0)\,]_{=} &\subseteq&
%	[\,\forall^2\exists, (\omega, {\colHi 0}), (0)\,]_{=} &\subseteq&[\,all, (\omega), (1)\,]_{=}
%	\\{}
%	%
%	[\,\exists^{∗}\forall^2\exists, (0, {\colLo 1}), (0)\,]_{=}&\supsetneq&
%		[\,\exists^{∗}\forall^2\exists, (0, {\colHi 0}), (0)\,]_{=}&\subseteq&[\,all, (\omega), (1)\,]_{=}
%	\\{}
%	[\,\forall^2\exists^{∗}, (0, {\colLo 1}), (0)\,]_{=}&\supsetneq&
%	[\,\forall^2\exists^{∗}, (0, {\colHi 0}), (0)\,]_{=}&\subseteq& [\,all, (\omega), (1)\,]_{=}
%	\end{array}
%	\end{gather*}
%\end{table}




\subsection{Skolemization and Herbrandization}



\subsection{Superposition}

% !TeX root = ../mthesis.tex
% !TeX encoding = UTF-8
% !TeX spellcheck = en_US

%%%
\begin{definition}\label{def:superposition-calculus}
	Let \( A,B \) be predicates (not equations),
	\( \mcC,\mcC',\mcD \) clauses, and \( s,s', t,u,v \) terms.
	The \coloremph{superposition calculus} includes the following inference rules
	\begin{itemize}
		%
		\item ordered resolution and ordered factoring
		%
		\begin{gather*}
		\infer
		[\mathsf{(oR)}]
		{(\mcC \lor \mcD)\sigma}
		{{ A} \lor\mcC & { \lnot B} \lor\mcD}
		\hspace{3cm}
		\infer
		[\mathsf{(oF)}]
		{(A \lor \mcC')\sigma}
		{{ A}\lor B\lor\mcC}
		\end{gather*}
		where unifier \( \sigma=\mgu(A,B) \) exists,
		instance \( A\sigma \) is \txtSTRICTLY{} in \( \mcC \sigma \),
		and	instance \( \lnot B\sigma \) is \txtMAXIMAL{} in \( \mcD\sigma \).
			%
		\item ordered paramodulation
		\begin{gather*}
		\infer
		[\mathsf{()}]
		{(\mcC\lor \lnot A[t]\lor\mcD)\,\sigma}
		{{ s\mEQ t} \lor\mcC\quad { \lnot A[s']}\lor\mcD}
		\hspace{3cm}
		\infer
		[\mathsf{()}]
		{(\mcC\lor A[t]\lor\mcD')\,\sigma}
		{{ s\mEQ t} \lor\mcC\quad { A[s']}\lor\mcD'}
		%
%		where unifier \( \sigma=\mgu(s,s') \) exists,
%		\( (s\mEQ t)\sigma \) {is strictly maximal} in \( \mcC\sigma \),
%		\( \lnot A[s'] \) { is maximal} in \( \mcD\sigma \),
%		\( A[s'] \) {is strictly maximal} in \( \mcD'\sigma \);
\end{gather*}
\item superposition
\begin{gather*}
		\infer
		[()]
		{(\mcC\lor u[t]\mNE v\lor\mcD)\,\sigma}
		{{ s\mEQ t}\lor\mcC\quad { u[s']\mNE v}\lor\mcD}
		\hspace{3cm}
		\infer
		[(S_+)]
		{(\mcC\lor u[t]\mEQ v\lor\mcD)\,\sigma}
		{{ s\mEQ t}\lor\mcC\quad { u[s']\mEQ v}\lor\mcD}
		\end{gather*}
		%
		where unifier \( \sigma=\mgu(s,s') \) exists,
		\( s'\not\in\mcV \),
		\( t\sigma\not\succcurlyeq s\sigma \),
		\( v\sigma\not\succcurlyeq u[s']\sigma \),
		\( (s\mEQ t)\sigma \) is \txtSTRICTLY{} in \( \mcC\sigma \),
		\( \lnot A[s'] \) and \( u[s']\mNE v \) are \txtMAXIMAL{} in \( \mcC\sigma \),
		\( A[s'] \) and \( u[s']\mEQ v \) are \txtSTRICTLY{} in \( \mcD\sigma \),
		\( (s\mEQ t)\sigma\not\succcurlyeq(u[s']\mEQ v)\sigma \).
		\item
		equality resolution and equality factoring
		\[
		\infer
		[\mathsf{(R_\mEQ)}]
		{\mcC\sigma}
		{{s\mNE s'}\lor\mcC}
		\hspace{3cm}
		\infer
		[\mathsf{(F_\mEQ)}]
		{(v\mNE s'\lor u\mEQ s'\lor\mcC')\sigma}
		{{ u\mEQ v}\lor{s\mEQ s'}\lor\mcC'}
		\]
		where
		\( \sigma=\mgu(s,s') \) exists,
		\( (s\mNE s')\sigma \) is \txtMAXIMAL{} in \( \mcC \),
		\( (s\mEQ s')\sigma \) is \txtSTRICTLY{} in \( \mcC' \),
		\( (s\mEQ s')\sigma\not\succcurlyeq(u\mEQ v) \). (????)
	\end{itemize}
	%
\end{definition}


\subsection{Conventions}

\begin{figure}[hbt]
	\begin{align*}
	x,y,z &\ \in\mcV \tag*{variables} \\
	\ma,\mb,\mc &\ \in\mcFfn[{0}]\tag*{constant symbols} \\
	\mf, \mg, \mh &\ \in\mcFfn[{n>0}]\tag*{function symbols} \\
	\mpp, \mpq&\ \in\mcFPn[{0}]\tag*{propositional symbols}\\
	\mP, \mQ, \mR &\ \in\mcFPn[{n>0}]\tag*{predicate symbols}\\
	\mcf, \mcg, \mch &\ \in\mcF^{n>0} \tag*{predicate or functions symbols}\\
	\ell, r, s,t,u &\ \in\mcTf = \mcTFfV\tag*{(first order) terms}\\
	\mcs, \mct, \mcu &\ \in \mcPT \cup \mcET \cup \mcTf \tag*{predicates, equations, terms}\\
	\mu, \nu, \sigma, \tau &\ \tag*{substitutions, unifiers}\\
	A,B,C,D &\ \tag*{atoms} \\
	\mcL, \mcE, E, L &\ \tag*{literals} \\
	\mcC, \mcD &\ \tag*{clauses} \\
	S &\ \tag*{sets of clauses, i.e.~\CNF}\\
	F, G &\ \tag*{first order formulae, i.e.~\FOF}
	\end{align*}
	\caption{Typographical conventions}
	\label{fig:conventions}
\end{figure}












