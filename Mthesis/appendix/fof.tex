% !TeX root = ../mthesis.tex
% !TeX encoding = UTF-8
% !TeX spellcheck = en_US


In Section \ref{sec:syntax} we have already defined atomic formulae in Definition \ref{def:atoms},
but we can only build formulae in clausal normal form (\CNF) with Definition \vref{def:syntax:CNF}.
Now we will define arbitrary first order formulae (\FOF).

\CNF-Syntax Definition \vref{def:syntax:CNF},
\CNF-Semantics Definition \vref{def:semantics:CNF},
\FOF-Syntax Definition \vref{def:syntax:FOF}, and
\FOF-Semantics Definition \vref{def:semantics:FOF}



We've already defined when an atom holds for an assignment $\alpha_\mcI$
in an interpretation $\mcI$ within Definition \vref{def:model}.
Now we extend these definitions to first order formulae.


%
	Beside the obvious semantically indistinguishable formulae with double negations, conjunctions, and disjunctions
	we have introduced new ones.
	\begin{enumerate}
		\item $\forall x F$, $\exists x F$, and $F$ are equivalent if $x\not\in\fvar(F)$.
		We usually omit quantifiers with variables that do not occur free in the subformula.
		\item $\exists x F$ and $F$ are not equivalent if $x\in\fvar(F)$,
		e.g. $\exists x(x\mNE\ma)$ is satisfiable and $x\mNE\ma$ isn't.
		\item $\forall x F$ and $F$ are equivalent even if $x\in\fvar(F)$,
		because in both cases we demand that $F$ holds in all assignments in our model.
		Usually we keep these universal quantifiers in \FOF.

		\item A first order formulae without quantifiers is in \coloremph{clausal form},
		but not necessarily in \CNF, e.g.~a weakened version of symmetry $(x\mEQ \ma)\limp (\ma\mEQ x)$
		is equisatisfiable to $\forall x ((x\mEQ \ma)\limp (\ma\mEQ x))$
		and $\exists a (\forall x ((x\mEQ a)\limp (a\mEQ x))$.
		\item A proposition $p$, i.e.~a nullary predicate symbol holds if $()\in p_\mcI \subseteq A^0$.
	\end{enumerate}