In Section \ref{sec:syntax} we have already defined atomic formulae in Definition \ref{def:atoms}, 
but we can only build formulae in clausal normal form (\CNF) with Definition \vref{def:syntax:CNF}.
Now we will define arbitrary first order formulae (\FOF).

\begin{definition}[\FOF]\label{def:syntax:FOF}
	Predicates and equations are (atomic) first order formulae. 
	The negation $(\lnot F)$, 
	the universal quantification $(\forall x F)$, 
	and the existential quantification $(\exists x F)$ 
	of a given formula $F$ are (composite) first order formulae.
	Further, the disjunction $(F \lor F')$, 
	the conjunction $(F \land F') $, 
	and the implication $(F \limp F') \equiv (\lnot F) \lor F'$ 
	of two given formulae $F$ and $F'$ 
	are (composite) first order formulae.
\end{definition}

We've already defined when an atom holds for an assignment $\alpha_\mcI$ 
in an interpretation $\mcI$ within Definition \vref{def:model}.
Now we extend these definitions to first order formulae.

\begin{definition}[Semantics of \FOF]\label{def:semantics:FOF}
	
	A universally quantified formula $\forall x F$ holds in $\mcI$ if its subformula $F$ holds for all assignments for $x$.
	An existential quantified formula $\exists xF$ holds if its subformula $F$ holds for at least one assignment for $x$.
	A negation $\lnot F$ holds if its subformula $F$ does not hold, 
	a disjunction $F\lor F'$ holds if one or both of its subformulae $F$ or $F'$ hold,
	a conjunction $F\land F'$ holds, if both of its subformualae $F$ and $F'$ hold, 
	an implication $F\limp F'$ holds if its first subformula $F$ does not hold or its second subformula $F'$ holds (or both).
\end{definition}


\begin{remark}Usually we us precedences on connectives to omit parentheses 
	and some heuristics to structure the formulae for readability 
	without introducing semantic ambiguity.
%
	Beside the obvious semantically indistinguishable formulae with double negations, conjunctions, and disjunctions 
	we have introduced new ones.
	\begin{enumerate}
		\item $\forall x F$, $\exists x F$, and $F$ are indistinguishable if $x\not\in\var(F)$. 
		We usually omit quantifiers with variables that do not occur in the subformulae.
		\item In general $\exists x F$ is different from $F$ if $x\in\var(F)$, e.g. $\exists x(x\mNE\ma)$ is satisfiable and $x\mNE\ma$ isn't.
		\item $\forall x F$ and $F$ are equivalent even if $x\in\var(F)$, 
		because in both cases we demand that $F$ holds in all assignments in our model.
		Usually we keep these universal quantifiers in \FOF.
		
		A first order formulae without quantifiers is in {\myem clausal form}, 
		but not necessarily in \CNF, e.g.~a weakened version of symmetry $(x\mEQ \ma)\limp (\ma\mEQ x)$ 
		is equisatisfiable to $\forall x ((x\mEQ \ma)\limp (\ma\mEQ x))$ 
		or $\exists a (\forall x ((x\mEQ a)\limp (a\mEQ x))$. 
		\item A proposition $p$, i.e.~a nullary predicate symbol holds if $()\in p_\mcI \subseteq A^0$.
	\end{enumerate}

\end{remark}