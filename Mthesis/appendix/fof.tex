In Section \ref{sec:syntax} we have already defined atomic formulae in Definition \ref{def:atoms}, 
but we can only build formulae in clausal normal form (\CNF) with Definition \vref{def:syntax:CNF}.
Now we will define arbitrary first order formulae (\FOF).

\begin{definition}[\FOF]\label{def:syntax:FOF}
	Predicates and equations are (atomic) first order formulae. 
	The negation $(\lnot F)$, 
	the universal quantification $(\forall x F)$, 
	and the existential quantification $(\exists x F)$ 
	of a given formula $F$ are (composite) first order formulae.
	Further, the disjunction $(F \lor F')$, 
	the conjunction $(F \land F') $, 
	and the implication $(F \limp F') \equiv (\lnot F) \lor F'$ 
	of two given formulae $F$ and $F'$ 
	are (composite) first order formulae.
\end{definition}

We've already defined when an atom holds for an assignment $\alpha_\mcI$ 
in an interpretation $\mcI$ within Definition \vref{def:model}.
Now we extend these definitions to first order formulae.

\begin{definition}[Semantics of \FOF]\label{def:semantics:FOF}
	
	A universally quantified formula $\forall x F$ holds in $\mcI$ if its subformula $F$ holds for all assignments for $x$.
	An existential quantified formula $\exists xF$ holds if its subformula $F$ holds for at least one assignment for $x$.
	A negation $\lnot F$ holds if its subformula $F$ does not hold, 
	a disjunction $F\lor F'$ holds if one or both of its subformulae $F$ or $F'$ hold,
	a conjunction $F\land F'$ holds, if both of its subformualae $F$ and $F'$ hold, 
	an implication $F\limp F'$ holds if its first subformula $F$ does not hold or its second subformula $F'$ holds (or both).
\end{definition}

\noindent We assume that each variable of a formula is quantified exactly once.

\begin{definition}\label{def:syntax:PNF}
	A first order formula $F = Q_1 x_1 \ldots Q_n x_n\, F'$ 
	with quantifiers $Q_i\in\{\forall,\exists\}$, 
	quantifier-free subformula $F'$ with $\var(F') = \{ x_1,\ldots,x_n \}$
	is in {\myem prenex normal form}.
\end{definition}

\begin{lemma}
	Any first order formula can be transformed 
	into an equivalent prenex normal form 
	by exhaustively using the following equivalences from left to right.
	\begin{align*}
	\lnot(\forall x F) &\equiv \exists x (\lnot F)
	&
	\lnot(\exists x F) &\equiv \forall x (\lnot F)
	\\
	{\colG G \land (\forall x F) \equiv{}} (\forall x F) \land G &\equiv \forall x (F \land G)  
	&
	{\colG G \land (\exists x F) \equiv{}} (\exists x F) \land G &\equiv \exists x (F \land G)  
	\\
	{\colG G \lor (\forall x F) \equiv{}} (\forall x F) \lor G &\equiv \forall x (F \lor G) 
	&
	{\colG G \lor (\exists x F) \equiv{}} (\exists x F) \lor G &\equiv \exists x (F \lor G)
	\end{align*}
	\begin{remark}
		The equivalences depend on $y\not\in\var(F)$ and $x\not\in\var(G)$.
	\end{remark}
\end{lemma}

\begin{lemma}We just state the following useful equivalences.
	\begin{align*}
		(\forall x F) \land (\forall y G) &\equiv \forall x (F\land G\{y\mapsto x\})
		&
		(\exists x F) \lor (\exists y G) &\equiv \exists x (F\lor G\{y\mapsto x\})
		\\
		\forall x \forall y F &\equiv \forall y \forall x F
		&
		\exists x \exists y F &\equiv \exists y \exists x F
		\\
		\forall x G &\equiv G & \exists x G &\equiv G
		\\
		(\forall x F) \land (\exists y G) &\equiv \exists y \forall x (F\land G)
		&
		(\forall x F) \lor (\exists y G) &\equiv \exists y \forall x (F\lor G)
	\end{align*}
\end{lemma}

\begin{definition}
	Monadic predicate logic is first order logic 
	where the signature of the formulae just contains
	propositional and unary predicate symbols,
	but no function symbols.
\end{definition}

\begin{lemma}
	In monadic predicate logic the following equivalence holds.
	\[
		\forall x \exists y F \equiv \exists y \forall x F
	\]
	
\end{lemma}

\begin{proof}By induction on the structure of F. 
	\begin{itemize}
		\item(base) Let $F$ be a literal with unary predicate symbol $\mP\in\mcFPn[1]$.
		\begin{gather*}
		\forall x \exists y\,{\colG\lnot}\mP(x) \equiv \forall x\, {\colG\lnot}\mP(x) 
%		{\colG{}\equiv \exists y \forall x {\colG\lnot}\mP(x)}
		\qquad 
		\forall x \exists y\, {\colG\lnot}\mP(y) \equiv \exists y\, {\colG\lnot}\mP(y)  
%		{\colG{}\equiv \exists y \forall x {\colG\lnot}\mP(y)}
		\end{gather*}
		\item(base) Let $F$ be a junction $\mcL \star \mcL'$ of two literals 
		where $\star\in\{\lor,\land\}$.
		\begin{align*}
			\forall x(\exists y (\mcL\star\mcL'))
			\equiv \forall x((\mcL\star(\exists y\mcL')))    )
			\equiv (\forall x \mcL) \star (\exists y\mcL') 
			\tag*{if $\mcL\in\{\lnot\mP(x),\mP(x)\}$}
			\\
			\exists y(\forall x (\mcL\star\mcL'))
			\equiv \exists y((\forall x\mcL)\star \mcL')))    )
			\equiv (\forall x \mcL) \star (\exists y\mcL')
			\tag*{otherwise}
		\end{align*}
		\item Let $F$ be a junction $G\star G'$ of two sub-formulae
		where $\star\in\{\lor,\land\}$.
		\begin{gather*}
	\forall x\exists y (G\lor G') 
	\equiv \forall x (\exists y G \lor \exists y G')
	\equiv \forall x \exists y G \lor \forall x \exists y G'
\\
	\defEV[IH] \exists y\forall x  G \lor \exists y\forall x G'
	\equiv \exists y(\forall x  G \lor \forall x G')
	\equiv \exists y \forall x (G \lor G')
		\end{gather*}
		
	\end{itemize}
\end{proof}


\begin{lemma}[Monadic first order logic]
	Any first order formulae with constant or unary predicate symbols only but no function symbols 
	is equisatisfiable to a set of clauses with unary predicates and constant function symbols.
\end{lemma}

\begin{lemma}[Ackermann class]
	Any formula in prenex normal form with prefix $\exists^* \forall \exists^*$ 
	and arbitrary predicate symbols but without function symbols is equisatisfiable to 
	a set of clauses with the same predicate symbols, but additional Skolem constants and unary Skolem functions.
\end{lemma}


\begin{lemma}[Gödel class]
	Any formula in prefix normal form with prefix $\exists^* \forall\forall \exists^*$ 
	with predicate symbols but without function symbols 
	is equisatisfiable to a set of clauses 
	but additional Skolem constants and binary Skolem functions.
\end{lemma}


\begin{remark}Usually we us precedences on connectives to omit parentheses 
	and some heuristics to structure the formulae for readability 
	without introducing semantic ambiguity.
%
	Beside the obvious semantically indistinguishable formulae with double negations, conjunctions, and disjunctions 
	we have introduced new ones.
	\begin{enumerate}
		\item $\forall x F$, $\exists x F$, and $F$ are indistinguishable if $x\not\in\var(F)$. 
		We usually omit quantifiers with variables that do not occur in subformulae.
		\item In general $\exists x F$ is different from $F$ if $x\in\var(F)$, e.g. $\exists x(x\mNE\ma)$ is satisfiable and $x\mNE\ma$ isn't.
		\item $\forall x F$ and $F$ are equivalent even if $x\in\var(F)$, 
		because in both cases we demand that $F$ holds in all assignments in our model.
		Usually we keep these universal quantifiers in \FOF.
		
		A first order formulae without quantifiers is in {\myem clausal form}, 
		but not necessarily in \CNF, e.g.~a weakened version of symmetry $(x\mEQ \ma)\limp (\ma\mEQ x)$ 
		is equisatisfiable to $\forall x ((x\mEQ \ma)\limp (\ma\mEQ x))$ 
		or $\exists a (\forall x ((x\mEQ a)\limp (a\mEQ x))$. 
	\end{enumerate}

\end{remark}