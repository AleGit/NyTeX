% !TeX root = ../mthesis.tex
% !TeX encoding = UTF-8
% !TeX spellcheck = en_US

\begin{definition}For a term $t$ we define the set of variables
	
	\begin{align*}
		\var(t) &= \left\{\begin{array}{ll}
			\{ x \} & \text{if } t = x \in \mcV \\
			\bigcup_{i=1}^n \var(t_i) & \text{if }  t = \mf(t_1, \ldots t_n), \mf \in \mcFfn
%			\emptyset &\text{if } \mkt \in \mcFO
		\end{array}\right.	
	\end{align*}
	the set of function symbols
	\begin{align*}
	\fun(t) &= \left\{\begin{array}{ll}
	\emptyset & \text{if } t = x \in \mcV \\
	\{ \mf \} \cup \bigcup_{i=1}^n \fun(t_i) & \text{if }  t = \mf(t_1, \ldots t_n), \mf \in \mcFfn
	%			\emptyset &\text{if } \mkt \in \mcFO
	\end{array}\right.	
	\end{align*}
	and the set of subterms
	\begin{align*}
	\subterms(t) &= \left\{\begin{array}{ll}
	\{ t \} & \text{if } t \in \mcV \\
	\{ t \} \cup \bigcup_{i=1}^n \subterms(t_i) & \text{if }  t = f(t_1, \ldots t_n), f \in \mcFn
	%			\emptyset &\text{if } \mkt \in \mcFO
	\end{array}\right.	
	\end{align*}
	
\end{definition}


%\begin{definiton}
	\begin{align*}
	\forall x\ &\ x = x \\
	\forall x,y\ &\ x = y \Leftrightarrow y = x \\
	\forall x,y,z\ &\ x = y \land y = z \Rightarrow x = y \\
	\forall x_1,\ldots,x_n,y_1,\ldots,y_n\ &\ x_1=y_1\land\ldots x_n=y_n\Rightarrow f(x_1,\ldots,x_n)=f(y_1,\ldots,y_n) \\
	\forall x_1,\ldots,x_n,y_1,\ldots,y_n\ &\ x_1=y_1\land\ldots x_n=y_n\Rightarrow P(x_1,\ldots,x_n)\Leftrightarrow P(y_1,\ldots,y_n) \\
	\end{align*}
%\end{definition}

\begin{definition}\label{def:orders-on-literals}
	We extend a well-founded and total order $\gtpre$ on general ground terms, i.e. general atoms 
	to a well-founded proper order $\gtpre_\mL$ on literals such that for all atoms $A$ and $B$ with $A\gtpre B$ 
	the relations $A\gtpre_\mL B$ , 
	$\lnot A\gtpre_\mL\lnot B$ and 
	$\lnot A\gtpre_\mL A$ hold. 
	%
	A (non-ground) literal $L$ is {\myem (strictly) maximal} if there exists a ground substitution $\tau$ 
	such for no other literal $L'$ the relation $L'\tau\gtpre L\tau$ (strictly: $\succcurlyeq$) holds.
	We write $\gtpre_{gr}$ to suggest the existence of such a ground substitution $\tau$.
\end{definition}
%%%
\begin{definition}\label{def:superposition-calculus}
	The {\myem superposition calculus} includes
	\begin{itemize}
		%
		\item ordered resolution $(R)$,
		ordered factoring $(F)$
		%
		\begin{gather*}
		\infer
		[(R)]
		{(\mcC \lor \mcD)\sigma}
		{{ A} \lor\mcC & { \lnot B} \lor\mcD}
		\hspace{3cm}
		\infer
		[(F)]
		{(A \lor \mcC)\sigma}
		{{ A}\lor B'\lor\mcC}
		\end{gather*}
		%
		where
		$\sigma=\mgu(A,B)$ is defined, 
		$A\sigma$ { is strictly maximal} in $\mcC\sigma$,
		$\lnot B\sigma$ { is maximal} in $\mcD\sigma$, 
		$A\sigma$ is maximal in $\mcD\sigma$;
		%
		\item ordered paramodulation $(P_-,P_+)$
		\[
		\infer
		[(P_-)]
		{(\mcC\lor \lnot A[t]\lor\mcD)\,\sigma}
		{{ s\mEQ t} \lor\mcC\quad { \lnot A[s']}\lor\mcD}
		\hspace{3cm}
		\infer
		[(P_+)]
		{(\mcC\lor A[t]\lor\mcD)\,\sigma}
		{{ s\mEQ t} \lor\mcC\quad { A[s']}\lor\mcD}
		\]
		%
		where $\sigma=\mgu(s,s')$ is defined, 
		$(s\mEQ t)\sigma$ { is strictly maximal} in $\mcC\sigma$, 
		$\lnot A[s']$ { is maximal} in $\mcD\sigma$,
		$A[s']$ { is strictly maximal} in $\mcD\sigma$;
		%
		\item superposition $(S_-,S_+)$
		\[
		\infer
		[(S_-)]
		{(\mcC\lor u[t]\mNE v\lor\mcD)\,\sigma}
		{{ s\mEQ t}\lor\mcC\quad { u[s']\mNE v}\lor\mcD}
		\hspace{3cm}
		\infer
		[(S_+)]
		{(\mcC\lor u[t]\mEQ v\lor\mcD)\,\sigma}
		{{ s\mEQ t}\lor\mcC\quad { u[s']\mEQ v}\lor\mcD}
		\]
		%
		where $\sigma=\mgu(s,s')$ is defined,
		$s'\not\in\mcV$, 
		$t\sigma\not\succcurlyeq s\sigma$,
		$v\sigma\not\succcurlyeq u[s']\sigma$;
		%
		\item 
		equality resolution $(R_\mEQ )$,
		and equality factoring $(F_\mEQ )$
		\[
		\infer
		[(R_\mEQ)]
		{\mcC\sigma}
		{{ s\mNE s'}\lor\mcC}
		\hspace{3cm}
		\infer
		[(F_\mEQ)]
		{(v\mNE s'\lor u\mEQ s'\lor\mcC)\sigma}
		{{s\mEQ s'}\lor { u\mEQ v}\lor\mcC}
		\]
		where 
		$\sigma=\mgu(s,s')$ is defined,
		$(s=s')\sigma\not\succcurlyeq(u\mEQ v)$.
	\end{itemize}
	%
\end{definition}

\begin{definition}\label{def:unit-superpositin-calculus}
	The {\myem unit superposition calculus} includes 
	\begin{itemize}
		\item unit paramodulation $(U\!P)$
		\begin{gather*}
		\infer
		[(U\!P)]
		{(L[t])\,\sigma}
		{s\mEQ t&L[s']}
		\end{gather*}
		where $\sigma=\mgu(s,s')$ is defined,
		$s'\not\in\mcV$,
		$t\sigma\not\succcurlyeq s\sigma$;
		%
		\item unit superposition $(U\!S_-,U\!S_+)$
		\begin{gather*}
		\infer
		[(U\!S_-)]
		{(u[t]\mNE v)\,\sigma}
		{s\mEQ t&u[s']\mNE v}
		\qquad\qquad
		\infer
		[(U\!S_+)]
		{(u[t]\mEQ v)\,\sigma}
		{s\mEQ t&u[s']\mEQ v}
		\end{gather*}
		where $\sigma=\mgu(s,s')$ is defined,
		$s'\not\in\mathcal{V},$
		$t\sigma\not\succcurlyeq s\sigma,$
		$v\sigma\not\succcurlyeq u[s']\sigma$;
		%
		\item unit equality resolution $(U\!R_\mEQ)$, and unit resolution $(U\!R)$
		\begin{gather*}
		\infer
		[(U\!R_\mEQ)]
		{\emptyclause}
		{s\mNE t}
		\quad\quad\quad\qquad
		\infer
		[(U\!R)]
		{\emptyclause}
		{A&\lnot B}
		\end{gather*}
		where $s$ and $t$ ($A$ and $B$ respectively) are unifiable.
	\end{itemize}
\end{definition}

