% !TeX root = ../mthesis.tex
% !TeX encoding = UTF-8
% !TeX spellcheck = en_US

%\ref{tab:undecidable:PurePredicateLogic}, and
%\ref{tab:undecidable:FunctionsAndEquations}


We look how formulae from these decidable fragments of first order logic
(Tables \ref{tab:decidedable:FiniteModelProperty} and \ref{tab:decidable:InfinityAxioms})
can be transformed nicely into equisatisfiable set of clauses.
The transformation is easy for formulae where the quantifiers follow a specific pattern where the appearances of the universal quantifier is limited. First we remove leading existential quantifiers and replace their variables with Skolem constants.
Then we remove the trailing existential quantifiers and replace there variables with Skolem functions where the variables of the universal quantifiers are the arguments. After that we remove the remaining universal quantifiers. We transform the remaining formula with constants, functions and a restricted number of free variables into (an equisatisfiable) conjunctive normal form with a suitable procedure.

\begin{definition}\label{def:prefix:class}
	We describe classes of first order formulae in clausal normal form with triples
	\[
	_{_\CNF}\langle\, \pi, (p_1,p_2,\ldots), (f_1,f_2,\ldots)\,\rangle_{(=)}
	\]
	where \( \pi \) is the maximal number of distinct variables (with multiple occurrences),
	\( p_i \) the maximal number of predicates symbols of arity \( i \),
	\( f_i \) the maximal number of function symbols of arity \( i \).
	The equality symbol is not counted as binary predicate symbol.
	Instead the subscript \( = \) indicates its presence.
\end{definition}

\begin{table}[hbt]
\begin{align}
\PNF& &&& \CNF& \tag*{}\\
[\,\exists^{∗}\forall^{∗}, all, (0)&\,]_{=}
&&\checkmark& _{_\CNF}\langle\,\omega,all, (0)&\,\rangle_{=}
\label{tab:decidable:CNF:1}
\\
[\,\exists^{∗}\forall^2\exists^{∗} , all, (0)&\,]
&&\checkmark& _{_\CNF}\langle\,2, all, (0,\omega)&\,\rangle
\label{tab:decidable:CNF:2}
\\
[\,all, (\omega), (\omega)&\,]
&&?& _{_\CNF}\langle\,\omega, (\omega), (\omega)&\,\rangle
\label{tab:decidable:CNF:3}
\\
[\,\exists^{∗}\forall\exists^{∗}, all, all&\,]
&&\checkmark& _{_\CNF}\langle\,1,all, all&\,\rangle
\label{tab:decidable:CNF:4}
\\
[\,\exists^{∗}, all, all&\,]_{=}
&&\checkmark& _{_\CNF}\langle\, 0, all, all&\,\rangle_{=}
\label{tab:decidable:CNF:5}
\\
 [\,all, (\omega), (1)&\,]_{=}
&&\supsetneq& {} _{_\CNF}\langle\,\omega, (\omega), (1)&\,\rangle_{=}
\label{tab:decidable:CNF:6}
\\
[\,\exists^{∗}\forall\exists^{∗}, all, (1)&\,]_{=}
&&\checkmark& _{_\CNF}\langle\,1, all, (\omega)&\,\rangle_{=}
\label{tab:decidable:CNF:7}
\end{align}
\caption[Transformation]{Clausal representation of decidable prefix classes}
\label{tab:decidedable:CNF}
\end{table}

First we transform the matrix of a formula in \PNF into an equisatisfiable conjunctive normal form.
With the following transformation into clausal normal form
we get one skolem constant per leading existential quantifier. And we get \ldots

\begin{itemize}
	\item[\ref{tab:decidable:CNF:1}]
	\ldots one free variable per trailing universal quantifier.

	\item[\ref{tab:decidable:CNF:2}]
	\ldots one binary skolem function per trailing existential quantifier,
	and two free variables for the two universal quantifiers in between.

	\item[\ref{tab:decidable:CNF:3}]
	??? With Lemma \ref{lem:unaries:unaries} the class \( [\,all, (\omega), (\omega)\,] \)
	is equivalent to \( [\,\exists^{∗}\forall^{∗}, (\omega), (\omega)\,] \).

	\item[\ref{tab:decidable:CNF:4}]
	\ldots one unary skolem function per trailing existential quantifier,
	and one free variable for the one universal quantifier in between.

	\item[\ref{tab:decidable:CNF:5}]
	\ldots no free variables at all.

	\item[\ref{tab:decidable:CNF:6}]
	The class \( [\,all, (\omega), (1)\,]_= \)
	is bigger than class \( [\,\exists^{∗}\forall^{∗}, (\omega), (1)\,]_= \).
\begin{align*}
	\forall x \exists y\,&(\msucc(x)\mNE x\land \msucc(x)\mEQ y)\tag*{satisfiable}
	\\
	\exists y \forall x\,&(\msucc(x)\mNE x\land \msucc(x)\mEQ y)\tag*{unsatisfiable}.
	\end{align*}
	\item[\ref{tab:decidable:CNF:7}]
\end{itemize}

\begin{lemma}\label{lem:unaries:unaries}
	??? A formula in \( [\,all, (\omega), (\omega)\,] \) is equivalent / equisatisfiable
	to a formula in \( [\,\exists^*\forall^*, (\omega), (\omega)\,] \).
\end{lemma}

\begin{proof} ??? It is sufficient to show
	that \( \forall x\exists y\,F\equiv\exists y\forall y\,F \)
%	where \( \fvar(F)\subseteq\{x,y\} \)
	for unary predicate and function symbols.
	We use structural induction on formula \( F \).
	\begin{enumerate}
		\item (base) Let \( F \) be a literal,
		i.e.~a (negated) predicate with one argument term \( t \).
		We abbreviate the literal with \( L(z) \) if \( \var(t) = \{ z \} \).
		\begin{gather*}
			\forall x \exists y L(x)  \equiv \forall x L(x) \equiv {\colG \exists y} \forall x L(x)
			\qquad\qquad
			\forall x \exists y L(y)  \equiv \exists y L(y) \equiv \exists y{\colG \forall x}  L(y)
		\end{gather*}
		\item (step) Let \( F \) be the negation of a formula.
		\begin{gather*}
		\forall x \exists y\, (\lnot G)
		\equiv \lnot (\exists x \forall y\, G)
		\defEV[IH] \lnot (\forall y\exists x \, G)
		\equiv \exists y \forall x\, (\lnot G)
		\end{gather*}
		\item (step) Let \( F \) be the junction \( \landlor{}\in\{\land,\lor\} \) of a “free” literal, i.e.~\( x,y\not\in\var(L) \), and a formula.
		\begin{gather*}
		\forall x\exists y\,( L \landlor G)
		\equiv L \landlor \forall x\exists y\, G
		\defEV[IH]L \landlor \exists y\forall x\, G
		\equiv\exists y\forall x\,( L \landlor G )
		\end{gather*}
		\item (step) Let \( F \) be the conjunction of a “universal” literal and a formula.
		\begin{align*}
		\forall x \exists y\,({L(x)\land G})
		&\equiv \forall x\,({L(x)} \land \exists y\,G)
		\equiv \forall x\,L(x)\land \forall x\exists y\,G
		\\&\defEV[IH] \forall x\,{L(x)} \land \exists y\forall x\,G
		\equiv\exists y\,(\forall x\,L(x)\land \forall x\,G)
		\equiv\exists y \forall x\,({L(x)\land G})
		\end{align*}
		\item (step) Let \( F \) be the disjunction of an “existential” literal and a formula.
		\begin{align*}
		\exists x \forall y\,(L(x)\lor H)
		&\equiv \lnot(\forall x\exists y\,(L^c(x)\land \lnot H)) \\
		&\defEV[4.]\lnot( \exists y\forall x\,(L^c(x)\land \lnot H) )
		\equiv \forall y \exists x\,(L(x)\lor H)
		\end{align*}


		\item (step)
		Let \( F \) be the disjunction of an “universal” literal and a formula.
		\begin{align*}
		\forall x \exists y\,(L(x)\lor G)
		&\equiv
		\end{align*}

		\item (step)
		Let \( F \) be the conjunction of an “existential” literal and a formula.
		\begin{align*}
		\exists x \forall y\,(L(x)\land G)
		&\equiv
		\end{align*}


	\end{enumerate}
\end{proof}



%\begin{lemma}\label{lem:unaries:equals}
%	??? A formula in \( [\,all, (\omega), (1)\,]_= \) is equivalent / equisatisfiable
%	to a formula in \( [\,\exists^*\forall^*, (\omega), (1)\,]_= \) ???
%\end{lemma}
%
%\begin{proof}
%	\begin{enumerate}
%		\item \( \forall x \exists y\, x\mEQ y \)
%		\item \( \forall x \exists y\, y\mEQ x \)
%	\end{enumerate}
%
%\end{proof}


%% ADDITIONAL MATERIAL %%%%

\begin{table}[hbt]
	\begin{align*}
	[\,\forall\exists\forall, (\omega, 1), (0)&\,] \tag{Kahr 1962}
	\\
	[\,\forall^3 \exists, (\omega, 1), (0)&\,] \tag{Surányi 1959}
	\\
	[\,\forall^{∗} \exists, (0, 1), (0)&\,] \tag{Kalmár-Surányi 1950}
	\\
	[\,\forall\exists\forall^{∗} , (0, 1), (0)&\,]  \tag{Denton 1960}
	\\
	[\,\forall\exists\forall\exists^{∗}, (0, 1), (0)&\,] \tag{Gurevich 1966}
	\\
	[\,\forall^3 \exists^{∗} , (0, 1), (0)&\,] \tag{Kalmár-Surányi 1947}
	\\
	[\,\forall\exists^{∗} \forall, (0, 1), (0)&\,] \tag{Kostyrko-Genenz 1964}
	\\
	[\,\exists^{∗} \forall\exists\forall, (0, 1), (0)&\,] \tag{Surányi 1959}
	\\
	[\,\exists^{∗} \forall^3 \exists, (0, 1), (0)&\,] \tag{Surányi 1959}
	\end{align*}
	\caption{Undecidable prefix classes in pure predicate logic}
	\label{tab:undecidable:PurePredicateLogic}
\end{table}

\begin{table}[hbt]
	\begin{align*}
	[\,\forall, (0), (2)&\,]_{=} \tag{Gurevich 1976}
	\\
	[\,\forall, (0), (0, 1)&\,]_{=} \tag{Gurevich 1976}
	\\
	[\,\forall^2 , (0, 1), (1)&\,] \tag{Gurevich 1969}
	\\
	[\,\forall^2 , (1), (0, 1)&\,] \tag{Gurevich 1969}
	\\
	[\,\forall^2\exists, (\omega, 1), (0)&\,]_{=} \tag{Goldfarb 1984}
	\\
	[\,\exists^{∗}\forall^2\exists, (0, 1), (0)&\,]_{=} \tag{Goldfarb 1984}
	\\
	[\,\forall^2\exists^{∗}, (0, 1), (0)&\,]_{=} \tag{Goldfarb 1984}
	\end{align*}
	\caption{Undecidable prefix classes with functions or equality}
	\label{tab:undecidable:FunctionsAndEquations}
\end{table}


\begin{table}[hbt]
	\begin{gather*}
		\begin{array}{rcccl}
		\text{undecidable} &&\multicolumn{1}{c}{decidable}  & &  \text{classification}
		\\ \\{}
[\,\Pi_?, (p_1, {\colLo 1}), (0)\,] &\supsetneq&
[\,\Pi_?, (p_1, {\colHi 0}), (0)\,]&\subseteq&
[\,all,(\omega),(\omega)\,]
\\[0.5em]
		\\ \\{}
		[\,\forall\exists\forall, (\omega, 1), (0)\,] &\multirow{9}{*}{\( \supsetneq \)}&
		\multirow{9}{*}{\( [\,\Pi, (p_1, {\colHi 0}), (0)\,] \)} &\multirow{9}{*}{\( \subseteq \)}&
		\multirow{9}{*}{\( [\,all,(\omega),(\omega)\,] \)}
	\\{}
	[\,\forall^3 \exists, (\omega, 1), (0)\,] &&
	&&
	\\{}
	[\,\forall^{∗} \exists, (0, 1), (0)\,] &&
	&&
	\\{}
	[\,\forall\exists\forall^{∗} , (0, 1), (0)\,]&&
	&&
	\\{}
	[\,\forall\exists\forall\exists^{∗}, (0, 1), (0)\,]&&
	&&
	\\{}
	[\,\forall^3 \exists^{∗} , (0, 1), (0)\,] &&
	&&
	\\{}
	[\,\forall\exists^{∗} \forall, (0, 1), (0)\,] &&
	&&
	\\{}
	[\,\exists^{∗} \forall\exists\forall, (0, 1), (0)\,]&&
	&&
	\\{}
	[\,\exists^{∗} \forall^3 \exists, (0, 1), (0)\,] &&
	&&
\\{}
\\[0.5em]
	[\,\forall, (0), ({\colLo 2})\,]_{=} &\supsetneq&
	[\,\forall, (0), ({\colHi 1})\,]_{=} &\subseteq&[\,all, (\omega), (1)\,]_{=}
	\\{}
	[\,\forall, (0), (0, {\colLo 1})\,]_{=} &\supsetneq&
	[\,\forall, (0), (0, {\colHi 0})\,]_{=}	&\subseteq&[\,all, (\omega), (1)\,]_{=}
	\\[0.5em]
	[\,\forall^2 , (0, {\colN 1}), ({\colLo 1})\,]_{\color{white}=} &\supsetneq&
	[\,\forall^2 , (0, {\colN 1}), ({\colHi 0})\,]_{\color{white}=} &\subseteq&[\,\exists^{∗}\forall^2\exists^{∗} , all, (0)\,]_{\color{white}=}
	\\{}
	%
	{[\,\forall^2 , ({\colN 1}), (0, {\colLo 1})\,]_{\color{white}=}}&\supsetneq&
	[\,\forall^2 , ({\colN 1}), (0, {\colHi 0})\,]_{\color{white}=}&\subseteq&[\,\exists^{∗}\forall^2\exists^{∗} , all, (0)\,]_{\color{white}=}
	\\[0.5em]
	[\,\forall^2\exists, (\omega, {\colLo 1}), (0)\,]_{=} &\subseteq&
	[\,\forall^2\exists, (\omega, {\colHi 0}), (0)\,]_{=} &\subseteq&[\,all, (\omega), (1)\,]_{=}
	\\{}
	%
	[\,\exists^{∗}\forall^2\exists, (0, {\colLo 1}), (0)\,]_{=}&\supsetneq&
		[\,\exists^{∗}\forall^2\exists, (0, {\colHi 0}), (0)\,]_{=}&\subseteq&[\,all, (\omega), (1)\,]_{=}
	\\{}
	[\,\forall^2\exists^{∗}, (0, {\colLo 1}), (0)\,]_{=}&\supsetneq&
	[\,\forall^2\exists^{∗}, (0, {\colHi 0}), (0)\,]_{=}&\subseteq& [\,all, (\omega), (1)\,]_{=}
	\end{array}
	\end{gather*}
\end{table}


