% !TeX root = ../mthesis.tex
% !TeX encoding = UTF-8
% !TeX spellcheck = en_US

%\ref{tab:decidedable:FiniteModelProperty} and
%\ref{tab:decidable:InfinityAxioms}
%\ref{tab:undecidable:PurePredicateLogic}, and
%\ref{tab:undecidable:FunctionsAndEquations}
This subsection presents fragments of first order logic where satisfiability is decidable.\footnote{
	Definition and compact overview from presentation
	“\href{http://logic.rwth-aachen.de/~graedel/kalmar.pdf}{Decidable fragments of first-order and fixed-point logic}”
	by E.~Grädel (\url{http://logic.rwth-aachen.de/~graedel/}).	
}  


\begin{definition}\label{def:prefix:class}
	We describe classes of first order formulae in \PNF with triples
	\[
		[\, \Pi, (p_1,p_2,\ldots), (f_1,f_2,\ldots)\,]_{(=)}
	\]
	where $\Pi$ describes the quantifier prefix,
	$p_i$ the maximal number of predicates symbols of arity $i$,
	$p_j$ the maximal number of function symbols of arity $j$,
	and the presence of the equality predicate is indicated by subscript $=$,
	i.e. the equality symbol is not counted as binary predicate symbol.
\end{definition}

\begin{example}
	The monadic predicate calculus includes formulae with arbitrary quantifier prefixes, 
	arbitrary many unary predicate symbols, the equality symbol, but no function symbols.
	\begin{align*}
	\colG [\,all, (\omega), (1)\,]_= 
	\quad\supsetneq& &
		[\,all, (\omega), (0 )&\,]_=
		 \tag{Löwenheim 1925, Kalmár 1929}
	\\
	\colG [\,\exists^*\forall\exists^*,all,(1)\,]_=
	\quad\supsetneq& &
	[\,\exists^*\forall\exists^*,all,(0)&\,]_{=} \tag{Ackermann 1928}
	\end{align*}
	The Ackermann prefix class contains formulae with arbitrary many existential quantifiers, 
	but just one universal quantifier. It contains arbitrary many predicate symbols
	with arbitrary arities, the equality symbol, but no function symbols. 
\end{example}

%Goldfarb, Gurevich, Rabin, Shelah completly characterized decidable and undecidable prefix classes.

\begin{table}[hbt]
\begin{align*}
%\colG [\,all, (\omega), (0 )&\colG\,]_= \tag{Löwenheim 1925, Kalmár 1929}
%\\
[\,\exists^{∗}\forall^{∗}, all, (0)&\,]_{=} \tag{Bernays, Schönfinkel 1928, Ramsey 1932}
\\
[\,\exists^{∗}\forall^2\exists^{∗} , all, (0)&\,] \tag{Gödel 1932, Kalmár 1933, Schütte 1934}
\\
[\,all, (\omega), (\omega)&\,] \tag{Löb 1967, Gurevich 1969}
\\
[\,\exists^{∗}\forall\exists^{∗}, all, all&\,] \tag{Gurevich 1973}
\\
[\,\exists^{∗}, all, all&\,]_{=} \tag{Gurevich 1976}
\end{align*}
%\caption[Decidable prefix classes]{Decidable prefix classes in first order logic}
\caption[Decidable prefix classes]{Decidable prefix classes with final model property }
\label{tab:decidedable:FiniteModelProperty}
\end{table}

\begin{table}[hbt]
	\begin{align*}
	[\,all, (\omega), (1)&\,]_{=} \tag{Rabin 1969} 
	\\
	[\,\exists^{∗}\forall\exists^{∗}, all, (1)&\,]_{=} \tag{Shelah 1977}
	\end{align*}
	%\caption[Decidable prefix classes]{Decidable prefix classes in first order logic}
	\caption[Decidable prefix classes]{Decidable prefix classes with infinity axioms. }
	\label{tab:decidable:InfinityAxioms}
\end{table}

\begin{definition}\label{def:prefix:class}
	We describe classes of first order formulae in \CNF with triples
	\[
	_{_\CNF}[\, \pi, (p_1,p_2,\ldots), (f_1,f_2,\ldots)\,]_{(=)}
	\]
	where $\pi$ is the maximal number of variables per clause,
	$p_i$ the maximal number of predicates symbols of arity $i$,
	$p_j$ the maximal number of function symbols of arity $j$,
	and the presence of the equality predicate is indicated by subscript $=$,
	i.e. the equality symbol is not counted as binary predicate symbol.
\end{definition}

\begin{table}[hbt]
\begin{align*}
\PNF& & \CNF& \\
[\,\exists^{∗}\forall^{∗}, all, (0)&\,]_{=} 
& _{_\CNF}[\,\omega,all, (0)&\,]_{=}
\\
[\,\exists^{∗}\forall^2\exists^{∗} , all, (0)&\,] 
& _{_\CNF}[\,2, all, (0,\omega)&\,]
\\
[\,all, (\omega), (\omega)&\,]
& _{_\CNF}[\,\omega, (\omega), (\omega)&\,]
\\
[\,\exists^{∗}\forall\exists^{∗}, all, all&\,]
& _{_\CNF}[\,1,all, all&\,]
\\
[\,\exists^{∗}, all, all&\,]_{=}
& _{_\CNF}[\, 0, all, all&\,]_{=}
\\
[\,all, (\omega), (1)&\,]_{=} 
& _{_\CNF}[\,\omega, (\omega), (1)&\,]_{=} 
\\
[\,\exists^{∗}\forall\exists^{∗}, all, (1)&\,]_{=}
& _{_\CNF}[\,1, all, (1)&\,]_{=}
\end{align*}
\caption[Transformation]{Transformation into equisatisfiable clausal forms}
\label{tab:decidedable:CNF}
\end{table}




%%% ADDITIONAL MATERIAL %%%%

%\begin{table}[hbt]
%	\begin{align*}
%	[\,\forall\exists\forall, (\omega, 1), (0)&\,] \tag{Kahr 1962}
%	\\
%	[\,\forall^3 \exists, (\omega, 1), (0)&\,] \tag{Surányi 1959}
%	\\
%	[\,\forall^{∗} \exists, (0, 1), (0)&\,] \tag{Kalmár-Surányi 1950}
%	\\
%	[\,\forall\exists\forall^{∗} , (0, 1), (0)&\,]  \tag{Denton 1960}
%	\\
%	[\,\forall\exists\forall\exists^{∗}, (0, 1), (0)&\,] \tag{Gurevich 1966}
%	\\
%	[\,\forall^3 \exists^{∗} , (0, 1), (0)&\,] \tag{Kalmár-Surányi 1947}
%	\\
%	[\,\forall\exists^{∗} \forall, (0, 1), (0)&\,] \tag{Kostyrko-Genenz 1964}
%	\\
%	[\,\exists^{∗} \forall\exists\forall, (0, 1), (0)&\,] \tag{Surányi 1959}
%	\\
%	[\,\exists^{∗} \forall^3 \exists, (0, 1), (0)&\,] \tag{Surányi 1959}
%	\end{align*}
%	\caption{Undecidable prefix classes in pure predicate logic}
%	\label{tab:undecidable:PurePredicateLogic}
%\end{table}

%\begin{table}[hbt]
%	\begin{align*}
%	[\,\forall, (0), (2)&\,]_{=} \tag{Gurevich 1976}
%	\\
%	[\,\forall, (0), (0, 1)&\,]_{=} \tag{Gurevich 1976}
%	\\
%	[\,\forall^2 , (0, 1), (1)&\,] \tag{Gurevich 1969}
%	\\
%	[\,\forall^2 , (1), (0, 1)&\,] \tag{Gurevich 1969}
%	\\
%	[\,\forall^2\exists, (\omega, 1), (0)&\,]_{=} \tag{Goldfarb 1984}
%	\\
%	[\,\exists^{∗}\forall^2\exists, (0, 1), (0)&\,]_{=} \tag{Goldfarb 1984}
%	\\
%	[\,\forall^2\exists^{∗}, (0, 1), (0)&\,]_{=} \tag{Goldfarb 1984}
%	\end{align*}
%	\caption{Undecidable prefix classes with functions or equality}
%	\label{tab:undecidable:FunctionsAndEquations}
%\end{table}


%\begin{table}[hbt]
%	\begin{gather*}
%		\begin{array}{rcccl}
%		\text{undecidable} &&\multicolumn{1}{c}{decidable}  & &  \text{classification}
%		\\ \\{}
%[\,\Pi_?, (p_1, {\colLo 1}), (0)\,] &\supsetneq&
%[\,\Pi_?, (p_1, {\colHi 0}), (0)\,]&\subseteq&
%[\,all,(\omega),(\omega)\,]
%%\\[0.5em]		
%%		\\ \\{}
%%		[\,\forall\exists\forall, (\omega, 1), (0)\,] &\multirow{9}{*}{$\supsetneq$}&
%%		\multirow{9}{*}{$[\,\Pi, (p_1, {\colHi 0}), (0)\,]$} &\multirow{9}{*}{$\subseteq$}&
%%		\multirow{9}{*}{$[\,all,(\omega),(\omega)\,]$}
%%	\\{}
%%	[\,\forall^3 \exists, (\omega, 1), (0)\,] &&
%%	&&
%%	\\{}
%%	[\,\forall^{∗} \exists, (0, 1), (0)\,] &&
%%	&&
%%	\\{}
%%	[\,\forall\exists\forall^{∗} , (0, 1), (0)\,]&&
%%	&&
%%	\\{}
%%	[\,\forall\exists\forall\exists^{∗}, (0, 1), (0)\,]&&
%%	&&
%%	\\{}
%%	[\,\forall^3 \exists^{∗} , (0, 1), (0)\,] &&
%%	&&
%%	\\{}
%%	[\,\forall\exists^{∗} \forall, (0, 1), (0)\,] &&
%%	&&
%%	\\{}
%%	[\,\exists^{∗} \forall\exists\forall, (0, 1), (0)\,]&&
%%	&&
%%	\\{}
%%	[\,\exists^{∗} \forall^3 \exists, (0, 1), (0)\,] &&
%%	&&
%%\\{}
%\\[0.5em]
%	[\,\forall, (0), ({\colLo 2})\,]_{=} &\supsetneq&
%	[\,\forall, (0), ({\colHi 1})\,]_{=} &\subseteq&[\,all, (\omega), (1)\,]_{=}
%	\\{}
%	[\,\forall, (0), (0, {\colLo 1})\,]_{=} &\supsetneq&
%	[\,\forall, (0), (0, {\colHi 0})\,]_{=}	&\subseteq&[\,all, (\omega), (1)\,]_{=}
%	\\[0.5em]
%	[\,\forall^2 , (0, {\colN 1}), ({\colLo 1})\,]_{\color{white}=} &\supsetneq&
%	[\,\forall^2 , (0, {\colN 1}), ({\colHi 0})\,]_{\color{white}=} &\subseteq&[\,\exists^{∗}\forall^2\exists^{∗} , all, (0)\,]_{\color{white}=}
%	\\{}
%	%
%	{[\,\forall^2 , ({\colN 1}), (0, {\colLo 1})\,]_{\color{white}=}}&\supsetneq&
%	[\,\forall^2 , ({\colN 1}), (0, {\colHi 0})\,]_{\color{white}=}&\subseteq&[\,\exists^{∗}\forall^2\exists^{∗} , all, (0)\,]_{\color{white}=}
%	\\[0.5em]
%	[\,\forall^2\exists, (\omega, {\colLo 1}), (0)\,]_{=} &\subseteq&
%	[\,\forall^2\exists, (\omega, {\colHi 0}), (0)\,]_{=} &\subseteq&[\,all, (\omega), (1)\,]_{=}
%	\\{}
%	%
%	[\,\exists^{∗}\forall^2\exists, (0, {\colLo 1}), (0)\,]_{=}&\supsetneq&
%		[\,\exists^{∗}\forall^2\exists, (0, {\colHi 0}), (0)\,]_{=}&\subseteq&[\,all, (\omega), (1)\,]_{=}
%	\\{}
%	[\,\forall^2\exists^{∗}, (0, {\colLo 1}), (0)\,]_{=}&\supsetneq&
%	[\,\forall^2\exists^{∗}, (0, {\colHi 0}), (0)\,]_{=}&\subseteq& [\,all, (\omega), (1)\,]_{=}
%	\end{array}
%	\end{gather*}
%\end{table}


