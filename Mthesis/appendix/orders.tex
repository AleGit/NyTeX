% !TeX root = ../mthesis.tex
% !TeX encoding = UTF-8
% !TeX spellcheck = en_US


\begin{definition}
	A {\myem partial order} is a reflexive, transitive, and antisymmetric binary relation, 
	i.e.~$\sqsubseteq$ is a partial order on $A$ if
	the following three clauses hold for arbitrary elements $x,y,z$ in $A$.
	\begin{align*}
	x\sqsubseteq x
	\tag*{reflexivivity}
	\\
	x\not\sqsubseteq y \lor  y \not\sqsubseteq z \lor x\sqsubseteq z
	\tag*{transitivity}
	\\
	x\not\sqsubseteq y \lor y\not\sqsubseteq x \lor x = y
	\tag*{antisymmetry}
	\end{align*}
\end{definition}

\begin{definition}
	A {\myem proper order} is a irreflexive and transitive binary relation, 
	i.e.~$\sqsubset$ is a proper order on $A$ if the following two clauses hold.
	\begin{align*}
	x\not\sqsubset x
	\tag*{irreflexivivity}
	\\
	x\not\sqsubset y \lor  y \not\sqsubset z \lor x\sqsubset z
	\tag*{transitivity}
	\end{align*}
	always hold for arbitrary elements $x,y,z$ in $A$.
\end{definition}

\begin{definition}
	A {\myem total order} is a proper order where the following clause holds.
	\[
		x \sqsubset y \lor y \sqsubset x  \lor x=y \tag*{totality}
	\]
\end{definition}

\begin{example}
	By definition the empty relation $\emptyset \subseteq A \times A$ is a
	irreflexive,
	transitive,
	symmetric,
	antisymmetric,
	and asymmetric
	relation. 
	Hence it is a proper order, but not total.
\end{example}

\begin{lemma}
	Any proper order is asymmetric, e.g.~the clause
	\begin{align*}
	x\not\sqsubset y &\lor y\not\sqsubset x \tag*{asymmetry}
	\end{align*}
	always holds for arbitrary $x$ and $y$ in $A$.
\end{lemma}

\begin{proof} We use resolution to derive asymmetry from irreflectivity and transitivity.
	\[
			\infer[\{x'\mapsto x, z\mapsto x \}]{
			x\not\sqsubset y \lor y\not\sqsubset x }{
			{\colHi x\not\sqsubset x} & x'\not\sqsubset y \lor  y \not\sqsubset z \lor {\colLo x'\sqsubset z}
		}
	\]
\end{proof}

\begin{example}The strict subset relation $\subsetneq$ over a power set is a proper order, but not a total, 
	e.g. for the power set over natural numbers we have 
	\[
	\{ 1 \} \not\subsetneq \{ 2 \}
	\land \{ 2 \} \not\subsetneq \{ 1 \}
	\land 	\{ 1 \} \neq \{ 2 \}
	\tag*{non-totality}
	\]
\end{example}



\begin{definition}
	A binary relation $\supset$ is {\myem well-founded} on a set $A$ if there is no infinite sequence 
	$(a_i)_{i\in\mathbb{N}}$ of $a_i\in A$
	with $a_i\supset a_{i+1}$ for all $i\in\mathbb{N}$.
\end{definition}

\begin{example}
	The strict superset relation $\supsetneq$ on finite sets is well-founded.
	The canonical greater than relation $>$ is well founded on natural numbers,
	but not on integers or positive rational numbers.
	\begin{gather*}
	-1 > -2 > \ldots > -(2^i) > -(2^{i+1}) > \ldots\\
	1 > \frac{1}{2} > \frac{1}{4} > \ldots > \frac{1}{2^i} > \frac{1}{2^{i+1}} > \ldots
	\end{gather*}
\end{example}


