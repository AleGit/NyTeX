% !TeX root = tlk16jun15h.tex
% !TeX encoding = UTF-8
% !TeX spellcheck = en_US


% !TeX encoding = UTF-8
% !TeX spellcheck = en_US

% !TeX root = ../m3Headout.tex
% !TeX encoding = UTF-8
% !TeX spellcheck = en_US

%% ==============================================================
%% ### BEAMER ###

\usetheme[]{CambridgeUS}	
\usecolortheme{seagull}			% seahorse, fly, dolphin, dove, beetle, seagull ...
\useinnertheme{circles}			% tree, smoothtree, infolines, smoothbars	
\usefonttheme{professionalfonts} 	% professionalfonts serif structurebold structureitalicserif structuresmallcapsserif
	
\beamertemplatenavigationsymbolsempty

%% ==============================================================
%% ### "PACKAGES" ###

% BY BEAMER: xcolor, amsmath, amsthm, calc, geometry, hyperref, extsizes

\usepackage{calc}

\usepackage[utf8x]{inputenc} 		% Eingabekodierung	
\usepackage[english]{babel}	
\usepackage[T1]{fontenc}	 		% Ausgabekodierung (PDF)
%\usepackage[
	%usenames,
	%dvipsnames,
	%svgnames,
	%table]{xcolor} 				% BEAMER, Farben
%\usepackage{amsmath}				% BEAMTER
%\usepackage{amsthm}				% BEAMTER
\usepackage{amssymb}				% \ngtpreq

\usepackage{proof} 					%\infer, \deduce
%\usepackage{marvosym}				% \Lightning

%\usepackage{soul}					% \so\caps\ul\st\hl
%\usepackage[normalem]{ulem} 		% \uline\uuline\sout\xout\uwave
%\usepackage{transparent}
\usepackage{listings} 			% are there some issues ?
%\usepackage{multirow}
%\usepackage{pdfpages}

%\usepackage[weather]{ifsym} 	% does not work with tex writer
%\let\Sun\relax 				% defined in marvosym too
%\let\Lightning\relax 			% defined in marvosym too

%\usepackage{ulsy}				% \blitza \blitzb ... \blitze (does not exist anymore? just not included?)

\usepackage{tabularx}

\usepackage{pifont} 			% dingbats

% setup listing

\lstdefinelanguage{smtlib}{
	comment={[l];},
	keywords={assert,xor, or, and},
	otherkeywords={declare-fun, set-logic},
	emph={Int,QF_LIA},
}

\lstset {
	backgroundcolor=\color{white},     	% choose the background color; you must add \usepackage{color} or \usepackage{xcolor}
	basicstyle=\ttfamily\footnotesize, 	% the size of the fonts that are used for the code
	breakatwhitespace=false,         	% sets if automatic breaks should only happen at whitespace
	breaklines=true,                 	% sets automatic line breaking
	captionpos=b,                    	% sets the caption-position to bottom
	commentstyle=\color{gray},    		% comment style
	deletekeywords={...},            	% if you want to delete keywords from the given language
	emphstyle=\color{orange},
	escapeinside={\%*}{*)},         % if you want to add LaTeX within your code
	extendedchars=true,             % lets you use non-ASCII characters; for 8-bits encodings only, does not work with UTF-8
	frame=none,		% frame=single, % adds a frame around the code
	keepspaces=true,                % keeps spaces in text, useful for keeping indentation of code (possibly needs columns=flexible)
	keywordstyle=\color{blue},      % keyword style
	% language=smtlib,	%Octave,    % the language of the code
	% literate={;},
	% otherkeywords={declare-fun,set-logic,assert,xor,or,and},            % if you want to add more keywords to the set
	%morecomment=[l]{;}				% line comment
	numbers=left,                   % where to put the line-numbers; possible values are (none, left, right)
	numbersep=5pt,                  % how far the line-numbers are from the code
	numberstyle=\tiny\color{gray}, 	% the style that is used for the line-numbers
	rulecolor=\color{black},        % if not set, the frame-color may be changed on line-breaks within not-black text (e.g. comments (green here))
	showspaces=false,               % show spaces everywhere adding particular underscores; it overrides 'showstringspaces'
	showstringspaces=false,         % underline spaces within strings only
	showtabs=false,                 % show tabs within strings adding particular underscores
	stepnumber=2,                   % the step between two line-numbers. If it's 1, each line will be numbered
	stringstyle=\color{orange},     % string literal style
	tabsize=2,	                   	% sets default tabsize to 2 spaces
	title=\lstname                  % show the filename of files included with \lstinputlisting; also try caption instead of title
}

% !TeX root = ../m3Handout.tex
% !TeX encoding = UTF-8
% !TeX spellcheck = en_US

%% color definitions

%\colorlet{col:a}{Fuchsia}
%\colorlet{col:b}{Blue}
\colorlet{colG}{Gray}
\colorlet{colO}{Orange}
\colorlet{colHi}{Green}
\colorlet{colLo}{Red}
\colorlet{colNa}{Gray}
\colorlet{colN}{Blue}
\colorlet{colEm}{RoyalBlue}
\colorlet{colMax}{Violet}
\colorlet{colSmx}{RoyalBlue}

%\newcommand{\colA}{\color{col:a}}	% example a
%\newcommand{\colB}{\color{col:b}}	% example b
\newcommand{\colG}{\color{colG}}	% neutral
%\newcommand{\colO}{\color{col:o}}	% old
\newcommand{\colN}{\color{colN}}	% new
\newcommand{\colHi}{\color{colHi}}	% hi, true
\newcommand{\colLo}{\color{colLo}}	% lo, false
\newcommand{\colNA}{\color{colNa}}	% not available		
\newcommand{\colMax}{\color{colMax}}	% maximal
\newcommand{\colSmx}{\color{colSmx}}	% strictly maximal		

%\newcommand{\MYa}{\color{Fuchsia}}
%\newcommand{\MYb}{\color{Blue}}
%\newcommand{\MYg}{\color{Gray}}
%\newcommand{\MYo}{\color{Orange}}
%
%\newcommand{\MYhi}{\color{Green}}
%\newcommand{\MYlo}{\color{Red}}
%\newcommand{\MYna}{\color{Gray}}
%
%\newcommand{\MYA}[1]{{\MYa#1}}
%\newcommand{\MYB}[1]{{\MYb#1}}
%\newcommand{\MYG}[1]{{\MYg#1}}
%\newcommand{\MYO}[1]{{\MYo#1}}
%
%\newcommand{\MYHI}[1]{{\MYhi#1}}
%\newcommand{\MYLO}[1]{{\MYlo#1}}
%\newcommand{\MYNA}[1]{{\MYna#1}}
%
%\newcommand{\MYS}[1]{{\color{Violet}#1}}
%\newcommand{\MYM}[1]{{\color{RoyalBlue}#1}}
%
%\newcommand{\mLightning}{{\text{\Lightning}}}

\newcommand{\BEGINX}{\begin{example}}
\newcommand{\ENDX}{\end{example}}

\newcommand{\BEGIN}{\begin{exampleblock}}
\newcommand{\END}{\end{exampleblock}}

\newcommand{\FOL}{$_\mathtt{FOL}$}
% !TeX root = ../m3Handout.tex
% !TeX encoding = UTF-8
% !TeX spellcheck = en_US

%% ==============================================================
%% ### MISC ###

%\newcommand{\EQ}{\simeq}
%\newcommand{\NEQ}{\not\simeq}
\newcommand{\foEQ}{\approx}		%	{\simeq}
\newcommand{\foNEQ}{\not\foEQ}

\newcommand\TOP[2]{\genfrac{}{}{0pt}{}{#1}{#2}}
\newcommand\TOPTEXT[2]{\TOP{\text{#1}}{\text{#2}}}
%\newcommand{\mygreek}[1]{\selectlanguage{polutonikogreek}#1\selectlanguage{english}}
%\newcommand{\mygreek}[1]{{\selectlanguage{polutonikogreek}#1}\selectlanguage{english}}
%\renewcommand{\mygreek}[1]{\foreignlanguage{polutonikogreek}{#1}}

%\newcommand{\iSUB}[2]{#2\!\mapsto\!#1}
%\newcommand{\BgSyntaxTree}{\usebackgroundtemplate{\transparent{0.1}\includegraphics[width=\paperwidth]{SyntaxTreeBackground.png}}}

\newcommand{\EMPH}[1]{\emph{\textcolor{colEm}{#1}}}

%% ==============================================================
%% ### MY MATH ENVIRONMENTS ###

\theoremstyle{plain}
%\newtheorem{theorem}{Theorem}
%\newtheorem{proposition}{Proposition}
%\newtheorem{lemma}{Lemma}
%\newtheorem*{corollary}{Corollary}

\theoremstyle{definition}
%\newtheorem{definition}{Definition}
\newtheorem{Conjecture}{Conjecture}
%\newtheorem*{example}{Example}
%\newtheorem{algorithm}{Algorithm}
\newtheorem{procedure}{Procedure}
\newtheorem{goal}{Goal}
\newtheorem{notation}{Notation}

\theoremstyle{remark}
%\newtheorem*{remark}{Remark}
\newtheorem*{observation}{Observation}
%\newtheorem*{note}{Note}
%\newtheorem{case}{Case}

%% ==============================================================
%% ### MY MATH DEFINITIONS ###

% math operators

%\DeclareMathOperator{\arity}{arity}
%\DeclareMathOperator{\var}{var}
%\DeclareMathOperator{\pos}{pos}
%\DeclareMathOperator{\T}{T}
%\DeclareMathOperator{\dom}{dom}
%\DeclareMathOperator{\rng}{rng}
%\DeclareMathOperator{\img}{img}
\DeclareMathOperator{\mgu}{mgu}	% most general unifier
%\DeclareMathOperator{\wgt}{W\!}
\DeclareMathOperator{\sel}{sel}
%\DeclareMathOperator{\mul}{mul}
%\DeclareMathOperator{\add}{add}

\DeclareMathOperator{\UNIF}{unifiable}
\DeclareMathOperator{\INST}{instance}
\DeclareMathOperator{\GNRL}{generalization}
\DeclareMathOperator{\VRNT}{variant}
\DeclareMathOperator{\PSTR}{pstr}

\newcommand{\NGTPREQ}{\not\succcurlyeq}

% constant (function, predicate) symbols

\newcommand{\mf}{{\mathsf f}}
\newcommand{\mg}{{\mathsf g}}
\newcommand{\mh}{{\mathsf h}}
\newcommand{\ma}{{\mathsf a}}
\newcommand{\mb}{{\mathsf b}}
\newcommand{\mc}{{\mathsf c}}
\newcommand{\md}{{\mathsf d}}
\newcommand{\mx}{{\mathsf x}}
\newcommand{\my}{{\mathsf y}}
\newcommand{\msucc}{{\mathsf s}}
\newcommand{\mpred}{{\mathsf p}}
\newcommand{\mA}{{\mathsf A}}
\newcommand{\mB}{{\mathsf B}}
\newcommand{\mL}{{\mathsf L}}
\newcommand{\mP}{{\mathsf P}}
\newcommand{\mQ}{{\mathsf Q}}

% caligraphic symbols

\newcommand{\mcC}{{\mathcal C}}
\newcommand{\mcD}{{\mathcal D}}
%\newcommand{\mcE}{{\mathcal E}}
%\newcommand{\mcF}{{\mathcal F}}
%\newcommand{\mcM}{{\mathcal M}}
%\newcommand{\mcO}{{\mathcal O}}
\newcommand{\mcP}{{\mathcal P}}
%\newcommand{\mcR}{{\mathcal R}}
%\newcommand{\mcT}{{\mathcal T}}
\newcommand{\mcV}{{\mathcal V}}

\newcommand{\Var}{{}\mcV\mathsf{ar}}
\newcommand{\Dom}{{}\mcD\mathsf{om}}
\newcommand{\Pos}{{}\mcP\mathsf{os}}
\newcommand{\PosStr}{\Pos^\Sigma}

% fraktal symbols

\newcommand{\mfC}{{\mathfrak C}}
\newcommand{\mfL}{{\mathfrak L}}
\newcommand{\mfR}{{\mathfrak R}}
\newcommand{\mfT}{{\mathfrak T}}

\newcommand{\SIGA}{{\mathcal A}}
\newcommand{\SIGC}{{\mathcal C}}
\newcommand{\SIGE}{{\mathcal E}}
\newcommand{\SIGF}{{\mathcal F}}
\newcommand{\SIGL}{{\mathcal L}}
\newcommand{\SIGP}{{\mathcal P}}
\newcommand{\SIGS}{{\mathcal S}}
\newcommand{\SIGT}{{\mathcal T}}
\newcommand{\SIGV}{{\mathcal V}}
% tt symbols

\newcommand{\mtS}{{\mathtt S}}
\newcommand{\sgr}{\succ_{\mathsf gr}}

% 

\newcommand{\TI}[1]{^{^{#1:}}\!}
\newcommand{\ANGLES}[1]{\langle#1\rangle}

\newcommand{\joins}{\rightarrow^*\cdot^*\!\!\leftarrow}
\newcommand{\meets}{^*\!\!\leftarrow \cdot \rightarrow^* }


\newcommand{\mCP}[1]{\mathsf{CP}(#1)}		% Critical Pair
\newcommand{\mCPR}{\mCP{\mcR}}		% CP(R)

\newcommand{\MUL}[2]	% multiplication
{\mf(#1,#2)}			% mul(x,y)
%{#1\cdot #2}			% x.y

\newcommand{\ADD}[2]	% addition
{\add(#1,#2)}			% add(x,y)
%{#1+#2}				% x+y

\newcommand{\MYPOS}[1]{{\tt #1}}
\newcommand{\overlap}[3]{\langle #1,\MYPOS{#2}, #3 \rangle}
\newcommand{\overlapN}[4]{{_{\overlap{#1}{#2}{#3}}}^{#4:\;}}

%\newcommand{\GTKBO}{>_{\tt kbo}}
\newcommand{\GTKBOW}[2]{\texttt{SMT}(#1\!>_\texttt{kbo}\!#2)}
\newcommand{\GTKBOP}[2]{\texttt{SMT}(#1\!>_\texttt{kbo}'\!#2)}

\newcommand{\UPL}{\infer
	[(\sigma)
		\quad\sigma=\mgu(l,l'), l'\!\not\in\mcV, l\sigma\rho\sgr r\sigma\rho
	]
	{L[r]\sigma}
	{l=r & L[l']}	
}

\newcommand{\emptyclause}{\square}

\newcommand{\cmark}{\ding{51}}
\newcommand{\xmark}{\ding{55}}

% !TeX root = ../m3Handout.tex
% !TeX encoding = UTF-8
% !TeX spellcheck = en_US

%% ### TIKZ ###

\usepackage{tikz}
\usetikzlibrary{
	automata,
	arrows,
	% backgrounds
	% decorations % --> pdflatex error
	% decorations.pathmorphing
	% fit,
	% graphs,p
	% petri,
	% positioning,
	% snakes -> decorations
	% shadows,
	shapes,
	trees,
}

\tikzset{
%	->,
%	>=stealth', 
%	shorten >=1pt, 
%	auto,
 	node distance=2.1cm, 
%	semithick,
 	minimum size=0,
 	inner sep=1,
 	outer sep=1mm,
%
 	initial text=$\varepsilon$,
%	
 	every state/.style={
 		fill=red,
%		draw=none,
%		text=white,
 		radius=0.1em
 	},
 	my/.style={ rectangle, draw=red	},
%	sloped,below
 }


\tikzstyle{myrect} = [rectangle,draw=black,rounded corners, minimum height=3em, thick, text centered,text width=5.5em]
\tikzstyle{mykaro} = [diamond,draw=black,rounded corners, thick, text centered,text width=4em]
\tikzstyle{mycircle} = [circle,draw=black,thick, text centered, minimum height=3.5em,text width=4em,text width=3.5em]
\tikzstyle{myarrow} = [thick,->,>=stealth]
%
\tikzstyle{myframe} = [rectangle,draw=black,rounded corners, minimum height=3em, thick, text centered,text width=5.5em]

%% ==============================================================

 \newcommand{\ORIGIN}{
 	\node[orange](ORIGIN) at (0,0) {\scriptsize$\odot$};
 	\node[orange](XONE) at (1,0) {\scriptsize$\times$};
 	}
%% ==============================================================

%copied from  m2report

 \tikzstyle{myrect} = [rectangle,draw=black,rounded corners, minimum height=3em, thick, text centered,text width=5.5em]
 \tikzstyle{mykaro} = [diamond,draw=black,rounded corners, thick, text centered,text width=4em]
 \tikzstyle{mycircle} = [circle,draw=black,thick, text centered, minimum height=3.5em,text width=4em,text width=3.5em]
 \tikzstyle{myarrow} = [thick,->,>=stealth]

 \tikzstyle{myframe} = [rectangle,draw=black,rounded corners, minimum height=3em, thick, text centered,text width=5.5em]

% clear command for final version

\renewcommand{\ORIGIN}{}
\providecommand{\PAUSE}{}






% ********************************

\author{Alexander Maringele}
\title{flea\\
}
%\subtitle{of an instantiation-based first order theorem prover with equality}
\subtitle{bit(e)s and pieces}
%\subtitle{in easy examples}
\date{June 15th, 2016}
%======

%\includeonly{DiscriminationTree}
\begin{document}

\selectlanguage{english}

\frame[<+->]{
\maketitle
} 

\frame{
	
	
	\begin{quote}
		Hofstadter's Law: It always takes longer than you expect, 
		
		even when you take into account Hofstadter's Law.
	\end{quote}
	 \hfill--- Douglas Hofstadter, Gödel, Escher, Bach: An Eternal Golden Braid
	}

\frame{
	\setcounter{tocdepth}{1}
	\tableofcontents}

%\include{Title}
\section*{References}
\frame[<+->]{
\frametitle{References}

\nocite{NHRV2001ote}
%\nocite{ SRV2001ti,ZHM2009jar,RV2003eir,NHRV2001ote}	% ZHM2009jar, RV2003eir, NHRV2001ote
\bibliographystyle{amsalpha}
\bibliography{biblio}

}
%\include{Overview}

%====================================================================
% BEGIN: CONTENT ----------------------------------------------------
%====================================================================




%% !TeX spellcheck = en_US
% !TeX root = ../m3Handout.tex
% !TeX encoding = UTF-8
\begin{example}[forward subsumption]
			\vspace{-1em}
	\begin{gather*}
		S = \{ \TI{1}\mP(x,y), \TI{2}\lnot \mP(\ma,z)\} \cup \{\TI{3}\colG\mP(\ma,z') \}
		\tag*{$t_1$ subsumes $t_3$}
		\\[0.7em]
		\infer[\{x\mapsto\ma,y\mapsto z\}]
		{\square}
		{\mP(x,y) & \lnot \mP(\ma,z)}
		\tag*{Resolution}
		\\[0.7em]
		S\bot = \{ \mP(\bot,\bot), {\colLo\lnot \mP(\ma,\bot), \mP(\ma,\bot)} \}
		\tag*{InstGen / SMT}
	\end{gather*}
	
\end{example}

\section{Previously}

\begin{frame}
	% We transform a first formula in to a equisatisfiable set of clauses,
	% i.e. the conjunction of universally quantified disjunction of variable distinct first order literals
	% then we check if the set is satisfiable, if not then the formula is a theorem.
	\frametitle{Goal}
	
% !TeX spellcheck = en_US
% !TeX encoding = UTF-8

\begin{tikzpicture}[scale = 1, transform shape, draw=black, fill=black, thick, sloped]

	\draw[myarrow, ultra thick] (0,0) -- 
	node[pos=0, above] {$F$} 
	(2,0);
		
	% outer rectangle
	\draw[rounded corners=1.5mm,dotted] (0.5,3) rectangle (8.5,-3);
	% is F a theorem?
	\draw(1.9,2.7) node {Is $F$ a theorem?};
	
\pause
	% SLIDE Is S satisfiable?
	\node[colG] (S) at (1.1,-2.7) {\scriptsize$\lnot F \approx S$};
\pause
		\node (S) at (2.5,0) {$S$};
		

			% SLIDE 2
			\draw[thin,dashed,draw=colO] (2.5,0) ellipse (0.4 and 1.2); % S
			
			% inner rectangle
			\draw[very thick,draw=DarkGray]  (1.5,-2.25) rectangle (8,2.25); 
			% is S satisfaible?
			\draw (2.9,1.9) node {Is $S$ satisfiable?};
	
\pause
		% SLIDE unsatisfiable
		\draw[thin,dashed,draw=colO] (2.6,0) ellipse (0.6 and 1.44);  % S
		\draw[dashed, draw=colG, thick] 
		decorate[decoration={snake}] 
		{(1.4, 1) -- (8.2,0.6)};
		\draw[myarrow, draw=colHi, ultra thick] (6.5,1.8) -- 
			node[pos=0,below] {unsatisfiable}
			node[pos=0.85, above] {theorem} 
			(10,1.8) ;

\pause
		% SLIDE satisfiable
		\draw[thin,dashed,draw=colO] (2.8,0) ellipse (0.9 and 1.73);  % S
		\draw[dashed, draw=colG, thick]  
		decorate[decoration={snake}] 
		{ (1.4,-1)  --  (8.2,-0.6) };
		\draw[myarrow,draw=colLo, ultra thick] (7,-1.3) -- 
			node[pos=0, below] {satisfiable}
			node[pos=0.75, above] {not a theorem} (11,-1.3) ;
		 
\pause
		% SLIDE 5
		 \draw[thin,dashed,draw=colO] (3.2,0) ellipse (1.35 and 2.07); % S
		 	\draw[myarrow,draw=colNa, ultra thick] (7,0.15) -- 
		 	%	node[pos=0,above] {space out}
		 	node[pos=0,below] {time out}
		 	node[pos=0.85, above] {maybe} (10.5,0.15) ;

	\onslide<1->
\end{tikzpicture}

\end{frame}

\subsection{First-order refutation}

\begin{frame}
	% We transform a first formula in to a equisatisfiable set of clauses,
	% i.e. the conjunction of universally quantified disjunction of variable distinct first order literals
	% then we check if the set is satisfiable, if not then the formula is a theorem.
	\frametitle{Goal}
	
	% !TeX spellcheck = en_US
% !TeX encoding = UTF-8

\begin{tikzpicture}[scale = 1, transform shape, draw=black, fill=black, thick, sloped]

	\draw[myarrow, ultra thick] (0,0) -- 
	node[pos=0, above] {$F$} 
	(2,0);
		
	% outer rectangle
	\draw[rounded corners=1.5mm,dotted] (0.5,3) rectangle (8.5,-3);
	% is F a theorem?
	\draw(1.9,2.7) node {Is $F$ a theorem?};
	
\pause
	% SLIDE Is S satisfiable?
	\node[colG] (S) at (1.1,-2.7) {\scriptsize$\lnot F \approx S$};
\pause
		\node (S) at (2.5,0) {$S$};
		

			% SLIDE 2
			\draw[thin,dashed,draw=colO] (2.5,0) ellipse (0.4 and 1.2); % S
			
			% inner rectangle
			\draw[very thick,draw=DarkGray]  (1.5,-2.25) rectangle (8,2.25); 
			% is S satisfaible?
			\draw (2.9,1.9) node {Is $S$ satisfiable?};
	
\pause
		% SLIDE unsatisfiable
		\draw[thin,dashed,draw=colO] (2.6,0) ellipse (0.6 and 1.44);  % S
		\draw[dashed, draw=colG, thick] 
		decorate[decoration={snake}] 
		{(1.4, 1) -- (8.2,0.6)};
		\draw[myarrow, draw=colHi, ultra thick] (6.5,1.8) -- 
			node[pos=0,below] {unsatisfiable}
			node[pos=0.85, above] {theorem} 
			(10,1.8) ;

\pause
		% SLIDE satisfiable
		\draw[thin,dashed,draw=colO] (2.8,0) ellipse (0.9 and 1.73);  % S
		\draw[dashed, draw=colG, thick]  
		decorate[decoration={snake}] 
		{ (1.4,-1)  --  (8.2,-0.6) };
		\draw[myarrow,draw=colLo, ultra thick] (7,-1.3) -- 
			node[pos=0, below] {satisfiable}
			node[pos=0.75, above] {not a theorem} (11,-1.3) ;
		 
\pause
		% SLIDE 5
		 \draw[thin,dashed,draw=colO] (3.2,0) ellipse (1.35 and 2.07); % S
		 	\draw[myarrow,draw=colNa, ultra thick] (7,0.15) -- 
		 	%	node[pos=0,above] {space out}
		 	node[pos=0,below] {time out}
		 	node[pos=0.85, above] {maybe} (10.5,0.15) ;

	\onslide<1->
\end{tikzpicture}
	
\end{frame}

\subsection{Resolution and InstGen}

\frame{
	\begin{Definition}[Ordered Resolution]
		% !TeX spellcheck = en_US
% !TeX encoding = UTF-8


\[
	\infer
	[]
	{(C \lor D)\sigma}
	{A \lor C& \lnot B\lor D}
%		\qquad
%		\infer[]
%		{C\sigma}
%		{A\lor\lnot B\lor C}
		\]
		where 
		\begin{center}
		$A\sigma$ strictly maximal in $\mcC\sigma$, $\lnot B\sigma$ maximal in $\mcD\sigma$, $\sigma=\mgu(A,B)$.
		\end{center}
		\end{Definition}
	
	\begin{Definition}[Inst-Gen]
	% !TeX spellcheck = en_US
% !TeX encoding = UTF-8


\[
	\infer
	[]
	{(A \lor \mcC)\sigma\quad(\lnot B\lor \mcD)\sigma}
	{A \lor \mcC& \lnot \mc\lor D}
	\]
	where
	\[
		\sel(A\lor\mcC) = A\qquad\sel(\lnot B\lor \mcD) = \lnot{B}\qquad\sigma=\mgu(A,B)
	\]
\end{Definition}

%\begin{Example}[Unsatisfiable set of clauses]
%	$S = \{ \mP(x)\lor\lnot\mP(y), \lnot\mP(\ma),\mP(\mb)\}$
%\end{Example}
}

\subsection{Examples}
\frame{
	\begin{Example}[Resolution]
		\vspace{-1em}
		% !TeX root = tlk16jun15h.tex

	\vspace{-1.5em}
\begin{gather*}
	S = 
	\{ 
	\mP(x) \lor \lnot\mP(y), 
	\lnot \mP(\ma), 
	\mP(\mb) 
	\}\\[0.5em]
\infer[y\mapsto\mb]{\emptyclause}{
\infer[x\mapsto\ma]{\lnot\mP(y)}{\mP(x)\lor \lnot \mP(y) & \lnot\mP(\ma)} & \mP(\mb)
	}
			\end{gather*}
		\end{Example}
		
			\begin{Example}[Inst-Gen]
				\vspace{-1em}
				\begin{align*}
	S_0\bot =&\  
	\{ 
	{\colHi\mP(\bot)} \lor \lnot\mP(\bot), 
	{\colHi\lnot \mP(\ma)}, 
	{\colHi\mP(\mb)} 
		\} \tag{satisfiable} \\
		&\quad 
		\infer[x\mapsto\ma]{
			\mP(\ma)\lor\lnot\mP(y)
			}{\mP(x)\lor\lnot\mP(y))&\lnot\mP(\ma)}
			\\
			S_1\bot \supsetneq&\  
			\{  
			{\colHi\lnot \mP(\ma)}, 
			{\colHi\mP(\mb)},
				{\colLo\mP(\ma)} \lor {\colHi \lnot \mP(\bot) } \}
				\tag{satisfiable}
				\\
				&\quad
				\infer[y\mapsto\mb]{
					\mP(\ma)\lor\mP(\mb)}{
					\mP(\mb)&\mP(\ma)\lor\lnot\mP(y)
					}
					\\
					S_2\bot \supsetneq&\ 
					\{ \lnot\mP(\ma), \mP(\mb), \mP(\ma) \lor \lnot\mP(\mb) \} \tag{unsatisfiable}
	\end{align*}
			\end{Example}
		}
		
	

		



\section{Implementation}

\subsection{Tools}

\subsection{Structure}



\subsection{Indexing}


%\include{Motivation}
%
%\include{Position}
%
%\include{PathIndexing}
%
%\ORIGIN

\PAUSE\node (root) at (1,0) {.};
\node (h) at (0,-1) {.};
\path (root) edge node {$\mh$} (h);
\PAUSE\node (hf) at (-1,-2) {.};
\path (h) edge node {$\mf$} (hf);
\PAUSE\node (hfx) at (-2,-3) {.};
\path (hf) edge node {$*$} (hfx);
\PAUSE\node (hfxx) at (-3,-4) {.};
\path (hfx) edge node {$*$} (hfxx);
\PAUSE\node (1) at (-3,-4.2) {\scriptsize$t_1$};

\PAUSE\node (hfxh) at (-2,-4) {.};
\path (hfx) edge node {$\mh$} (hfxh);
\PAUSE\node (hfxha) at (-2,-5) {.};
\path (hfxh) edge node {$\ma$} (hfxha);
\PAUSE\node (2) at (-2,-5.2) {\scriptsize$t_2$};

\PAUSE\node (hfh) at (-1,-3) {.};
\path (hf) edge node {$\mh$} (hfh);
\PAUSE\node (hfha) at (-1,-4) {.};
\path (hfh) edge node {$\ma$} (hfha);
\PAUSE\path[dashdotted,bend left=45,gray](hf) edge (hfha);
\PAUSE\node (hfhaa) at (-1,-5) {.};
\path (hfha) edge node {$\ma$} (hfhaa);
\PAUSE\node (3) at (-1,-5.2) {\scriptsize$t_3$};


%
%\include{SubstitutionTrees}
%
%\include{Experiences}

\section{Equality}
%\subsection{Schemata}

\begin{frame}
	\begin{block}{Schemata}
		\vspace{-1em}
		\begin{align*}		
			&x=x &s\neq s
			\\
			{\colG x\neq y\ \lor}\ 
				&y=x &s \neq t
			\\
			{\colG x\neq y \lor y\neq z\ \lor}\ 
				&x=z &s\neq t
			\\
			{\colG x_1\neq y_1 \lor x_2\neq y_2 \ \lor}\ 
			&\mf(x_1,x_2) = \mf(y_1,y_2) 
				&\mf(s)\neq\mf(t)
			\\
		    {\colG x\neq y\ \lor}\ &\lnot\mP(x) \lor \mP(y) 
			    &\mP(s), \lnot\mP(t)
			\end{align*}
		\end{block}
%		
		\begin{example}
			\vspace{-1em}
			\begin{align*}
			S =&\ \{ \mP(\ma), \lnot\mP(\mf(\ma,\mb)), \mf(x,\mb)= x \} 
			\\[0.5em]
%			&\quad\infer[
%			]{
%				\ma\neq y \lor \lnot \mP(\ma) \lor \mP(y)
%				}{
%				x\neq y \lor \lnot\mP(x) \lor \mP(y)  & \mP(\ma)
%			}
%				\\[0.5em]
%				&\quad\infer[
%				]{
%					x\neq \mf(\ma,\mb) \lor \lnot \mP(x) \lor \mP(\mf(\ma,\mb))
%				}{
%				x\neq y \lor \lnot\mP(x) \lor \mP(y) & \lnot\mP(\mf(\ma,\mb))
%			}
&\quad \ma\neq y \lor \lnot \mP(\ma) \lor \mP(y) \tag*{congruence}\\
&\quad x\neq \mf(\ma,\mb) \lor \lnot \mP(x) \lor \mP(\mf(\ma,\mb)) \tag*{congruence}
\end{align*}
			\end{example}
			
			\end{frame}


%\frame{
%	$x = \ma \lor x\neq \ma$\hfill  $\bot = \bot \lor \bot \neq \bot$
%	
%	$\mf(\ma) \neq \mf(\mb)$\hfill  $\mf(\bot) \neq \mf(\ma)$
%	
%	$R = \{ x = \ma \}$ is ground complete
%	
%	$\sigma= \{x\mapsto\mb\}$ $(x=\ma) \sigma = \ma \to \mb$ with $\ma > \mb$ $\mf(a) \neq \mb$
%	
%	
%	
%	
%	}
%	
%	\frame {
%		% see PUZ063-1.p 
%		
%		$\mP(\ma), \lnot\mP(\mf(\ma,\mb)), \mf(x,\mb)=x$
%		
%		$\mP(\ma), \lnot\mP(\mf(\ma,\mb)), \mf(\bot,\mb)=\bot$
%		
%		$\{\mf(x,\mb)=x\}$ is ground complete and with $\{ x\mapsto \ma \}$ we get $\lnot\mP(\ma)$
%		}
 
\end{document}