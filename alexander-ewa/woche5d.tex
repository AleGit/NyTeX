\lecture{Thema analysieren und in Form bringen}{woche 5}

\section{Zusammenfassung}
\begin{frame}
\frametitle{Zusammenfassung der letzten LVA}  

\onslide<2->  
\begin{Definition}
  \begin{itemize}
  \item<3-> Ein \RED{Zitat} ist die wortwörtliche Wiederholung
  \item<4-> Eine \RED{Paraphrase} bezeichnet die Darstellung des Gedanken eines 
    anderen in eigenen Worten
  \end{itemize}
\end{Definition}
\onslide<5->
\medskip
\begin{mybox}{Verweis auf Webseiten}
\begin{itemize}
\item<6-> Quellen die nur online verfügbar sind können unter der Angabe
des Links zitiert werden
\item<7-> Einzelne Webseiten nur dann zitieren, wenn diese stabil sind (und dann
als Fußnote)
\item<8-> Wenn auf den Inhalt von fluktuierenden Seiten verwiesen wird, muss
das Datum des  Zugriffs beigefügt werden
\item<9-> Es gibt keine verbindlichen Regeln, ob einzelne Webseiten auch im
Literaturverzeichnis aufgenommen werden können
\end{itemize}
\end{mybox}
\end{frame}

\begin{frame}[fragile]
\frametitle{Proseminaraufgaben (für den 7.~April)}    

\begin{itemize}
\item<2-> Plagiatsfälle:
\begin{enumerate}
\item<3-> \href{http://de.wikipedia.org/w/index.php?title=Annette_Schavan}{Annette Schavan, Deutsche Ministerin für Bildung und Forschung.}
\item<4-> \href{https://de.wikipedia.org/wiki/Christian_Buchmann}{Christian Buchmann, Steirischer Landesrat für Wirtschaft.}
\end{enumerate}
\end{itemize}

\uncover<5->{%
\begin{quote}
  \GREEN{„{Im Gegensatz zu Österreich hat das wissenschaftliche Plagiat in Amerika stärkere Konsequenzen: 
%  		„In den USA werden solche Fälle in der Regel von „Honor Boards“, deren Mitglieder Studenten sind behandelt. 
%  		Auf Basis des an der jeweiligen Hochschule geltenden „Code of Honor“ schlagen diese geeignete Strafen 
[\ldots] (bis hin zur Exmatrikulation) [\ldots].“
%vor.“ 
  		(vgl. Schlonsok, Bernadette (9.~2005): Zur Problematik der Plagiate.}}%
\only<6->{\footnote{Nach~\url{http://www.unet.univie.ac.at/~a0301287/Strafrecht.htm}, 2.~April, 2014; Orginallink existiert nicht mehr. (eventuell im Google Cache)}}
\end{quote}
}


\begin{itemize}
\item<7-> Nennen Sie zumindest 3 Schreibhürden
  \begin{enumerate}
  \item<8-> Schreiben kann man oder nicht
  \item<9-> Perfekt oder gar nicht
  \item<10-> Ich kann nicht Englisch
  \end{enumerate}
\end{itemize}

\end{frame}

\section{Inhalte}
\begin{frame}
\frametitle{Inhalte der Lehrveranstaltung}

\begin{mybox}{Erarbeiten und Verstehen von Texten}
Texte verstehen bzw.\ in eigenen Worten zusammenfassen,
Literaturrecherche, Recherchen im Internet, richtig zitieren
\end{mybox}

\bigskip

\begin{mybox}{Form und Struktur einer Arbeit} 
\alert<2->{Textsorten: Seminar-, Bachelor- und Masterarbeiten, Thema analysieren und in Form bringen}
\end{mybox}

\bigskip

\begin{mybox}{\LaTeX}
Eingabefile, Setzen von Text, bzw. von Bildern, 
Setzen von mathematischen Formeln,  Seitenaufbau,
Schriften, Spezialfälle
\end{mybox}

\bigskip

\begin{mybox}{Bewertung, Prüfung und Präsentation von Arbeiten}
Bewerten von anderen Arbeiten, Das review System in der Informatik,
Präsentieren: eine Einführung
\end{mybox}
\end{frame}

\section{Form und Struktur einer Arbeit}
\begin{frame}
\frametitle{Textsorte: Seminararbeit}

\bigskip
\begin{tabular}{c@{\hspace{5mm}}l}
  \begin{minipage}[b]{.35\linewidth}
    \fbox{\includegraphics[height=6cm,width=4cm]{cls-template}}
  \end{minipage}

&
\begin{minipage}[b]{.6\linewidth}
  \begin{itemize}
  \item<2-> 15--30 Seiten
  \item<3-> Zusammenfassung/Erläuterung bestehender wissenschaftlicher Arbeiten
  \item<4-> Kein Anspruch auf Originalität, aber Vollständigkeit
  \item<5-> Eigener Beitrag besteht meist in der Aufbereitung (= gefälliger Darstellung) der Arbeiten
  \end{itemize}
\end{minipage}
\end{tabular}
\end{frame}

\begin{frame}
\frametitle{Textsorte: Bachelorarbeit}

\begin{tabular}{c@{\hspace{5mm}}l}
  \begin{minipage}[b]{.35\linewidth}
    \fbox{\includegraphics[height=6cm,width=4cm]{clb-template}}
  \end{minipage}

&
\begin{minipage}[b]{.6\linewidth}
  \begin{itemize}
  \item<2-> 30--60 Seiten
  \item<3-> Im Rahmen der Bachelorarbeit wird ein Projekt mit einem Arbeitsaufwand
    von 500 Stunden abgewickelt, die Bachelorarbeit beschreibt dieses Projekt
  \item<4-> Üblicherweise ist das Bachelorprojekt ein Programmierprojekt
  \item<5-> Kein Anspruch auf Originalität, aber Darstellung der erzielten Ergebnisse
  \item<6-> Der Übergang von einer Seminararbeit zur Bachelorarbeit kann, je nach Thema, fließend sein
  \end{itemize}
\end{minipage}
\end{tabular}
\end{frame}

\begin{frame}
\frametitle{Textsorte: Masterarbeit}

\begin{tabular}{c@{\hspace{5mm}}l}
  \begin{minipage}[b]{.35\linewidth}
    \fbox{\includegraphics[height=6cm,width=4cm]{clm-template}}
  \end{minipage}

&
\begin{minipage}[b]{.6\linewidth}
  \begin{itemize}
  \item<2-> 60--100 Seiten
  \item<3-> Zusammenfassung, Erläuterung, und eventuell Implementierung bestehender 
    wissenschaftlicher Arbeiten
  \item<4-> Im Gegensatz zu einer Seminarabeit wird in der Masterarbeit erwartet, dass
    neue Erkenntnisse eingebracht werden
  \item<5-> Eigener Beitrag besteht meist in der Aufbereitung, aber auch
    Verallgemeinerung der Arbeiten
  \item<6-> Idealerweise führen Masterarbeiten direkt zu (wissenschaftlichen) Veröffentlichungen
  \end{itemize}
\end{minipage}
\end{tabular}
\end{frame}

\section{Thema analysieren und in Form bringen}
\begin{frame}[fragile]
\frametitle{\enquote{Dem Inhalt eine Struktur geben}}

%\uncover<2->{
%\begin{mybox}{Inhaltsverzeichnis}
%  \small
%  \color{green!30!black}
%  \begin{verbatim}
%    \tableofcontents
%  \end{verbatim}
%\vspace{-8mm}
%\end{mybox}
%}
%
%\bigskip
%\begin{mybox}<2->{Einleitung}
%Hier wird die Arbeit in Kurzform vorgestellt und motiviert 
%\end{mybox}
%
%\bigskip
%\begin{mybox}{Hauptteil}
%Beschreibung und Analyse des Themas  
%\end{mybox}
%
%\bigskip
%\begin{mybox}{Schlussfolgerung}
%Wiederholung des Themas und Analyse in Bezug auf die Motivation  
%\end{mybox}
%
%\bigskip
%\begin{mybox}{Literaturverzeichnis}
%  \small
%  \color{green!30!black}
%  \begin{verbatim}
%    \bibliographystyle{plain}
%    \bibliography{references}
%  \end{verbatim}
%\vspace{-8mm}
%\end{mybox}

\begin{mybox}{}
	\onslide<2->
	\begin{itemize}
		\item<2-> Titelseite
		\item<3-> Abstract
		\item<4-> Inhaltsverzeichnis
		\item<5-> Einleitung
		\item<6-> \textbf{Hauptteil}
		\item<7-> Schlussfolgerung
		\item<8-> Literaturverzeichnis
		\item<9-> Anhang
	\end{itemize}
\end{mybox}
\textbf{}
\onslide<10->
	\begin{mybox}{}
	Nach dem (teilweisen) Lesen und Verstehen der für Ihre Arbeit relevanten Literatur 
	beginnen Sie mit dem Verfassen des Hauptteils.
\end{mybox}
\end{frame}

\begin{frame}
	\frametitle{Titelseite, Abstract und Inhaltsverzeichnis}
	\framesubtitle{Der erste Eindruck zählt}

\onslide<2->	
	\begin{mybox}{}
		\begin{itemize}
			\item Die Titelseite enhält zumindest den Titel, 
			das Datum und die Namen der AutorInnen und BetreuerInnen der Arbeit.
			\item<3-> \texttt{\textbackslash title\{...\} 
				\textbackslash date\{...\} 
				\textbackslash author\{...\}
				\textbackslash supervisor\{...\}
				\textbackslash maketitle}
		\end{itemize}
	\end{mybox}
\bigskip
\onslide<4->
\begin{mybox}{}
	\begin{itemize}
		\item Das Abstract ist ein kurze und prägnante Zusammenfassung der Arbeit ohne Wertung oder Referenzen.
		\item<5-> Schreiben Sie das Abstract \emph{nach} dem fertigestellten Hauptteil und auch
		nach Einleitung und Zusammenfassung.
		\item<6-> \texttt{\textbackslash begin\{abstract\} ... \textbackslash end\{abstract\}}
	\end{itemize}
\end{mybox}
\bigskip
\onslide<7->
\begin{mybox}{}
	\begin{itemize}
		\item Das Inhalstverzeichnis verweist auf (Unter-) Kapitel und Abschnitte.
		\item<8> \texttt{\textbackslash tableofcontents}
	\end{itemize}
\end{mybox}
\end{frame}

\begin{frame}
\frametitle{Einleitung}  
\framesubtitle{Hier wird die Arbeit in Kurzform vorgestellt und motiviert}

\begin{itemize}
\item<2-> Seien Sie sehr präzise, wenn Sie die Einleitung schreiben
\item<3-> Die Leserin muss eine Idee dafür bekommen, welche Themen die Arbeit behandelt
\item<4-> Die Einleitung endet mit einer detaillierten Beschreibung der Struktur
der Arbeit
\item<5-> Schreiben Sie die Einleitung nach dem fertiggestellten Hauptteil
\end{itemize}

\medskip
\onslide<7->
\begin{Beispiel}  
\GREEN{% 
This  document gives some  hints on  how to  structure and  organize a
thesis.   It does  not contain  explicit help  on \LaTeX.   For that
issue please refer to a short introduction in German~\cite{LAT01} or a
not  so  short  introduction  in English~\cite{LAT02}.   To  ensure  a
uniform  layout   this  note  further  fixes   some  conventions  when
typesetting in \LaTeX\ and lists some useful packages.%
}
\end{Beispiel}
\end{frame}

\begin{frame}
\frametitle{Hauptteil}  
\framesubtitle{Beschreibung und Analyse des Themas}

\onslide<2->
\begin{mybox}{Strukturierung}
\begin{itemize}
\item<3-> Strukturieren Sie die Arbeit in Kapitel und Unterkapitel, sodass ein Kapitel
  eine logische Einheit beschreibt
\item<4-> Beginnen Sie Sektionen mit einem kurzen Absatz, der den Inhalt
  beschreibt
\item<5-> Vermeiden Sie zu lange beziehungsweise zu kurze Kapitel
\end{itemize}
\end{mybox}

\onslide<6->
\bigskip
\begin{mybox}{Formatierung}
  \begin{itemize}
  \item<7-> Auch im Englischen werden die Worte in Überschriften 
groß geschrieben
  \item<8-> Verwenden Sie dedizierte Umgebungen für Programmlistings, Tabellen, Grafiken, etc.
  \end{itemize}
\end{mybox}
\end{frame}

\begin{frame}
\frametitle{Schlussfolgerung}
\framesubtitle{Wiederholung des Themas und Analyse in Bezug auf die Motivation}

\begin{itemize}
\item<2-> Die Themen der Arbeit werden noch einmal vorgestellt
\item<3-> Die Ergebnisse der Arbeit werden mit der Motivation in der Einleitung
  verglichen
\item<4-> Beschreiben Sie die eigenen Arbeit
\item<5-> Eventuell gehen Sie auf zukünftige Arbeit und ähnliche Arbeiten ein
\item<6-> Schreiben Sie die Schlussfolgerung nach dem fertiggestellten Hauptteil
\end{itemize}

\onslide<7->
\bigskip
\begin{Beispiel}
\GREEN{
This note gives a comprehensive guide for computational logic students on
how to organize their scientific documents. In order to get started
with \LaTeX\ some useful packages are mentioned.
}  
\end{Beispiel}
\end{frame}

\begin{frame}
\frametitle{Literaturverzeichnis}
\small
% \bibliographystyle{plain}
% \bibliography{biblio}
\begin{thebibliography}{1}

\bibitem<2->{LAT02}
T.~Oetiker, H.~Partl, I.~Hyna, and E.~Schlegl.
\newblock The not so short introduction to {L}a{T}e{X}, 2015.
\newblock \url{http://ctan.org/tex-archive/info/lshort/english}.

\bibitem<3->{LAT01}
M.~Daniel, P.~Gundlach, W.~Schmidt, J.~Knappen, H.~Partl, and I.~Hyna.
\newblock {L}a{T}e{X}-{K}urzbeschreibung, 2016.
\newblock \url{http://ctan.org/tex-archive/info/german/LaTeX2e-Kurzbeschreibung}.

\end{thebibliography}

\end{frame}

\begin{frame}
\frametitle{Proseminaraufgaben (für den 28.~April)}    

\begin{enumerate}
\item<2-> Lesen Sie das Kapitel \enquote{Lust statt Last: Wissenschaftliche Texte schreiben}
von Norbert Frank, Sektion 4
\item<3-> Lesen Sie \enquote{How to Write a Thesis} von Harald Zankl
\end{enumerate}
\end{frame}
%%% Local Variables:
%%% mode: latex
%%% TeX-master: "slides"
%%% End:

