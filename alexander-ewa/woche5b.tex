\lecture{Thema analysieren und in Form bringen}{woche 5}

\section{Zusammenfassung}
\begin{frame}
\frametitle{Zusammenfassung der letzten LVA}  
  
\begin{Definition}
  \begin{itemize}
  \item Ein \RED{Zitat} ist die wortwörtliche Wiederholung
  \item Eine \RED{Paraphrase} bezeichnet die Darstellung des Gedanken eines 
    Anderen in eigenen Worten
  \end{itemize}
\end{Definition}

\medskip
\begin{mybox}{Verweis auf Webseiten}
\begin{itemize}
\item Quellen die nur online verfügbar sind können unter der Angabe
des Links zitiert werden
\item Einzelne Webseiten nur dann zitieren, wenn diese stabil sind (und dann
als Fußnote)
\item Wenn auf den Inhalt von fluktuierenden Seiten verwiesen wird, muss
das Datum des  Zugriffs beigefügt werden
\item Es gibt keine verbindlichen Regeln, ob einzelne Webseiten auch im
Literaturverzeichnis aufgenommen werden können
\end{itemize}
\end{mybox}
\end{frame}

\begin{frame}[fragile]
\frametitle{Proseminaraufgabe (für den 20.~April)}    

\begin{itemize}
\item Zwei Plagiatsfällen:
\begin{enumerate}
\item<2-> \href{http://de.wikipedia.org/w/index.php?title=Annette_Schavan&oldid=139635798}{Annette Schavan}
\item<2-> \href{http://en.wikipedia.org/w/index.php?title=Hello,_I_Love_You&oldid=649051368}{``Hello, I Love You'' von The Doors}
\end{enumerate}
\end{itemize}

\uncover<3->{%
\begin{quote}
  \GREEN{\enquote{Im Gegensatz zu Österreich hat das wissenschaftliche Plagiat in Amerika stärkere Konsequenzen: „In den USA werden solche Fälle in der Regel von „Honor Boards“, deren Mitglieder Studenten sind behandelt. Auf Basis des an der jeweiligen Hochschule geltenden „Code of Honor“ schlagen diese geeignete Strafen (bis hin zur Exmatrikulation) vor.“ (vgl. Schlonsok, Bernadette (9.~2005): Zur Problematik der Plagiate.}}%
\uncover<3->{\footnote{Nach~\url{http://www.unet.univie.ac.at/~a0301287/Strafrecht.htm}, 2.~April, 2014; Orginallink existiert nicht mehr.}}
\end{quote}}

\begin{itemize}
\item Nennen Sie zumindest 3 Schreibhürden
  \begin{enumerate}
  \item<4-> Schreiben kann man oder nicht
  \item<4-> Perfekt oder gar nicht
  \item<4-> Ich kann nicht Englisch
  \end{enumerate}
\end{itemize}
\end{frame}

\section{Inhalte}
\begin{frame}
\frametitle{Inhalte der Lehrveranstaltung}

\begin{mybox}{Erarbeiten und Verstehen von Texten}
Texte verstehen bzw.\ in eigenen Worten zusammenfassen,
Literaturrecherche, Recherchen im Internet, richtig zitieren
\end{mybox}

\bigskip

\begin{mybox}{Form und Struktur einer Arbeit} 
\alert<2->{Textsorten: Seminar-, Bachelor- und Masterarbeiten, Thema analysieren und in Form bringen}
\end{mybox}

\bigskip

\begin{mybox}{\LaTeX}
Eingabefile, Setzen von Text, bzw. von Bildern, 
Setzen von mathematischen Formeln,  Seitenaufbau,
Schriften, Spezialfälle
\end{mybox}

\bigskip

\begin{mybox}{Bewertung, Prüfung und Präsentation von Arbeiten}
Bewerten von anderen Arbeiten, Das review System in der Informatik,
Präsentieren: eine Einführung
\end{mybox}
\end{frame}

\section{Form und Struktur einer Arbeit}
\begin{frame}
\frametitle{Textsorte: Seminararbeit}

\bigskip
\begin{tabular}{c@{\hspace{5mm}}l}
  \begin{minipage}[b]{.35\linewidth}
    \fbox{\includegraphics[height=6cm,width=4cm]{cls-template}}
  \end{minipage}

&
\begin{minipage}[b]{.6\linewidth}
  \begin{itemize}
  \item 15--30 Seiten
  \item Zusammenfassung/Erläuterung bestehender wissenschaftlicher Arbeiten
  \item Kein Anspruch auf Originalität, aber Vollständigkeit
  \item Eigener Beitrag besteht meist in der Aufbereitung (= gefälliger Darstellung) der Arbeiten
  \end{itemize}
\end{minipage}
\end{tabular}
\end{frame}

\begin{frame}
\frametitle{Textsorte: Bachelorarbeit}

\begin{tabular}{c@{\hspace{5mm}}l}
  \begin{minipage}[b]{.35\linewidth}
    \fbox{\includegraphics[height=6cm,width=4cm]{clb-template}}
  \end{minipage}

&
\begin{minipage}[b]{.6\linewidth}
  \begin{itemize}
  \item 15--30 Seiten
  \item Im Rahmen der Bachelorarbeit wird ein Projekt mit einem Arbeitsaufwand
    von 500 Stunden abgewickelt, die Bachelorarbeit beschreibt dieses Projekt
  \item Üblicherweise ist das Bachelorprojekt ein Programmierprojekt
  \item Kein Anspruch auf Originalität, aber Darstellung der erzielten Ergebnisse
  % \item Jede Arbeitsgruppe @ IFI hat ihre eigenen Regeln, in CL wird besonders Wert
  %   auf eine knappe, aber präzise Darstellung der Probleme und ihre Lösung gelegt
  % \item Dabei wird erwartet, dass nur Lösungen/Implementierungen dargestellt werden,
  %   die sich von den offensichtlichen Lösungen abheben
  \item Der Übergang von einer Seminararbeit zur Bachelorarbeit kann, je nach Thema, fließend sein
  \end{itemize}
\end{minipage}
\end{tabular}
\end{frame}

\begin{frame}
\frametitle{Textsorte: Masterarbeit}

\begin{tabular}{c@{\hspace{5mm}}l}
  \begin{minipage}[b]{.35\linewidth}
    \fbox{\includegraphics[height=6cm,width=4cm]{clm-template}}
  \end{minipage}

&
\begin{minipage}[b]{.6\linewidth}
  \begin{itemize}
  \item 60--100 Seiten
  \item Zusammenfassung, Erläuterung, und eventuell Implementierung bestehender 
    wissenschaftlicher Arbeiten
  \item Im Gegensatz zu einer Seminarabeit wird in der Masterarbeit erwartet, dass
    neue Erkenntnisse eingebracht werden
  \item Eigener Beitrag besteht meist in der Aufbereitung, aber auch
    Verallgemeinerung der Arbeiten
  \item Idealerweise führen Masterarbeiten direkt zu (wissenschaftlichen) Veröffentlichungen
  \end{itemize}
\end{minipage}
\end{tabular}
\end{frame}

\section{Thema analysieren und in Form bringen}
\begin{frame}[fragile]
\frametitle{\enquote{Dem Inhalt eine Struktur geben}}

\begin{mybox}{Inhaltsverzeichnis}
  \small
  \color{green!30!black}
  \begin{verbatim}
    \tableofcontents
  \end{verbatim}
\vspace{-8mm}
\end{mybox}

\bigskip
\begin{mybox}{Einleitung}
Hier wird die Arbeit in Kurzform vorgestellt und motiviert 
\end{mybox}

\bigskip
\begin{mybox}{Hauptteil}
Beschreibung und Analyse des Themas  
\end{mybox}

\bigskip
\begin{mybox}{Schlussfolgerung}
Wiederholung des Themas und Analyse in Bezug auf die Motivation  
\end{mybox}

\bigskip
\begin{mybox}{Literaturverzeichnis}
  \small
  \color{green!30!black}
  \begin{verbatim}
    \bibliographystyle{plain}
    \bibliography{references}
  \end{verbatim}
\vspace{-8mm}
\end{mybox}
\end{frame}

\begin{frame}
\frametitle{Einleitung}  
\framesubtitle{Hier wird die Arbeit in Kurzform vorgestellt und motiviert}

\begin{itemize}
\item<2-> Seien Sie sehr präzise, wenn Sie die Einleitung schreiben
\item<3-> Die Leserin muss eine Idee dafür bekommen, welche Themen die Arbeit behandelt
\item<4-> Die Einleitung endet mit einer detaillierten Beschreibung der Struktur
der Arbeit
\item<5-> Schreiben Sie die Einleitung zuletzt
\item<6-> Gleiches gilt für die Zusammenfassung
\end{itemize}

\medskip
\onslide<7->
\begin{Beispiel}  
\GREEN{% 
This  document gives some  hints on  how to  structure and  organize a
thesis.   It does  not contain  explicit help  on \LaTeX.   For that
issue please refer to a short introduction in German~\cite{LAT01} or a
not  so  short  introduction  in English~\cite{LAT02}.   To  ensure  a
uniform  layout   this  note  further  fixes   some  conventions  when
typesetting in \LaTeX\ and lists some useful packages.%
}
\end{Beispiel}
\end{frame}

\begin{frame}
\frametitle{Hauptteil}  
\framesubtitle{Beschreibung und Analyse des Themas}

\begin{mybox}{Strukturierung}
\begin{itemize}
\item<2-> Strukturieren Sie die Arbeit in Kapitel und Unterkapitel, sodass ein Kapitel
  eine logische Einheit beschreibt
\item<3-> Beginnen Sie Sektionen mit einem kurzen Absatz, der den Inhalt
  beschreibt
\item<4-> Vermeiden Sie zu lange beziehungsweise zu kurze Kapitel
\end{itemize}
\end{mybox}

\onslide<5->
\bigskip
\begin{mybox}{Formatierung}
  \begin{itemize}
  \item<5-> Auch im Englischen werden die Worte in Überschriften 
groß geschrieben
  \item<6-> Verwenden Sie dedizierte Umgebungen für Programmlistings, Tabellen, Grafiken, etc.
  \end{itemize}
\end{mybox}
\end{frame}

\begin{frame}
\frametitle{Schlussfolgerung}
\framesubtitle{Wiederholung des Themas und Analyse in Bezug auf die Motivation}

\begin{itemize}
\item<2-> Die Themen der Arbeit werden noch einmal vorgestellt
\item<3-> Die Ergebnisse der Arbeit werden mit der Motivation in der Einleitung
  verglichen
\item<4-> Beschreiben Sie die eigenen Arbeit
\item<5-> Eventuell gehen Sie auf zukünftige Arbeit und ähnliche Arbeiten ein
\item<6-> Schreiben Sie die Schlussfolgerung zuletzt
\end{itemize}

\onslide<7->
\bigskip
\begin{Beispiel}
\GREEN{
This note gives a comprehensive guide for computational logic students on
how to organize their scientific documents. In order to get started
with \LaTeX\ some useful packages are mentioned.
}  
\end{Beispiel}
\end{frame}

\begin{frame}
\frametitle{Literaturverzeichnis}
\small
% \bibliographystyle{plain}
% \bibliography{biblio}
\begin{thebibliography}{1}

\bibitem{LAT02}
T.~Oetiker, H.~Partl, I.~Hyna, and E.~Schlegl.
\newblock The not so short introduction to {L}a{T}e{X}, 2007.
\newblock \texttt{ctan.org/tex-archive/info/lshort/english}.

\bibitem{LAT01}
W.~Schmidt, J.~Knappen, H.~Partl, and I.~Hyna.
\newblock {L}a{T}e{X}-{K}urzbeschreibung, 2003.
\newblock \texttt{ctan.org/tex-archive/info/german/LaTeX2e-Kurzbeschreibung}.

\end{thebibliography}

\end{frame}

\begin{frame}
\frametitle{Proseminaraufgabe (für den 27.~April)}    

\begin{enumerate}
\item Lesen Sie das Kapitel \enquote{Lust statt Last: Wissenschaftliche Texte schreiben}
von Norbert Frank, Sektion 4
\item Lesen Sie \enquote{How to Write a Thesis} von Harald Zankl
\end{enumerate}
\end{frame}
%%% Local Variables:
%%% mode: latex
%%% TeX-master: "slides"
%%% End:

